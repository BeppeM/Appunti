\documentclass[a4paper,12pt, oneside]{book}

%\usepackage{fullpage}
\usepackage[italian]{babel}
\usepackage[utf8]{inputenc}
\usepackage{amssymb}
\usepackage{amsthm}
\usepackage{graphics}
\usepackage{amsfonts}
\usepackage{amsmath}
\usepackage{amstext}
\usepackage{engrec}
\usepackage{rotating}
\usepackage{verbatim}
\usepackage[safe,extra]{tipa}
\usepackage{showkeys}
\usepackage{multirow}
\usepackage{hyperref}
\usepackage{microtype}
\usepackage{enumerate}
\usepackage{braket}
\usepackage{marginnote}
\usepackage{pgfplots}
\usepackage{cancel}
\usepackage{polynom}
\usepackage{booktabs}
\usepackage{enumitem}
\usepackage{framed}
\usepackage{pdfpages}
\usepackage{pgfplots}
\usepackage{fancyhdr}
\pagestyle{fancy}
\fancyhead[LE,RO]{\slshape \rightmark}
\fancyhead[LO,RE]{\slshape \leftmark}
\fancyfoot[C]{\thepage}


\title{Formulario di Fisica}
\author{UniShare\\\\Davide Cozzi\\\href{https://t.me/dlcgold}{@dlcgold}}
\date{}

\pgfplotsset{compat=1.13}
\begin{document}
\maketitle

\definecolor{shadecolor}{gray}{0.80}

\newtheorem{teorema}{Teorema}
\newtheorem{definizione}{Definizione}
\newtheorem{esempio}{Esempio}
\newtheorem{corollario}{Corollario}
\newtheorem{lemma}{Lemma}
\newtheorem{osservazione}{Osservazione}
\newtheorem{nota}{Nota}
\newtheorem{esercizio}{Esercizio}
\tableofcontents

\renewcommand{\chaptermark}[1]{%
\markboth{\chaptername
\ \thechapter.\ #1}{}}
\renewcommand{\sectionmark}[1]{\markright{\thesection.\ #1}}
\chapter{Meccanica}
\subsection{Cinematica}
\subsubsection{Moto rettilineo}
\begin{itemize}
\item \textbf{velocità media: }$v_m=\frac{\Delta \vec{x}}{\Delta t}=\frac{x_2-x_1}{t_2-t_1}=\frac{\vec{v_2}-\vec{v_1}}{2}$
\item  \textbf{velocità istantanea: }$v(t)=\frac{d\vec{x}(t)}{dt}$
\item \textbf{equazione del moto rettilineo uniforme: }$x(t)=x_0+v(t-t_0)$
\item \textbf{accelerazione media:} $a_m=\frac{\vec{v_2}-\vec{v_1}}{t_2-t_1}=\frac{\Delta v}{\Delta t}$
\item \textbf{velocità moto uniformemente accelerato:} $v(t)=v_0+at$
\item \textbf{equazione del moto rettilineo uniformemente accelerato:} $$x(t)=x_0+v_0t+\frac{a}{2}t^2$$
\item \textbf{velocità finale moto uniformemente accelerato:}
$$v_{fin}^2=v_0^2+2a\Delta x$$
\end{itemize}
\newpage
\subsubsection{Moto verticale}
\begin{itemize}
\item \textbf{punto ad altezza h lasciato cadere:}
$$\vec{x}(t)=h-\frac{1}{2} g t^2$$
$$\vec{v}(t)=-gt$$
$$t_{caduta}=\sqrt{\frac{2 h}{g}}$$
$$\vec{v}_{suolo}=-\sqrt{2 g h}$$
\item \textbf{punto ad altezza h spinto in basso con una certa velocità verso il basso:}
$$\vec{x}(t)=h-\vec{v}_1t-\frac{1}{2} g t^2$$
$$\vec{v}(t)=-\vec{v}_1-gt$$
$$t_{caduta}=-\frac{\vec{v}_1}{g}+\frac{1}{g}\sqrt{\vec{v}_1^2+2gh}$$
$$v_{suolo}=-\sqrt{\vec{v_1}^2+2gh}$$
\item \textbf{punto ad altezza 0 spinto in alto con una certa velocità:}
$$\vec{x}(t)=\vec{v_2}t-\frac{1}{2} g t^2$$
$$\vec{v}(t)=\vec{v_2}-gt$$
con $v=0$ si ha l'altezza massima:
$$t_{x_{max}}=\frac{\vec{v_2}}{g}$$
e quindi:
$$x(t_{max})=\frac{1}{2}\frac{\vec{v_2}^2}{g}$$
$$t_{caduta}=\frac{\vec{v_2}}{g}$$
$$t_{tot}=t_{max}+t_c=\frac{2\vec{v_2}}{g}$$
\end{itemize}
\subsubsection{Moto nel Piano}\textbf{da sistemare}
\begin{itemize}
\item \textbf{modulo della velocità in componenti cartesiane}: $$v=|\vec{v}|=\sqrt{v_x^2+v_y^2}$$
\item \textbf{modulo della velocità in componenti cartesiane}: 
$$v=|\vec{v}|=\sqrt{v_r^2+v_q^2}$$
\item \textbf{accelerazione nel piano: }$\vec{a}=\vec{a}_T+\vec{a}_n$
\end{itemize}
\subsubsection{Moto Circolare}
\begin{itemize}
\item \textbf{velocità angolare media nel moto uniforme:} $\omega_m=\frac{\Delta\theta}{\Delta t}$
\item \textbf{velocità angolare istantanea nel moto uniforme:} $\omega=\frac{v}{R}$
\item \textbf{accelerazione centripeta (quella tangenziale è nulla) nel moto uniforme:} 
$$a=\frac{v^2}{R}=\omega^2R=\omega v$$
\item \textbf{equazioni del moto uniforme:}
$$s(t)=s_0+vt$$
$$\theta(t)=\theta_0+\omega t$$
\item \textbf{periodo:}
$$T=\frac{2\pi R}{v}=\frac{2\pi R}{\omega R}=\frac{2\pi}{\omega}$$
\item \textbf{accelerazione nel caso di moto non uniforme:}
$$\vec{a}=\vec{a}_T+\vec{a}_N$$
$$\alpha_{media}=\frac{\Delta\omega}{\Delta t}$$
$$\alpha_{istantanea}=\frac{1}{R}a_T$$
$$a_N=\omega^2 R$$
$$a_T=\alpha R$$
\item \textbf{equazioni del moto circolare non uniforme:}
$$\omega(t)=\omega_0+\alpha t$$
$$\theta(t)=\theta_0+\omega_0t+\frac{1}{2}\alpha t^2$$
$$a_N=\omega^2 R=(\omega_0+\alpha t)^2 R$$
$$|\vec{v}=\omega R$$
\end{itemize}
\subsubsection{Moto Parabolico}
\begin{itemize}
\item \textbf{moto parabolico da terra, con angolo e velocità iniziale:}
$$
\begin{cases}
v_x=v_0cos\theta_0\\
v_y=v_0sin\theta_0-gt
\end{cases}
$$
$$\begin{cases}
x(t)=(v_0cos\theta_0)t\\
y(t)=(v_0sin\theta_0)t-\frac{1}{2}gt^2
\end{cases}$$
$$t=\frac{x}{v_0cos\theta_0}$$
$$y(x)=(tan\theta_0)x-\frac{g}{2v_0^2cos^2\theta_0}x^2\mbox{ (traiettoria)}$$
$$x_G=\frac{v_0^2}{g}sin(2\theta_0)\mbox{ (gittata)}$$
$$x_G=\frac{v_0^2}{g}\mbox{ (gitatta massima)}$$
$$x_M=\frac{1}{2}\frac{v_0^2}{g}sin(2\theta_0)\mbox{ (altezza massima)}$$
$$y_M=\frac{v_0^2}{2g}sin^2\theta_0 \mbox{ (altezza massima lungo la traiettoria)}$$
$$Y_{M_{max}}=\frac{v_0^2}{2g}\mbox{ (altezza massima, la verticale)}$$
$$t_{volo}=\frac{2v_0}{g}sin\theta_0$$
$$t_{{volo}_{max}}=\frac{2v_0}{g}$$
$$\begin{cases}
v_x(t_G)=v_x(t_0)=v_0cos\theta_0\\
v_y(t_G)=-v_y(t_0)=-v_0sin\theta_0
\end{cases}\mbox{ (velocità finali)}$$
\item \textbf{moto parabolico da altezza h:}
$$\begin{cases}
x(t)=v_0t\\
y(t)=h-\frac{1}{2}gt^2
\end{cases}$$
$$\begin{cases}
v_x(t)=v_0\\
v_y(t)=-gt
\end{cases}$$
$$t_{volo}=\frac{x}{v_0}$$
$$y(x)=h-\frac{g}{2v_0^2}x^2\mbox{ (traiettoria)}$$
$$t_{caduta}=\sqrt{\frac{2h}{g}}$$
$$x(t_c)=x_G=v_0t_c=v_0\sqrt{\frac{2h}{g}}\mbox{ (gittata)}$$
$$\begin{cases}v_x(t_c)=v_0\\
v_y(t_c)=-\sqrt{2gh}\end{cases} \mbox{( velocità finali)}$$
$$v_{caduta}=\sqrt{v_0^2+2gh}$$
\end{itemize}
\end{document}