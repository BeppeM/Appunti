\documentclass[a4paper,12pt, oneside]{book}

%\usepackage{fullpage}
\usepackage[italian]{babel}
\usepackage[utf8]{inputenc}
\usepackage{amssymb}
\usepackage{amsthm}
\usepackage{graphics}
\usepackage{amsfonts}
\usepackage{amsmath}
\usepackage{amstext}
\usepackage{engrec}
\usepackage{rotating}
\usepackage[safe,extra]{tipa}
\usepackage{showkeys}
\usepackage{multirow}
\usepackage{hyperref}
\usepackage{microtype}
\usepackage{enumerate}
\usepackage{braket}
\usepackage{marginnote}
\usepackage{pgfplots}
\usepackage{cancel}
\usepackage{polynom}
\usepackage{booktabs}
\usepackage{enumitem}
\usepackage{framed}
\usepackage{pdfpages}
\usepackage{pgfplots}
\usepackage{minted}
\usepackage{fancyhdr}
\pagestyle{fancy}
\fancyhead[LE,RO]{\slshape \rightmark}
\fancyhead[LO,RE]{\slshape \leftmark}
\fancyfoot[C]{\thepage}



\title{Linguaggi di Programmazione}
\author{UniShare\\\\Davide Cozzi\\\href{https://t.me/dlcgold}{@dlcgold}\\\\Gabriele De Rosa\\\href{https://t.me/derogab}{@derogab} \\\\Federica Di Lauro\\\href{https://t.me/f_dila}{@f\textunderscore dila}}
\date{}

\pgfplotsset{compat=1.13}
\begin{document}
\maketitle

\definecolor{shadecolor}{gray}{0.80}

\newtheorem{teorema}{Teorema}
\newtheorem{definizione}{Definizione}
\newtheorem{esempio}{Esempio}
\newtheorem{corollario}{Corollario}
\newtheorem{lemma}{Lemma}
\newtheorem{osservazione}{Osservazione}
\newtheorem{nota}{Nota}
\tableofcontents
\renewcommand{\chaptermark}[1]{%
\markboth{\chaptername
\ \thechapter.\ #1}{}}
\renewcommand{\sectionmark}[1]{\markright{\thesection.\ #1}}
\chapter{Introduzione}
\textbf{Questi appunti sono presi a lezione. Per quanto sia stata fatta una revisione è altamente probabile (praticamente certo) che possano contenere errori, sia di stampa che di vero e proprio contenuto. Per eventuali proposte di correzione effettuare una pull request. Link: } \url{https://github.com/dlcgold/Appunti}.\\
\textbf{Grazie mille e buono studio!}
\chapter{I linguaggi}
Si possono classificare in 3 gruppi i linguaggi di programmazione:
\begin{enumerate}
\item \textbf{Linguaggi imperativi}, come \textit{C, Assembler, Python etc...}. Le caratteristiche dei linguaggi imperativi sono legate all'architettura di Von Neumann, composta da una componente passiva (la memoria) e una attiva (il processore). Il processore esegue calcoli e assegna valori a varie celle di memoria. Si ha quindi il concetto di \textit{astrazione}. Una variabile non è altro che un'astrazione di una cella di memoria fisica. Ogni linguaggio ha diversi livelli di astrazione dell'archiettura di Von Neumann (che ricordiamo usare il "ciclo" formato da \textit{Fetch instruction, Execute e Store result}), con i cosiddetti linguaggi di \textit{alto} e \textit{basso} livello. Possono essere sia linguaggi compilati (come \textit{C}) che interpretati (come \textit{Python}).\\
I linguaggi imperativi usano quindi il \textit{Paradigma Imperativo}, detto anche \textit{Procedurale}. In questo paradigma si adotta uno \textit{stile prescrittivo}, si prescrivono infatti operazioni che il processore deve eseguire e le istruzioni vengono eseguite in ordine, al più di strutture di controllo, e per questo è il miglior paradigma per rappresentare gli algoritmi. Questi linguaggi sono tra i più vecchi e tutt'ora tra i più usati soprattutto per la manipolazione numerica. Si ha la seguente formula che ben descrive il paradigma imperativo:
\begin{center}
\textit{Programma = Algoritmi + Strutture Dati}
\end{center}
In un linguaggio imperativo si ha sia una parte dedicata alla dichiarazione di variabili che una parte dedicata agli algoritmi risolutivi del problema. Inoltre le istruzioni possono essere così divise:
\begin{itemize}
\item istruzioni di I/O
\item istruzioni di assegnamento
\item istruzioni di controllo
\end{itemize}
La ricerca di gestire applicazioni ancora a più alto livello con codice più conciso e semplice, che affrontano i problemi in maniera più logica (o comunque in maniera differente) ha portato alla nascita di altri paradigmi. Linguaggi logici e funzionali sono accomunati dall'essere di altissimo livello, dall'essere generati per manipolazione simbolica e non numerica, dal non distinguere perfettamente programma e strutture dati, dall'essere basati su concetti matematici e sull'adottare uno \textit{stile dichiarativo}
\item \textbf{Linguaggi a oggetti}, come \textit{C++} utilizzano il paradigma ad oggetti con l'uso di classi etc. Non vengono affrontati nel corso.
\item \textbf{Linguaggi Logici}, come il \textit{Prolog}. Si basano sul concetto della deduzione logica e hanno come base la logica formale e come obbiettivo la formalizzazione del ragionamento. Programmare con un linguaggio logico significa descrivere un problema con frasi del linguaggio (ovvero con formule logiche) e interrogare il sistema che effettua deduzioni sulla base della conoscenza rappresentata. Il paradigma logico si può rappresentare con la seguente formula:
\begin{center}
\textit{Programma = Conoscenza + Controllo}
\end{center}
Si ha uno \textit{stile dichiarativo} in quanto la conoscenza del problema è espressa indipendentemente dal suo utilizzo (si usa il \textbf{cosa} e non il \textbf{come}). Si ha quindi un'alta modularità e flessibilità ma si ha la problematica della rappresentazione della conoscenza, infatti definire un linguaggio logico significa definire come il programmatore può esprimere la conoscenza e quale tipo di controllo si può utilizzare nel processo di deduzione.\\
Analizziamo le basi del Prolog:
\begin{itemize}
\item dopo ogni asserzione si mette un \textit{.} mentre le \textbf{,} sono degli \textit{and} logici
\item le costanti si indicano in minuscolo e le variabili in maiuscolo
\item \textbf{Asserzioni Incondizionate (fatti)} così indicate:
\begin{minted}{prolog}
A.
\end{minted}
\item \textbf{Asserzioni Condizionate (regole)}, che ricordiamo non essere regole di inferenza, così indicate:
\begin{minted}{prolog}
A :- B, C, D, ..., Z.
\end{minted}
dove A è il \textit{conseguente}, ovvero la conclusione, mentre le altre sono gli antecedenti, ovvero le premesse. Il simbolo ":-" è un implica che per ragioni di interprete si legge al contrario rispetto al solito: seguendo l'esempio si ha che \textit{B, C, D, ..., Z implicano A} ovvero $B, C, D, ..., Z \to A$
\end{itemize}
\item \textbf{Interrogazione}, che rappresenta l'input utente, è così espressa:
\begin{minted}{prolog}
:- K, L, M, ..., P.
\end{minted}
e indica che si chiede cosa implicano quei dati antecedenti.

Vediamo un esempio più completo (anche se non del tutto):\\
\textit{due individui sono colleghi se lavorano per la stessa 
ditta:}
\begin{minted}{prolog}
collega(X, Y) :- 
     lavora(X, Z), 
     lavora(Y, Z), 
     diverso(X, Y).

lavora(ciro, ibm).
lavora(ugo, ibm).
lavora(olivia, samsung).
lavora(ernesto, olivetti).
lavora(enrica, samsung).

:- collega(X, Y).
\end{minted}
dove la prima asserzione rappresenta la regola, le successive 5 i fatti e l'ultima riga è l'interrogazione. Il programma non è completo in quanto non si definisce concretamente \textit{diverso}. La logica di risoluzione è la seguente: L'interprete cerca un X e un Y (che sono variabili) in grado di rappresentare quella regola, infatti l'interrogazione è la conseguenza, e li cerca tra i fatti partendo dal primo ("ciro, ibm") che viene messo come X = ciro e Z = ibm. Parte il confronto con se stesso (si ha tanto la funzione \textit{diverso}) e con gli altri (che mano a mano diventeranno gli Y e Z del secondo lavora) dando alla fine come risultato solo i colleghi cercati.
\item \textbf{Linguaggi funzionali:}, come \textit{Lisp} i suoi "dialetti" come \textit{Common Lisp}, hanno come concetto primitivo la \textit{funzione} che è una regola di associazione tra due insiemi (dominio e codominio). La regola di una funzione ne specifica dominio, codominio e regola di associazione. Una funzione può essere applicata ad un elemento del dominio (detto \textit{argomento}) per restituire l'elemento del codominio associato (mediante il processo di \textit{valutazione} o \textit{esecuzione}). Nel paradigma funzionale puro l'unica applicazione è l'applicazione di funzioni e il ruolo dell'esecutore si esaurisce nel valutare l'applicazione di una funzione e produrre un valore. In questo paradigma "puro" il valore di una funzione è determinato soltanto 
dal valore degli argomenti che riceve al momento della sua applicazione, e non dallo stato del sistema rappresentato dall’insieme complessivo dei valori associati a variabili (e/o locazioni di memoria) in quel momento, comportando l'assenza di effetti collaterali. Il concetto di variabile è qui quello di \textit{costante matematica} con valori immutabili (non si ha l'operazione di assegnamento). La programmazione funzionale consiste nel combinare funzioni mediante composizioni e utilizzare la \textbf{ricorsione}. Il paradigma è ben rappresentato da questa formula:
\begin{center}
\textit{Programma = Composizione di Funzioni + Ricorsione}
\end{center}
Si ha quindi un insieme di funzioni mutualmente ricorsive e l'esecuzione del programma consiste nella valutazione dell'applicazione di una funzione principale a degli argomenti.\\
Il linguaggio \textit{Lisp}, inizialmente proposto da John McCarthy nel '58 era un linguaggio funzionale puro. Si sono poi sviluppati molti ambienti di programmazione Lisp come: \textit{Common Lisp, Scheme e Emacs Lisp}.\\
Si analizza un esempio di codice in \textit{Lisp}:
\textit{Controllare se un elemento (item) appartiene ad un insieme (rappresentato con una lista);}
\begin{minted}{lisp}
(defun member (item list)
	(cond((null list) nil)
		((equal item (first list)) T)
		(T (member item (rest list)))))
(member 42 (list 12 34 42))
\end{minted}
Si ha che tutto è rappresentato da una lista, si hanno delle funzioni standard (\textit{defun, equal, first, rest e list}) e \textit{member} che viene definita dal programmatore. L'ultima linea definisce il numero da cercare e la lista, sempre con la logica di funzioni dentro ad altre. L'esecuzione è la seguente: Si definisce, nella prima riga, la funzione che cerca un valore (\textit{item}) in una lista (\textit{list}). Nella seconda riga cominciano le condizioni: 
\begin{enumerate}
\item prima si controlla se la lista è nulla con \textit{nil} (che sarebbe falso) si ha il risultato di questo controllo. Questo rappresenta anche il caso base della nostra funzione ricorsiva.
\item si controlla se l'elemento cercato è uguale al primo della lista e con \textit{T} si indica \textit{true}
\item con l'ultima parte si ha la vera e propria ricorsione, forzata da \textit{T} iniziale, che ripete l'operazione togliendo ogni volta il primo elemento, facendo ricominciare i controlli con l'elemento successivo finché non si trova o non si ha il caso base della lista vuota
\end{enumerate}
Si nota l'assenza di assegnamenti.
\end{enumerate}
\end{document}