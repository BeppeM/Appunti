\documentclass[a4paper,12pt, oneside]{book}

%\usepackage{fullpage}
\usepackage[italian]{babel}
\usepackage[utf8]{inputenc}
\usepackage{amssymb}
\usepackage{amsthm}
\usepackage{graphics}
\usepackage{amsfonts}
\usepackage{amsmath}
\usepackage{amstext}
\usepackage{engrec}
\usepackage{rotating}
\usepackage[safe,extra]{tipa}
\usepackage{showkeys}
\usepackage{multirow}
\usepackage{hyperref}
\usepackage{microtype}
\usepackage{enumerate}
\usepackage{braket}
\usepackage{marginnote}
\usepackage{pgfplots}
\usepackage{cancel}
\usepackage{polynom}
\usepackage{booktabs}
\usepackage{enumitem}
\usepackage{framed}
\usepackage{pdfpages}
\usepackage{pgfplots}
\usepackage{fancyhdr}
\pagestyle{fancy}
\fancyhead[LE,RO]{\slshape \rightmark}
\fancyhead[LO,RE]{\slshape \leftmark}
\fancyfoot[C]{\thepage}



\title{Linguaggi e Computabilità}
\author{UniShare\\\\Davide Cozzi\\\href{https://t.me/dlcgold}{@dlcgold}\\\\Gabriele De Rosa\\\href{https://t.me/derogab}{@derogab} \\\\Federica Di Lauro\\\href{https://t.me/f_dila}{@f\textunderscore dila}}
\date{}

\pgfplotsset{compat=1.13}
\begin{document}
\maketitle

\definecolor{shadecolor}{gray}{0.80}

\newtheorem{teorema}{Teorema}
\newtheorem{definizione}{Definizione}
\newtheorem{esempio}{Esempio}
\newtheorem{corollario}{Corollario}
\newtheorem{lemma}{Lemma}
\newtheorem{osservazione}{Osservazione}
\newtheorem{nota}{Nota}
\tableofcontents
\renewcommand{\chaptermark}[1]{%
\markboth{\chaptername
\ \thechapter.\ #1}{}}
\renewcommand{\sectionmark}[1]{\markright{\thesection.\ #1}}

\chapter{Introduzione}
\textbf{Questi appunti sono presi a lezione. Per quanto sia stata fatta una revisione è altamente probabile (praticamente certo) che possano contenere errori, sia di stampa che di vero e proprio contenuto. Per eventuali proposte di correzione effettuare una pull request. Link: } \url{https://github.com/dlcgold/Appunti}.\\
\textbf{Grazie mille e buono studio!}
\section{Definizioni}
\begin{itemize}
\item un \textbf{linguaggio }è un insieme di stringhe che può essere generato mediante un dato meccanismo con delle date caratteristiche; un linguaggio può essere riconosciuto, ovvero dando in input una stringa un meccanismo può dirmi se appartiene o meno ad un linguaggio. I meccanismi che generano linguaggi si chiamano \textit{grammatiche}, quelli che li riconoscono \textit{automi}. I linguaggi formali fanno parte dell'informatica teorica \textit{(TCS)}
\item si definisce \textbf{alfabeto} come un insieme finito e non vuoto di simbolo (come per esempio il nostro alfabeto o le cifre da 0 a 9). Solitamente si indica con $\Sigma$ o $\Gamma$
\item si definisce \textbf{stringa} come una sequenza finita di simboli (come per esempio una parola o una sequenza numerica). La stringa vuota è una sequenza di 0 simboli, e si indica con $\varepsilon$ o $\lambda$
\item si definisce \textbf{lunghezza di una stringa} il numero di simboli che la compone (ovviamente contando ogni molteplicità). Se si ha $w\in \Sigma^*$ è una stringa $w$ con elementi da $\Sigma^*$ (insieme di tutte le stringhe di tutte le lunghezze possibili fatte da $\Sigma$), allora $|w|$ è la lunghezza di $w$, inoltre $|\varepsilon|=0$.
\item si definisce \textbf{potenza di un alfabeto} $\Sigma^k$ come l'insieme di tutte le sequenze (espressi come stringhe e non simboli) di lunghezza $k\in\mathbb{N},\, k>0$ ottenibili da quell'alfabeto (se $\Sigma^2$ si avranno tutte le sequenza di 2 elementi etc...). Se ho $k=1$ si ha $\Sigma^1\neq \Sigma$ in quanto ora ho stringhe e non simboli. Se ho $k=0$ ho $\Sigma^0=\varepsilon$. Dato $k$ ho $|\Sigma|$ che è la cardinalità dell'insieme $\Sigma$ (e non la sua lunghezza come nel caso delle stringhe); sia $w\in\Sigma^k=a_1,a_2,...,a_k,\,a_i\in\Sigma$ e $|\Sigma|=q$ ora: $$|\Sigma^k|=q^k$$
\item si definisce $\Sigma^*$ come\textbf{ chiusura di Kleene} che è l'unione infinita di $\Sigma^k$ ovvero $$\Sigma*=\Sigma^0\cup \Sigma^1\cup...\cup \Sigma^k$$
\item si ha che $\Sigma^+$ è l'unione per $k\geq 1$ di $\Sigma^k$ ovvero:
$$\Sigma+=\Sigma^1\cup \Sigma^2\cup...\cup \Sigma^k= \Sigma^*-\Sigma^0$$
per esempio, per l'insieme $\{0,1\}$ si ha:
$$\Sigma^*=\{\varepsilon,0,1,00,01,10,100,000,...\}$$
\item quindi un \textbf{linguaggio} \textit{L} è un insieme di stringhe e:
$$L\subseteq \Sigma^*$$ 
si hanno sottoinsiemi particolari, come l'insieme vuoto, che resta però un linguaggio, il \textbf{linguaggio vuoto} e $\emptyset\in\Sigma^k,\,|\emptyset|=0$ che è diverso dal linguaggio che contiene la stringa vuota $|\varepsilon|=1$ (che conta come una stringa). Inoltre $\Sigma^*\subseteq \Sigma^*$ che ha lunghezza infinita. Posso concatenare due stringhe con un punto: $a\cdot b\cdot c=abc$ e $a\cdot \varepsilon=a$. Ovviamente la stringa concatenata è lunga come la somma delle lunghezze delle stringhe che la compongono. Vediamo qualche esempio di linguaggio:
\begin{itemize}
\item il linguaggio di tutte le stringhe che consistono in $n$ 0 seguiti da $n$ 1:
$$\{\varepsilon,01,0011,000111,...\}$$
\item l'insieme delle stringhe con un uguale numero di 0 e di 1:
$$\{\varepsilon,01,10.0011,0101.1001,..\}$$
\item l'insieme dei numeri binari il cui valore è un numero primo:
$$\{\varepsilon,10 , 11, 101, 111,1011,...\}$$
\item $\Sigma^*$ è un linguaggio per ogni alfabeto $\Sigma$
\item $\emptyset$, il linguaggio vuoto, e $\{\varepsilon\}$ sono un linguaggio rispetto a qualunque alfabeto
\end{itemize}
\end{itemize}
\end{document}