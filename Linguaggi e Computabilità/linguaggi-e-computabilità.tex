\documentclass[a4paper,12pt, oneside]{book}

%\usepackage{fullpage}
\usepackage[italian]{babel}
\usepackage[utf8]{inputenc}
\usepackage{amssymb}
\usepackage{amsthm}
\usepackage{graphics}
\usepackage{amsfonts}
\usepackage{amsmath}
\usepackage{amstext}
\usepackage{engrec}
\usepackage{rotating}
\usepackage[safe,extra]{tipa}
\usepackage{showkeys}
\usepackage{multirow}
\usepackage{hyperref}
\usepackage{microtype}
\usepackage{enumerate}
\usepackage{braket}
\usepackage{marginnote}
\usepackage{pgfplots}
\usepackage{cancel}
\usepackage{polynom}
\usepackage{booktabs}
\usepackage{enumitem}
\usepackage{framed}
\usepackage{pdfpages}
\usepackage{pgfplots}
\usepackage{fancyhdr}

\usepackage{varwidth,pst-tree,realscripts}
\usepackage{bidi}
\psset{showbbox=false,treemode=D,linewidth=0.3pt,treesep=2ex,levelsep=0.5cm}
\newcommand{\LFTw}[2]{%
\Tr[ref=#1]{\psframebox[linestyle=none,framesep=4pt]{%
\begin{varwidth}{15ex}\center #2\end{varwidth}}}}
\newcommand{\LFTr}[2]{\Tr[ref=#1]{\psframebox[linestyle=none,framesep=4pt]{#2}}}

\def\pstreehooki{\psset{thislevelsep=*0pt}}
\def\pstreehookiii{\psset{thislevelsep=*0pt}}
\def\pstreehookv{\psset{thislevelsep=*0pt}}

\pagestyle{fancy}
\fancyhead[LE,RO]{\slshape \rightmark}
\fancyhead[LO,RE]{\slshape \leftmark}
\fancyfoot[C]{\thepage}



\title{Linguaggi e Computabilità}
\author{UniShare\\\\Davide Cozzi\\\href{https://t.me/dlcgold}{@dlcgold}\\\\Gabriele De Rosa\\\href{https://t.me/derogab}{@derogab} \\\\Federica Di Lauro\\\href{https://t.me/f_dila}{@f\textunderscore dila}}
\date{}

\pgfplotsset{compat=1.13}
\begin{document}
\maketitle

\definecolor{shadecolor}{gray}{0.80}

\newtheorem{teorema}{Teorema}
\newtheorem{definizione}{Definizione}
\newtheorem{esempio}{Esempio}
\newtheorem{corollario}{Corollario}
\newtheorem{lemma}{Lemma}
\newtheorem{osservazione}{Osservazione}
\newtheorem{nota}{Nota}
\tableofcontents
\renewcommand{\chaptermark}[1]{%
\markboth{\chaptername
\ \thechapter.\ #1}{}}
\renewcommand{\sectionmark}[1]{\markright{\thesection.\ #1}}

\chapter{Introduzione}
\textbf{Questi appunti sono presi a lezione. Per quanto sia stata fatta una revisione è altamente probabile (praticamente certo) che possano contenere errori, sia di stampa che di vero e proprio contenuto. Per eventuali proposte di correzione effettuare una pull request. Link: } \url{https://github.com/dlcgold/Appunti}.\\
\textbf{Grazie mille e buono studio!}
\section{Definizioni}
\begin{itemize}
\item un \textbf{linguaggio }è un insieme di stringhe che può essere generato mediante un dato meccanismo con delle date caratteristiche; un linguaggio può essere riconosciuto, ovvero dando in input una stringa un meccanismo può dirmi se appartiene o meno ad un linguaggio. I meccanismi che generano linguaggi si chiamano \textit{grammatiche}, quelli che li riconoscono \textit{automi}. I linguaggi formali fanno parte dell'informatica teorica \textit{(TCS)}
\item si definisce \textbf{alfabeto} come un insieme finito e non vuoto di simbolo (come per esempio il nostro alfabeto o le cifre da 0 a 9). Solitamente si indica con $\Sigma$ o $\Gamma$
\item si definisce \textbf{stringa} come una sequenza finita di simboli (come per esempio una parola o una sequenza numerica). La stringa vuota è una sequenza di 0 simboli, e si indica con $\varepsilon$ o $\lambda$
\item si definisce \textbf{lunghezza di una stringa} il numero di simboli che la compone (ovviamente contando ogni molteplicità). Se si ha $w\in \Sigma^*$ è una stringa $w$ con elementi da $\Sigma^*$ (insieme di tutte le stringhe di tutte le lunghezze possibili fatte da $\Sigma$), allora $|w|$ è la lunghezza di $w$, inoltre $|\varepsilon|=0$.
\item si definisce \textbf{potenza di un alfabeto} $\Sigma^k$ come l'insieme di tutte le sequenze (espressi come stringhe e non simboli) di lunghezza $k\in\mathbb{N},\, k>0$ ottenibili da quell'alfabeto (se $\Sigma^2$ si avranno tutte le sequenza di 2 elementi etc...). Se ho $k=1$ si ha $\Sigma^1\neq \Sigma$ in quanto ora ho stringhe e non simboli. Se ho $k=0$ ho $\Sigma^0=\varepsilon$. Dato $k$ ho $|\Sigma|$ che è la cardinalità dell'insieme $\Sigma$ (e non la sua lunghezza come nel caso delle stringhe); sia $w\in\Sigma^k=a_1,a_2,...,a_k,\,a_i\in\Sigma$ e $|\Sigma|=q$ ora: $$|\Sigma^k|=q^k$$
\item si definisce $\Sigma^*$ come\textbf{ chiusura di Kleene} che è l'unione infinita di $\Sigma^k$ ovvero $$\Sigma*=\Sigma^0\cup \Sigma^1\cup...\cup \Sigma^k$$
\item si ha che $\Sigma^+$ è l'unione per $k\geq 1$ di $\Sigma^k$ ovvero:
$$\Sigma+=\Sigma^1\cup \Sigma^2\cup...\cup \Sigma^k= \Sigma^*-\Sigma^0$$
per esempio, per l'insieme $\{0,1\}$ si ha:
$$\Sigma^*=\{\varepsilon,0,1,00,01,10,100,000,...\}$$
\item quindi un \textbf{linguaggio} \textit{L} è un insieme di stringhe e:
$$L\subseteq \Sigma^*$$ 
si hanno sottoinsiemi particolari, come l'insieme vuoto, che resta però un linguaggio, il \textbf{linguaggio vuoto} e $\emptyset\in\Sigma^k,\,|\emptyset|=0$ che è diverso dal linguaggio che contiene la stringa vuota $|\varepsilon|=1$ (che conta come una stringa). Inoltre $\Sigma^*\subseteq \Sigma^*$ che ha lunghezza infinita. Posso concatenare due stringhe con un punto: $a\cdot b\cdot c=abc$ e $a\cdot \varepsilon=a$. Ovviamente la stringa concatenata è lunga come la somma delle lunghezze delle stringhe che la compongono. Vediamo qualche esempio di linguaggio:
\begin{itemize}
\item il linguaggio di tutte le stringhe che consistono in $n$ 0 seguiti da $n$ 1:
$$\{\varepsilon,01,0011,000111,...\}$$
\item l'insieme delle stringhe con un uguale numero di 0 e di 1:
$$\{\varepsilon,01,10.0011,0101.1001,..\}$$
\item l'insieme dei numeri binari il cui valore è un numero primo:
$$\{\varepsilon,10 , 11, 101, 111,1011,...\}$$
\item $\Sigma^*$ è un linguaggio per ogni alfabeto $\Sigma$
\item $\emptyset$, il linguaggio vuoto, e $\{\varepsilon\}$ sono un linguaggio rispetto a qualunque alfabeto
\end{itemize}
\end{itemize}
Prendiamo un alfabeto $\Sigma=\{0, 1\}$ con la sua chiusura di Kleen $\Sigma=\{0, 1\}^*$. Quando si ha un input si può avere un problema di decisione, \textit{P}, che dia come output "si" o "no". Posso avere un problema di decisione (o \textit{membership}) su $w\in\Sigma=\{0, 1\}^*$, con \textit{w} stringa, che dia in output "si" o "no". Un linguaggio \textit{L} sarà:
$$L=\{w\in\{0, 1\}^*\,|\,\, P(w)=si$$
quindi si ha che:
$$\Sigma^*\backslash L=\{P(w)=no\}$$
Vediamo ora un esempio di \textit{Context Free Language (CFL)}, costruito a partire da una \textit{Context Free Grammar (CFG)}:
\begin{esempio}
Sia $\Sigma=\{0, 1\}$ e $L_{pal}="stringhe\,\, palindrome\,\, binarie"$.
Quindi, per esempio, $0110\in L,\,\, 11011\in L$ ma $10010\not\in L$. Si ha che $\varepsilon$, la stringa vuota, appartiene a $L$. Diamo una definizione ricorsiva:
\begin{itemize}
\item \textbf{base:} $\varepsilon,\, 0\,\ 1\in L_{pal}$
\item \textbf{passo:} se $w$ è palindroma allora $0w0$ è palindromo e $1w1$ è palindromo
\end{itemize}
una variabile generica $S$ può sottostare alle \textit{regole di produzione} di una certa grammatica. In questo caso si ha uno dei seguenti:
$$S\to\varepsilon,\, S\to 0,\, S\to 1,\, S\to 0S0,\, S\to 1S1$$
\end{esempio}
Si ha che una grammatica $G$ è una quadrupla $G=(V,\,T,\,P,\,S$ con:
\begin{itemize}
\item $V$ simboli variabili
\item $T$ simboli terminali, ovvero i simboli con cui si scrivono le stringhe alla fine
\item $P$ regole di produzione
\item $S$ variabile di partenza \textit{start}
\end{itemize}
riprendiamo l'esempio sopra:
\begin{esempio}
$$G_{pal}=(V=\{S\},\, T=\{0, 1\},\, P,\, S)$$
con:
$$P=\{S\to\varepsilon,\, S\to 0,\, S\to 1,\, S\to 0S0,\, S\to 1S1\}$$
Si può ora costruire un algoritmo per creare una stringa palindroma a partire dalla grammatica $G$:
$$\underbrace{S}_{\mbox{start}}\underbrace{\to}_{\mbox{applico una regola}} 1S1 \to 01S10\to \underbrace{01010}_{\mbox{sostituisco variabile}}$$

con $S,\, 1S1\,\, e\,\, 01S10$ che sono \textit{forme sentenziali}. Posso così ottenere tutte le possibili stringhe. Esiste anche una forma abbreviata:
$$S\to \varepsilon|o|1|0S0|1S1$$
Non si fanno sostituzioni in parallelo, prima una $S$ e poi un'altra
\end{esempio}
%aggiungi esempio parentesi
Si hanno 4 grammatiche formali, \textit{gerarchia di Chomsky}:
\begin{itemize}
\item \textbf{tipo 0:} non si hanno restrizioni sulle regole di produzione, $\alpha\to\beta$. Sono linguaggi ricorsivamente numerabili e sono rappresentati dalle \textit{macchine di Turing}, deterministiche o non deterministiche (la macchina di Turing è un automa)
\item \textbf{tipo 1:}  il lato sinistro della produzione (\textit{testo}) ha lunghezza uguale a quello destro (\textit{corpo}). Sono grammatiche dipendenti dal contesto (\textit{contestuali}) e come automa hanno\textit{ la macchina di Turing che lavora in spazio lineare}:
$$\alpha_1A\alpha_2\to \alpha_1B\alpha_2$$
con $\alpha_1$ e $\alpha_2$ detti \textit{contesto} e $\alpha_1,\,\alpha_2,\, \beta\in (V\cup T)^*$
\item \textbf{tipo 2:} sono quelle libere dal contesto, context free. Come regola ha $A\to\beta$ con $A\in V$ e $\beta\in V\cup T)^*$ e come automa ha gli \textit{automi a pila non deterministici}
\item \textbf{tipo 3:} sono le grammatiche \textit{regolari}. Come regole ha $A\to\alpha B$ (o $A\to B\alpha$) e $A\to\alpha$  con $A,B\in V$ e $\alpha\in T$. Come automi ha gli \textit{automi a stato finito deterministici o non deterministici}
\end{itemize}
%aggiungi esercizio
\newpage
\begin{esempio}
Sia $G=(V,T,O,E)$, con $V=\{E,I\}$ e $T=\{a,b,0,1,(,),+,*\}$ 
quindi ho le seguenti regole, è di tipo 3:
\begin{enumerate}
\item $E\to I$
\item $E\to E+E$
\item $E\to E*E$
\item $E\to (E)$
\item $I\to a$
\item $I\to b$
\item $I\to Ia$
\item $I\to Ib$
\item $I\to I0$
\item $I\to I1$
\end{enumerate}
voglio ottenere $a*(a+b00)$ 
sostituisco sempre a destra (right most derivation)
$$E\to E*E\to E*(E)\to E*(E+E)\to E*(E+I)\to E+(E+I0)$$
$$\to R+(I+b00)\to E*(a+b00)\to I*(a+b00)\to a*(a+b00)$$

usiamo ora \textit{l'inferenza ricorsiva}:
\begin{center}
\begin{tabular}{|c|c|c|c|c|}
\hline
passo & stringa ricorsiva & var & prod & passo stringa impiegata\\
1 & a & I & 5 & $\backslash$ \\
\hline
2 & b & I & 6 & $\backslash$ \\ 
\hline
3 & b0 & I & 9 & 2\\
\hline
4 & b00 & I & 9 & 3\\
\hline
5 & a & E & 1 & 1 \\
\hline
6 & b00 & E & 1 & 4\\
\hline
7 & a+b00 & E & 2 & 5,6\\
\hline
8 & (a+b00) & E & 4 & 7\\
\hline
9 &a*(a+b00) & E & 3 & 5, 8\\
\hline
\end{tabular}
\end{center}
\end{esempio}
definisco formalmente la derivazione $\to$:
\begin{definizione}
Prendo una grammatica $G=(V,T,P,S)$, grammatica CFG. Se $\alpha A \beta$ è una stringa tale che $\alpha,\beta\in (V\cup T)^*$, appartiene sia a variabili che terminali. Sia $A\in V$ e sia $a\to \gamma$ una produzione di $G$. Allora 
scriviamo:
$$\alpha A \beta \to \alpha\gamma\beta$$
con $\gamma\in (V\cup T)^*$.\\
Le sostituzioni si fanno indipendentemente da $\alpha$ e $\beta$.
Questa è quindi la definizione di derivazione.
\end{definizione}
\begin{definizione}
Definisco il simbolo $\to +$, ovvero il simbolo di \textit{derivazioni in 0 o più passi}. Può essere definito in modo ricorsivo. Per induzione sul numero di passi. 

\begin{itemize}
\item la base dice che  $\forall \alpha\in (V\cup T)^*,\, \alpha\to * \,\alpha$
\item il passo è: se $\alpha\to_G * \,\beta $ e $ \beta \to_G * \,\gamma$ allora $\alpha\to * \,\gamma$
\end{itemize}
Si può anche dire che $\alpha\to_G +\, \beta$ sse esiste una sequenza di stringhe $\gamma_1,...,\gamma_n$ con $n\geq 1$ tale che $\alpha=\gamma_1$, $\beta=\gamma_n$ e $\forall i,\, 1<i<n-1$ si ha che $\gamma_1\to \gamma_{i+1}$
la derivazione in 0 o più passi è la chiusura transitiva della derivazione
\end{definizione}
\begin{definizione}
avendo ora definito questi simboli possiamo definire una forma sentenziale. Infatti è una stringa $\alpha$ tale che:
$$\forall \alpha\in (V\cup T)^* \mbox{ tale che }S\to_G *\, \alpha$$
\end{definizione}
\begin{definizione}
data $G=(V,T,P,S)$ si ha che $L(G)=\{w\in T^* |\, S\to_G *\, w\}$ ovvero composto da stringhe terminali che sono derivabili o 0 o più passi.
\end{definizione}
\begin{esempio}
formare una grammatica CFG per il linguaggio:
$$L=\{0^n 1^n| n\geq 1\}=\{01, 0011, 000111,...\}$$
con $x^n$ intendo una concatenazione di $n$ volte $x$ (che nel nostro caso sono 0 e 1).\\
posso scrivere:
$$0^n 1^n =00^{n-1} 1^{n-1}1$$
il nostro caso base sarà la stringa $01$, Poi si ha:
$G=/V,T,P,S)$, $T=\{0,1\}$, $V=\{S\}$, il caso base $S\to 01$  e $S\to 0S1$
il caso passo è quindi: se $w= 0^{n-1}1^{n-1}\in L$ allora $0w1\in L$.\\
Ora voglio dimostare che $000111\in L$, ovvero $S\to*\, 000111$:\\
$$S\to\, 0S1 \to 00S11\to 000S111$$
\end{esempio}
\begin{teorema}
data la grammatica $G=\{V,T,P,S)$ CFG e $\alpha\in (V\cup T)^*$. Si ha che vale $S\to*\, \alpha$ sse $S\to_{lm}*\, \alpha$ sse $S\to_{rm}*\, \alpha$. Con $to_{lm}*$ simbolo di \textit{left most derivation }e $to_{rm}*$ simbolo di \textit{right most derivation }
\end{teorema}
\begin{esempio}
formare una grammatica CFG per il linguaggio:
$$L=\{0^n 1^n| n\geq 0\}=\{\varepsilon, 01, 0011, 000111,...\}$$
stavolta abbiamo anche la stringa vuota. Il caso base stavolta è $S\to\varepsilon| \, 0S1$ 
\end{esempio}
\begin{esempio}
Fornisco una CFG per $L=\{a^n|n\geq 1\}=\{a, aa, aaa,...\}$.
La base è $a$ \\il passo è che se $a^{n-1}\in L$ allora $a^{n-1}a\in L$ ( o che $aa^{n-1}\in L$).\\
Si ha la grammatica $G=\{V,T,P,S)$, $V=\{S\}$, $T=\{a\}$ e si hanno $S\to a|\,Sa$ (o  $S\to a|\,aS$). Dimostro che $a^3\in L$.
$$S\to Sa \to Saa\to aaa$$
oppure 
$$S\to aS\to aaS\to aaa$$
\end{esempio}
\begin{esempio}
trovo una CFG per $L=\{(ab)^n|n\geq 1\}=\{ab, abab, ababab,...\}$\\
La base è $ab$ \\il passo è che se $(ab)^{n-1}\in L$ allora $(ab)^{n-1}ab\in L$.\\
Si ha la grammatica $G=\{V,T,P,S)$, $V=\{S\}$, $T=\{a,b\}$ (anche se in realtà $T=\{ab\}$) e si hanno $S\to ab|\,Aab$. Poi dimostro come l'esempio sopra
\end{esempio}
\begin{esempio}
trovo una CFG per $L=\{a^n c b^n|n\geq 1\}=acb,aacbb,aaacbbb,...\}$\\
Il caso base è $acb$ il passo è che se $a^{n-1}cb^{n-1}\in L$ allora $a^{n-1}cb^{n-1}acb\in L$ 
Si ha la grammatica $G=\{V,T,P,S)$, $V=\{S\}$, $T=\{a,b,c\}$ e si hanno $S\to aSb|acb$.\\
dimostro che $aaaacbbbbb\in L$:
$$S\to aSb\to aaSbb\to aaaaSbbb\to aaaacbbbb$$

provo a usare anche una grammatica regolare, con le regole $S\to aS|c$, $c\to cB$ e $B\to bB|b$;
$$S\to aS\to aaS\to aaC\to aacB\to aacb...$$
non si può dimostrare in quanto non si può imporre una regola adatta
\end{esempio}
\begin{esempio}
$L=\{a^n c b^{n-1}|n\geq 2\}$, con $a^n c b^{n-1}=a^{n-1}acb^{n-1}$. $S\to aSb|aacb$. Quindi:
$$S\to aSb\to aaaccbb\in L$$
\end{esempio}
\begin{esempio}
cerco CFG per $L=\{a^n c^k b^n|\,n,\,k>0\}$. $a$ e $b$ devono essere uguali, uso quindi una grammatica context free, mentre $c$ genera un linguaggio regolare.\\
Si ha la grammatica $G=\{V,T,P,S)$, $V=\{S,C\}$, $T=\{a,b,c\}$ e si hanno $S\to aSb|aCb$ e $C\to cC|c$. dimostro che $aaaccbbb\in L, n=3,\, k=2$:
$$S\to aSb \to aaSbb\to aaaCbbb\to aaacCbbb\to aaaccbbb$$
\end{esempio}
\begin{esempio}
scrivere CFG per $L=\{a^nb^nc^kb^k|\, n,\,k\geq 0\}
$
$$=\{w\in\{a,b,c,d\}^*|\,a^nb^nc^kb^k|\, n,\,k\geq 0\}$$
quindi L concatena due linguaggi $L1$ e $L2$, $X=\{a^nb^n\}$ e $Y=\{c^kd^k\}$: 
$$X\to aXb | \varepsilon$$
$$Y\to cYd | \varepsilon$$
$$S\to XY$$
voglio derivare $abcd$:
$$S\to XY \to XcYd\to aXbcYd\to aXbc\varepsilon d\to a\varepsilon bc\varepsilon d\to abcd$$
voglio derivare $cd$
$$S\to XY\to Y\to cYd\to cd$$
\end{esempio}
Quindi se ho $w\in L1, L2$, ovvero appartenente ad una concatenazione di linguaggi prima uso le regole di un linguaggio, poi dell'altro e infine ottengo il risultato finale.\\
\begin{esempio}
scrivere CFG per $L=\{a^nb^kc^kd^n|\, n>0,\, k\geq 0\}
$.
$$S\to aSd|\, aXd$$
$$X\to bXc| \varepsilon$$
derivo $aabcdd$:
$$S\to aSd\to aaXdd\to aabXcdd\to aabcdd$$
\end{esempio}
\begin{esempio}
scrivere CFG per $L=\{a^ncb^nc^mad^m|\, n>0,\, m\geq 1\}
$.
$$S\to XY$$
$$X\to aXb|c$$
$$Y\to cUd| cad$$
$$S\to XY\to cY\to ccad$$
\end{esempio}
\begin{esempio}
scrivere CFG per $L=\{a^{n+m}xc^nyd^m|\, n,\, m\geq 0\}
$. $a^{n+m}=a^na^m \mbox{ o } a^ma^n$. Si hanno 2 casi:
\begin{enumerate}
\item $L=\{a^na^m xc^nyd^m|\, n,\, m\geq 0\}
$
\item $L=\{a^ma^n xc^nyd^m|\, n,\, m\geq 0\}
$
\end{enumerate}
ma solo  $L=\{a^ma^n xc^nyd^m|\, n,\, m\geq 0\}
$ può generare una CFG (dove non si possono fare incroci, solo concatenazioni e inclusioni/innesti). 
$$S\to aSd| Y$$
$$Y\to Xy$$
$$X\to aXc|x$$ 
si può fare in 2:
$$S\to aSd| Xy$$
$$X\to aXc|x$$ 
derivo con $m=n=1$, $aaxcyd$:
$$S\to aSd\to aXyd\to aaXcyd\to aaxcyd$$
\end{esempio}
\begin{esempio}
scrivere CFG per $L=\{a^nb^m|\, n\geq m \geq 0\}
$.$$L=\{\varepsilon, a, ab, aa, aab, aabb, aaa, aaab, aaabb, aaabbb,...\}$$
Se $n\geq m$ allora $\exists k\geq 0 \to n=m+k$. Quindi:
$$l=\{a^{m+k}b^m|m,k\geq0\}$$ si può scrivere in 2 modi:
\begin{enumerate}
\item $l=\{a^ma^kb^m|m,k\geq0\}$ quindi con innesto
\item $l=\{a^ka^mb^m|m,k\geq0\}$quindi con concatenazione
\end{enumerate}
entrambi possibili per una CFG:
\begin{enumerate}
\item 
$$S\to XY$$
$$X\to aX|\varepsilon \mbox{ si può anche scrivere } X\to Xa|\varepsilon$$
$$Y\to aYb|\varepsilon$$ 
oppure 
$$S\to aS|X$$
$$X\to aXb| \varepsilon$$
\item 
$$S\to aSb|\varepsilon$$
$$X\to aX|\varepsilon$$
\end{enumerate}
\end{esempio}
\begin{esempio}
scrivere CFG per $L=\{a^nb^{m+n}c^h|\, m>h\geq0,\, n\geq0\}
$.\\
Se $n>h$ allora $\exists k \to n= h+k$, quindi:
$$L=\{a^nb^{m+h+k}c^h|\, m>h\geq0,\, n\geq0\}$$. ovvero:
$$L=\{a^nb^nb^kb^hc^h|\, m\geq 0, k>0, h\geq 0\}$$
si ha:
$$S\to XYZ$$
$$X\to aXb|\varepsilon$$
$$Y\to Yb|b$$
$$Z\to bZc|\varepsilon$$
si può anche fare:
$$S\to XY$$
$$X\to aXb|\varepsilon$$
$$Y\to bYc|Z$$
$$Z\to bZ|b$$
\end{esempio}
\begin{esempio}
scrivere CFG per $L=\{a^nb^mc^k|\, k>n+m,\, n,m\geq 0\}
$.\\
per $n=m=0,\, k=1$ avrò la stringa $c$.
se $k>n+m$ allora $\exists l>0\to k=n+m+l$ quindi:
$$L=\{a^nb^mc^{n+m+l}|\, l>0,\, n,m\geq 0\}
$$
$$=L=\{a^nb^mc^nc^mc^l|\, l>0,\, n,m\geq 0\}$$
sistemando:
$$=L=\{a^nb^mc^lc^mcnl|\, l>0,\, n,m\geq 0\}$$
quindi:
$$S\to aSc|X$$
$$X\to bXc|Y$$
$$Y\to cY|c$$
\end{esempio}
\newpage
\begin{esempio}
scrivere CFG per $L=\{a^nxc^{n+m}y^hz^kd^{m+h}|\, n,m,k,h\geq 0\}
$.\\
ovvero:
$$L=\{a^nxc^nc^my^hz^kd^hd^m|\, n,m,k,h\geq 0\}$$
quindi avrò:
$$S\to XY$$
$$X\to aXc|x$$
$$Y\to cYd|W$$
$$W\to yWd|X$$
$$Z\to zZ|\varepsilon$$
\end{esempio}
\begin{esempio}
vediamo un esempio di grammatica dipendente dal contesto:
$$L=\{a^nb^nc^n|\, n\geq 1\}$$
$G=\{V,T,P,S\}=\{(S,B,C,X)\}=\{(a,b,c),P,S\}$
ecco le regole di produzione (qui posso scambiare variabili a differenza delle context free):
\begin{enumerate}
\item $S\to aSBC$
\item $S\to aBC$
\item $CB\to XB$
\item $XB\to XC$
\item $XC\to BC$
\item $aB\to ab$
\item $bB\to bb$
\item $bC\to bc$
\item $cC\to cc$
\end{enumerate}
vediamo un esempio di derivazione:
per $n=1$ ho $abc$ ovvero:
$$S\to aBC\to abC\to abc$$
con $n=2$ ho $aabbcc$:
$S\to aSBC\to aaBCBC\to aaBXBC\to aaBXCC\to aaBBCC\to aabBCC\to aabbCC\to aabbcC\to aabbcc$\\
%vedere dimostrazione pag 14 soligo
\end{esempio}
\newpage
\begin{esempio}
vediamo un esempio di grammatica dipendente dal contesto:
$$L=\{a^nb^mc^nd^m|\, n,m\geq 1\}$$
Si ha:
$$G=(\{S,X,C,D,Z\},\{a,b,c,d\},P,S)$$
con le seguenti regole di produzione:
\begin{itemize}
\item $S\to aSc|\, aXc$
\item $X\to bXD|\, bD$
\item $DC\to CD$
\item $DC\to DZ$
\item $DZ\to CZ$
\item $XZ\to CD$
\item $bC\to bc$
\item $cC\to cc$
\item $cD\to cd$
\item $dD\to dd$
\end{itemize}
provo a derivare $aabbbccddd$ quindi con $n=2,\,m=3$:\\
$$S\to aSC\to aaXCC\to aabXDCC\to aabbXDDCC\to $$
$$aabbbDDDCC\to aabbbCCDDD\to aabbbccddd$$
\end{esempio}
\begin{esempio}
Sia $L=\{w\in\{a,b\}^*|\, \mbox{ w contiene lo stesso numero di a e b}\}$:
$$S\to aSbS|\,bSaS|\, \varepsilon$$
dimostro per induzione che è corretto:
\begin{itemize}
\item \textbf{caso base:} $|w|=0\to w=\varepsilon$
\item \textbf{caso passo:} si supponga che $G$ produca tutte le stringhe (di lunghezza $<$ di $n$) di $\{a,b\}^*$ con lo stesso numero di \textit{a} e \textit{b} e dimostro che produce anche quelle di lunghezza $n$, sia:
$$w\in \{a,b\}^* \mid\, |w|=n \mbox{ con\textit{ a} e \textit{b} in egual numero, }m(a)=m(b) \mbox{ con m() che indica il numero di caratteri}$$
quindi si ha che:
$$w=aw_1bw_2\mbox{ o } w=bw_1aw_2$$
sia.
$$k_1=m(a)\in w_1=m(b)\in w_1$$
$$k_2=m(a)\in w_2=m(b)\in w_2$$
allora:
$$k_1+k_2+1=m(a)\in w= m(b)\in W$$
sapendo che $|w_1|<n$ e $|w_2|<n$ allora $w_1$ e $w_2$ sono egnerati da G per ipotesi induttiva
\end{itemize}
\end{esempio}
\subsection{Alberi Sintatici}
\begin{definizione}
Data una grammatica CFG, $G=\{V,T,P,S\}$ un \textbf{albero sintattico} per $G$ soddisfa le seguenti condizioni:
\begin{itemize}
\item ogni nodo interno è etichettato con una variabile
\item ogni foglia è anch'essa etichettata con una variabile o col simbolo di terminale T o con la stringa vuota $\varepsilon$ (in questo caso la foglia è l'unico figlio del padre)
\item se un nodo interno è etichettato con A i suoi figli saranno etichettati con X1, ..., Xk e $A\to  X1, ..., Xk$ sarà una produzione di $G$. Se un Xi è $\varepsilon$ sarà l'unica figlio e $A\to \varepsilon$ sarà comunque una produzione di $G$
\end{itemize}
La concatenazione in ordine delle foglie viene detto \textbf{prodotto dell'albero}
\end{definizione}
\newpage
\begin{esempio}
Usiamo l'esempio delle stringhe palindrome:
$$P\to 0P0|\,1P1|\varepsilon$$
sia il seguente albero sintatico:
\begin{center}
\psframebox[linestyle=none,framesep=10pt]{%
\pstree{\LFTw{t}{\fontspec{Noto Sans}[Script=Latin]P}}{\pstree{\Tp[edge=none]}{%
  \LFTw{t}{\fontspec{Noto Sans}[Script=Latin]0}
  \pstree{\LFTw{t}{\fontspec{Noto Sans}[Script=Latin]P}}{\pstree{\Tp[edge=none]}{%
    \LFTw{t}{\fontspec{Noto Sans}[Script=Latin]1}
    \pstree{\LFTw{t}{\fontspec{Noto Sans}[Script=Latin]P}}{\pstree{\Tp[edge=none]}{%
      \LFTw{t}{\fontspec{Noto Sans}[Script=Latin]$\varepsilon$}}}
    \LFTw{t}{\fontspec{Noto Sans}[Script=Latin]1}}}
  \LFTw{t}{\fontspec{Noto Sans}[Script=Latin]0}}}}
\end{center}
\end{esempio}
\begin{esempio}
Si ha:
$$E\to I|\, E+E|\, E*E|\, (E)$$
$$I\to a|\,b|\,Ia|\,Ib|\,I0|\,I1$$
un albero sintattico per $a*(a+b00)$ può essere:
\begin{center}

\psframebox[linestyle=none,framesep=10pt]{%
\pstree{\LFTw{t}{\fontspec{Noto Sans}[Script=Latin]E}}{\pstree{\Tp[edge=none]}{%
  \pstree{\LFTw{t}{\fontspec{Noto Sans}[Script=Latin]E}}{\pstree{\Tp[edge=none]}{%
    \pstree{\LFTw{t}{\fontspec{Noto Sans}[Script=Latin]I}}{\pstree{\Tp[edge=none]}{%
      \LFTw{t}{\fontspec{Noto Sans}[Script=Latin]a}}}}}
  \LFTw{t}{\fontspec{Noto Sans}[Script=Latin]*}
  \pstree{\LFTw{t}{\fontspec{Noto Sans}[Script=Latin]E}}{\pstree{\Tp[edge=none]}{%
    \LFTw{t}{\fontspec{Noto Sans}[Script=Latin](}
    \pstree{\LFTw{t}{\fontspec{Noto Sans}[Script=Latin]E}}{\pstree{\Tp[edge=none]}{%
      \pstree{\LFTw{t}{\fontspec{Noto Sans}[Script=Latin]E}}{\pstree{\Tp[edge=none]}{%
        \pstree{\LFTw{t}{\fontspec{Noto Sans}[Script=Latin]I}}{\pstree{\Tp[edge=none]}{%
          \LFTw{t}{\fontspec{Noto Sans}[Script=Latin]a}}}}}
      \LFTw{t}{\fontspec{Noto Sans}[Script=Latin]+}
      \pstree{\LFTw{t}{\fontspec{Noto Sans}[Script=Latin]E}}{\pstree{\Tp[edge=none]}{%
        \pstree{\LFTw{t}{\fontspec{Noto Sans}[Script=Latin]I}}{\pstree{\Tp[edge=none]}{%
          \pstree{\LFTw{t}{\fontspec{Noto Sans}[Script=Latin]I}}{\pstree{\Tp[edge=none]}{%
            \LFTw{t}{\fontspec{Noto Sans}[Script=Latin]b}}}
          \LFTw{t}{\fontspec{Noto Sans}[Script=Latin]0}}}
        \LFTw{t}{\fontspec{Noto Sans}[Script=Latin]0}}}}}
    \LFTw{t}{\fontspec{Noto Sans}[Script=Latin])}}}}}}
\end{center}
\end{esempio}
\newpage
Data una CFG si ha che i seguenti cinque enunciati si equivalgono:
\begin{enumerate}
\item la procedura di inferenza ricorsiva stailisce che una stringa $w$ di simboli terminali appartiene al linguaggio $L(A)$ con $A$ variabile
\item $A\to ^*w$
\item $A\to^*_{lm}w$
\item $A\to^*_{rm}w$
\item esiste un albero sintattico con radice $A$ e prodotto $w$
\end{enumerate}
queste 5 proposizioni si implicano l'uni l'altra:
\begin{center}
\begin{tikzpicture}
    \node (top) at (0,0) {5};
 	\node (a) at(-1,-0.5) {3};
 	\node (b) at(0,-1) {4};
 	\node (c) at(-2.0,-1.85) {2};
 	\node (d) at(1.5,-2) {1};
    \draw [->] (top) -- (a);
    \draw [->] (top) -- (b);
    \draw [->] (a) -- (c);
    \draw [->] (b) -- (c);
    \draw [->] (c) -- (d);
    \draw [->] (d) -- (top);
\end{tikzpicture}
\end{center}
vediamo qualche dimostrazione di implicazione tra queste proposizioni:
\begin{proof}[da 1 a 5]
si procede per induzione:
\begin{itemize}
\item \textbf{caso base:} ho un livello solo (una sola riga), $\exists A\to w$:
$$\overset{A}{\overset{\triangle}w}$$
\item \textbf{caso passo:} suppongo vero per un numero di righe $\leq n$, lo dimsotro per $n+1$ righe:
$$A\to X_1,X_2,...,X_k$$
$$w=w_1,w_2,...,w_k$$
ovvero, in meno di $n+1$ livelli:
\begin{center}

\psframebox[linestyle=none,framesep=10pt]{%
\pstree{\LFTw{t}{\fontspec{Noto Sans}[Script=Latin]A}}{\pstree{\Tp[edge=none]}{%
  \LFTw{t}{\fontspec{Noto Sans}[Script=Latin]$\overset{X_1}{\overset{\triangle}w_1}$}
  \LFTw{t}{\fontspec{Noto Sans}[Script=Latin]$\overset{X_2}{\overset{\triangle}w_2}$}
  \LFTw{t}{\fontspec{Noto Sans}[Script=Latin]$\vdots$}
  \LFTw{t}{\fontspec{Noto Sans}[Script=Latin]$\overset{X_k}{\overset{\triangle}w_k}$}}}}
\end{center}
\end{itemize}
\end{proof}
\begin{proof}[da 5 a 3]
procedo per induzione:
\begin{itemize}
\item \textbf{caso base (n=1): }$\exists A\to w\mbox{ quindi } A\to_{lm}w$, come prima si ha un solo livello:
$$\overset{A}{\overset{\triangle}w}$$
\item \textbf{caso passo: }suppongo che la proprierà valga per ogni albero di profondità minore uguale a $n$, dimostro che valga per gli alberi profondi $n+1$:
$$A\to X_1,X_2,...,X_k$$
$$w=w_1,w_2,...,w_k$$
ovvero, in meno di $n+1$ livelli:
\begin{center}

\psframebox[linestyle=none,framesep=10pt]{%
\pstree{\LFTw{t}{\fontspec{Noto Sans}[Script=Latin]A}}{\pstree{\Tp[edge=none]}{%
  \LFTw{t}{\fontspec{Noto Sans}[Script=Latin]$\overset{X_1}{\overset{\triangle}w_1}$}
  \LFTw{t}{\fontspec{Noto Sans}[Script=Latin]$\overset{X_2}{\overset{\triangle}w_2}$}
  \LFTw{t}{\fontspec{Noto Sans}[Script=Latin]$\vdots$}
  \LFTw{t}{\fontspec{Noto Sans}[Script=Latin]$\overset{X_k}{\overset{\triangle}w_k}$}}}}
\end{center}
$$A\to_{lm} X_1,X_2,...,X_k$$
$$x_1\to^*_{lm}w_1 \mbox{ per ipotesi induttiva si ha un albero al più di n livelli}$$
quindi:
$$A\to_{lm}X_1,...,X_k\to^*_{lm}w_1,X_2,...,X_k\to^*_{lm}...\to^*_{lm}w_1,...,w_k=w$$
\end{itemize}
\begin{esempio}
$$E\to I\to Ib\to ab$$
$$\alpha E\beta\to\alpha I\beta\to \alpha Ib\beta\to \alpha ab\beta,\,\,\,\alpha,\beta\in(V\cup T)^*$$
\end{esempio}
\end{proof}
\begin{esempio}
Mostro l'esistenza di una derivazione sinistra dell'albero sintattico di $a*(a+b00)$:
$$E\to^*_{lm}E*E\to^*_{lm}I*E\to^*_{lm}a*E\to^*_{lm}a*(E)\to^*_{lm}a*(E+E)\to^*_{lm}$$
$$a*(I+E)\to^*_{lm}a*(a+E)\to^*_{lm}a*(a+I)\to^*_{lm}a+(a+I0)\to^*_{lm}a*(a+I00)\to^*_{lm}a*(a+b00)$$
\end{esempio}
\subsection{Grammatiche ambigue}
\begin{definizione}
Una grammatica è definita ambigua se esiste una stringa $w$ di terminali che ha più di un albero sintattico
\end{definizione}
\begin{esempio}
vediamo un esempio:
\begin{enumerate}
\item $E\to E+E\to E+E*E$
ovvero:
\begin{center}

\psframebox[linestyle=none,framesep=10pt]{%
\pstree{\LFTw{t}{\fontspec{Noto Sans}[Script=Latin]E}}{\pstree{\Tp[edge=none]}{%
  \LFTw{t}{\fontspec{Noto Sans}[Script=Latin]E}
  \LFTw{t}{\fontspec{Noto Sans}[Script=Latin]+}
  \pstree{\LFTw{t}{\fontspec{Noto Sans}[Script=Latin]E}}{\pstree{\Tp[edge=none]}{%
    \LFTw{t}{\fontspec{Noto Sans}[Script=Latin]E}
    \LFTw{t}{\fontspec{Noto Sans}[Script=Latin]*}
    \LFTw{t}{\fontspec{Noto Sans}[Script=Latin]E}}}}}}
\end{center}
\item $E\to E*E\to E+E*E$
ovvero:
\begin{center}

\psframebox[linestyle=none,framesep=10pt]{%
\pstree{\LFTw{t}{\fontspec{Noto Sans}[Script=Latin]E}}{\pstree{\Tp[edge=none]}{%
  \pstree{\LFTw{t}{\fontspec{Noto Sans}[Script=Latin]E}}{\pstree{\Tp[edge=none]}{%
    \LFTw{t}{\fontspec{Noto Sans}[Script=Latin]E}
    \LFTw{t}{\fontspec{Noto Sans}[Script=Latin]+}
    \LFTw{t}{\fontspec{Noto Sans}[Script=Latin]E}}}
  \LFTw{t}{\fontspec{Noto Sans}[Script=Latin]*}
  \LFTw{t}{\fontspec{Noto Sans}[Script=Latin]E}}}}
\end{center}
\end{enumerate}
si arriva a due stringhe uguali ma con alberi diversi. Introduciamo delle categorie sintatiche, dei vincoli alla produzione delle regole:
\begin{enumerate}
\item $E\to T|\, E+T$
\item $T\to F|\, T+F$
\item $F\to I|\, (E)$
\item $I\to a|\,b|\,Ia|,Ib|\,I0|\,I1$
\end{enumerate}
\end{esempio}
Possono esserci più derivazioni di una stringa ma l'importante è che non ci siano alberi sintattici diversi. Capire se una CFG è ambigua è un problema indecidibile
\begin{esempio}
vediamo un esempio:
$$S\to \varepsilon|\,SS|\, iS|\, iSeS$$
con S=statement, i=if e e=else. Considero due derivazioni:
\begin{enumerate}
\item $S\to iSeS\to iiSeS\to iie$:
\begin{center}

\psframebox[linestyle=none,framesep=10pt]{%
\pstree{\LFTw{t}{\fontspec{Noto Sans}[Script=Latin]S}}{\pstree{\Tp[edge=none]}{%
  \LFTw{t}{\fontspec{Noto Sans}[Script=Latin]i}
  \pstree{\LFTw{t}{\fontspec{Noto Sans}[Script=Latin]S}}{\pstree{\Tp[edge=none]}{%
    \LFTw{t}{\fontspec{Noto Sans}[Script=Latin]i}
    \pstree{\LFTw{t}{\fontspec{Noto Sans}[Script=Latin]S}}{\pstree{\Tp[edge=none]}{%
      \LFTw{t}{\fontspec{Noto Sans}[Script=Latin]$\varepsilon$}}}}}
  \LFTw{t}{\fontspec{Noto Sans}[Script=Latin]e}
  \pstree{\LFTw{t}{\fontspec{Noto Sans}[Script=Latin]S}}{\pstree{\Tp[edge=none]}{%
    \LFTw{t}{\fontspec{Noto Sans}[Script=Latin]$\varepsilon$}}}}}}\end{center}
\item $S\to iS\to iiSeS\to iieS\to iie$:
\begin{center}

\psframebox[linestyle=none,framesep=10pt]{%
\pstree{\LFTw{t}{\fontspec{Noto Sans}[Script=Latin]S}}{\pstree{\Tp[edge=none]}{%
  \LFTw{t}{\fontspec{Noto Sans}[Script=Latin]i}
  \pstree{\LFTw{t}{\fontspec{Noto Sans}[Script=Latin]S}}{\pstree{\Tp[edge=none]}{%
    \LFTw{t}{\fontspec{Noto Sans}[Script=Latin]i}
    \pstree{\LFTw{t}{\fontspec{Noto Sans}[Script=Latin]S}}{\pstree{\Tp[edge=none]}{%
      \LFTw{t}{\fontspec{Noto Sans}[Script=Latin]$\varepsilon$}}}
    \LFTw{t}{\fontspec{Noto Sans}[Script=Latin]e}
    \pstree{\LFTw{t}{\fontspec{Noto Sans}[Script=Latin]S}}{\pstree{\Tp[edge=none]}{%
      \LFTw{t}{\fontspec{Noto Sans}[Script=Latin]$\varepsilon$}}}}}}}}
\end{center}
\end{enumerate}
Si ha quindi una grammatica ambigua
\end{esempio}
\begin{teorema}
Per ogni CFG, con $G=(V, T, P, S)$, per ogni stringa $w$ di terminali si ha che $w$ ha due alberi sintattici distinti sse ha due derivazioni sinistre da S distinte.\\
Se la grammatica non è ambigua allora esiste un'unica derivazione sinistra da $S$
\end{teorema}
\subsubsection{Linguaggi inerentemente ambigui}
\begin{definizione}
Un linguaggio $L$ è inerentemente ambiguo se tutte le grammatiche CFG per tale linguaggio sono a loro volta ambigue
\end{definizione}
\begin{esempio}
Sia $L=\{a^nb^nc^md^m|\, n,m\geq 1\}\cup \{a^nbmnc^md^n|\, n,m\geq 1\}$\\
si ha quindi un CFL formato dall'unione di due CFL. $L$ è inerentemente ambiguo e generato dalla seguente grammatica:
\begin{itemize}
\item $S\to AB|\,C$
\item $A\to aAb|\,ab$
\item $B\to cBd|\, cd$
\item $C\to aCd|\, aDd$
\item $D\to bDc|\, bc$
\end{itemize}
si possono avere due derivazioni:
\begin{enumerate}
\item $S\to_{lm}AB\to_{lm} aAbB\to_{lm} aabbB\to_{lm}aabbcBd\to_{lm}aabbccdd$
\item $S\to_{lm} C\to_{lm} aCd\to_{lm}aaBdd\to_{lm}aabBcdd\to_{lm}aabbccdd$
\end{enumerate}
a generare problemi sono le stringhe con n=m perché possono essere prodotte in due modi diversi da entrambi i sottolinguaggi. Dato che l'intersezione tra i due sottolinguaggi non è buota si ha che $L$ è ambiguo
\end{esempio}
\subsection{Grammatiche Regolari}
Sono le grammatiche che generano i linguaggi regolari (quelli del terzo tipo) che sono casi particolari dei CFL.\\
Si ha la solita grammatica $G = (V, T, P, S)$ con però vincoli su $P$:
\begin{itemize}
\item $\varepsilon$ si può ottenere solo con $S\to \varepsilon$
\item le produzioni sono tutte lineari a destra ($A\to aA$ o $A\to a$) o a sinistra ($A\to Ba$ o $A\to a$)
\end{itemize}
\begin{esempio}
$I\to a|\,b|\,Ia|\,Ib|\,I0|\,I1$ è una grammatica con le produzioni lineari a sinistra.\\
Potremmo pensarlo a destra $I\to a|\,b|\,aI|\,bI|\,0I|\,1I$.\\
Vediamo esempi di produzione con queste grammatiche:
\begin{itemize}
\item con $I\to a|\,b|\,Ia|\,Ib|\,I0|\,I1$ possiamo derivare $ab01b0$:
$$I\to I0\to Ib0\to I1b0\to I01b0\to Ib01b0\to ab01b0$$
\item con $I\to a|\,b|\,aI|\,bI|\,0I|\,1I$ invece non riusciamo a generare nulla:
$$I\to 0I\to 0a$$
\end{itemize}
definisco quindi un'altra grammatica (con una nuova categoria sintattica):
$$I\to aJ|\, bJ$$
$$J\to a|\,b|\,aJ|\,bJ|\,0J|\,1J$$
che però non mi permette di terminare le stringhe con 0 e 1, la modifico ancora otterdendo:
$$I\to aJ|\, bJ$$
$$J\to a|\,b|\,aJ|\,bJ|\,0J|\,1J|\,0|\,1$$
e questo è il modo corretto per passare da lineare sinistra a lineare destra
\end{esempio}
\begin{esempio}
Sia $G=(\{S\},\{0,1\},P,S)$ con $S\to \varepsilon|\,0|\,1|\,0S|\,1S$. Si ha quindi:
$$L(G)=\{0,1\}^*$$
si hanno comunque due proposizioni ridondanti, riducendo trovo:
$$S\to \varepsilon|\, 0S|\,1S$$
con solo produzioni lineari a destra. Con produzioni lineari a sinistra ottengo:
$$S\to \varepsilon|\, S0|\,S1$$
\end{esempio}
\begin{esempio}
Trovo una grammatica lineare destra e una sinistra per $L=\{a^nb^m|\,n,m\geq 0\}$:
\begin{itemize}
\item \textbf{lineare a destra:} si ha $G=(\{S,B\},\{a,b\},P,S)$ e quindi:
$$S\to \varepsilon|\,aS|\,bB$$
$$B\to bB|\,b$$
ma non si possono generare stringhe di sole $b$, infatti:
$$S\to aS\to abB\to abbB\to abbb$$
ma aggiungere $\varepsilon$ a B \textbf{non è lecito}. posso però produrre la stessa stringa da due derivazioni diverse:
$$S\to \varepsilon|\,aS|\,bB|\,b$$
$$B\to bB|\,b$$
che risulta quindi la nostra lineare a destra
\item \textbf{lineare a sinistra:} si ha $G=(\{S,A\},\{a,b\},P,S)$ e quindi:
$$S\to \varepsilon|\,Sb|\,Ab|\,a$$
$$A\to Aa|\,a$$
\end{itemize}
\end{esempio}
\begin{esempio}
Trovo una grammatica lineare destra e una sinistra per $L=\{ab^ncd^me|\,n\geq 0\,,m> 0\}$:
\begin{itemize}
\item \textbf{lineare a destra:} si ha  si ha $G=(\{S,A,B,E\},\{a,b,c,d,e\},P,S)$ e quindi:
$$S\to aA$$
$$A\to bA|\,cB$$
$$B\to dB|\, dE$$
$$E\to e$$
\item \textbf{lineare a sinistra:} si ha  si ha $G=(\{S,X,Y,Z\},\{a,b,c,d,e\},P,S)$ e quindi:
$$S\to Xe$$
$$A\to Xd|\,Yd$$
$$B\to Zc$$
$$E\to a|\,Zb$$
\end{itemize}
quindi se per esempio ho la stringa "ciao" si ha:
\begin{itemize}
\item \textbf{lineare a destra:}
$$S\to Ao$$
$$A\to Ba$$
$$B\to Ei$$
$$E\to c$$
\item \textbf{lineare a sinistra:} 
$$S\to cA$$
$$A\to iB$$
$$B\to aE$$
$$E\to o$$
\end{itemize}
\end{esempio}
\begin{esempio}
A partire da $G=(\{S,T\},\{0,1\},P,S)$ con:
$$S\to\varepsilon|\,0S|\,1T$$
$$T\to 0T|\,1S$$

trovo come è fatto $L(G)$:
$$L(G)=\{w\in\{0,1\}^*|\, w \mbox{ ha un numero di 1 pari}\}$$
\end{esempio}
\begin{esempio}
fornire una grammatica regolare a destra e sinistra per:
$$L=\{w\in\{0,1\}^*|\, w \mbox{ ha almeno uno 0 o almeno un 1}\}$$
Si ah che tutte le stringhe tranne quella vuota ciontengono uno 0 o un 1
quindi  $G=(\{S\},\{0,1\},P,S)$:
\begin{itemize}
\item \textbf{lineare a destra:}
$$S\to 0|\,1|\,0S|\,1S$$
\item \textbf{lineare a sinistra:} 
$$S\to 0|\,1|\,S0|\,S1$$
\end{itemize}
\end{esempio}
\subsection{Espressioni Regolari (Regex)}
le regex sono usate per la ricerca di un pattern in un testo o negli analizzatori lessicali. Una regex denota il linguaggio e non la grammatica. Si hanno le seguenti operazioni tra due linguaggi $L$ e $M$:
\begin{itemize}
\item \textbf{unione:} dati $L,\, M\in \Sigma^*$, l'unione $L\cup M$ è l'insieme delle stringhe che si trovano in entrambi i linguaggi o solo in uno dei due
\begin{esempio}
$$L=\{001,10,111\}$$
$$M=\{\varepsilon,001\}$$
$$L\cup M=\{01,10,111,\varepsilon\}$$
\end{esempio}
si ha che:
$$L\cup M=M\cup L$$
\item \textbf{concatenazione:} dati $L,\, M\in \Sigma^*$, la concatenazione $L\cdot M$ (o $LM$) è lisieme di tutte le stringhe ottenibili concatenandone una di $L$ a una di $M$
\begin{esempio}
$$L=\{001,10,111\}$$
$$M=\{\varepsilon,001\}$$
$$L\cdot M=\{001,001001,10,...\}$$
\end{esempio}
si ha che:
$$L\cdot M\neq M\cdot L$$
\item si definiscono:
\begin{itemize}
\item $L\cdot L=L^2$, $L\ cdot L\cdot L=L^3$ etc...
\item $L^1=L$
\item $L^0=\varepsilon$
\end{itemize}
\item \textbf{chiusura di Kleene:}
\end{itemize}
\end{document}
