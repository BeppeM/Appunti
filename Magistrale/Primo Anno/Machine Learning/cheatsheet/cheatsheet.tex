\documentclass[a4paper,12pt, oneside]{article}
% \usepackage{fullpage}
\usepackage[italian]{babel}
\usepackage[utf8]{inputenc}
\usepackage{amssymb}
\usepackage{amsthm}
\usepackage{graphics}
\usepackage{amsfonts}
\usepackage{listings}
\usepackage{amsmath}
\usepackage{amstext}
\usepackage{engrec}
\usepackage{rotating}
\usepackage[safe,extra]{tipa}
\usepackage{showkeys}
\usepackage{multirow}
\usepackage{hyperref}
\usepackage{sectsty}
\usepackage{mathtools}
\usepackage{microtype}
\usepackage{enumerate}
\usepackage{braket}
\usepackage{marginnote}
\usepackage{pgfplots}
\usepackage{cancel}
\usepackage{polynom}
\usepackage{booktabs}
\usepackage{enumitem}
\usepackage{framed}
\usepackage{algorithm}
\usepackage{algpseudocode}
\usepackage{pdfpages}
\usepackage{pgfplots}
\usepackage[cache=false]{minted}


\title{}

\author{}
\date{}

\pgfplotsset{compat=1.13}
\begin{document}
%\maketitle

\definecolor{shadecolor}{gray}{0.80}
\setlist{leftmargin = 2cm}
\newtheorem{teorema}{Teorema}
\newtheorem{definizione}{Definizione}
\newtheorem{esempio}{Esempio}
\newtheorem{corollario}{Corollario}
\newtheorem{lemma}{Lemma}
\newtheorem{osservazione}{Osservazione}
\newtheorem{nota}{Nota}
\newtheorem{esercizio}{Esercizio}

\renewcommand{\chaptermark}[1]{%
  \markboth{\chaptername
    \ \thechapter.\ #1}{}}
\renewcommand{\sectionmark}[1]{\markright{\thesection.\ #1}}
\allsectionsfont{\centering}
\section*{Formule}
\begin{itemize}
  \item Cardinalità spazio delle ipotesi:
  \[|X|=\prod |A_i|\]
  \item Cardinalità spazio dei concetti:
  \[|C|=|\mathcal{P}(X)|=2^{|X|}\]
  \item Cardinalità spazio delle ipotesi, semanticamente:
  \[|H|_{sem}=1+\prod_{A} (|A_i|+1)\]
  \item Cardinalità spazio delle ipotesi, sintatticamente:
  \[|H|_{sint}=\prod (|A_i|+2)\]
  \item Aspettativa di $G(X)$ su $P$ ($Val(X)$ = range di valori di $X$):
  \[E_P[g(X)]=\sum_{x\in Val(X)} g(x)\cdot P_X(x)\]
  \item Entropia di una variabile $X$:
  \[H[X]=-\sum_{i=1}^n p_i\cdot\log_2 p_i=E_P[\log_2(p)]\]
  \item Entropia di una distribuzione condizionale, con target $T$:
  \[H[T|X=x_i]=-\sum_{j=1}^m P_{T|X}(t_j|x_i)\cdot \log_2 P_{T|X}(t_j|x_i)\]
  \item Entropia condizionale, con target $T$:
  \[H[T|X]=\sum P(x)\cdot H(T|X=x)\]
  \item Information Gain su variabile $X$ e target $T$:
  \[IG[T|X]=H[T]-H[T|X]\]

\end{itemize}
\newpage
\section*{Definizioni}
\section*{Procedimenti comodi}
\end{document}