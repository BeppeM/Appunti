\documentclass[a4paper,12pt, oneside]{book}

% \usepackage{fullpage}
\usepackage[italian]{babel}
\usepackage[utf8]{inputenc}
\usepackage{amssymb}
\usepackage{amsthm}
\usepackage{graphics}
\usepackage{amsfonts}
\usepackage{listings}
\usepackage{amsmath}
\usepackage{amstext}
\usepackage{engrec}
\usepackage{rotating}
\usepackage{verbatim}
\usepackage[safe,extra]{tipa}
\usepackage{showkeys}
\usepackage{multirow}
\usepackage{hyperref}
\usepackage{microtype}
\usepackage{fontspec}
\usepackage{enumerate}
\usepackage{braket}
\usepackage{marginnote}
\usepackage{pgfplots}
\usepackage{cancel}
\usepackage{polynom}
\usepackage{booktabs}
\usepackage{enumitem}
\usepackage{framed}
\usepackage{pdfpages}
\usepackage{pgfplots}
\usepackage{algorithm}
% \usepackage{algpseudocode}
\usepackage[cache=false]{minted}
\usepackage{mathtools}
\usepackage[noend]{algpseudocode}

\usepackage{tikz}\usetikzlibrary{er}\tikzset{multi  attribute /.style={attribute
    ,double  distance =1.5pt}}\tikzset{derived  attribute /.style={attribute
    ,dashed}}\tikzset{total /.style={double  distance =1.5pt}}\tikzset{every
  entity /.style={draw=orange , fill=orange!20}}\tikzset{every  attribute
  /.style={draw=MediumPurple1, fill=MediumPurple1!20}}\tikzset{every
  relationship /.style={draw=Chartreuse2,
    fill=Chartreuse2!20}}\newcommand{\key}[1]{\underline{#1}}
  \usetikzlibrary{arrows.meta}
  \usetikzlibrary{decorations.markings}
  \usetikzlibrary{arrows,shapes,backgrounds,petri}
\tikzset{
  place/.style={
        circle,
        thick,
        draw=black,
        minimum size=6mm,
    },
  transition/.style={
    rectangle,
    thick,
    fill=black,
    minimum width=8mm,
    inner ysep=2pt
  },
  transitionv/.style={
    rectangle,
    thick,
    fill=black,
    minimum height=8mm,
    inner xsep=2pt
    }
  } 
\usetikzlibrary{automata,positioning}
\usepackage{fancyhdr}
\pagestyle{fancy}
\fancyhead[LE,RO]{\slshape \rightmark}
\fancyhead[LO,RE]{\slshape \leftmark}
\fancyfoot[C]{\thepage}


\title{Machine Learning}
\author{UniShare\\\\Davide Cozzi\\\href{https://t.me/dlcgold}{@dlcgold}}
\date{}

\pgfplotsset{compat=1.13}
\begin{document}
\maketitle

\definecolor{shadecolor}{gray}{0.80}
\setlist{leftmargin = 2cm}
\newtheorem{teorema}{Teorema}
\newtheorem{definizione}{Definizione}
\newtheorem{esempio}{Esempio}
\newtheorem{corollario}{Corollario}
\newtheorem{lemma}{Lemma}
\newtheorem{osservazione}{Osservazione}
\newtheorem{nota}{Nota}
\newtheorem{esercizio}{Esercizio}
\algdef{SE}[DOWHILE]{Do}{doWhile}{\algorithmicdo}[1]{\algorithmicwhile\ #1}
\tableofcontents
\renewcommand{\chaptermark}[1]{%
  \markboth{\chaptername
    \ \thechapter.\ #1}{}}
\renewcommand{\sectionmark}[1]{\markright{\thesection.\ #1}}
\newcommand{\floor}[1]{\lfloor #1 \rfloor}
\newcommand{\MYhref}[3][blue]{\href{#2}{\color{#1}{#3}}}%
\chapter{Introduzione}
\textbf{Questi appunti sono presi a lezione. Per quanto sia stata fatta
  una revisione è altamente probabile (praticamente certo) che possano
  contenere errori, sia di stampa che di vero e proprio contenuto. Per
  eventuali proposte di correzione effettuare una pull request. Link: }
\url{https://github.com/dlcgold/Appunti}.\\
\chapter{Introduzione al ML}
Il \textbf{Machine Learning (\textit{ML})} è sempre più diffuso nonostante sia
nato diversi anni fa.\\
Un \textbf{sistema di apprendimento automatico} ricava da un \textit{dataset}
una conoscenza non fornita a priori, descrivendo dati non forniti in
precedenza. Si estrapolano informazioni facendo assunzioni sulle informazioni
sistema già conosciute, creando una \textbf{classe delle ipotesi H}. Si cercano
ipotesi coerenti per guidare il sistema di apprendimento automatico. Bisogna
però mettere in conto anche eventuali errori, cercando di capire se esiste
davvero un'ipotesi coerente e, in caso di assenza, si cerca di approssimare. In
quest'ottica bisogna mediare tra \textbf{fit} e \textbf{complessità}. Ogni
sistema dovrà cercare di mediare tra questi due aspetti, un \textit{fit}
migliore comporta alta \textit{complessità}. Si ha sempre il rischio di
\textbf{overfitting}, cercando una precisione dei dati che magari non esiste. Si
ha un \textbf{generatore di dati} ma il sistema non ha conoscenza della totalità
degli stessi.\\
Definiamo alcuni concetti base:
\begin{itemize}
  \item \textbf{task (\textit{T})}, il compito da apprendere. È più acile
  apprendere attraverso esempi che codificare conoscenza o definire alcuni
  compiti. Inoltre il comportamento della macchina in un ambiente può essere
  diverso da quello desiderato, a causa della mutabilità dell'ambiente ed è più
  semplice cambiare gli esempi che ridisegnare un sistema 
  \item \textbf{performance (\textit{P})}, la misura della bontà
  dell'apprendimento (e bisognerà capire come misurare la cosa)
  \item \textbf{experience (\textit{E})}, l'esperienza sui cui basare
  l'apprendimento. Il tipo di esperienza scelto può variare molto il risultato e
  il successo dell'apprendimento
\end{itemize}
In merito alle parti ``software'' distinguiamo:
\begin{itemize}
  \item \textbf{learner}, la parte di programma che impara dagli esempi in modo
  automatico
  \item \textbf{trainer}, il \textit{dataset} che fornisce esperienza al
  \textit{learner}
\end{itemize}
Durante l'\textbf{apprendimento} si estrapolano dati da \textbf{istanze di
  addestramento o test}. Quindi:
\begin{itemize}
  \item si ricevono i dati di addestramento
  \item il sistema impara ad estrapolare partendo da quei dati
  \item si ricevono dati di test su cui si estrapola
\end{itemize}
L'ipotesi da apprendere viene chiamata \textbf{concetto target} (tra tutte le
ipotesi possibili identifico quella giusta dai dati di addestramento).\\
Approfondiamo il discorso relativo all'\textit{esperienza}. Innanzitutto nel
momento della scelta bisogna valutare la rappresentatività esperienza. SI ha
inoltre un controllo dell'esperienza da parte del \textit{learner}:
\begin{itemize}
  \item l'esperienza può essere fornita al learner senza che esso possa
  interagire
  \item il learner può porre domande su quegli esempi che non risultano chiari 
\end{itemize}
\textbf{L'esperienza deve essere presentata in modo causale.}\\
Si hanno due tipi di esperienza:
\begin{enumerate}
  \item \textbf{diretta}, dove i learner può acquisire informazione utile
  direttamente dagli esempi o dover inferire indirettamente da essi
  l’informazione necessaria (può essere chiaramente più complicato)
  \item \textbf{indiretta}
\end{enumerate}
Il tipo di dato che studieremo comunemente sarà il \textbf{vettore booleano} e
la risposta sarà anch'essa di tipo booleano. In questo contesto l'ipotesi è una
\textbf{congiunzione di variabili}.\\
Per ogni istanza di addestramento cerchiamo una risposta eventualmente
corrispondente al nostro \textit{target} (ovvero 1), qualora esista.\\
Si hanno tre tipi di apprendimento:
\begin{enumerate}
  \item \textbf{apprendimento supervisionato}, dove vengono forniti a priori
  esempi di comportamento e si suppone che il \textit{trainer} dia la risposta
  corretta per ogni input (mentre il learner usa gli esempi forniti per
  apprendere). L'esperienza è fornita da un insieme di coppie:
  \[S\equiv\{(x_1,y_1),(x_2,y_2),\ldots,(x_n,y_n)\}\]
  e, per ogni input ipotetico $x_i$ l'ipotetico trainer restituisce il corretto
  $y_i$
  \item \textbf{apprendimento non supervisionato}, dove si riconosce
  \textit{schemi} nell'input senza indicazioni sui valori in uscita. Non c'è
  target e si ha \textit{libertà di classificazione}. Si cerca una
  \textit{regolarità} e una \textit{struttura} insita nei dati. In questo caso
  si ha: 
  \[S\equiv\{x_1,x_2,\ldots,x_n\}\]
  Il clustering è un tipico problema di apprendimento non supervisionato. Non si
  ha spesso un metodo oggettivo per stabilire le prestazioni che vengono quindi
  valutate da umani
  \item \textbf{apprendimento per rinforzo}, dove bisogna apprendere, tramite il
  \textit{learner} sulla base
  della risposta dell’ambiente alle proprie azioni. Si lavora con
  un\textit{addestramento continuo}, aggiornando le ipotesi con l'arrivo dei
  dati (ad esempio per una macchina che deve giocare ad un gioco). Durante la
  fase di test bisogna conoscere le prestazioni e valutare la correttezza di
  quanto appreso. Il learner viene addestrato tramite \textit{rewards} e quindi
  apprende una strategia per massimizzare i \textit{rewards}, detta
  \textbf{strategia di comportamento} e per valutare la prestazione si cerca di
  massimizzare ``a lungo termine'' la ricompensa complessivamente ottenuta
\end{enumerate}
Possiamo inoltre distinguere due tipi di apprendimento:
\begin{enumerate}
  \item \textbf{attivo}, dove il \textit{learner} può ``domandare'' sui dati
  disponibili 
  \item \textbf{passivo}, dove il \textit{learner} apprende solo a partire dai
  dati disponibili 
\end{enumerate}
Si parla di \textbf{inductive learning} quando voglio apprendere una funzione da
un esempio (banalmente una funzione target $f$ con esempio $(x, f(x))$, ovvero
una coppia). Si cerca quindi un'ipotesi $h$, a partire da un insieme d'esempi di
apprendimento, tale per cui $h\approx f$. Questo è un modello semplificato
dell'apprendimento reale in quanto si ignorano a priori conoscenze e si assume
di avere un insieme di dati.Viene usato un approccio che sfrutta anche il
\textit{Rasoio di Occam}.
\begin{shaded}
  Terminologia:
  \begin{itemize}
    \item $X$, \textbf{spazio delle istanze}
    \item $D$, \textbf{training set}
    \item $c$, \textbf{concetto}, $c\subseteq X$
    \item $h$, \textbf{ipotesi}, $h\subseteq X$
    \item $(x, f(x))$, \textbf{esempio}, tale per cui:
    \[f(x)=
      \begin{cases}
        1& \mbox{se } x\subseteq X\\
        0& \mbox{altrimenti}
      \end{cases}
    \]
    \item $\{(x_1',f(x_1')),\ldots,(x_n',f(x_n'))\}$, \textbf{test}
    \item $\{(x_1,f(x_1)),\ldots,(x_n,f(x_n))\}$, \textbf{training set}
    \item \textbf{cross validation}, ovvero ripeto $m$ volte la validazione su
    campioni diversi di input per evitare che un certo risultato derivi dalla
    fortuna 
    \item \textbf{ipotesi H}, ovvero una congiunzione $\land$ di vincoli sugli
    attributi. Tale ipotesi è \textbf{consistente}, ovvero è coerente con tutti
    gli esempi
  \end{itemize}
\end{shaded}
Si avrà, in realtà, a che fare con dati, di target e ipotesi,
booleani e questo ambito è propriamente chiamato \textbf{concept learning}. In
questo contesto si cerca di capire quale funzione booleana è adatta al mio
addestramento. In altre parole si cerca di apprendere un'ipotesi booleana
partendo da esempi di training composti da input e output della
funzione. Qualora nel concept learning si abbia a che fare con più di due 
possibilità si aumentano i bit usati.\\
Nel concept learning un'ipotesi è un insieme di valori di attributi e ogni
valore può essere:
\begin{itemize}
  \item specificato
  \item non importante e si indica con ?
  \item nullo e si indica con $\emptyset$
\end{itemize}
Quindi, dato un training set $D$, cerco di determinare un'ipotesi $h\in H$ tale
che: 
\[h(x)=c(x),\,\forall x\in X\]
Si ha la teoria delle \textbf{ipotesi di apprendimento induttivo} che dice che
se la mia $h$ approssima bene nel \textit{training set} allora approssima bene
su tutti gli esempi non ancora osservati.\\
Il concept learning è quindi una ricerca del \textit{fit} migliore.
\begin{definizione}
  Date $h_j,h_k\in H$ booleane e definite su $X$. Si ha che $h_j$ è \textbf{più
    generale o uguale a} $h_k$ (e si scrive con $h_j\geq h_k$) sse:
  \[(h_k(x)=1)\longrightarrow (h_j(x)=1),\,\,\forall x\in X\]
  \textbf{Si impone quindi un ordine parziale}.\\
  Si ha che $h_j$ è \textbf{più generale di} $h_k$ (e si scrive con $h_j> h_k$)
  sse:
  \[(h_j\geq h_k)\land (h_k\not\geq h_j)\]
  Riscrivendo dal punto di vista insiemistico si ha che $h_j$ è \textbf{più
    generale o uguale a} $h_k$ sse:
  \[h_k\supseteq h_j\]
  e che è \textbf{più generale di} $h_k$ sse:
  \[h_k\supset h_j\]
  Dal punto di vista logico si ha che $h_j$ è \textbf{più generale di} $h_k$ sse
  impone meno vincoli di $h_k$
\end{definizione}
\textit{Lo spazio delle ipotesi è descritto da una congiunzione di attributi.}\\
Parliamo ora algoritmo \textbf{Find-S}. Questo algoritmo permette di partire
dall'ipotesi più specifica (attributi nulli, ovvero$\emptyset,\ldots,\emptyset$)
e generalizzarla, 
trovando ad ogni passo un'ipotesi più specifica e consistente con il training
set $D$. L'ipotesi in uscita sarà anche consistente con gli esempi negativi
dando prova che il target è effettivamente in $H$. Con questo algoritmo non si
può dimostrare di aver trovato l'unica ipotesi consistente con gli esempi e,
ignorando gli esempi negativi non posso capire se $D$ contiene dati
inconsistenti. Inoltre non ho l'ipotesi più generale.
\begin{algorithm}[H]
  \begin{algorithmic}
    \Function{findS}{}
    \State $h\gets \mbox{ l'ipotesi più specifica in } H$
    \For {\textit{ogni istanza di training positiva $x$}}
    \For {\textit{ogni vincolo di attributo $a_i$ in $h$}}
    \If { il vincolo di attributo $a_i$ in $h$ è soddisfatto da $x$}
    \State \textit{non fare nulla}
    \Else
    \State \textit{sostituisci $a_i$ in $h$ con il successivo vincolo più}
    \State \textit{generale che è soddisfatto da $x$}
    \EndIf
    \EndFor
    \EndFor
    \Return \textit{ipotesi $h$}
    \EndFunction
  \end{algorithmic}
  \caption{Algoritmo Find-S}
\end{algorithm}
\begin{definizione}
  Si dice che $h$ è \textbf{consistente} con il training set $D$ di concetti
  target sse: 
  \[Consistent(h,D):=h(x)=c(x),\,\,\forall \langle x,c(x)\rangle\in D\]
\end{definizione}
\begin{definizione}
  Si definisce \textbf{version space}, rispetto ad $H$ e $D$, come il
  sottoinsieme delle ipotesi da $H$ consistenti con $D$ e si indica con:
  \[VS_{H,D}=\{h\in H|\,Consistent(h,D)\]
\end{definizione}
Vediamo quindi algoritmo \textbf{List-Then Eliminate}:
\begin{algorithm}[H]
  \begin{algorithmic}
    \Function{LTE}{}
    \State $vs \gets$ \textit{una lista connettente tutte le ipotesi di } $H$
    \For {\textit{ogni esempio di training $\langle x,c(x)\rangle$}}
    \State \textit{rimuovi da $vs$ ogni ipotesi $h$ non consistente con}
    \State \textit{l'esempio di training, ovvero $h(x)\neq c(x)$}
    \EndFor
    \Return \textit{la lista delle ipotesi in $vs$}
    \EndFunction
  \end{algorithmic}
  \caption{Algoritmo List-Then Eliminate}
\end{algorithm}
Questo algoritmo è irrealistico in quanto richiese un numero per forza esaustivo
di ipotesi.\\
\begin{definizione}
  Definiamo:
  \begin{itemize}
    \item $G$ come il confine generale di $VS_{H,D}$, ovvero l'insieme dei
    membri generici al massimo. È l'insieme delle ipotesi più generali:
    \[G=\{g\in H|\, g\mbox{ è consistente con }D \land\]
    \[ (\nexists g'\in H \mbox{ t.c } g'\geq q \land g'\mbox{ è consistente con
      }D\})\]
    Possiamo dire che $G=\langle ?,?,?,\ldots ?\rangle$
    \item $S$ come il confine specifico di $VS_{H,D}$, ovvero l'insieme dei
    membri specifici al massimo. È l'insieme delle ipotesi più specifiche:
     \[S=\{s\in H|\, s\mbox{ è consistente con }D \land\]
    \[ (\nexists s'\in H \mbox{ t.c } s'\geq s \land s'\mbox{ è consistente con
      }D\})\]
    Possiamo dire che $S=\langle \emptyset,\emptyset,\emptyset,\ldots \emptyset
    \rangle$ 
  \end{itemize}
  Ogni elemento di $VS_{H,D}$ si trova tra questi confini:
  \[VS_{H,D}=\{h\inH|\,(\exists s\in S)\,\,(\exists g\in G)\,\,(g\geq h\geq
    s)\}\]
  con $\geq$ che specifica che è \textit{più generale o uguale}
\end{definizione}
Vediamo quindi algoritmo \textbf{candidate eliminate}
\begin{algorithm}[H]
  \begin{algorithmic}
    \Function{CE}{}
    \State $G\gets$ \textit{insieme delle ipotesi più generali in $H$}
    \State $S\gets$ \textit{insieme delle ipotesi più specifiche in $H$}
    \For {\textit{ogni esempio di training $d=\langle x,c(x)\rangle$}}
    \If {\textit{d è un esempio positivo}}
    \State \textit{rimuovi da $G$ ogni ipotesi inconsistente con $d$}
    \For {\textit{ogni ipotesi $s$ in $S$ inconsistente con $d$}}
    \State \textit{rimuovi $s$ da $S$}
    \State
    \State \textit{aggiungi a $S$ tutte le generalizzazioni minime $h$ di $s$}
    \State \textit{tali che $h$ sia consistente con $d$ e qualche membro di $G$}
    \State \textit{sia più generale di $h$}
    \EndFor
    \State \textit{rimuovi da $S$ ogni ipotesi più generale di un'altra in $S$}
    \Else
    \State \textit{rimuovi da $S$ ogni ipotesi inconsistente con $d$}
    \For {\textit{ogni ipotesi $g$ in $G$ inconsistente con $d$}}
    \State \textit{rimuovi $g$ da $G$}
    \State
    \State \textit{aggiungi a $G$ tutte le generalizzazioni minime $h$ di $g$}
    \State \textit{tali che $h$ sia consistente con $d$ e qualche membro di $S$}
    \State \textit{sia più generale di $h$}
    \EndFor
    \State \textit{rimuovi da $G$ ogni ipotesi più generale di un'altra in $G$}
    \EndIf
    \EndFor
    \Return \textit{la lista delle ipotesi in $vs$}
    \EndFunction
  \end{algorithmic}
  \caption{Algoritmo Candidate Eliminate}
\end{algorithm}
Questo algoritmo ha alcune proprietà:
\begin{itemize}
  \item converge all'ipotesi $h$ corretta provando che non ci sono errori in $D$
  e che $c\in H$
  \item se $D$ contiene errori allora l'ipotesi corretta sarà eliminata dal
  \textit{version space}
  \item si possono apprendere solo le congiunzioni
  \item se $H$ non contiene il concetto corretto $c$, verrà trovata l'ipotesi
  vuota
\end{itemize}
Il nostro spazio delle ipotesi non è in grado di rappresentare un semplice
concetto di target disgiuntivo, si parla infatti di \textbf{Biased Hypothesis
  Space}. \\
Studiamo quindi un \textbf{unbiased learner}. Si vuole scegliere un $H$ che
esprime ogni concetto insegnabile, ciò significa che $H$ è l'insieme di tutti i
possibili sottoinsiemi di $X$. $H$ sicuramente contiene il concetto target. $S$
diventa l'unione degli esempi positivi e $G$ la negazione dell'unione di quelli
negativi. Per apprendere il concetto di target bisognerebbe presentare ogni
singola istanza in $X$ come esempio di training.\\
Un learner che non fa assunzioni a priori in merito al concetto target non ha
basi ``razionali'' per classificare istanze che non vede.\\
Introduciamo quindi il \textbf{bias induttivo} considerando:
\begin{itemize}
  \item un algoritmo di learning del concetto $L$
  \item degli esempi di training $D_C=\{\langle x,c(x)\rangle\}$
\end{itemize}
Si ha che $L(x_i,D_c$) denota la classificazione assegnata all'istanza $x_1$, da
$L$, dopo il training con $D_c$.
\begin{definizione}
  Il \textbf{bias induttivo} di $L$ è un insieme minimale di asserzioni $B$ tale
  che, per ogni concetto target $c$ e $D_c$ corrispondente si ha che:
  \[[B\land D_c\land x_i]\.\vdash\,L(x_i,D_c),\,\,\forall x_i\in X\]
  con $\vdash$ che rappresenta l'implicazione logica
\end{definizione}
Possiamo quindi distinguere:
\begin{itemize}
  \item \textbf{sistema induttivo}, dove si hanno in input gli esempi di
  training e la nuova istanza, viene usato l'algoritmo \textit{candidate
    eliminate} con $H$ e si ottiene o la classificazione della nuova istanza
  nulla
  \item \textbf{sistema deduttivo} equivalente al sistema induttivo sopra
  descritto dove in input si aggiunge l'asserzione ``$H$ contiene il concetto
  target'' e si produce lo stesso output tramite un \textbf{prover di teoremi}
\end{itemize}
Abbiamo quindi visto tre tipi di \textit{learner}:
\begin{enumerate}
  \item il \textbf{rote learner}, dove si ha classificazione sse $x$ corrisponde
  ad un esempio osservato precedentemente. Non si ha \textit{bias induttivo}
  \item l'algoritmo \textbf{candidare eliminate} con \textbf{version space},
  dove il \textit{bias} corrisponde al fatto che lo spazio delle ipotesi
  contiene il concetto target 
  \item l'algoritmo \textbf{Find-S}, dove il \textit{bias} corrisponde al fatto
  che lo spazio delle ipotesi contiene il concetto target e tutte le istanze
  sono negative a meno che il target opposto sia implicato in un altro modo 
\end{enumerate}

\end{document}

% LocalWords:  Machine Learning machine learning dataset fit overfitting sse 
% LocalWords:  Concept concept experience learner rewards inductive validation
% LocalWords:  find Find findS version List biased Hypothesis unbiased bias
% LocalWords:  bias prover
