\documentclass[a4paper,12pt, oneside]{book}

% \usepackage{fullpage}
\usepackage[italian]{babel}
\usepackage[utf8]{inputenc}
\usepackage{amssymb}
\usepackage{amsthm}
\usepackage{graphics}
\usepackage{amsfonts}
\usepackage{listings}
\usepackage{amsmath}
\usepackage{amstext}
\usepackage{engrec}
\usepackage{rotating}
\usepackage{verbatim}
\usepackage[safe,extra]{tipa}
\usepackage{showkeys}
\usepackage{multirow}
\usepackage{hyperref}
\usepackage{microtype}
\usepackage{fontspec}
\usepackage{enumerate}
\usepackage{braket}
\usepackage{marginnote}
\usepackage{pgfplots}
\usepackage{cancel}
\usepackage{polynom}
\usepackage{booktabs}
\usepackage{enumitem}
\usepackage{framed}
\usepackage{pdfpages}
\usepackage{pgfplots}
\usepackage{algorithm}
% \usepackage{algpseudocode}
\usepackage[cache=false]{minted}
\usepackage{mathtools}
\usepackage[noend]{algpseudocode}

\usepackage{tikz}\usetikzlibrary{er}\tikzset{multi  attribute /.style={attribute
    ,double  distance =1.5pt}}\tikzset{derived  attribute /.style={attribute
    ,dashed}}\tikzset{total /.style={double  distance =1.5pt}}\tikzset{every
  entity /.style={draw=orange , fill=orange!20}}\tikzset{every  attribute
  /.style={draw=MediumPurple1, fill=MediumPurple1!20}}\tikzset{every
  relationship /.style={draw=Chartreuse2,
    fill=Chartreuse2!20}}\newcommand{\key}[1]{\underline{#1}}
  \usetikzlibrary{arrows.meta}
  \usetikzlibrary{decorations.markings}
  \usetikzlibrary{arrows,shapes,backgrounds,petri}
\tikzset{
  place/.style={
        circle,
        thick,
        draw=black,
        minimum size=6mm,
    },
  transition/.style={
    rectangle,
    thick,
    fill=black,
    minimum width=8mm,
    inner ysep=2pt
  },
  transitionv/.style={
    rectangle,
    thick,
    fill=black,
    minimum height=8mm,
    inner xsep=2pt
    }
  } 
\usetikzlibrary{automata,positioning}
\usepackage{fancyhdr}
\pagestyle{fancy}
\fancyhead[LE,RO]{\slshape \rightmark}
\fancyhead[LO,RE]{\slshape \leftmark}
\fancyfoot[C]{\thepage}


\title{Modelli della Concorrenza}
\author{UniShare\\\\Davide Cozzi\\\href{https://t.me/dlcgold}{@dlcgold}}
\date{}

\pgfplotsset{compat=1.13}
\begin{document}
\maketitle

\definecolor{shadecolor}{gray}{0.80}
\setlist{leftmargin = 2cm}
\newtheorem{teorema}{Teorema}
\newtheorem{definizione}{Definizione}
\newtheorem{esempio}{Esempio}
\newtheorem{corollario}{Corollario}
\newtheorem{lemma}{Lemma}
\newtheorem{osservazione}{Osservazione}
\newtheorem{nota}{Nota}
\newtheorem{esercizio}{Esercizio}
\algdef{SE}[DOWHILE]{Do}{doWhile}{\algorithmicdo}[1]{\algorithmicwhile\ #1}
\tableofcontents
\renewcommand{\chaptermark}[1]{%
  \markboth{\chaptername
    \ \thechapter.\ #1}{}}
\renewcommand{\sectionmark}[1]{\markright{\thesection.\ #1}}
\newcommand{\floor}[1]{\lfloor #1 \rfloor}
\newcommand{\MYhref}[3][blue]{\href{#2}{\color{#1}{#3}}}%
\chapter{Introduzione}
\textbf{Questi appunti sono presi a lezione. Per quanto sia stata fatta
  una revisione è altamente probabile (praticamente certo) che possano
  contenere errori, sia di stampa che di vero e proprio contenuto. Per
  eventuali proposte di correzione effettuare una pull request. Link: }
\url{https://github.com/dlcgold/Appunti}.\\
\textbf{Le immagini presenti in questi appunti sono tratte dalle slides del
  corso e tutti i diritti delle stesse sono da destinarsi ai docenti del corso
  stesso}.
\chapter{Introduzione alla concorrenza}
La \textbf{concorrenza} è presente in diversi aspetti della quotidianità (anche
non informatica) e un primo esempio di \textbf{sistema concorrente} è quella
della \textit{cellula vivente}, che può essere vista come un dispositivo che
trasforma e manipola dati per ottenere un risultato. I vari processi all'interno
di una cellula avvengono in modo concorrente. La cellula è un \textit{sistema
  asincrono}. Un secondo esempio non informatico 
è quello dell'\textit{orchestra musicale} dove i vari componenti suonano spesso
simultaneamente, rappresentando un \textit{sistema sincrono} (ovvero un sistema
che funziona avendo una sorta di ``cronometro'' condiviso dai vari attori del
sistema). Un esempio informatico è un \textit{processore multicore} (anche se in
realtà anche se fosse \textit{monocore} sarebbe comunque un sistema concorrente
per ovvie ragioni). Anche un \textit{rete di calcolatori} è un modello
concorrente. Anche i \textit{modelli sociali umani} sono modelli concorrenti.\\
I modelli concorrenti hanno alcuni aspetti comuni, semplificando molto:
\begin{itemize}
  \item competizione per l’accesso a risorse condivise
  \item cooperazione per un fine comune (che può portare a competizione)
  \item coordinamento di attività diverse
  \item sincronia e asincronia
\end{itemize}
Lo studio e la progettazione di sistemi concorrenti si hanno diversi problemi
peculiari che rendono la progettazione degli stessi molto difficile. Un sistema
concorrente mal progettato può avere effetti catastrofici.\\
Per poter sviluppare modelli concorrenti se necessità innanzitutto di:
\begin{itemize}
  \item \textbf{linguaggi}, per specificare sistemi concorrenti. Tra questi si
  hanno innanzitutto i \textbf{linguaggi di programmazione} (con l'uso di
  \textit{thread}, \textit{mutex}, scambio di messaggi etc$\ldots$ con i vari
  problemi di \textit{race condition}, uso di variabili condivise
  etc$\ldots$). Un altro linguaggio (non informatico) può essere quello di una
  \textit{partitura musicale} (dove si visualizza bene la natura
  \textit{sincrona}). Un linguaggio per rappresentare un \textit{processo
    concorrente} è quello di usare un \textbf{task graph (\textit{grafo delle
      attività})}, dove i nodi sono le attività (o eventi) mentre gli archi
  rappresentano una \textit{relazione d'ordine parziale}, come per esempio una
  \textit{relazione di precedenza}, sui nodi. Un altro linguaggio è dato dalle
  \textbf{algebre di processi}, simile ad un sistema di equazioni, con simboli
  che rappresentano eventi del sistema concorrente e operatori atti a comporre
  fra loro i vari sottoprocessi del sistema concorrente. Ogni ``equazione''
  descrive un processo che costituisce un elemento di un sistema concorrente
  \item \textbf{modelli}, per modellare sistemi concorrenti in astratto. Un
  esempio è dato dalle \textbf{reti di Petri}, che modellano un sistema
  concorrente partendo dalle nozioni di \textit{stato locale} di uno dei
  componenti del sistema e di \textit{evento locale} che ha un effetto su alcune
  componenti (e non tutte). Si ha quindi rappresentato un \textit{sistema
    dinamico} che si evolve nel tempo (evoluzione rappresentata tramite
 \textit{relazioni di flusso})
  \item \textbf{logica}, per analizzare e specificare sistemi concorrenti
  \item \textbf{model-checking}, per validare formule relative a proprietà di
  sistemi concorrenti 
\end{itemize}
\chapter{Correttezza di programmi sequenziali}
\end{document}
% LocalWords:  Machine Learning dell multicore monocore checking mutex thread
% LocalWords:  race condition graph sottoprocessi Petri
