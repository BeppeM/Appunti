\documentclass[a4paper,12pt, oneside]{book}

% \usepackage{fullpage}
\usepackage[italian]{babel}
\usepackage[utf8]{inputenc}
\usepackage[margin=0.5in]{geometry}
\usepackage{amssymb}
\usepackage{amsthm}
\usepackage{graphics}
\usepackage{amsfonts}
\usepackage{listings}
\usepackage{amsmath}
\usepackage{amstext}
\usepackage{engrec}
\usepackage{blindtext}
\usepackage{sectsty}
\usepackage{rotating}
\usepackage{verbatim}
\usepackage[safe,extra]{tipa}
% \usepackage{showkeys}
\usepackage{multirow}
\usepackage{hyperref}
\usepackage{microtype}
\usepackage{fontspec}
\usepackage{enumerate}
\usepackage{physics}
\usepackage{braket}
\usepackage{marginnote}
\usepackage{pgfplots}
\usepackage{cancel}
\usepackage{polynom}
\usepackage{booktabs}
\usepackage{enumitem}
\usepackage{framed}
\usepackage{pdfpages}
\usepackage{pgfplots}
\usepackage{algorithm}
% \usepackage{algpseudocode}
\usepackage[cache=false]{minted}
\usepackage{mathtools}
\usepackage[noend]{algpseudocode}
\newcommand*{\bfrac}[2]{\genfrac{}{}{0pt}{}{#1}{#2}}

\usepackage{tikz}\usetikzlibrary{er}\tikzset{multi  attribute /.style={attribute
    ,double  distance =1.5pt}}\tikzset{derived  attribute /.style={attribute
    ,dashed}}\tikzset{total /.style={double  distance =1.5pt}}\tikzset{every
  entity /.style={draw=orange , fill=orange!20}}\tikzset{every  attribute
  /.style={draw=MediumPurple1, fill=MediumPurple1!20}}\tikzset{every
  relationship /.style={draw=Chartreuse2,
    fill=Chartreuse2!20}}\newcommand{\key}[1]{\underline{#1}}
\usetikzlibrary{arrows.meta}
\usetikzlibrary{decorations.markings}
\usetikzlibrary{arrows,shapes,backgrounds,petri}
\tikzset{
  place/.style={
    circle,
    thick,
    draw=black,
    minimum size=6mm,
  },
  transition/.style={
    rectangle,
    thick,
    fill=black,
    minimum width=8mm,
    inner ysep=2pt
  },
  transitionv/.style={
    rectangle,
    thick,
    fill=black,
    minimum height=8mm,
    inner xsep=2pt
  }
} 
\usetikzlibrary{automata,positioning,chains,fit,shapes}
\usepackage{fancyhdr}
\pagestyle{fancy}
% \fancyhead[LE,RO]{\slshape \rightmark}
% \fancyhead[LO,RE]{\slshape \leftmark}
\fancyhf{} % sets both header and footer to nothing
\renewcommand{\headrulewidth}{0pt}
\fancyfoot[C]{\thepage}
\usepackage[usenames,dvipsnames]{pstricks}
\usepackage{epsfig}
\usepackage{pst-grad} % For gradients
\usepackage{pst-plot} % For axes
\usepackage[space]{grffile} % For spaces in paths
\usepackage{etoolbox} % For spaces in paths
\makeatletter % For spaces in paths
\patchcmd\Gread@eps{\@inputcheck#1 }{\@inputcheck"#1"\relax}{}{}
\makeatother
\usepackage{lipsum}
\DeclareSymbolFont{symbolsC}{U}{txsyc}{m}{n}
\DeclareMathSymbol{\strictif}{\mathrel}{symbolsC}{74}
\title{Fondamenti Logico Matematici dell'Informatica}
\author{UniShare\\\\Davide Cozzi\\\href{https://t.me/dlcgold}{@dlcgold}}
\date{}

\pgfplotsset{compat=1.13}
\sectionfont{\centering}
\chapterfont{\centering}

\begin{document}
%\maketitle

\definecolor{shadecolor}{gray}{0.80}
\setlist{leftmargin = 2cm}
\newtheorem{teorema}{Teorema}
\newtheorem{definizione}{Definizione}
\newtheorem{esempio}{Esempio}
\newtheorem{corollario}{Corollario}
\newtheorem{lemma}{Lemma}
\newtheorem{osservazione}{Osservazione}
\newtheorem{nota}{Nota}
\newtheorem{esercizio}{Esercizio}
\algdef{SE}[DOWHILE]{Do}{doWhile}{\algorithmicdo}[1]{\algorithmicwhile\ #1}
%\tableofcontents
\renewcommand{\chaptermark}[1]{%
  \markboth{\chaptername
    \ \thechapter.\ #1}{}}
\renewcommand{\sectionmark}[1]{\markright{\thesection.\ #1}}
\newcommand{\floor}[1]{\lfloor #1 \rfloor}
\newcommand{\MYhref}[3][blue]{\href{#2}{\color{#1}{#3}}}%
\chapter*{Tabelle Fondamenti Logico Matematici}
\section*{Deduzione Naturale Proposizionale Varie Logiche}
\begin{table}[H]
  \Large
  \centering
  \begin{tabular}{c||c|c}
    connettivo/falso& introduzione & eliminazione\\
    \hline
    \hline
    $\land$ & $\frac{A,B}{A\land B}i\land$&$\frac{A\land B}{A}e\land$
                                            $\frac{A\land B}{B}e\land$\\
    \hline
    $\lor$ &$\frac{A}{A\lor B}i\lor$
             $\frac{B}{A\lor B}i\lor$&$\frac{A\lor B,
                                       \bfrac{\bfrac{\cancel{A}}{\pi'}}{C},
                                       \bfrac{\bfrac{\cancel{B}}{\pi''}}{C}}{C}
                                       e\lor$\\
    \hline
    $\to$ & $\frac{\bfrac{\cancel{A}}
            {\bfrac{\pi}{B}}}{A\to B}i\to$ & $\frac{A,A\to B}{B}e\to$\\
   
    \hline
    $\neg$ & $\frac{\bfrac{\bfrac{\cancel{A}}{\pi}}
             {\bot}}{\neg A}i\neg$& $\frac{\bfrac{\bfrac{\cancel{\neg A}}{\pi}}
             {\bot}}{ A}e\neg$\\
    \hline
    \hline
    $\bot$ & $\frac{A,\neg A}{\bot}i\bot$& $\frac{\bot}{B}e\bot$\\
  \end{tabular}
\end{table}
\begin{itemize}
  \item \textbf{La regola dell'eliminazione del $\bot$ non si può usare in
    logica minimale}
  \item \textbf{La regola dell'introduzione del $\neg$ non si può usare in
    logica minimale}
  \item \textbf{La regola dell'eliminazione del $\neg$ non si può usare in
    logica intuizionistica}
  \item \textbf{Le altre regole sono valide sia per la logica classica che per
    quella intuizionistica che per quella modale}
\end{itemize}
\section*{Deduzione Naturale Predicativa Varie Logiche}
\begin{table}[H]
  \Large
  \centering
  \begin{tabular}{c|c|c}
    quantificatore & introduzione & eliminazione\\
    \hline
    $\exists$ & $\frac{P(a)}{\exists xP(x)}i\exists$
                                  &$\frac{\stackrel{\pi}{\exists xP(x}),
                                    \stackrel{
                                    \stackrel{\cancel{P(a)}}{\pi_1}}{C}}{C}
                                    e\exists$\\
    \hline
    $\forall$ & $\frac{\stackrel{\pi}{P(a)}}{\forall xP(x)}i\forall$
                                  &$\frac{\forall xP(x)}{P(a)}e\forall$\\
  \end{tabular}
\end{table}
\begin{itemize}
  \item \textbf{Le regole valgono sia per logica classica che intuizionistica}
\end{itemize}
\section*{Tableaux Logica Intuizionistica Proposizionale}
\begin{table}[H]
  \Large
  \centering
  \begin{tabular}{c||c|c}
    connettivo& T-regola& F-regola\\
    \hline
    \hline
    $\land$ & $\frac{S,T(A\land B)}{S,TA,TB}T\land$&
                        $\frac{S,F(A\land B)}{S,FA/S,FB}F\land$\\
    \hline
    $\lor$ & $\frac{S,T(A\lor B)}{S,TA/S,TB}T\lor$&
                        $\frac{S,F(A\lor B)}{S,FA,FB}F\lor$\\
    \hline
    $\to$ & $\frac{S,T(A\to B)}{S,FA/S,TB}T\to$&
                        $\frac{S,F(A\to B)}{S,TA,FB}F\to$\\
    \hline
    $\neg$ & $\frac{S,T(\neg A)}{S,FA}T\neg$&
                        $\frac{S,F(\neg A)}{S,TA}F\neg$\\
    \hline
  \end{tabular}
\end{table}
\begin{table}[H]
  \Large
  \centering
  \begin{tabular}{c||c}
    connettivo& T-regola con eventuale ripetizione\\
    \hline
    \hline
    $\to$ & $\frac{S,T(A\to B)}{S,FA,T(A\to B)/S,TB}T\to$\\
    \hline
    $\neg$ & $\frac{S,T(\neg A)}{S,FA,T(\neg A)}T\neg$
  \end{tabular}
\end{table}
\section*{Tableaux Ottimizzati Logica Intuizionistica Proposizionale}
\begin{table}[H]
  \Large
  \centering
  \begin{tabular}{c||c|c|c}
    & {\small{$T$-regola}}& {\small{$F$-regola}} & {\small{$F_C$-regola}}\\
    \hline
    \hline
    $\land$ & $T\land=\frac{S,T(A\land B)}{S,TA,TB}$&
              $F\land=\frac{S,F(A\land B)}{S,FA/S,FB}$&
              $F_C\land=\frac{S,F_C(A\land B)}{S_C,F_CA/S_C,F_CB}$\\ 
    \hline
    $\lor$ & $T\lor=\frac{S,T(A\lor B)}{S,TA/S,TB}$&
                        $F\lor=\frac{S,F(A\lor B)}{S,FA,FB}$&
                        $F_C\lor=\frac{S,F_C(A\lor B)}{S,F_CA,F_CB}$\\
    \hline
    $\to$ & $T\to=\frac{S,T(A\to B)}{S,FA, T(A\to B)/S,TB}$&
                        $F\to=\frac{S,F(A\to B)}{S_C,TA,FB}$&
                        $F_C\to=\frac{S,F_C(A\to B)}{S_C,TA,F_CB}$\\
    \hline
    $\neg$ & $T\neg=\frac{S,T(\neg A)}{S,F_CA}$&
                        $F\neg=\frac{S,F(\neg A)}{S_C,TA}$&
                        $F_C\neg=\frac{S,F_C(\neg A)}{S_C,TA}$\\
    \hline
  \end{tabular}
\end{table}
\begin{itemize}
  \item \textbf{$S_C$ è definito come l'insieme $S$ meno l'insieme delle formule
    segnate con $F$}
\end{itemize}
\begin{center}
  \textbf{Ottimizzazioni Implicazione Logica Intuizionistica Proposizionale}
\end{center}
\begin{table}[H]
  \Large
  \centering
  \begin{tabular}{c|c}
    Antecedente $Ant$ & $\mathbf{T\to}$\\
    \hline
    $Ant=A$ \textit{o} $Ant=\neg A$ & $\frac{S,TA\to B}{S,FA/S,TB}T\to AN$\\
    \hline
    $Ant=A\land B $ & $\frac{S,T(A\land B)\to C}{S,T(A\to(B\to C))}T\to \land$\\
    \hline
    $Ant=A\lor B$ & $\frac{S,T(A\lor B)\to C}{S,TA\to C, TB\to C}T\to \lor$\\
    \hline
    $Ant =A\to B$ & $\frac{S,T(A\to B)\to C}{S, FA\to B, TB\to C/S, TC}T\to \to$\\
    \hline
  \end{tabular}
\end{table}
\begin{center}
  \textbf{Implicazione segnata (\textit{versione corretta ma non completa di
      $T\to$})} 
\end{center}
\[\frac{S, TA\to B}{S,F_CA/S_C, TB}\overline{T\to}\]

\end{document}