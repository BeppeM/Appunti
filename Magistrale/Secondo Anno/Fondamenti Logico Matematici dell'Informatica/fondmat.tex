\documentclass[a4paper,12pt, oneside]{book}

% \usepackage{fullpage}
\usepackage[italian]{babel}
\usepackage[utf8]{inputenc}
\usepackage{amssymb}
\usepackage{amsthm}
\usepackage{graphics}
\usepackage{amsfonts}
\usepackage{listings}
\usepackage{amsmath}
\usepackage{amstext}
\usepackage{engrec}
\usepackage{rotating}
\usepackage{verbatim}
\usepackage[safe,extra]{tipa}
% \usepackage{showkeys}
\usepackage{multirow}
\usepackage{hyperref}
\usepackage{microtype}
\usepackage{fontspec}
\usepackage{enumerate}
\usepackage{physics}
\usepackage{braket}
\usepackage{marginnote}
\usepackage{pgfplots}
\usepackage{cancel}
\usepackage{polynom}
\usepackage{booktabs}
\usepackage{enumitem}
\usepackage{framed}
\usepackage{pdfpages}
\usepackage{pgfplots}
\usepackage{algorithm}
% \usepackage{algpseudocode}
\usepackage[cache=false]{minted}
\usepackage{mathtools}
\usepackage[noend]{algpseudocode}

\usepackage{tikz}\usetikzlibrary{er}\tikzset{multi  attribute /.style={attribute
    ,double  distance =1.5pt}}\tikzset{derived  attribute /.style={attribute
    ,dashed}}\tikzset{total /.style={double  distance =1.5pt}}\tikzset{every
  entity /.style={draw=orange , fill=orange!20}}\tikzset{every  attribute
  /.style={draw=MediumPurple1, fill=MediumPurple1!20}}\tikzset{every
  relationship /.style={draw=Chartreuse2,
    fill=Chartreuse2!20}}\newcommand{\key}[1]{\underline{#1}}
\usetikzlibrary{arrows.meta}
\usetikzlibrary{decorations.markings}
\usetikzlibrary{arrows,shapes,backgrounds,petri}
\tikzset{
  place/.style={
    circle,
    thick,
    draw=black,
    minimum size=6mm,
  },
  transition/.style={
    rectangle,
    thick,
    fill=black,
    minimum width=8mm,
    inner ysep=2pt
  },
  transitionv/.style={
    rectangle,
    thick,
    fill=black,
    minimum height=8mm,
    inner xsep=2pt
  }
} 
\usetikzlibrary{automata,positioning,chains,fit,shapes}
\usepackage{fancyhdr}
\pagestyle{fancy}
\fancyhead[LE,RO]{\slshape \rightmark}
\fancyhead[LO,RE]{\slshape \leftmark}
\fancyfoot[C]{\thepage}
\usepackage[usenames,dvipsnames]{pstricks}
\usepackage{epsfig}
\usepackage{pst-grad} % For gradients
\usepackage{pst-plot} % For axes
\usepackage[space]{grffile} % For spaces in paths
\usepackage{etoolbox} % For spaces in paths
\makeatletter % For spaces in paths
\patchcmd\Gread@eps{\@inputcheck#1 }{\@inputcheck"#1"\relax}{}{}
\makeatother

\title{Fondamenti Logico Matematici dell'Informatica}
\author{UniShare\\\\Davide Cozzi\\\href{https://t.me/dlcgold}{@dlcgold}}
\date{}

\pgfplotsset{compat=1.13}
\begin{document}
\maketitle

\definecolor{shadecolor}{gray}{0.80}
\setlist{leftmargin = 2cm}
\newtheorem{teorema}{Teorema}
\newtheorem{definizione}{Definizione}
\newtheorem{esempio}{Esempio}
\newtheorem{corollario}{Corollario}
\newtheorem{lemma}{Lemma}
\newtheorem{osservazione}{Osservazione}
\newtheorem{nota}{Nota}
\newtheorem{esercizio}{Esercizio}
\algdef{SE}[DOWHILE]{Do}{doWhile}{\algorithmicdo}[1]{\algorithmicwhile\ #1}
\tableofcontents
\renewcommand{\chaptermark}[1]{%
  \markboth{\chaptername
    \ \thechapter.\ #1}{}}
\renewcommand{\sectionmark}[1]{\markright{\thesection.\ #1}}
\newcommand{\floor}[1]{\lfloor #1 \rfloor}
\newcommand{\MYhref}[3][blue]{\href{#2}{\color{#1}{#3}}}%
\chapter{Introduzione}
\textbf{Questi appunti sono presi a lezione. Per quanto sia stata fatta
  una revisione è altamente probabile (praticamente certo) che possano
  contenere errori, sia di stampa che di vero e proprio contenuto. Per
  eventuali proposte di correzione effettuare una pull request. Link: }
\url{https://github.com/dlcgold/Appunti}.\\
\chapter{Il Paradigma Dimostrazioni = Algoritmi}
Prendiamo come \textit{linguaggio di specifica} un \textbf{linguaggio
  del prim'ordine con identità}.
\begin{shaded}
  Si ricorda che, in \textit{logica matematica}:
  \begin{definizione}
    Definiamo \textbf{linguaggio del primo ordine} come un linguaggio formale
    che serve per gestire meccanicamente enunciati e ragionamenti che
    coinvolgono i connettivi logici, le relazioni e i quantificatori $\forall$ e
    $\exists$.\\
    Si ha che ``del primo ordine'' ndica che c'è un insieme di riferimento e i
    quantificatori possano riguardare solo gli elementi di tale insieme e non i
    sottoinsiemi (posso dire ``per tutti gli elementi'' ma non ``per tutti i
    sottoinsiemi''). Tale linguaggio è caratterizzato da:
    \begin{itemize}
      \item un \textbf{alfabeto di simboli} per variabili, costanti, predicati,
      funzioni, connettivi, quantificatori o punteggiatura 
      \item un \textbf{insieme di termini} per denotare gli elementi
      dell'insieme in analisi
      \item un \textbf{insieme di formule ben formate (\textit{FBF})} ovvero un
      insieme di stringhe composte di simboli dell'alfabeto che vengono
      considerate sintatticamente corrette 
    \end{itemize}
  \end{definizione}
  \begin{definizione}
    Definiamo \textbf{sistema assiomatico} come un insieme di assiomi che
    possono essere usati per dimostrare teoremi. Una teoria matematica consiste
    quindi in una assiomatica e tutti i teoremi che ne derivano.  
  \end{definizione}
  \begin{definizione}
    Definiamo un sistema formale come una formalizzazione rigorosa e completa
    della nozione di sistema assiomatico costituito da:
    \begin{itemize}
      \item un alfabeto
      \item una grammatica che specifica quali sequenze finite dei simboli
      dell'alfabeto corrispondono ad una FBF. La grammatica deve essere
      ricorsiva, nel senso che deve esistere un algoritmo per decidere se una
      sequenza di simboli è o meno una formula ben formata 
      \item un sottoinsieme delle FBF che sono gli assiomi. L'insieme degli
      assiomi è ricorsivo
      \item le regole di inferenze che associano formule ben formate ad n-uple
      di formule ben formate
    \end{itemize}
  \end{definizione}
  \begin{definizione}
    Definiamo gli \textbf{assiomi di Peano} come un gruppo di assiomi ideati al
    fine di definire assiomaticamente l'insieme dei numeri naturali:
    \begin{itemize}
      \item esiste un numero naturale: 0 (alternativamente 1 se si vuole escludere
      0):
      \[0/1\in \mathbb{N}\]
      \item ogni naturale ha un naturale come successore. Ho quindi una funzione
      ``successore'' tale che:
      \[S:\mathbb{N}\to\mathbb{N}\]
      \item numeri diversi hanno successori diversi, ovvero:
      \[x\neq y\implies S(x)\neq S(y)\]
      \item 0 (o alternativamente 1) non è il successore di alcun naturale,
      ovvero:
      \[S(x)\neq 0, \forall x\in \mathbb{N}\]
      \item ogni sottoinsieme di numeri naturali che contenga lo zero e il
      successore di ogni proprio elemento coincide con l'intero insieme dei
      numeri 
      naturali. Ovvero dato $U\subseteq \mathbb{N}$ tale che:
      \begin{itemize}
        \item $0\in U$
        \item $x\in U\implies S(x)\in U$
      \end{itemize}
      allora:
      \[U=\mathbb{N}\]
      Tale assioma è detto \textbf{assioma dell'induzione} o \textbf{principio di
        induzione} 
    \end{itemize}
  \end{definizione}
  \begin{definizione}
    In una teoria del primo ordine si chiama \textbf{chiusura universale} di una
    formula ben formata $A(x_1,\ldots,x_n)$, con $x_1,\ldots,x_n$ variabili
    libere, la formula:
    \[\forall x_1\forall x_2\forall x_n\,\,A(x_1,\ldots,x_n)\]
    ottenuta premettendo un quantificatore universale su ogni variabile libera. 
  \end{definizione}
  \begin{definizione}
    Definiamo, in logica matematica, \textbf{aritmetica di Peano (\textit{PA})}
    come una teoria del primo 
    ordine che ha come assiomi propri una versione degli \textbf{assiomi di
      Peano} 
    espressi nel linguaggio del primo ordine. Si ha quindi che il linguaggio di
    PA 
    è il linguaggio dell'aritmetica del primo ordine con i seguenti simboli:
    \begin{itemize}
      \item vari simboli per le variabili: $x$, $y$, $z$, $x_1$ etc$\ldots$
      \item costanti individuali: $0$ etc$\ldots$
      \item simboli per funzioni unarie: $S$
      \item simboli per funzioni binarie $+$, $\times$ ($+(x,y)$ si indica anche
      con $x+y$ e analogamente si fa per $\times$)
      \item simboli per relazioni unarie: $=$
      \item simboli per connettivi logici, quantificatori e parentesi
    \end{itemize}
    Gli assiomi di PA sono costituiti da:
    \begin{itemize}
      \item gli assiomi logici
      \item gli assiomi per l'uguaglianza
      \item i seguenti assiomi propri (che ``traducono'' nella logica di Peano
      gli assiomi di Peano):
      \begin{itemize}
        \item $\forall x\neg(S(x)=0)$
        \item $\forall x\forall y(S(x)=S(y)\implies x=y)$
        \item $\forall x(x+0=x)$
        \item $\forall x\forall y(x+S(y)=S(x+y))$
        \item $\forall x(x\times 0 =0)$
        \item $\forall x\forall y(x\times S(y)=(x\times y)+x)$
      \end{itemize}
    \end{itemize}
    Agli assiomi propri si aggiunge anche il seguente assioma proprio:
    {\footnotesize{\[(\phi(0,x_1,\ldots,x_n)\land(\forall
      x(\phi(x,x_1,\ldots,x_n)\implies\phi(S(x),x_1,\ldots,x_n))\implies\forall 
      x\phi(x,x_1,\ldots,x_n)\]}}
    per ogni FBF $\phi(x,x_1,\ldots,x_n)$ di cui
    $x,x_1,\ldots,x_n$ sono variabili libere. Questo è uno schema di assiomi
    detto \textbf{schema di induzione} e si ha un assioma per ogni FBF
    $\phi$ 
  \end{definizione}
  \begin{definizione}
    Definiamo, in logica classica, il \textbf{principio del terzo escluso} che
    stabilisce che una proposizione e la sua negazione hanno valore opposto, non
    avendo una ``terza opzione''. In logica classica è una \textbf{tautologia}.
  \end{definizione}
  \begin{definizione}
    Un termine è un \textbf{termine chiuso} sse non contiene delle variabili
    individuali.  
  \end{definizione}
  \begin{definizione}
    Una \textbf{formula chiusa} è una formula costruita nel linguaggio dei
    predicati in cui o non compaiono variabili o tutte le variabili presenti
    sono vincolate a un quantificatore e sono dunque variabili legate. 
  \end{definizione}
\end{shaded}
\begin{esempio}
  Vediamo qualche esempio:
  \[\forall x,y\in\mathbb{N},\,\,\exists z\in\mathbb{N}\mbox{ t.c. }
    mcd(x,y,z)\]
  ovvero $z$ è l'\textit{mcd} di $x$ e $y$.\\
  Un altro esempio:
  \[\forall x\in \mathbb{N},\,\,\exists y\in \mathbb{N}\mbox{ t.c. }
    fatt(x,y)\]
  ovvero $y$ è il fattoriale di $z$.
\end{esempio}
Formule come quelle dell'esempio possono essere lette come \textbf{specifiche
  del problema di trovare un algoritmo totalmente corretto} che calcoli il
risultato di tale problema per ogni input valido. Questa lettura non è implicita
nella logica classica, dove non è richiesto di stabilire come viene prodotto il
risultato. Si ha quindi a che fare con una lettura di un problema algoritmico di
interesse per un informatico.\\
Le \textbf{dimostrazioni} di questa tipologia di formule, nell'ambito
dell'\textbf{aritmetica di Peano (\textit{PA})}, sono quindi interpretabili come
gli algoritmi che calcolano le funzioni specificate. \\
Come \textit{vantaggi} di questa ``atteggiamento'' si ha che:
\begin{itemize}
  \item l'attenzione si concentra su costruire la dimostrazione, sui passi
  dimostrativi, e non sulla stesura del codice
  \item i passi elementari della dimostrazione sono automatici
  \item la correttezza della dimostrazione è verificabile in modo automatico
  \item l'estrazione/sintesi dell'algoritmo dalla dimostrazione è diretta. Una
  volta che si ha la dimostrazione corretta si può estrarre direttamente
  l'algoritmo. Tale algoritmo è totalmente corretto rispetto alla specifica
\end{itemize}
La difficoltà si trasferisce dall'ambito convenzionale della programmazione e
codifica dell'algoritmo in se alla costruzione dimostrazione e dei passi
dimostrativi.\\
Si hanno quindi anche degli \textit{svantaggi}, abbastanza problematici:
\begin{itemize}
  \item l'algoritmo ottenuto non è ottimale rispetto al problema. Rispetto a
  questo bisognerebbe capire come incorporare ``più semantica'' del problema da
  risolvere nella dimostrazione stessa
  \item il formalismo e il linguaggio delle dimostrazioni sono ``lontani'' da
  quelli usati usualmente nella pratica informatica
\end{itemize}
\textbf{Non tutte le dimostrazioni sono direttamente interpretabili come
  algoritmi}. Per vedere questa cosa prendiamo un esempio famoso di formula da
dimostrare in analisi.
\begin{esempio}[esempio di Troelstra]
  Esistono due numeri irrazionali $n$ e $m$ tali che $n^m$ è razionale. In
  termini di formula del primo ordine si ha quindi:
  \[\exists n,m\in\{\mathbb{R}/\mathbb{Q}\}\mbox{ t.c. }n^m\in\mathbb{Q}\]
  Cerchiamo di capire se:
  \[\sqrt{2}^{\sqrt{2}}\in \mathbb{Q}\mbox{ o }\sqrt{2}^{\sqrt{2}}\not\in
    \mathbb{Q}\]
  Vediamo quindi i due casi (sono solo due per il principio del terzo escluso):
  \begin{enumerate}
    \item assumo $\sqrt{2}^{\sqrt{2}}\in \mathbb{Q}$ e pongo $n=m=\sqrt{2}$
    avendo trovato due numeri irrazionali $n$ e $m$ tali per cui
    $n^m\in\mathbb{Q}$ 
    \item assumo $\sqrt{2}^{\sqrt{2}}\not\in \mathbb{Q}$ e pongo
    $n=\sqrt{2}^{\sqrt{2}}$ e $m=\sqrt{2}$. Ne segue che:
    \[n^m=\left(\sqrt{2}^{\sqrt{2}}\right)^{\sqrt{2}}=(\sqrt{2})^2=2\in\mathbb{Q}\] 
    e quindi ho due numeri irrazionali $n$ e $m$ tali per cui $n^m\in\mathbb{Q}$
  \end{enumerate}
  Non possiamo essere soddisfatti di questa dimostrazione. Non veniamo a
  conoscenza, tramite la dimostrazione, che $\sqrt{2}^{\sqrt{2}}$ sia o meno
  razionale. Non possiamo capirlo in quanto assumo il terzo escluso e quindi non
  so quale dei due casi sia valido, non abbiamo un ``esiste'' costruttivo
  ($\exists n,m\in \{\mathbb{R}/\mathbb{Q}\}$) in quanto non sappiamo se
  $\sqrt{2}^{\sqrt{2}}$ è razionale o meno. Nonostante ciò al dimostrazione sta
  perfettamente ``in piedi'' ma non esibisce $n$ e $m$ in quanto non determina
  se $\sqrt{2}^{\sqrt{2}}\in\mathbb{Q}$.
\end{esempio}
Quanto successo nell'esempio di Troelstra non può succedere in un
\textbf{sistema costruttivo}.
\begin{definizione}
  Definiamo \textbf{sistema costruttivo} un sistema dove si hanno come
  \textit{requisiti minimali}: 
\begin{itemize}
  \item $S\vdash A\lor B\implies S\vdash A\mbox{ oppure } S\vdash B$ quindi se
  nel sistema dimostro $A\lor B$ allora nel sistema dimostro $A$ o dimostro $B$,
  con $A$ e $B$ formule chiuse. Questa è la \textbf{disjunction property
    (\textit{DP})}  
  \item $S\vdash\exists xA(x)\implies S\vdash A(t)$ quindi se ho dimostrato un
  esistenziale allora deve esistere un termine chiuso $t$ per cui dimostro
  $A(t)$ nel sistema. Questa è la \textbf{explicitly definibility property
    (\textit{EDP})}, detta anche \textbf{existence/witness property}
\end{itemize}
\end{definizione}
La logica classica quindi \textbf{non è un sistema costruttivo} perché in logica
classica riesco sempre a dimostrare $A\lor\neg A$ mentre nell'esempio di
Troelstra si nota come non si possa dimostrare né $A$ né $\neg A$. Quindi da una
dimostrazione classica di $A\lor\neg A$ io non tiro fuori una dimostrazione
classica di $A$ oppure una dimostrazione classica di $\neg A$ quindi non vale la
DP. Inoltre non vale nemmeno la EDP, ho dimostrato l'esistenza di $n$ e $m$ (che
sono termini chiusi) ma non ho trovato se vale la proprietà che siano
irrazionali. La logica classica quindi non è una logica costruttiva.\\
%20.00
\end{document}  
% LocalWords:  clock  Bayesana machine learning Bayes riscalare
% LocalWords:  condizionalmente
