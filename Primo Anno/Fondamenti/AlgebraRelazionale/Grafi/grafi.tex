\section{Grafi}
Si definisce come \emph{grafo $G$} una coppia $(V,E)$ in cui $V$ è l'insieme
dei \textbf{vertici} o \textbf{nodi}, indicanti gli elementi, invece $E$
 è l'insieme degli \textbf{archi}, indicanti la relazione esistente tra i vertici del grafo.\newline
Numero di archi = numero di nodi - 1

Se in grafo tutti gli archi presentano un ordinamento, ossia si definisce una direzione
tra i 2 vertici, si definisce il grafo \emph{orientato} altrimenti il grafo è \emph{non orientato}.
%Inserire Esempi

Un arco che congiunge $V_i$ a $V_j$ si dice \emph{uscente} da $V_i$ ed \emph{entrante} in $V_j$.

\subsection{Nomenclatura}
In un grafo si definisce:
\begin{description}
    \item[NODO SORGENTE]: nodi in cui non si hanno archi entranti
    \item[NODO POZZO]: nodi in cui non si hanno archi uscenti
    \item[NODO ISOLATO]: nodi in cui non si hanno archi entranti né uscenti
    \item[GRADO DI ENTRATA]: è il numero di archi entranti in un nodo
    \item[GRADO DI USCITA]: è il numero degli archi uscenti da un nodo
    \item[CAMMINO da $V_{in}$ a $V_{fin}$]:è una sequenza finita di nodi $(V_1,V_2,\dots,V_n)$
     con $V_1 = V_{in}$ e $V_n = V_{fin}$, dove ciascun nodo è collegato al successivo da un arco orientato
    \item[SEMICAMMINO da $V_{in}$ a $V_{fin}$]: è una sequenza finita di nodi
     $(V_1,V_2,\dots,V_n)$ con $V_1 = V_{in}$ e $V_n = V_{fin}$, dove ciascun nodo
     è collegato al successivo da un arco non orientato.
    \item[CONNESSO]:un grafo in cui dati due nodi $V_a$ e $V_b$, con $V_a \neq V_b$,
                    esiste un semicammino tra di essi.
    \item[CICLO]intorno un nodo $V$ è un cammino in cui $V = V_{in} = V_{fin}$
    \item[SEMICICLO]intorno un nodo $V$ è un semicammino in cui $V = V_{in} = V_{fin}$
    \item[CAPPIO]intorno ad un nodo è un cammino di lunghezza 1 in cui $V_in = V_{fin}$
\end{description}

%Inserire esempi ed esercizi
%Esempio dimostrazione per induzione
\begin{thm}
$\sum _{k=0} ^ n (4k+1)= (2n+1)(n+1)$
\end{thm}

\begin{proof}
Caso Base $n = 0$:
\begin{equation*}
    \sum _{k = 0} ^ 0 (4k+1) = (2*0 +1)(0+1) \quad 1 = 1 \ \text{vero}
\end{equation*}
Caso passo:
\quad Ipotesi:$\sum _{k = 0} ^ n (4k+1) = (2n+1)(n+1)$\newline
\quad Tesi: $\sum _{k = 0} ^ {n+1} (4k+1) = (2n+3)(n+2)$
\begin{equation*}
\begin{split}
\sum _{k = 0} ^ {n+1} (4k+1) & = \sum _{k = 0} ^ n (4k+1) + 4(n+1) + 1 \\
                             & = (2n+1)(n+1) + 4n + 5\\
                             & = 2n^2+3n+1+4n+5 \\
                             & = (2n+3)(n+2)\\
\end{split}
\end{equation*}
\end{proof}


% Proprietà Grafi
\subsection{Proprietà dei Grafi}
Le proprietà delle relazioni si riflettono in proprietà dei grafi.

\begin{defi}
Sia $G$ una relazione binaria su un insieme $V$
\end{defi}
\begin{enumerate}
    \item Se $G$ è riflessiva allora il corrispettivo grafo avrà un cappio intorno ogni nodo
    \item Se $G$ è una relazione irriflessiva allora nel grafo non ci sono cappi
    \item Se $G$ è una relazione simmetrica allora il grafo non è orientato
    \item Se $G$ è una relazione asimmetrica allora tra due nodi non ci sarà mai un arco e il suo inverso
    \item Se $G$ è una relazione transitiva allora nel grafo qualora vi siano gli archi
          tra $x_1 \mapsto x_2$ e tra $x_2 \mapsto x_3$ vi è l'arco tra $x_1 \mapsto x_3$
\end{enumerate}

%Paragrafo sui Dag e gli Alberi
%Esempi di Alberi
\begin{forest}
    [20
        [10
            [3]
            [5]
        ]
        [40
            [25]
            [80]
        ]
    ]
\end{forest}

%Esempio Alberi di Ricerca
\begin{forest}
    [20
        [10
            [3]
            [5]
        ]
        [40
            [25]
            [80]
        ]
    ]
\end{forest}

