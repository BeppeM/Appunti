%Ricorsione definizione
\section{Ricorsione}
La ricorsione è funzione definitoria che consiste nel definire un insieme
di elementi base e di definire gli altri elementi mediante il richiamo di se stessa,
fino ad arrivare ai casi base.

Esempio:
la definizione ricorsiva del fattoriale è definita come segue:
\begin{equation*}
    n! = \begin{cases} 1 \quad n = 0 \lor n = 1 \\ n * (n-1)! \quad n > 1\\
\end{cases}
\end{equation*}

la definizione del coefficiente binomiale è definita come segue:
\begin{equation*}
    \bigl( ^ n _ k \bigr) = \begin{cases} 1 \quad n = k \lor k = 0 \\
                             n \quad k = n-1 \lor k = 1 \\
                             (^{n-1} _{k-1}) + (^{n-1} _k) \quad \text{altrimenti} \\
                \end{cases}
\end{equation*}

la definizione di $somma:Z x Z \mapsto Z$ è la seguente:
\begin{equation*}
    somma(a,b) = \begin{cases} a \quad b = 0 \\
                               somma(b,a) \quad a < b \\
                               1 + somma(b-1) \quad b > 0\\
                               -1 + somma(b+1) \quad b < 0 \\
                  \end{cases}
\end{equation*}
