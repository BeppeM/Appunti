Gli insiemi sono collezioni di oggetti detti elementi, in cui si prescinde dall'
ordine e dalla ripetizione degli elementi e questa è la sua definizione ingenua.

Si dice $x \in A$ se l'elemento appartiene all'insieme altrimenti si usa $x \not \in A$.
L'insieme vuoto si indica con $\emptyset$ mentre due insiemi con gli stessi elementi si usa $A = B$.\newline
Si definisce \textit{cardinalità} il numero degli elementi e si indica con $|A|$.

Gli insiemi possono essere definiti in due maniere:
\begin{itemize}
  \item \textbf{Estensionale}:si elencano gli elementi di un insieme\newline
        Esempio:
        \begin{equation*}
        \begin{split}
            A = \{1,2,3\} & |A| = 3 \\
            B = \{2,2,4,6\} & |B| = 3 \\
        \end{split}
        \end{equation*}
  \item \textbf{Intensionale}:si descrivono gli elementi che soddisfano una determinata proprietà\newline
        Esempio: \newline
        $D = \{x \in \N | x < 100\}$
\end{itemize}

Dati due insiemi $S$ e $T$ si ha:
\begin{align*}
  S \subset T & \ \{x | (x \in S \rightarrow x \in T) \land S \not = T \} \\
  S = T & \  \{x | x \in S \ \text{se e solo se} \ x \in T \} \\
  S \subseteq T & \ \{x | S \subset Q \ \text{oppure} \ S = T \}
\end{align*}

Esempio:
\begin{align*}
A = \{1,2,3,5,6,7,9,10 \}  & \quad A = B  \ \text{Falso}\\
B = \{1,3,5,6,9 \}  & \quad B \subseteq A \ \text{Vero}\\
C = \{1,2,3,5,6,7,9,10 \} & \quad A \subset C \ \text{Falso}\\
\end{align*}

Due insiemi $S$ e $T$ si dicono \textit{equipotenti}, indicato con $S \sim T$, se
essi sono in corrispondenza univoca, ossia hanno la stessa cardinalità.
\begin{equation*}
\begin{split}
A = \{ 1,4,7,10,43,52,4 \} & \ |A| = 6 \\
B = \{1,1,2,3,4,5,6,7,8,9 \} & \ |B| = 9 \\
\end{split}
\end{equation*}
$A$ e $B$ non sono equipotenti in quanto la loro cardinalità non coincide mentre
nel seguente esempio i due insiemi sono equipotenti:
\begin{equation*}
\begin{split}
C = \{0,2,4,6,8,10,12,14,16,18\} & \ |C| = 10 \\
D = \{1,3,5,7,9,11,13,15,17,19\} & \ |D| = 10 \\
\end{split}
\end{equation*}

%Esempi utili sull'insieme vuoto

L'insieme vuoto $\emptyset$ viene rappresentato come $\{ \ \}$ mentre l'insieme contentente
l'insieme vuoto $\{ \emptyset \}$ viene rappresentato come $\{ \{ \ \} \}$ per cui,
anche se può risultare controintuitivo, l'insieme vuoto è diverso dall'insieme
contenente soltanto l'insieme vuoto.

