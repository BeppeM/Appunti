
Gli insiemi numerici definiti nella Teoria degli insiemi sono i seguenti:
\begin{itemize}
  \item $\N$: insieme dei numeri naturali, comprendente anche lo $O$
  \item $\Z$: insieme dei numeri interi
  \item $\Q$: insieme dei numeri razionali
  \item $\R$: insieme dei numeri reali
  \item $\C$: insieme dei numeri complessi
\end{itemize}
Gli insiemi $\N, \Z, \Q$ hanno la stessa cardinalità indicata con $\aleph_0$,dimostrato da Georg Cantor.\newline
Gli insiemi $\R$ e $\C$ hanno la stessa cardinalità indicata con $ 2 ^ {\aleph_0}$,
dimostrato da Georg Cantor mediante il principio di diagonalizzazione.

%Tecnica di diagonalizzazione
\subsection{Tecnica di diagonalizzazione}
La tecnica di diagonalizzazione è una tecnica, inventata da Georg Cantor, per dimostrare la
non numerabilità dei Numeri Reali.\newline
Essa consiste nel tentativo di costruire una biiezione tra un insieme $X$ ed $\N$
e verificare che qualche elemento di $X$ sfugge alla biiezione.

\begin{thm}
    $2^{\aleph_0}$ è strettamente maggiore di $\aleph_0$
\end{thm}

%DA DIMOSTRARE
