\section{Prodotto Cartesiano e Coppie Ordinate}
Le coppie Ordinate sono una collezione di 2 oggetti in cui non si prescinde
dall'ordine e dalla ripetizione infatti $(a,b) \neq (b,a)$ e $(1,2,1,4) \neq (1,2,4)$.
Dal punto di visto insiemistico la coppia ordinata è $\{\{x\},\{x,y\}\}$.

Una \emph{n-upla} ordinata $x_1,\dots,x_n$ è definita come $(x1,\dots,x_n) = ( (x_1,\dots,x_{n-1}),x_n)$
dove $(x_1,\dots,x_{n-1})$ è una $(n-1)$-upla ordinata.

Si definisce come \emph{lista}, una sequenza di oggetti in cui non si prescinde
dalla multiplicità degli elementi.

Dati 2 insiemi $A$ e $B$, non necessariamente  distinti, si definisce come \textit{Prodotto Cartesiano},
indicato con $A \times B$, l'insieme delle coppie in cui il primo elemento appartiene ad $A$
e il secondo elemento della coppia appartiene ad $B$.\newline
$A \times B = \{(a,b) | a \in A \ \text{e} \ b \in B\} $
\begin{align*}
A & = \{1,2,3\} \\
B & = \{1,4,5\} \\
A \times B & = \{(1,1),(1,4),(1,5),(2,1),(2,4),(2,5),(3,1),(3,4),(3,5)\} \\
S & = \{ 4,14,56 \} \\
T & = \{ 3,46,12 \} \\
S \times T & = \{(4,3),(4,46),(4,12),(14,3),(14,46),(14,12),(56,3),(56,46),(56,12) \} \\
\end{align*}

%MultiInsiemi
\section{Multiinsiemi}
Si definisce come \emph{multiinsiemi} una collezione di elementi in cui si prescinde
dall'ordine ma non dalla multiplicità degli elementi.\newline
Si puo anche definire come una funzione $M:E \mapsto \N$ che associa ad ogni elemento
di un insieme $E$ finito o numerabile, un numero, appartenente ad $\N$ indicante
il numero di occorrenze dell'elemento di $E$ nel multiinsieme $M$.\newline
La cardinalità di un multiinsieme $M$ è definita come
\begin{equation*}
\displaystyle |M| = \sum{_{e_i \in E} ^ {|E|}} M(e_i)
\end{equation*}
\begin{align*}
S = (1,2,1,3,4,4,2,3,3) \\
T = (2,1,3,1,4,4,3,2,3)
\end{align*}
Sono due multiinsiemi uguali con cardinalità 9

\subsection{Operazioni su Multiinsiemi}
\begin{description}
    \item[Intersezione]: $M_1 \cap M_2 = M_3$ dove $M_3(e) = min(M_1(e),M_2(e))$ per ogni $e \in E$.
    \item[Unione]: $M_1 \cup M_2 = M_3$ dove $M_3(e) = max(M_1(e),M_2(e))$ per ogni $e \in E$.
    \item[Unione Disgiunta]: $M_1 \uplus M_2 = M_3$ dove $M_3(e) = M_1(e) + M_2(e)$.
\end{description}

%Esempi
Esempio:
\begin{equation*}
\begin{split}
A & = (1,3,1,1,2,3,5,6,6) \\
B & = (2,3,5,6,3,6,3,7,3) \\
A \cup B & = (1,1,1,2,3,3,3,3,5,6,6,7) \\
A \cap B & = (2,3,3,5,6,6) \\
A \uplus B & = (1,1,1,2,2,3,3,3,3,3,3,5,5,6,6,6,6,7)\\
\end{split}
\end{equation*}
