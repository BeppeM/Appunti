\section{Unione ed Intersezione}
L'unione di due insiemi $S \cup T$ è l'insieme formato degli elementi di S e degli
elementi di T.\newline
L'intersezione di due insiemi $S \cap T$ è l'insieme degli elementi presenti in
tutti e due gli insiemi.

$S \cup T = \{x | x \in S \ \text{o} \ x \in T \} $ \newline
$S \cap T = \{x | x \in S \ \text{e} \ x \in T \} $

Esempio:
\begin{align*}
A = & \{Rosso,Arancio,Giallo \} \\
B = & \{Verde,Giallo,Marrone \} \\
A \cup B = & \{Rosso,Arancio,Giallo,Verde,Marrone \} \\
A \cap B = & \{Giallo \} \\
S = \{1,2,5,4,3,7,6,9\}  & \ S \cup T = \{1,2,3,4,5,6,7,9,11,34\} \\
T = \{5,4,2,9,11,34,6\}  & \ S \cap T = \{2,4,5,6,9\} \\
C = \{3,9,15,21,27\}  & \ C \cup D = \{3,6,9,15,21,24,27,33,42,51,60\} \\
D = \{6,15,24,33,42,51,60\} & \ C \cap D = \{15\}\\
\end{align*}

%Proprietà dell'Unione
\begin{prop}
    L'unione gode delle seguenti proprietà:
\end{prop}
\begin{enumerate}
\item $S \cup S = S$ \quad Idempotenza
\item $S \cup \emptyset = S$ \quad Elemento Neutro
\item $S_1 \cup S_2 = S_2 \cup S_1$ \quad Associatività
\item $S_1 \cup S_2 = S_2 \ \text{se e solo se} \ S_1 \subseteq S_2$
\item $(S_1 \cup S_2) \cup S_3 = S_1 \cup (S_2 \cup S_3)$ \quad Commutatività
\item $S_1 \subseteq S_1 \cup S_2$
\item $S_2 \subseteq S_1 \cup S_2$
\end{enumerate}

%dimostrazione
%\begin{proof}
%\begin{enumerate}
%    \item $S \cup S = \{ x | x \in S \lor x \in S \} = S$
%    \item $S \cup \emptyset = \{ x | x \in \lor x \in \emptyset \} = S$
%    \item $S_1 \cup S_2 = \{ x | x \in S_1 \lor x \in S_2 \}$ per definizione di Insieme
%           si può scrivere anche $\{ x | x \in S_2 \lor x \in S_1 \} = S_2 \cup S_1$
%    \item Prima dimostriamo che $S_1 \cup S_2 = S_2 \rightarrow S_1 \subseteq S_2$
%          $S_1 \cup S_2 = \{ x | x \in S_1 \lor x \in S_2 \}$
%          Per ipotesi sappiamo che $S_1 \cup S_2 = S_2$ per cui tutti gli elementi di $S_1$
%          sono anche elementi di $S_2$ per cui si dimostra che $S_1 \subseteq S_2$.
%          In maniera analoga si dimostra che $S_1 \subseteq S_2 \rightarrow S_1 \cup S_2 = S_2$.
%    \item Da Fare
%    \item Da Fare
%    \item Da fare
%\end{enumerate}
%\end{proof}

%Proprietà dell'Intersezione
\begin{prop}
    L'intersezione gode delle seguenti proprietà:
\end{prop}
\begin{enumerate}
  \item $S \cap S = S$
  \item $S \cap \emptyset = \emptyset$
  \item $S_1 \cap S_2 = S_2 \cap S_1$
  \item $(S_1 \cap S_2) \cap S_3 = S_1 \cap (S_2 \cap S_3)$
  \item $S_1 \cap S_2 = S_1 \ \text{se e solo se} \ S_1 \subseteq S_2$
  \item $S_1 \cap S_2 \subseteq S_1$
  \item $S_1 \cap S_2 \subseteq S_2$
\end{enumerate}

%dimostrazione
%\begin{proof}
%    \item $S \cap S = \{ x | x \in S \land x \in S \} = S$
%    \item $S \cap \emptyset = \{ x | x \in S \land x \in \emptyset \} = \emptyset$
%    \item $S_1 \cap S_2 = \{ x | x \in S_1 \land x \in S_2 \}$ per definizione di insieme
%           si può scrivere anche $\{ x | x \in S_2 \land x \in S_1 \} = S_2 \cap S_1 $
%    \item Da Fare
%    \item Da Fare
%    \item Da Fare
%    \item Da Fare
%\end{proof}

\begin{prop}
L'unione e l'intersezione hanno le prop. distributive:
\end{prop}
\begin{enumerate}
  \item $S_1 \cap (S_2 \cup S_3) = (S_1 \cap S_2) \cup (S_1 \cap S_3)$
  \item $S_1 \cup (S_2 \cap S_3) = (S_1 \cup S_2) \cap (S_1 \cup S_3)$
\end{enumerate}

%Dimostrazione
