\section{Composizione di Relazioni}
Data una relazione $R_1 \subseteq S \times T$ e una relazione $R_2 \subseteq T \times Q$,
si definisce come relazione composta $R_1 \circ R_2 \subseteq S \times Q$ come segue
$(a,c) \in R_1 \circ R_2$ se e solo se  esiste $b \in T$ tale che $(a,b) \in R_1$ e $(b,c) \in R_2$

Siano $S = \{ a,b \}, R_1 = \{ (a,a),(a,b),(b,b) \}$ e $R_2 = \{ (a,b),(b,a),(b,b) \}$
$R_1 \circ R_2 = \{ (a,a),(a,b),(b,a),(b,b) \}$ \newline
$R_2 \circ R_1 = \{ (a,b),(b,a),(b,b) \}$

\begin{prop}
La composizione di Relazioni è associativa
\end{prop}
%Fare la Dimostrazione

\begin{thm}
Se $R_1 \subseteq S \times T$ e $R_2 \subseteq T \times Q$, allora $(R_1 \circ R_2)^{-1} = R_1^{-1} \circ R_2^{-1}$
\end{thm}
%Fare Dimostrazione

%Potenza di una relazione

\begin{equation*}
\begin{split}
S & = \{ 1,3,5,4,6\}\\
T & = \{10,12,22,24,45\}\\ 
Q & = \{23,46,78,23,12\} \\
R_1 \subset S \times T & = \{(1,22),(3,24),(5,45),(6,24)\} \\
R_2 \subset T \times Q & = \{(10,46),(10,78),(22,46),(22,12),(45,23)\} \\
R_1 \circ R_2 & = \{(1,46),(1,23),(5,23)\} \\
\end{split}
\end{equation*}
