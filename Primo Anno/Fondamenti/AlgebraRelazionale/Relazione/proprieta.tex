Dubbio sulla definizione con dominio e Codominio diverso
Data una relazione $R$ definita su un dominio $S$ si definiscono le seguenti proprietà:
\begin{itemize}
  \item Riflessiva se e solo se $\forall x \in S$ risulta  $xRx$
  \item Irriflessiva se e solo se $\forall x \in S$ risulta $x \slashed{R} x$
  \item Simmetrica se e solo se $\forall x,y \in S$ risulta $xRy \rightarrow yRx$
  \item Asimmetrica: se e solo se $\forall x,y \in S$ risulta $xRy \rightarrow y \slashed{R} x$
  \item Antisimmetrica: se e solo se $\forall x,y \in S$ si ha $xRy \land yRx$ implica $x = y$
  \item Transitiva: se e solo se $\forall x,y,z \in S$ si ha $xRy \land yRz$ implica che $xRz$
\end{itemize}

%Esempi
Esempio:
\begin{itemize}
    \item Essere padre di: è una relazione non transitiva,irriflessiva e asimmetrica.
    \item Essere parenti: è una relazione simmetrica,transitiva e irriflessiva.
    \item Essere sposati: è una relazione non transitiva,irriflessiva e simmetrica.
    \item $< \ \subseteq N \times N$:è una relazione asimmetrica,transitiva e irriflessiva.
    \item $\leq \ \subseteq N \times N$:è una relazione riflessiva,transitiva e antisimmetrica.
\end{itemize}



Sulle relazioni si possono applicare le usuali operazioni insiemistiche quindi, ad esempio,
date $R_1 \subseteq S \times T$ e $R_2 \subseteq S \times T$ anche $R_1 \cup R_2$ è una relazione su $S \times T$.

%Definizione Relazione Complementare e Relazione Inversa
Data una relazione binaria $R \subseteq S \times T$ definiamo \emph{relazione complementare}
$\bar{R} \subseteq S \times T$ come $x \bar{R} y$ se e solo se $(x,y) \not \in R$.
Per definizione si ha $\bar{\bar{R}} = R$ e $R \cup \bar{R} = S \times T$.

Data una relazione binaria $R \subseteq S \times T$ esiste sempre la \emph{relazione inversa}
$R^-1 = \{(y,x) | (x,y) \in R\} \subseteq T \times S$.
Per definizione $(R ^ {-1}) ^ {-1} = R$
