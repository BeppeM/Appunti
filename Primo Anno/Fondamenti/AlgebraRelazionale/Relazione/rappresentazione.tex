\section{Rappresentazione di Relazioni}
Vi sono diverse modalità di rappresentazione delle relazioni,il cui metodo migliore
dipendono dall'arietà della relazione, che sono:
\begin{description}
    \item[Tabella a $n$ colonne] è una matrice a due dimensioni con righe,rappresentanti
          gli elementi, e colonne, indicanti gli insiemi; è conveniente utilizzare
          quando l'arietà della relazione è $n \geq 2$.
    \item[Grafo Bipartito] è un grafo in cui si elencano gli elementi di tutti gli insiemi
         e si usano delle frecce, chiamate archi, per indicare l'associazione tra gli elementi.
         E' meglio utilizzare il grafo bipartito soltanto per le relazioni binarie.
    \item[Matrice Booleana] è una matrice $M_R$ a valori \{0,1\} composta da $n$ righe e $m$ colonne.
    \item[Grafi] modalità di rappresentazione di relazioni binarie(spiegate in un paragrafo successivo)
\end{description}
Migliorare i Grafi bipartiti,i grafi e se possibile riuscire a disegnare le matrici booleane
\subsection{Tabelle}
Si vuole definire la relazione $anagrafica \subseteq Cognomi \times Nomi \times Date \times Luoghi$

\begin{tabular}{cccc}
\toprule
C & N & D & L \\
\midrule
Rossi & Arturo & 14/2/67 & Milano \\
Bianchi & Vincenzo & 13/5/68 & Udine \\
Verdi & Filippo & 23/5/78 & Crema \\
\bottomrule
\end{tabular}

\subsection{Grafo Bipartito}
Dato  $A = \{ 1,2,3,4 \}$ si ha $R \subseteq A \times A = \{ (1,1),(1,2),(2,3),(3,1),(3,4) \}$.
%Dimostrazione Tableaux Predicativo
Esempio:$\exists x (P(x) \lor Q(x)) \rightarrow (\exists x P(x) \lor \forall y Q(y))$
\begin{equation*}
\begin{prooftree}
\hypo{F \exists x (P(x) \lor Q(x)) \rightarrow (\exists x P(x) \lor \forall y Q(y))}
\infer1 {T \exists x (P(x) \lor Q(x)), F \exists x P(x) \lor \forall y Q(y)}
\infer1{T P(a) \lor Q(a), F \exists x P(x) \lor \forall y Q(y)}
\infer1 {T P(a) \lor Q(a), F \exists x P(x), F \forall y Q(y)}
\infer1 {T P(a) \lor Q(a), F \exists x P(x), F Q(b)}
\infer1 {T P(a),F \exists x P(x),F Q(b)/T Q(a),F Q(b),F \exists x P(x)}
\end{prooftree}
\end{equation*}
Il secondo ramo del tableau non potrà mai chiudere percui bisogna fare il T-Tableaux

\begin{equation*}
\begin{prooftree}
\hypo{T \exists x (P(x) \lor Q(x)) \rightarrow (\exists x P(x) \lor \forall y Q(y))}
\infer1 {F \exists x P(x) \lor Q(x)/T \exists x P(x) \lor \forall y Q(y)}
\infer1 {F \exists x P(x) \lor Q(x)/T \exists x P(x)/T \forall y Q(y)}
\end{prooftree}
\end{equation*}
Il tableaux non potrà mai chiudere in quanto il secondo e il terzo non generanno mai delle contraddizioni
per cui la formula è sodddisfacibile non tautologica.
%Grafo esempio


\subsection{Matrice Booleana}
La \emph{Matrice booleana} è una matrice $M_R$,composta da $n$ righe e $m$ colonne,
i cui elementi sono definiti come
\begin{equation*}
    m_{ij} = \begin{cases} 1 \ \text{se e solo se} \ (s_i,t_j) \in R \\
                           0 \quad \text{altrimenti} \\
             \end{cases}
\end{equation*}

%Esempio:\newline
\begin{thm}
 $\displaystyle \sum_{i = 1}^n 2i-1 = n^2$
\end{thm}

\begin{proof}
Per $n = 1 \quad \displaystyle \sum_{i = 1}^1 2i-1 = 1^2 \quad 1 = 1$ è vero

Se $\displaystyle \sum_{i = 1}^n 2i-1 = n^2$ allora
$\displaystyle \sum_{i = 1}^{n+1} 2i-1 = (n+1)^2$

\begin{equation*}
\begin{split}
  \sum_{i=1}^{n+1} 2i-1 & = \sum_{i=1}^n 2i-1 + 2(n+1) - 1 \\
         & = n^2 + 2n + 1 \\
         & = (n+1)^2 \\
\end{split}
\end{equation*}
\end{proof}
%Esempio Matrice Booleana

Da una matrice booleana si possono determinare facilmente le proprietà
di una relazione $R$,definita su $S$, soprattutto la proprietà simmetrica e la riflessiva.

La \emph{Matrice Complementare} $M_{\bar{R}}$ è costituita dai seguenti elementi
\begin{equation*}
    \bar{m}_{ij} = \begin{cases} 1 \ \text{se e solo se} \ m_{ij} = 0 \\
                                 0 \ \text{se e solo se} \ m_{ij} = 1 \\
                   \end{cases}
\end{equation*}

La \emph{Matrice inversa} $M_{R ^{-1}}$ è la trasporta della matrice $M_R$, ossia la matrice
in cui si scambiano le righe con le colonne e viceversa.
%Da fare la dimostrazione
Correggere definizione prodotto booleano
È sbagliata (Gianlo e  Davide fatelo voi ahahah)
Date due matrici $A$ e $B$,embrambe di $n \times m$ elementi, si definiscono 3 operazioni:
\begin{description}
    \item[MEET $A \sqcap B = C$]: è una matrice booleana i cui elementi sono:\newline
$c_{ij} =  \begin{cases} 1 \quad a_{ij} = 1 \land b_{ij} = 1 \\ 0 \quad a_{ij} = 0 \text{altrimenti} \end{cases}$
    \item[JOIN $A \sqcup B = C$]: è una matrice booleana i cui elementi sono:\newline
    $c{ij} = \begin{cases} 1 \quad a_{ij} = 1 \lor b_{ij} = 1 \\
                           0 \quad \text{altrimenti}\\ \end{cases}$
    \item[PRODOTTO BOOLEANO $A \odot B$]: è una matrice booleana $n \times p$, i cui elementi sono:\newline
    $c_{ij} = \begin{cases} 1 \quad \text{se per qualche} \ k(1 \leq k \leq m) \text{si ha} \ a_{ik} = 1 \land b_{kj} = 1\\
                            0 \quad \text{altrimenti} \\ \end{cases}$
\end{description}

%\begin{thm}
    $\displaystyle n! \geq 2^{(n-1)} \quad \forall n \in \N$
\end{thm}
%Dimostrazione
\begin{proof}
Per $n = 0$ si ha $0! \geq 2^{0-1}$ ossia $ 1 \geq 1/2$ che è sempre verificato

Se $n! \geq 2^{(n-1)}$ allora $(n+1)! \geq 2^n$
\begin{equation*}
\begin{split}
(n+1)! \geq 2^{n+1-1} & = n! (n+1) \geq 2^{n-1} * 2 \\
                      & = n! \frac{(n+1)}{2} \geq 2^{n-1} \\
\end{split}
\end{equation*}
Essendo $\frac{(n+1)}{2} \geq 0 \forall n \in \N$ e $n! \geq 2^{n-1}$ verificato per ipotesi
si ricava che la proposizione è verificata.
\end{proof}
Esempio sulle operazioni join,meet e prodotto booleano

%Grafi
\section{Grafi}
Si definisce come \emph{grafo $G$} una coppia $(V,E)$ in cui $V$ è l'insieme
dei \textbf{vertici} o \textbf{nodi}, indicanti gli elementi, invece $E$
 è l'insieme degli \textbf{archi}, indicanti la relazione esistente tra i vertici del grafo.\newline
Numero di archi = numero di nodi - 1

Se in grafo tutti gli archi presentano un ordinamento, ossia si definisce una direzione
tra i 2 vertici, si definisce il grafo \emph{orientato} altrimenti il grafo è \emph{non orientato}.
%Inserire Esempi

Un arco che congiunge $V_i$ a $V_j$ si dice \emph{uscente} da $V_i$ ed \emph{entrante} in $V_j$.

\subsection{Nomenclatura}
In un grafo si definisce:
\begin{description}
    \item[NODO SORGENTE]: nodi in cui non si hanno archi entranti
    \item[NODO POZZO]: nodi in cui non si hanno archi uscenti
    \item[NODO ISOLATO]: nodi in cui non si hanno archi entranti né uscenti
    \item[GRADO DI ENTRATA]: è il numero di archi entranti in un nodo
    \item[GRADO DI USCITA]: è il numero degli archi uscenti da un nodo
    \item[CAMMINO da $V_{in}$ a $V_{fin}$]:è una sequenza finita di nodi $(V_1,V_2,\dots,V_n)$
     con $V_1 = V_{in}$ e $V_n = V_{fin}$, dove ciascun nodo è collegato al successivo da un arco orientato
    \item[SEMICAMMINO da $V_{in}$ a $V_{fin}$]: è una sequenza finita di nodi
     $(V_1,V_2,\dots,V_n)$ con $V_1 = V_{in}$ e $V_n = V_{fin}$, dove ciascun nodo
     è collegato al successivo da un arco non orientato.
    \item[CONNESSO]:un grafo in cui dati due nodi $V_a$ e $V_b$, con $V_a \neq V_b$,
                    esiste un semicammino tra di essi.
    \item[CICLO]intorno un nodo $V$ è un cammino in cui $V = V_{in} = V_{fin}$
    \item[SEMICICLO]intorno un nodo $V$ è un semicammino in cui $V = V_{in} = V_{fin}$
    \item[CAPPIO]intorno ad un nodo è un cammino di lunghezza 1 in cui $V_in = V_{fin}$
\end{description}

%Inserire esempi ed esercizi
%Esempio dimostrazione per induzione
\begin{thm}
$\sum _{k=0} ^ n (4k+1)= (2n+1)(n+1)$
\end{thm}

\begin{proof}
Caso Base $n = 0$:
\begin{equation*}
    \sum _{k = 0} ^ 0 (4k+1) = (2*0 +1)(0+1) \quad 1 = 1 \ \text{vero}
\end{equation*}
Caso passo:
\quad Ipotesi:$\sum _{k = 0} ^ n (4k+1) = (2n+1)(n+1)$\newline
\quad Tesi: $\sum _{k = 0} ^ {n+1} (4k+1) = (2n+3)(n+2)$
\begin{equation*}
\begin{split}
\sum _{k = 0} ^ {n+1} (4k+1) & = \sum _{k = 0} ^ n (4k+1) + 4(n+1) + 1 \\
                             & = (2n+1)(n+1) + 4n + 5\\
                             & = 2n^2+3n+1+4n+5 \\
                             & = (2n+3)(n+2)\\
\end{split}
\end{equation*}
\end{proof}


% Proprietà Grafi
\subsection{Proprietà dei Grafi}
Le proprietà delle relazioni si riflettono in proprietà dei grafi.

\begin{defi}
Sia $G$ una relazione binaria su un insieme $V$
\end{defi}
\begin{enumerate}
    \item Se $G$ è riflessiva allora il corrispettivo grafo avrà un cappio intorno ogni nodo
    \item Se $G$ è una relazione irriflessiva allora nel grafo non ci sono cappi
    \item Se $G$ è una relazione simmetrica allora il grafo non è orientato
    \item Se $G$ è una relazione asimmetrica allora tra due nodi non ci sarà mai un arco e il suo inverso
    \item Se $G$ è una relazione transitiva allora nel grafo qualora vi siano gli archi
          tra $x_1 \mapsto x_2$ e tra $x_2 \mapsto x_3$ vi è l'arco tra $x_1 \mapsto x_3$
\end{enumerate}

%Paragrafo sui Dag e gli Alberi
%Esempi di Alberi
\begin{forest}
    [20
        [10
            [3]
            [5]
        ]
        [40
            [25]
            [80]
        ]
    ]
\end{forest}

%Esempio Alberi di Ricerca
\begin{forest}
    [20
        [10
            [3]
            [5]
        ]
        [40
            [25]
            [80]
        ]
    ]
\end{forest}


