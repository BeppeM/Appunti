%Capitolo sulle funzioni
\chapter{Funzioni}
Si definisce \textit{funzione $f:S \mapsto T$} una relazione $f \subseteq S \times T$
tale che $\forall x \in S$ esiste al più un $y \in T$ per cui $(x,y) \in f$.\newline
Se il $dom(f) = S$ la funzione si dice \emph{totale} altrimenti la funzione è \emph{parziale}.

%Tipologie di funzioni
\section{Tipologie di Funzioni}
Una funzione $f:S \mapsto T$ si dice:
\begin{description}
    \item[iniettiva]: $\forall x,y \in S$ con $x \neq y$ si verifica $f(x) \neq f(y)$.
    \item[suriettiva]: $\forall y \in T \exists x \in S$ tale che $f(x) = y$.
    \item[biettiva]: se la funzione è iniettiva e suriettiva
\end{description}
%Determinare se una funzione è iniettiva e suriettiva
Per determinare se una funzione è iniettiva bisogna verificare che $f(x) \neq f(y)$
comporta $x \neq y$.
Per determinare se una funzione è suriettiva bisogna risolvere l'equazione $f(x) = y$
e verificare se $y$ appartiene al codominio della funzione.

%Funzione inversa
Una funzione $f:S \mapsto T$ è detta \emph{invertibile} se la sua relazione inversa
$f ^ -1$ è essa stessa una funzione.\newline
\textbf{Condizione di invertibilità}: una funzione $f:S \mapsto T$ ammette una \emph{funzione inversa}
 $f ^{-1} :T \mapsto S$ se e solo se $f$ è una funzione iniettiva.

%Proprietà funzioni Inverse
\begin{thm}
Sia $f:A \mapsto B$ invertibile, con funzione inversa $f ^ -1$:
\end{thm}
\begin{enumerate}
    \item $f^{-1}$ è totale se e solo se $f$ è suriettiva
    \item $f$ è totale se e solo se $f^{-1}$ è suriettiva
\end{enumerate}

%Esempi
Esempi:\newline
$+:\N x \N \mapsto \N$ è una funzione totale,suriettiva ma non iniettiva \newline
$*:\N x \N \mapsto \N$ è una funzione totale,suriettiva ma non è iniettiva \newline
$successore:\N \mapsto \N$ è una funzione totale ed è iniettiva ma non suriettiva \newline
$successore:\Z \mapsto \Z$ è una funzione totale ed è biettiva \newline
$x^2:\N \mapsto \N$ è una funzione totale,iniettiva,non suriettiva,invertibile \newline
$x^2:\Z \mapsto \Z$ è una funzione totale,non iniettiva,non suriettiva,non invertibile. \newline

%Funzione Composta
Date due funzioni $f:S \mapsto T$ e $g:T \mapsto Q$ si definisce \emph{funzione composta}
$g \circ f:S \mapsto Q$ la funzione tale che $(g \circ f)(x) = g(f(x))$ per ogni $x \in S$.
La funzione composta $(g \circ f)(x)$ è definita se e solo se sono definite entrambe
$g(f(x))$ e $f(x)$.\newline
\textbf{Condizione di componibilità}: codominio della prima coincide col dominio della seconda.

%Esempi funzioni composte
Può capitare a volte di avere una funzione composta non definita in quanto non coincide
il dominio con il codominio ma la funzione risulta calcolabile;in quel caso si dice
che la funzione non è composta ma è calcolabile come ad esempio:

%Inserire esempi di funzioni composte

\begin{thm}
Siano $f:S \mapsto T$ e $g:T \mapsto Q$ invertibili. Allora $g \circ f$ è invertibile
e la sua inversa è $(g \circ f) ^{-1} = f^{-1} \circ g ^{-1}$.
\end{thm}
%Inserire dimostrazione

%Operazione
\section{Operazioni}
Si definisce come \emph{operazione n-aria} su un insieme $S$, una funzione
$f:S^n \mapsto S$ con $n \geq 1$.
Se $f$ è un'operazione binaria su $S$, essa si può rappresentare anche mediante
la notazione infissa $x_1 f x_2$ invece di $f(x_1,x_2)$

%Inserire esempi!!!!

%Definizione Funzioni Monotone
\subsection{Funzioni monotone}
\begin{defi}
    Siano $(S,\leq _S)$ e $(T,\leq _T)$ due poset e sia $f:S \mapsto T$ una funzione allora:
\end{defi}
\begin{enumerate}
    \item $f$ è detta \emph{monotona non decrescente} quando $f(x) \leq_T f(y) \iff x \leq_S y$
    \item $f$ è detta \emph{monotona crescente} quando $f(x) <_T f(y) \iff x <_S y$
    \item $f$ è detta \emph{monotona non crescente} quando $f(x) \leq_T f(y) \iff x \geq_S y$
    \item $f$ è detta \emph{monotona decrescente} quando $f(x) <_T f(y) \iff x >_S y$
\end{enumerate}
