%Esempio Tableaux proposizionali
Formula:$(A \rightarrow (B \land (C \lor D))) \iff ((E \land D) \rightarrow (A \lor B))$
\begin{proof}
\begin{equation*}
\begin{prooftree}
    \hypo{F (A \rightarrow (B \land (C \lor D))) \iff ((E \land D) \rightarrow (A \lor B))}
    \infer1{T A \rightarrow (B \land (C \lor D)),F (E \land D) \rightarrow (A \lor B)/}
    \infer1{F A \rightarrow (B \land (C \lor D)),T (E \land D) \rightarrow (A \lor B)}
\end{prooftree}
\end{equation*}
Decido di analizzare i due rami del tableaux in maniera separata per maggiore chiarezza
per cui analizzo prima $T A \rightarrow (B \land (C \lor D)),F (E \land D) \rightarrow (A \lor B)$
\begin{equation*}
\begin{prooftree}
\hypo{T A \rightarrow (B \land (C \lor D)),F (E \land D) \rightarrow (A \lor B)}
\infer1 {T A \rightarrow (B \land (C \lor D)),T E \land D,F A \lor B}
\infer1 {T A \rightarrow (B \land (C \lor D)),TE,TD,F A \lor B}
\infer1 {T A \rightarrow (B \land (C \lor D)),TE,TD,FA,FB}
\infer1 {FA,TE,TD,FA,FB/TE,TD,FA,FB,B \land (C \lor D)}
\end{prooftree}
\end{equation*}
Il primo ramo $FA,TE,TD,FA,FB$ non chiude per cui per stabilire la tipologia di
formula bisogna svolgere il T-Tableaux
\begin{equation*}
\begin{prooftree}
\hypo{T (A \rightarrow (B \land (C \lor D))) \iff ((E \land D) \rightarrow (A \lor B))}
\infer1 {T A \rightarrow (B \land (C \lor D)), T (E \land D) \rightarrow (A \lor B)/}
\infer1 {F A \rightarrow (B \land (C \lor D)),F (E \land D) \rightarrow (A \lor B)}
\end{prooftree}
\end{equation*}
Decido di analizzare i rami in maniera separata per chiarezza e decido di analizzare
per primo $F A \rightarrow (B \land (C \lor D)),F (E \land D) \rightarrow (A \lor B)$
\begin{equation*}
\begin{prooftree}
\hypo{F A \rightarrow (B \land (C \lor D)),F (E \land D) \rightarrow (A \lor B)}
\infer1{TA,F B \land (C \lor D),F (E \land D) \rightarrow (A \lor B)}
\infer1{TA,F B \land (C \lor D),T E \land D,F A \lor B}
\infer1{TA,F B \land (C \lor D),T E \land D,FA,FB}
\end{prooftree}
\end{equation*}
Il primo ramo del tableaux chiude per cui bisogna analizzare ora il secondo ramo
$T A \rightarrow (B \land (C \lor D)), T (E \land D) \rightarrow (A \lor B)$
\begin{equation*}
\begin{prooftree}
\hypo{T A \rightarrow (B \land (C \lor D)), T (E \land D) \rightarrow (A \lor B)}
\infer1{FA,T (E \land D) \rightarrow (A \lor B)/T B \land (C \lor D),T (E \land D) \rightarrow (A \lor B)}
\infer1{FA,F E \land D/FA,T A \lor B/TB,T C \lor D,T (E \land D) \rightarrow (A \lor B)}
\infer1{FA,FE/FA,FD/FA,T A \lor B/TB,T C \lor D,T (E \land D) \rightarrow (A \lor B)}
\end{prooftree}
\end{equation*}
I primi due rami non chiudono per cui la formula è soddisfacibile non tautologica.
\end{proof}
