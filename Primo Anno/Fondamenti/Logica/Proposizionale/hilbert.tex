%Appunti sul sistema deduttivo Hilbertiano
Il sistema deduttivo Hilbertiano è un sistema assiomatico in cui si individuano
una serie di proposizione, detti assiomi, da cui partire per dimostrare altre proposizioni,
limitando quindi il numero di regole di inferenza.\newline
Il sistema Hilbertiano è stato introdotto prima del sistema di Gentzen e formalizza
il sistema matematico di fare le dimostrazioni.

Nel sistema hilbertiano sono presenti 3 sistemi di Assiomi:
\begin{itemize}
    \item $(A \rightarrow (B \rightarrow A))$
    \item $(A \rightarrow (B \rightarrow C)) \rightarrow ((A \rightarrow B) \rightarrow (A \rightarrow C))$
    \item $(B \rightarrow \neg A) \rightarrow ((B \rightarrow A) \rightarrow \neg B)$
\end{itemize}
Nel sistema Hilbertiano c'è soltanto una regola di inferenza, detta modus ponens:
\begin{equation*}
\begin{prooftree}
\hypo{A \quad A \rightarrow B}
\infer1{B}
\end{prooftree}
\end{equation*}

%Esempio di dimostrazione |- A --> A


Già per fare questa banale dimostrazione si è dovuto ricorrere a una notevole mole
di lavoro per cui l'utilità del sistema di Hilbert dipende dall'introduzione di
regole più potenti di $MP$, dette \emph{regole derivate}.

Per ciascuna regola derivata dobbiamo dimostrare che sia corretta, ossia che l'uso di
una regola derivata non estende il numero di formule dimostrabili.
Per fare questo dobbiamo mostrare che è possibile trasformare la dimostrazione fatta
con la regola derivata in una dimostrazione fatta usando solo gli assiomi e il $MP$.

La prima e la più importante regola derivata è la \emph{regola di deduzione},
che è la comune tecnica d'inferenza matematica in cui si assume verificata la premessa
dell'implicazione da verificare.
\begin{equation*}
\begin{prooftree}
\hypo{U \cup \{A\} \vdash B}
\infer1{U \vdash A \rightarrow B}
\end{prooftree}
\end{equation*}

\begin{defi}
Sia $U$ un insieme di formule e $A$ una formula.
La notazione $U \vdash A$ indica che le formule in $U$ sono le ipotesi della dimostrazione di $A$,
cioè si possono considerare come degli assiomi nella dimostrazione.
\end{defi}

\begin{thm}
La regola di deduzione è una regola derivata corretta
\end{thm}

%Fare la dimostrazione

%Regola di Contrapposizione
Un altra regola derivata è la \emph{regola di contrapposizione}, che può essere
dedotta facilmente dal terzo assioma:
\begin{equation*}
\begin{prooftree}
\hypo{\vdash \neg B \rightarrow \neg A}
\infer1{\vdash A \rightarrow B}
\end{prooftree}
\end{equation*}

%Fare dimostrazione e/o inserire esempi

Un altra regola derivata è la \emph{regola di transitività}, che giustifica lo sviluppo
di un teorema per passi $A \rightarrow B$ utilizzando una serie di teoremi intermedi chiamati lemmi:
\begin{equation*}
\begin{prooftree}
\hypo{U \vdash A \rightarrow B}
\hypo{U \vdash B \rightarrow C}
\infer2{U \vdash A \rightarrow C}
\end{prooftree}
\end{equation*}
%Esercizi ed Esempio

%Regola dello scambio delle promesse
Un altra regola derivata è la regola dello scambio delle premesse che è la seguente:
\begin{equation*}
\hypo{U \vdash A \rightarrow (B \rightarrow C)}
\infer1{U \vdash B \rightarrow (A \rightarrow C)}
\end{equation*}
