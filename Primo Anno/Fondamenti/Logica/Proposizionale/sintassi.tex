\section{Linguaggio e Sintassi}
Il linguaggio di una logica proposizionale è composto dai seguenti elementi:

%Elementi linguaggio logica proposizionale
\begin{itemize}
  \item Variabili Proposizionali: $P,Q,R \dots$
  \item Connettivi Proposizionali: $\land, \lor, \neg, \rightarrow, \iff$
  \item Simboli Ausiliari: (,)
  \item Costanti: $T,F$
\end{itemize}

La sintassi di un linguaggio è composta da una serie di formule ben formate($FBF$) definite
induttivamente nel seguente modo:
%definizione formule ben formate
\begin{enumerate}
  \item Le costanti e le variabili proposizionali $\in FBF$.
  \item Se $A$ e $B \in FBF$ allora $(A \land B)$,$(A \lor B)$,$(\neg A)$,$(A \rightarrow B)$,
        $(A \iff B)$,$TA$ e $FA$ sono delle formule ben formate.
  \item nient'altro è una formula
\end{enumerate}

Esempio:\newline
$(P \land Q) \in Fbf$  è una formula ben formata\newline
$(PQ \land R) \not \in Fbf$ in quanto non si rispetta la sintassi del linguaggio definita.\newline

%Definizione delle sottoformule
Sia $A \in FBF$, l'insieme delle sottoformule di $A$ è definito come segue:
\begin{enumerate}
\item Se $A$ è una costante o variabile proposizionale allora A stessa è la sua sottoformula
\item Se $A$ è una formula del tipo $(\neg A')$ allora le sottoformule di A sono A stessa e le sottoformule di $A'$;
      $\neg$ è detto connettivo principale e $A'$ sottoformula immediata di A.
\item Se $A$ è una formula del tipo $B \circ C$, allora le sottoformule di A sono A stessa
      e le sottoformule di B e C;$\circ$ è il connettivo principale e B e C sono le due sottoformule immediate di A.
\end{enumerate}

É possibile ridurre ed eliminare delle parentesi attraverso l'introduzione della
precedenza tra gli operatori, che è definita come segue:\newline
$\neg, \land, \lor, \rightarrow,\iff$.

In assenza di parentesi una formula va parentizzata privileggiando le sottoformule
i cui connettivi principali hanno la precedenza più alta.\newline
In caso di parità di precedenza vi è la convenzione di associare da destra a sinistra.

Esempio:\newline
$\neg A \land (\neg B \rightarrow C) \lor D$ diventa
$((\neg A) \land ((\neg B) \rightarrow C) \lor D)$.

%Definizione di albero sintattico
\begin{defi}
    Un albero sintattico $T$ è un albero binario coi nodi etichettati da simboli
di $L$, che rappresenta la scomposizione di una formula ben formata $X$ definita come segue:
\end{defi}
\begin{enumerate}
    \item Se $X$ è una formula atomica,l'albero binario che la rappresenta è composto
          soltanto dal nodo etichettato con $X$
    \item Se $X = A \circ B$, $X$ è rappresentata da un albero binario che ha la radice
          etichettata con $\circ$, i cui figli sinistri e destri sono la rappresentazione di $A$ e $B$
    \item Se $X = \neg A$,$X$ è rappresentato dall'albero binario con radice etichettata
          con $\neg$, il cui figlio è la rappresentazione di $A$
\end{enumerate}

Poichè una formula è definita mediante un albero sintattico, le proprietà di una formula
possono essere dimostrate mediante induzione strutturale sulla formula, ossia dimostrare
che la proprietà di una formula soddisfi i seguenti 3 casi:
\begin{itemize}
  \item è verificata la proprietà per tutte le formule atomo $A$
  \item supposta verifica la proprietà per $A$, si verifica che la proprietà è verificata per $\neg A$
  \item supposta la proprietà verificata per $A_1$ e $A_2$, si verifica che la
        proprietà è verifica per $A_1 \circ A_2$, per ogni connettivo $\circ$.
\end{itemize}


%Inserire Esempio
