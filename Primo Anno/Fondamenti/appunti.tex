\documentclass[a4paper,11pt,fleqn]{report}
\usepackage[T1]{fontenc}%Gestione Font
\usepackage[utf8]{inputenc}%Gestione Charset Utf8
\usepackage[italian]{babel}%Gestione sillabazione Italiana
\usepackage{amsmath}%Package per la gestione della matematica
\usepackage{classicthesis}
\usepackage{arsclassica}
\usepackage{amsthm}%Package per la gestione dei Teoremi della matematica
\usepackage{amsfonts}%Package per i Font Matematici
\usepackage{booktabs}%Package per la gestione delle Tabelle
\usepackage{caption}%Package per la gestione delle Tabelle
\usepackage{slashed}%Package per fare il simbolo di non relazione
\usepackage{tkz-graph}%Package per disegnare grafi
\usepackage{forest}%package per disegnare alberi
\usepackage{ebproof}%Package per i Tableau
\setlength{\parindent}{0pt}%Toglie il rientro dei capoversi
\newtheorem{defi}{Definizione}%Definizione per avere la gestione delle definizioni
\newtheorem{prop}{Proposizione}[chapter]
\newtheorem{lem}{Lemma}
\newtheorem{thm}{Teorema}[chapter]
\newtheorem{corol}{Corollario}[chapter]
\newcommand{\numberset}{\mathbb}
\newcommand{\N}{\numberset{N}}
\newcommand{\Z}{\numberset{Z}}
\newcommand{\Q}{\numberset{Q}}
\newcommand{\R}{\numberset{R}}
\newcommand{\C}{\numberset{C}}

\begin{document}
\author{Marco Natali}
\title{Appunti di Fondamenti dell'Informatica}
\maketitle
\date{}
\tableofcontents

%Capitolo sugli insiemi
\chapter{Insiemi}

%Definizione di Insieme
\chapter{Relazioni}
Si definisce \textit{relazione n-aria} un sottoinsieme del prodotto cartesiano
rappresentato da tutte le coppie che rispettano la relazione voluta tra gli $n$ insiemi,
che può essere definita in maniera estensionale e/o intensionale. \newline
Si definisce \textit{arietà} di una relazione il numero e il tipo degli argomente
di una relazione.

%Definizione Dominio e Codominio di una relazione
\textbf{Dominio}:insieme degli elementi $x$ tali che $(x,y) \in R$ per qualsiasi $y$.\newline
\textbf{Codominio}:insieme degli elementi $y$ tali che $(x,y) \in R$ per qualsiasi $x$.

\begin{align*}
A \times B & = \{(1,1),(1,4),(1,5),(2,1),(2,4),(2,5),(3,1),(3,4),(3,5)\} \\
R \subseteq A \times B  & = \{(1,1),(1,4),(1,5),(2,4),(2,5),(3,4),(3,5)\}\\
R \subseteq B \times A & = \{(1,1),(1,2),(1,3)\} \\
\end{align*}


%Insiemi Numerici

Gli insiemi numerici definiti nella Teoria degli insiemi sono i seguenti:
\begin{itemize}
  \item $\N$: insieme dei numeri naturali, comprendente anche lo $O$
  \item $\Z$: insieme dei numeri interi
  \item $\Q$: insieme dei numeri razionali
  \item $\R$: insieme dei numeri reali
  \item $\C$: insieme dei numeri complessi
\end{itemize}
Gli insiemi $\N, \Z, \Q$ hanno la stessa cardinalità indicata con $\aleph_0$,dimostrato da Georg Cantor.\newline
Gli insiemi $\R$ e $\C$ hanno la stessa cardinalità indicata con $ 2 ^ {\aleph_0}$,
dimostrato da Georg Cantor mediante il principio di diagonalizzazione.

%Tecnica di diagonalizzazione
\subsection{Tecnica di diagonalizzazione}
La tecnica di diagonalizzazione è una tecnica, inventata da Georg Cantor, per dimostrare la
non numerabilità dei Numeri Reali.\newline
Essa consiste nel tentativo di costruire una biiezione tra un insieme $X$ ed $\N$
e verificare che qualche elemento di $X$ sfugge alla biiezione.

\begin{thm}
    $2^{\aleph_0}$ è strettamente maggiore di $\aleph_0$
\end{thm}

%DA DIMOSTRARE


%Unione ed Intersezione
\section{Unione ed Intersezione}
L'unione di due insiemi $S \cup T$ è l'insieme formato degli elementi di S e degli
elementi di T.\newline
L'intersezione di due insiemi $S \cap T$ è l'insieme degli elementi presenti in
tutti e due gli insiemi.

$S \cup T = \{x | x \in S \ \text{o} \ x \in T \} $ \newline
$S \cap T = \{x | x \in S \ \text{e} \ x \in T \} $

Esempio:
\begin{align*}
A = & \{Rosso,Arancio,Giallo \} \\
B = & \{Verde,Giallo,Marrone \} \\
A \cup B = & \{Rosso,Arancio,Giallo,Verde,Marrone \} \\
A \cap B = & \{Giallo \} \\
S = \{1,2,5,4,3,7,6,9\}  & \ S \cup T = \{1,2,3,4,5,6,7,9,11,34\} \\
T = \{5,4,2,9,11,34,6\}  & \ S \cap T = \{2,4,5,6,9\} \\
C = \{3,9,15,21,27\}  & \ C \cup D = \{3,6,9,15,21,24,27,33,42,51,60\} \\
D = \{6,15,24,33,42,51,60\} & \ C \cap D = \{15\}\\
\end{align*}

%Proprietà dell'Unione
\begin{prop}
    L'unione gode delle seguenti proprietà:
\end{prop}
\begin{enumerate}
\item $S \cup S = S$ \quad Idempotenza
\item $S \cup \emptyset = S$ \quad Elemento Neutro
\item $S_1 \cup S_2 = S_2 \cup S_1$ \quad Associatività
\item $S_1 \cup S_2 = S_2 \ \text{se e solo se} \ S_1 \subseteq S_2$
\item $(S_1 \cup S_2) \cup S_3 = S_1 \cup (S_2 \cup S_3)$ \quad Commutatività
\item $S_1 \subseteq S_1 \cup S_2$
\item $S_2 \subseteq S_1 \cup S_2$
\end{enumerate}

%dimostrazione
%\begin{proof}
%\begin{enumerate}
%    \item $S \cup S = \{ x | x \in S \lor x \in S \} = S$
%    \item $S \cup \emptyset = \{ x | x \in \lor x \in \emptyset \} = S$
%    \item $S_1 \cup S_2 = \{ x | x \in S_1 \lor x \in S_2 \}$ per definizione di Insieme
%           si può scrivere anche $\{ x | x \in S_2 \lor x \in S_1 \} = S_2 \cup S_1$
%    \item Prima dimostriamo che $S_1 \cup S_2 = S_2 \rightarrow S_1 \subseteq S_2$
%          $S_1 \cup S_2 = \{ x | x \in S_1 \lor x \in S_2 \}$
%          Per ipotesi sappiamo che $S_1 \cup S_2 = S_2$ per cui tutti gli elementi di $S_1$
%          sono anche elementi di $S_2$ per cui si dimostra che $S_1 \subseteq S_2$.
%          In maniera analoga si dimostra che $S_1 \subseteq S_2 \rightarrow S_1 \cup S_2 = S_2$.
%    \item Da Fare
%    \item Da Fare
%    \item Da fare
%\end{enumerate}
%\end{proof}

%Proprietà dell'Intersezione
\begin{prop}
    L'intersezione gode delle seguenti proprietà:
\end{prop}
\begin{enumerate}
  \item $S \cap S = S$
  \item $S \cap \emptyset = \emptyset$
  \item $S_1 \cap S_2 = S_2 \cap S_1$
  \item $(S_1 \cap S_2) \cap S_3 = S_1 \cap (S_2 \cap S_3)$
  \item $S_1 \cap S_2 = S_1 \ \text{se e solo se} \ S_1 \subseteq S_2$
  \item $S_1 \cap S_2 \subseteq S_1$
  \item $S_1 \cap S_2 \subseteq S_2$
\end{enumerate}

%dimostrazione
%\begin{proof}
%    \item $S \cap S = \{ x | x \in S \land x \in S \} = S$
%    \item $S \cap \emptyset = \{ x | x \in S \land x \in \emptyset \} = \emptyset$
%    \item $S_1 \cap S_2 = \{ x | x \in S_1 \land x \in S_2 \}$ per definizione di insieme
%           si può scrivere anche $\{ x | x \in S_2 \land x \in S_1 \} = S_2 \cap S_1 $
%    \item Da Fare
%    \item Da Fare
%    \item Da Fare
%    \item Da Fare
%\end{proof}

\begin{prop}
L'unione e l'intersezione hanno le prop. distributive:
\end{prop}
\begin{enumerate}
  \item $S_1 \cap (S_2 \cup S_3) = (S_1 \cap S_2) \cup (S_1 \cap S_3)$
  \item $S_1 \cup (S_2 \cap S_3) = (S_1 \cup S_2) \cap (S_1 \cup S_3)$
\end{enumerate}

%Dimostrazione


%Complementare e Differenza
\section{Complementare}
Dato un insieme $U$, detto Universo, si dice \textit{complemento} di $S$, indicato con $\bar{S}$,
la differenza di un sottoinsieme $S$ di $U$ rispetto ad $U$.\newline
$\bar{S} = \{x | x \in U \ \text{e} \ x \not \in S \} $
Migliorare rappresentazione simbolo complementare

Esempio:
\begin{equation*}
U = \{1,2,3,4,5,6,7\}
S = \{2,4,6\}
\bar{S} = \{1,3,5,7\}
\end{equation*}


\begin{prop}
    Le proprietà della complementazione di un insieme sono :
\end{prop}
\begin{enumerate}
  \item $\bar{U} = \emptyset $
  \item $\bar{\emptyset} = U$
  \item $\overline{\bar{S}} = S$
  \item $\overline{(S_1 \cup S_2)} = \bar{S_1} \cap \bar{S_2}$
  \item $\overline{(S_1 \cap S_2)} = \bar{S_1} \cup \bar{S_2}$
  \item $S \cap \bar{S} = \emptyset$
  \item $S \cup \bar{S} = U$
  \item $S_1 = S_2 \ \text{se e solo se} \ \bar{S_1} = \bar{S_2}$
  \item $S_1 \subseteq S_2 \ \text{se e solo se} \ \overline{S_2} \subseteq \overline{S_1}$
\end{enumerate}

\section{Differenza di Insiemi}
Dati 2 insiemi $S$ e $T$ chiamiamo $S \setminus T$ l'insieme \textit{differenza} costituito
da tutti gli elementi di S che non sono elementi di T. \newline
$S \setminus T = \{x | x \in S \ \text{e} \ x \not \in T\} $

Esempio:
\begin{align*}
S = \{a,b,c,d,e\} & \ T = \{a,c,f,g,e,h\} \\
S \setminus T = \{b,d\} \\
\end{align*}
%Proprietà differenza
\begin{prop}
    La differenza di insiemi gode delle seguenti proprietà:
\end{prop}
\begin{enumerate}
  \item $S \setminus S = \emptyset$
  \item $S \setminus \emptyset = S$
  \item $\emptyset \setminus S = \emptyset$
  \item $(S_1 \setminus S_2) \setminus S_3 =
         (S_1 \setminus S_3) \setminus S_2 = S_1 \setminus (S_2 \cup S_3)$
  \item $S_1 \setminus S_2 = S_1 \cap \bar{S_2}$
\end{enumerate}

La \textit{differenza simmetrica} di due insiemi $S_1$ e $S_2$, indicata con $S_1 \triangle S_2$,
è definita come $S_1 \triangle S_2 = (S_1 \setminus S_2) \cup (S_2 \setminus S_1) $

% proprietà diff. simmetrica
\begin{prop}
Proprietà Differenza simmetrica
\end{prop}
\begin{enumerate}
  \item $S \triangle S = \emptyset$
  \item $S \triangle \emptyset = S$
  \item $S_1 \triangle S_2 = S_2 \triangle S_1$
  \item $S_1 \triangle S_2 = (S_1 \cap \bar{S_2}) \cup (S_2 \cap \bar{S_1})$
  \item $S_1 \triangle S_2 = (S_1 \cup S_2) \ (S_2 \cap S_1)$
\end{enumerate}


%Partizione di un Insieme
\section{Partizione di un Insieme}
Dato un insieme non vuoto $S$, una partizione di $S$ è una famiglia $F$ di sottoinsiemi di $S$ tale che
\begin{enumerate}
    \item ogni elemento di $S$ appartiene a qualche elemento di $F$, ossia $\cup F = S$
    \item due elementi qualunque di $F$ sono disgiunti ossia $\cap F = \emptyset$
\end{enumerate}
La partizione non può avere come elemento l'insieme vuoto in quanto esso non
appartiene agli elementi dell'insieme A.
\begin{align*}
A = \{ 1,2,3 \}\\
Par(A) = \{ \{ 1 \},\{2,3 \} \}\\
B = \{5,10,\emptyset,23,45\}\\
Par(B) = \{ \{\emptyset,23\}, \{5,10\},\{45\}\} \\
\end{align*}


%Insieme delle Parti
\section{Funzione Caratteristica}
Sia $U$ un insieme assunto come Universo,si definisce come \emph{funzione caratteristica}
di un sottoinsieme $S \subseteq U$ come:
\begin{equation*}
    car(S,x) = \begin{cases} 1 \quad \text{se} \ x \in S \\
                             0 \quad \text{se} \ x \not \in S\\
               \end{cases}
\end{equation*}

\section{Insieme delle Parti}
L'insieme delle Parti di un insieme $S$, indicato con $\wp S$, è l'insieme formato
da tutti i sottoinsiemi dell'insieme $S$. \newline
$\wp S = \{X | X \subseteq S\} $
ridurre spazio tra i due esempi
\begin{align*}
A = \{\emptyset,3,20 \} \\
\wp A = \{A,\{\emptyset\},\{ 3\},\{ 20 \},\{ \emptyset,3 \},\{\emptyset,20 \},\{3,20\},\emptyset \} 
\end{align*}
\begin{align*}
S & = \{ \emptyset, \{ \emptyset \}, a,b\} \\
\wp S = & \{ \emptyset,\{\emptyset \},\{\{\emptyset\}\},\{a\},\{b\}, \\
        & \ \{ \emptyset,\{\emptyset\}\}, \{ \emptyset,a\},\{ \emptyset,b\} \\
        & \ \{ \emptyset,\{\emptyset\},a\},\{\emptyset,\{\emptyset\},b\}, \\
        & \ \{ \{\emptyset\},a,b\},\{ \emptyset,a,b\},\{\{\emptyset\},a\},\{\{\emptyset\},b\} \} \\
\end{align*}

\begin{defi}
    Se $S$ è composto da $n \geq 0$ elementi, il numero di elementi di $\wp(S)$ è $2 ^ n$.
\end{defi}
%Dimostrazione
\begin{proof}
Supponendo di avere una sequenza binaria di 3 bit,le cui possibili combinazioni
vengono rappresentate da $2 ^ k$ con $k = numBit$.\newline
Prendendo un insieme $A = \{'a','b','c' \}$ e utilizzando la funzione caratteristica,
con la convenzione di indicare il primo elemento dell'insieme $A$ a destra, si nota
che le sequenze di bit sono uguali alla sequenza ottenuta utilizzando la funzione caratteristica.
\end{proof}


%Paragrafo sul Prodotto Cartesiano
\section{Prodotto Cartesiano e Coppie Ordinate}
Le coppie Ordinate sono una collezione di 2 oggetti in cui non si prescinde
dall'ordine e dalla ripetizione infatti $(a,b) \neq (b,a)$ e $(1,2,1,4) \neq (1,2,4)$.
Dal punto di visto insiemistico la coppia ordinata è $\{\{x\},\{x,y\}\}$.

Una \emph{n-upla} ordinata $x_1,\dots,x_n$ è definita come $(x1,\dots,x_n) = ( (x_1,\dots,x_{n-1}),x_n)$
dove $(x_1,\dots,x_{n-1})$ è una $(n-1)$-upla ordinata.

Si definisce come \emph{lista}, una sequenza di oggetti in cui non si prescinde
dalla multiplicità degli elementi.

Dati 2 insiemi $A$ e $B$, non necessariamente  distinti, si definisce come \textit{Prodotto Cartesiano},
indicato con $A \times B$, l'insieme delle coppie in cui il primo elemento appartiene ad $A$
e il secondo elemento della coppia appartiene ad $B$.\newline
$A \times B = \{(a,b) | a \in A \ \text{e} \ b \in B\} $
\begin{align*}
A & = \{1,2,3\} \\
B & = \{1,4,5\} \\
A \times B & = \{(1,1),(1,4),(1,5),(2,1),(2,4),(2,5),(3,1),(3,4),(3,5)\} \\
S & = \{ 4,14,56 \} \\
T & = \{ 3,46,12 \} \\
S \times T & = \{(4,3),(4,46),(4,12),(14,3),(14,46),(14,12),(56,3),(56,46),(56,12) \} \\
\end{align*}

%MultiInsiemi
\section{Multiinsiemi}
Si definisce come \emph{multiinsiemi} una collezione di elementi in cui si prescinde
dall'ordine ma non dalla multiplicità degli elementi.\newline
Si puo anche definire come una funzione $M:E \mapsto \N$ che associa ad ogni elemento
di un insieme $E$ finito o numerabile, un numero, appartenente ad $\N$ indicante
il numero di occorrenze dell'elemento di $E$ nel multiinsieme $M$.\newline
La cardinalità di un multiinsieme $M$ è definita come
\begin{equation*}
\displaystyle |M| = \sum{_{e_i \in E} ^ {|E|}} M(e_i)
\end{equation*}
\begin{align*}
S = (1,2,1,3,4,4,2,3,3) \\
T = (2,1,3,1,4,4,3,2,3)
\end{align*}
Sono due multiinsiemi uguali con cardinalità 9

\subsection{Operazioni su Multiinsiemi}
\begin{description}
    \item[Intersezione]: $M_1 \cap M_2 = M_3$ dove $M_3(e) = min(M_1(e),M_2(e))$ per ogni $e \in E$.
    \item[Unione]: $M_1 \cup M_2 = M_3$ dove $M_3(e) = max(M_1(e),M_2(e))$ per ogni $e \in E$.
    \item[Unione Disgiunta]: $M_1 \uplus M_2 = M_3$ dove $M_3(e) = M_1(e) + M_2(e)$.
\end{description}

%Esempi
Esempio:
\begin{equation*}
\begin{split}
A & = (1,3,1,1,2,3,5,6,6) \\
B & = (2,3,5,6,3,6,3,7,3) \\
A \cup B & = (1,1,1,2,3,3,3,3,5,6,6,7) \\
A \cap B & = (2,3,3,5,6,6) \\
A \uplus B & = (1,1,1,2,2,3,3,3,3,3,3,5,5,6,6,6,6,7)\\
\end{split}
\end{equation*}

%Capitolo sugli Insiemi
%Capitolo sulle relazioni

%definizione
\chapter{Relazioni}
Si definisce \textit{relazione n-aria} un sottoinsieme del prodotto cartesiano
rappresentato da tutte le coppie che rispettano la relazione voluta tra gli $n$ insiemi,
che può essere definita in maniera estensionale e/o intensionale. \newline
Si definisce \textit{arietà} di una relazione il numero e il tipo degli argomente
di una relazione.

%Definizione Dominio e Codominio di una relazione
\textbf{Dominio}:insieme degli elementi $x$ tali che $(x,y) \in R$ per qualsiasi $y$.\newline
\textbf{Codominio}:insieme degli elementi $y$ tali che $(x,y) \in R$ per qualsiasi $x$.

\begin{align*}
A \times B & = \{(1,1),(1,4),(1,5),(2,1),(2,4),(2,5),(3,1),(3,4),(3,5)\} \\
R \subseteq A \times B  & = \{(1,1),(1,4),(1,5),(2,4),(2,5),(3,4),(3,5)\}\\
R \subseteq B \times A & = \{(1,1),(1,2),(1,3)\} \\
\end{align*}


%proprietà delle Relazioni
Dubbio sulla definizione con dominio e Codominio diverso
Data una relazione $R$ definita su un dominio $S$ si definiscono le seguenti proprietà:
\begin{itemize}
  \item Riflessiva se e solo se $\forall x \in S$ risulta  $xRx$
  \item Irriflessiva se e solo se $\forall x \in S$ risulta $x \slashed{R} x$
  \item Simmetrica se e solo se $\forall x,y \in S$ risulta $xRy \rightarrow yRx$
  \item Asimmetrica: se e solo se $\forall x,y \in S$ risulta $xRy \rightarrow y \slashed{R} x$
  \item Antisimmetrica: se e solo se $\forall x,y \in S$ si ha $xRy \land yRx$ implica $x = y$
  \item Transitiva: se e solo se $\forall x,y,z \in S$ si ha $xRy \land yRz$ implica che $xRz$
\end{itemize}

%Esempi
Esempio:
\begin{itemize}
    \item Essere padre di: è una relazione non transitiva,irriflessiva e asimmetrica.
    \item Essere parenti: è una relazione simmetrica,transitiva e irriflessiva.
    \item Essere sposati: è una relazione non transitiva,irriflessiva e simmetrica.
    \item $< \ \subseteq N \times N$:è una relazione asimmetrica,transitiva e irriflessiva.
    \item $\leq \ \subseteq N \times N$:è una relazione riflessiva,transitiva e antisimmetrica.
\end{itemize}



Sulle relazioni si possono applicare le usuali operazioni insiemistiche quindi, ad esempio,
date $R_1 \subseteq S \times T$ e $R_2 \subseteq S \times T$ anche $R_1 \cup R_2$ è una relazione su $S \times T$.

%Definizione Relazione Complementare e Relazione Inversa
Data una relazione binaria $R \subseteq S \times T$ definiamo \emph{relazione complementare}
$\bar{R} \subseteq S \times T$ come $x \bar{R} y$ se e solo se $(x,y) \not \in R$.
Per definizione si ha $\bar{\bar{R}} = R$ e $R \cup \bar{R} = S \times T$.

Data una relazione binaria $R \subseteq S \times T$ esiste sempre la \emph{relazione inversa}
$R^-1 = \{(y,x) | (x,y) \in R\} \subseteq T \times S$.
Per definizione $(R ^ {-1}) ^ {-1} = R$


%Modalità di Rappresentazione delle Relazioni
\section{Rappresentazione di Relazioni}
Vi sono diverse modalità di rappresentazione delle relazioni,il cui metodo migliore
dipendono dall'arietà della relazione, che sono:
\begin{description}
    \item[Tabella a $n$ colonne] è una matrice a due dimensioni con righe,rappresentanti
          gli elementi, e colonne, indicanti gli insiemi; è conveniente utilizzare
          quando l'arietà della relazione è $n \geq 2$.
    \item[Grafo Bipartito] è un grafo in cui si elencano gli elementi di tutti gli insiemi
         e si usano delle frecce, chiamate archi, per indicare l'associazione tra gli elementi.
         E' meglio utilizzare il grafo bipartito soltanto per le relazioni binarie.
    \item[Matrice Booleana] è una matrice $M_R$ a valori \{0,1\} composta da $n$ righe e $m$ colonne.
    \item[Grafi] modalità di rappresentazione di relazioni binarie(spiegate in un paragrafo successivo)
\end{description}
Migliorare i Grafi bipartiti,i grafi e se possibile riuscire a disegnare le matrici booleane
\subsection{Tabelle}
Si vuole definire la relazione $anagrafica \subseteq Cognomi \times Nomi \times Date \times Luoghi$

\begin{tabular}{cccc}
\toprule
C & N & D & L \\
\midrule
Rossi & Arturo & 14/2/67 & Milano \\
Bianchi & Vincenzo & 13/5/68 & Udine \\
Verdi & Filippo & 23/5/78 & Crema \\
\bottomrule
\end{tabular}

\subsection{Grafo Bipartito}
Dato  $A = \{ 1,2,3,4 \}$ si ha $R \subseteq A \times A = \{ (1,1),(1,2),(2,3),(3,1),(3,4) \}$.
%Dimostrazione Tableaux Predicativo
Esempio:$\exists x (P(x) \lor Q(x)) \rightarrow (\exists x P(x) \lor \forall y Q(y))$
\begin{equation*}
\begin{prooftree}
\hypo{F \exists x (P(x) \lor Q(x)) \rightarrow (\exists x P(x) \lor \forall y Q(y))}
\infer1 {T \exists x (P(x) \lor Q(x)), F \exists x P(x) \lor \forall y Q(y)}
\infer1{T P(a) \lor Q(a), F \exists x P(x) \lor \forall y Q(y)}
\infer1 {T P(a) \lor Q(a), F \exists x P(x), F \forall y Q(y)}
\infer1 {T P(a) \lor Q(a), F \exists x P(x), F Q(b)}
\infer1 {T P(a),F \exists x P(x),F Q(b)/T Q(a),F Q(b),F \exists x P(x)}
\end{prooftree}
\end{equation*}
Il secondo ramo del tableau non potrà mai chiudere percui bisogna fare il T-Tableaux

\begin{equation*}
\begin{prooftree}
\hypo{T \exists x (P(x) \lor Q(x)) \rightarrow (\exists x P(x) \lor \forall y Q(y))}
\infer1 {F \exists x P(x) \lor Q(x)/T \exists x P(x) \lor \forall y Q(y)}
\infer1 {F \exists x P(x) \lor Q(x)/T \exists x P(x)/T \forall y Q(y)}
\end{prooftree}
\end{equation*}
Il tableaux non potrà mai chiudere in quanto il secondo e il terzo non generanno mai delle contraddizioni
per cui la formula è sodddisfacibile non tautologica.
%Grafo esempio


\subsection{Matrice Booleana}
La \emph{Matrice booleana} è una matrice $M_R$,composta da $n$ righe e $m$ colonne,
i cui elementi sono definiti come
\begin{equation*}
    m_{ij} = \begin{cases} 1 \ \text{se e solo se} \ (s_i,t_j) \in R \\
                           0 \quad \text{altrimenti} \\
             \end{cases}
\end{equation*}

%Esempio:\newline
\begin{thm}
 $\displaystyle \sum_{i = 1}^n 2i-1 = n^2$
\end{thm}

\begin{proof}
Per $n = 1 \quad \displaystyle \sum_{i = 1}^1 2i-1 = 1^2 \quad 1 = 1$ è vero

Se $\displaystyle \sum_{i = 1}^n 2i-1 = n^2$ allora
$\displaystyle \sum_{i = 1}^{n+1} 2i-1 = (n+1)^2$

\begin{equation*}
\begin{split}
  \sum_{i=1}^{n+1} 2i-1 & = \sum_{i=1}^n 2i-1 + 2(n+1) - 1 \\
         & = n^2 + 2n + 1 \\
         & = (n+1)^2 \\
\end{split}
\end{equation*}
\end{proof}
%Esempio Matrice Booleana

Da una matrice booleana si possono determinare facilmente le proprietà
di una relazione $R$,definita su $S$, soprattutto la proprietà simmetrica e la riflessiva.

La \emph{Matrice Complementare} $M_{\bar{R}}$ è costituita dai seguenti elementi
\begin{equation*}
    \bar{m}_{ij} = \begin{cases} 1 \ \text{se e solo se} \ m_{ij} = 0 \\
                                 0 \ \text{se e solo se} \ m_{ij} = 1 \\
                   \end{cases}
\end{equation*}

La \emph{Matrice inversa} $M_{R ^{-1}}$ è la trasporta della matrice $M_R$, ossia la matrice
in cui si scambiano le righe con le colonne e viceversa.
%Da fare la dimostrazione
Correggere definizione prodotto booleano
È sbagliata (Gianlo e  Davide fatelo voi ahahah)
Date due matrici $A$ e $B$,embrambe di $n \times m$ elementi, si definiscono 3 operazioni:
\begin{description}
    \item[MEET $A \sqcap B = C$]: è una matrice booleana i cui elementi sono:\newline
$c_{ij} =  \begin{cases} 1 \quad a_{ij} = 1 \land b_{ij} = 1 \\ 0 \quad a_{ij} = 0 \text{altrimenti} \end{cases}$
    \item[JOIN $A \sqcup B = C$]: è una matrice booleana i cui elementi sono:\newline
    $c{ij} = \begin{cases} 1 \quad a_{ij} = 1 \lor b_{ij} = 1 \\
                           0 \quad \text{altrimenti}\\ \end{cases}$
    \item[PRODOTTO BOOLEANO $A \odot B$]: è una matrice booleana $n \times p$, i cui elementi sono:\newline
    $c_{ij} = \begin{cases} 1 \quad \text{se per qualche} \ k(1 \leq k \leq m) \text{si ha} \ a_{ik} = 1 \land b_{kj} = 1\\
                            0 \quad \text{altrimenti} \\ \end{cases}$
\end{description}

%\begin{thm}
    $\displaystyle n! \geq 2^{(n-1)} \quad \forall n \in \N$
\end{thm}
%Dimostrazione
\begin{proof}
Per $n = 0$ si ha $0! \geq 2^{0-1}$ ossia $ 1 \geq 1/2$ che è sempre verificato

Se $n! \geq 2^{(n-1)}$ allora $(n+1)! \geq 2^n$
\begin{equation*}
\begin{split}
(n+1)! \geq 2^{n+1-1} & = n! (n+1) \geq 2^{n-1} * 2 \\
                      & = n! \frac{(n+1)}{2} \geq 2^{n-1} \\
\end{split}
\end{equation*}
Essendo $\frac{(n+1)}{2} \geq 0 \forall n \in \N$ e $n! \geq 2^{n-1}$ verificato per ipotesi
si ricava che la proposizione è verificata.
\end{proof}
Esempio sulle operazioni join,meet e prodotto booleano

%Grafi
\section{Grafi}
Si definisce come \emph{grafo $G$} una coppia $(V,E)$ in cui $V$ è l'insieme
dei \textbf{vertici} o \textbf{nodi}, indicanti gli elementi, invece $E$
 è l'insieme degli \textbf{archi}, indicanti la relazione esistente tra i vertici del grafo.\newline
Numero di archi = numero di nodi - 1

Se in grafo tutti gli archi presentano un ordinamento, ossia si definisce una direzione
tra i 2 vertici, si definisce il grafo \emph{orientato} altrimenti il grafo è \emph{non orientato}.
%Inserire Esempi

Un arco che congiunge $V_i$ a $V_j$ si dice \emph{uscente} da $V_i$ ed \emph{entrante} in $V_j$.

\subsection{Nomenclatura}
In un grafo si definisce:
\begin{description}
    \item[NODO SORGENTE]: nodi in cui non si hanno archi entranti
    \item[NODO POZZO]: nodi in cui non si hanno archi uscenti
    \item[NODO ISOLATO]: nodi in cui non si hanno archi entranti né uscenti
    \item[GRADO DI ENTRATA]: è il numero di archi entranti in un nodo
    \item[GRADO DI USCITA]: è il numero degli archi uscenti da un nodo
    \item[CAMMINO da $V_{in}$ a $V_{fin}$]:è una sequenza finita di nodi $(V_1,V_2,\dots,V_n)$
     con $V_1 = V_{in}$ e $V_n = V_{fin}$, dove ciascun nodo è collegato al successivo da un arco orientato
    \item[SEMICAMMINO da $V_{in}$ a $V_{fin}$]: è una sequenza finita di nodi
     $(V_1,V_2,\dots,V_n)$ con $V_1 = V_{in}$ e $V_n = V_{fin}$, dove ciascun nodo
     è collegato al successivo da un arco non orientato.
    \item[CONNESSO]:un grafo in cui dati due nodi $V_a$ e $V_b$, con $V_a \neq V_b$,
                    esiste un semicammino tra di essi.
    \item[CICLO]intorno un nodo $V$ è un cammino in cui $V = V_{in} = V_{fin}$
    \item[SEMICICLO]intorno un nodo $V$ è un semicammino in cui $V = V_{in} = V_{fin}$
    \item[CAPPIO]intorno ad un nodo è un cammino di lunghezza 1 in cui $V_in = V_{fin}$
\end{description}

%Inserire esempi ed esercizi
\input{Esempi/Grafi/esempio1}

% Proprietà Grafi
\subsection{Proprietà dei Grafi}
Le proprietà delle relazioni si riflettono in proprietà dei grafi.

\begin{defi}
Sia $G$ una relazione binaria su un insieme $V$
\end{defi}
\begin{enumerate}
    \item Se $G$ è riflessiva allora il corrispettivo grafo avrà un cappio intorno ogni nodo
    \item Se $G$ è una relazione irriflessiva allora nel grafo non ci sono cappi
    \item Se $G$ è una relazione simmetrica allora il grafo non è orientato
    \item Se $G$ è una relazione asimmetrica allora tra due nodi non ci sarà mai un arco e il suo inverso
    \item Se $G$ è una relazione transitiva allora nel grafo qualora vi siano gli archi
          tra $x_1 \mapsto x_2$ e tra $x_2 \mapsto x_3$ vi è l'arco tra $x_1 \mapsto x_3$
\end{enumerate}

%Paragrafo sui Dag e gli Alberi
\input{AlgebraRelazionale/Grafi/alberi}



%Sezione Composizione di relazioni
\section{Composizione di Relazioni}
Data una relazione $R_1 \subseteq S \times T$ e una relazione $R_2 \subseteq T \times Q$,
si definisce come relazione composta $R_1 \circ R_2 \subseteq S \times Q$ come segue
$(a,c) \in R_1 \circ R_2$ se e solo se  esiste $b \in T$ tale che $(a,b) \in R_1$ e $(b,c) \in R_2$

Siano $S = \{ a,b \}, R_1 = \{ (a,a),(a,b),(b,b) \}$ e $R_2 = \{ (a,b),(b,a),(b,b) \}$
$R_1 \circ R_2 = \{ (a,a),(a,b),(b,a),(b,b) \}$ \newline
$R_2 \circ R_1 = \{ (a,b),(b,a),(b,b) \}$

\begin{prop}
La composizione di Relazioni è associativa
\end{prop}
%Fare la Dimostrazione

\begin{thm}
Se $R_1 \subseteq S \times T$ e $R_2 \subseteq T \times Q$, allora $(R_1 \circ R_2)^{-1} = R_1^{-1} \circ R_2^{-1}$
\end{thm}
%Fare Dimostrazione

%Potenza di una relazione

\begin{equation*}
\begin{split}
S & = \{ 1,3,5,4,6\}\\
T & = \{10,12,22,24,45\}\\ 
Q & = \{23,46,78,23,12\} \\
R_1 \subset S \times T & = \{(1,22),(3,24),(5,45),(6,24)\} \\
R_2 \subset T \times Q & = \{(10,46),(10,78),(22,46),(22,12),(45,23)\} \\
R_1 \circ R_2 & = \{(1,46),(1,23),(5,23)\} \\
\end{split}
\end{equation*}


%Relazioni di Equivalenza
\section{Relazioni di Equivalenza}
Si definisce $R$ una \emph{relazione di equivalenza} se e solo se la relazione binaria
$R$ è riflessiva, simmetrica e transitiva.

%Inserire Esempi

Data una relazione di equivalenza $R$ definita su un insieme $S$, si definisce
\emph{classe di equivalenza} di un elemento $x \in S$ come $[x] = \{y | (x,y) \in R \}$

\begin{thm}
Se $R$ è una relazione di equivalenza su $S$, allora le classi di Equivalenza
generate da $R$ partizionano $S$
\end{thm}
%Fare la Dimostrazione

Data una relazione di equivalenza in S, la partizione che essa determina si dice
\emph{insieme quoziente} di $S$ rispetto alla relazione di equivalenza e si indica con $S/$



%Ordinamento
\section{Struttura relazionale}
Una struttura relazionale $SR$ è una $n$-upla in cui il primo componente è un Insieme
non vuoto $S$ e le rimanenti componenti sono relazioni su $S^n$.

%Tipologie di Ordinamenti
\subsection{Tipologie di Ordinamenti}
\begin{description}
    \item[PREORDINE]: è una struttura relazionale $(S,R)$ in cui $S$ è un insieme non vuoto
          e $R$ è una relazione binaria \emph{riflessiva} e \emph{transitiva} su $S x S$
    \item[ORDINE STRETTO]: è una struttura relazionale $(S,R)$ in cui $R$ è una
          relazione binaria \emph{irriflessiva} e \emph{transitiva} su $S x S$
    \item[ORDINE LARGO(POSET)]: è una struttura relazionale $(S,R)$ in cui $R$ è una
          relazione binaria \emph{riflessiva, antisimmetrica} e \emph{transitiva} su $S x S$.
\end{description}

Una relazione $R$ è un ordinamento sull'insieme $S$ se e solo se $\forall x,y \in S$
vale solo una delle proprietà di \emph{tricotomia}:
\begin{itemize}
    \item $x = y \land x \slashed{R} y \land y \slashed{R} x$
    \item $xRy \land x \neq y \land y \slashed{R} x$
    \item $(yRx) \land x \neq y \land x \slashed{R} y$
\end{itemize}

Inserire Esempii!!!!!!

\begin{prop}
Se $(S,R)$ è una struttura relazionale anche $(S,R^{-1})$ rappresenta la stessa
struttura relazionale chimata e $(S,R^{-1})$ è il \emph{duale} di $(S,R)$.
\end{prop}

%Dimostrazione della Proposizione!!!!!

%Proposizione lunghezza cicli di un grafo del poset + dimostrazione
\begin{prop}
Il grafo di un Poset non ha cicli di lunghezza maggiore di 1
\end{prop}

%Ordine Lessicografico

%Diagramma di Hasse
\subsection{Diagramma di Hasse}
Il grafo di un poset può essere rappresentato dal \emph{diagramma di Hasse}, un grafo
con le seguenti proprietà:
\begin{itemize}
    \item si prescinde dal disegnare i cappi in quanto il poset è una relazione riflessiva
    \item si prescinde dal disegnare gli archi definiti per transitività
    \item il grafo si legge dal basso verso l'alto
\end{itemize}

esempio


%Elementi estremali
%Chiusura Transitiva
Sia $(S,R)$ una struttura relazionale con $R$ è una relazione binaria su $S$.\newline
La \emph{chiusura transitiva} di $R$ è la più piccola relazione transitiva $R'$ in cui $R \subseteq R'$.
La \emph{chiusura riflessiva} di $R$ è la più piccola relazione riflessiva $R'$ in cui $R \subseteq R'$

\begin{prop}
Dato un poset $(S,R)$ e un insieme $X \subseteq S$ si hanno le seguenti proprietà:
\end{prop}
\begin{enumerate}
    \item[elem. massimale]: un elemento $s \in S$ è massimale se $\not \exists s' \in S : s \leq s'$
    \item[elem. minimale]: un elemento $s \in S$ è minimale se $\not \exists s' \in S : s \geq s'$
    \item[maggiorante]: un elemento $s \in S$ è un maggiorante se $\forall x \in X$ si ha $s \geq x$
    \item[minorante]: un elemento $s \in S$ è un minorante se $\forall x \in X$ si ha $s \leq x$
    \item[minimo maggiorante]:un elemento $s \in S$ è un minimo maggiorante, indicato con $\sqcup X$,
          se è un maggiorante e per ogni altro maggiorante $s'$ di $X$ si ha $s \leq s'$.
    \item[massimo minorante]: un elemento $s \in S$ è un massimo minorante, indicato con $\sqcap X$,
          se è un minorante e per ogni altro minorante $s'$ di $X$ si ha $s' \leq s$
    \item[massimo]: se $\sqcup X \in X$ si dice che $\sqcup X$ è un massimo.
    \item[minimo]: se $\sqcap X \in X$ si dice che $\sqcap X$ è un minimo.
\end{enumerate}
%Proprietà con dimostrazioni

%Definizione di Buon Ordinamento
\begin{defi}
    Un poset è detto buon ordinamento se e solo se per ogni sottoinsieme non vuoto esiste
    $\cap X$ e tale poset è detto \emph{ben formato}
\end{defi}



%Reticoli
\subsection{Reticoli}
Un \emph{Reticolo} è un poset $(S,R)$ in cui per ogni $x,y \in S$ esistono
il massimo minorante(indicato con $x \sqcap y$) e il minimo maggiorante(indicato con $x \sqcup y$).
Se un poset $(S,R)$ è un reticolo anche il suo poset duale è un reticolo e se due
reticoli sono isomorfi come poset allora i reticoli sono detti \emph{isomorfi}.

\begin{prop}
    Se $(L_1,\leq)$ e $(L_2,\leq)$ sono reticoli, anche $(L_1 \times L_2,\leq)$ lo è,
    con ordine parziale prodotto
\end{prop}

%Proprietà del Reticolo
\begin{defi}
Sia $(L,\leq)$ un reticolo. Presi comunque $a,b,c \in L$ valgono le seguenti proprietà:
\end{defi}
\begin{enumerate}
    \item $a \leq a \cup b$ e $b \leq a \cup b$
    \item Se $a \leq c$ e $b \leq c$, allora $a \cup b \leq c$
    \item $a \cap b \leq a$ e $a \cap b \leq b$
    \item Se $c \leq a$ e $c \leq b$ allora $c \leq a \cap b$
    \item $a \cup a = a$ (Idempotenza)
    \item $a \cap a = a$ (Idempotenza)
    \item $a \cup b = b \cup a$ (Commutativa)
    \item $a \cap b = b \cap a$ (Commutativa)
    \item $a \cup (b \cup c) = (a \cup b) \cup c$ (Associativa)
    \item $a \cap (b \cap b) = (a \cap b) \cap c$ (Associativa)
    \item $a \cup(a \cap b) = a$ (Assorbimento)
    \item $a \cap (a \cup b) = a$ (Assorbimento)
\end{enumerate}

%Reticolo Completo
\begin{defi}
    Se ogni sottoinsieme di un reticolo possiede un minimo maggiorante e un massimo minorante
    allora il reticolo si definisce \emph{completo}.
\end{defi}

%Reticolo limitato
\begin{defi}
    Un reticolo è definito \emph{limitato} se esiste un minimo e un massimo assoluti.
\end{defi}

%Reticolo distribuitivo
\begin{defi}
    Un reticolo è detto \emph{distribuitivo} se valgono le seguenti proprietà:
\end{defi}
\begin{enumerate}
    \item $a \cap (b \cup c) = (a \cap b) \cup (a \cap c)$
    \item $a \cup (b \cap c) = (a \cup b) \cap (a \cup c)$
\end{enumerate}

%Reticolo Complementato
\begin{defi}
    Sia $(L,\leq)$ un reticolo distribuitivo limitato, con massimo $1$ e minimo $0$,
e sia $a \in L$, allora $a'$ è il \emph{complemento},il quale se esiste è unico,
 di $a$ se è rispettata la seguente proprietà: $a \cup a' = 1$ e $a \cap a' = 0$.
\end{defi}

\begin{defi}
Un reticolo $(L,\leq)$ è detto \emph{complementato} se è limitato e ogni suo elemento ha il complemento.
\end{defi}


 %Capitolo sulle relazioni
%Capitolo sulle funzioni
\chapter{Funzioni}
Si definisce \textit{funzione $f:S \mapsto T$} una relazione $f \subseteq S \times T$
tale che $\forall x \in S$ esiste al più un $y \in T$ per cui $(x,y) \in f$.\newline
Se il $dom(f) = S$ la funzione si dice \emph{totale} altrimenti la funzione è \emph{parziale}.

%Tipologie di funzioni
\section{Tipologie di Funzioni}
Una funzione $f:S \mapsto T$ si dice:
\begin{description}
    \item[iniettiva]: $\forall x,y \in S$ con $x \neq y$ si verifica $f(x) \neq f(y)$.
    \item[suriettiva]: $\forall y \in T \exists x \in S$ tale che $f(x) = y$.
    \item[biettiva]: se la funzione è iniettiva e suriettiva
\end{description}
%Determinare se una funzione è iniettiva e suriettiva
Per determinare se una funzione è iniettiva bisogna verificare che $f(x) \neq f(y)$
comporta $x \neq y$.
Per determinare se una funzione è suriettiva bisogna risolvere l'equazione $f(x) = y$
e verificare se $y$ appartiene al codominio della funzione.

%Funzione inversa
Una funzione $f:S \mapsto T$ è detta \emph{invertibile} se la sua relazione inversa
$f ^ -1$ è essa stessa una funzione.\newline
\textbf{Condizione di invertibilità}: una funzione $f:S \mapsto T$ ammette una \emph{funzione inversa}
 $f ^{-1} :T \mapsto S$ se e solo se $f$ è una funzione iniettiva.

%Proprietà funzioni Inverse
\begin{thm}
Sia $f:A \mapsto B$ invertibile, con funzione inversa $f ^ -1$:
\end{thm}
\begin{enumerate}
    \item $f^{-1}$ è totale se e solo se $f$ è suriettiva
    \item $f$ è totale se e solo se $f^{-1}$ è suriettiva
\end{enumerate}

%Esempi
Esempi:\newline
$+:\N x \N \mapsto \N$ è una funzione totale,suriettiva ma non iniettiva \newline
$*:\N x \N \mapsto \N$ è una funzione totale,suriettiva ma non è iniettiva \newline
$successore:\N \mapsto \N$ è una funzione totale ed è iniettiva ma non suriettiva \newline
$successore:\Z \mapsto \Z$ è una funzione totale ed è biettiva \newline
$x^2:\N \mapsto \N$ è una funzione totale,iniettiva,non suriettiva,invertibile \newline
$x^2:\Z \mapsto \Z$ è una funzione totale,non iniettiva,non suriettiva,non invertibile. \newline

%Funzione Composta
Date due funzioni $f:S \mapsto T$ e $g:T \mapsto Q$ si definisce \emph{funzione composta}
$g \circ f:S \mapsto Q$ la funzione tale che $(g \circ f)(x) = g(f(x))$ per ogni $x \in S$.
La funzione composta $(g \circ f)(x)$ è definita se e solo se sono definite entrambe
$g(f(x))$ e $f(x)$.\newline
\textbf{Condizione di componibilità}: codominio della prima coincide col dominio della seconda.

%Esempi funzioni composte
Può capitare a volte di avere una funzione composta non definita in quanto non coincide
il dominio con il codominio ma la funzione risulta calcolabile;in quel caso si dice
che la funzione non è composta ma è calcolabile come ad esempio:

%Inserire esempi di funzioni composte

\begin{thm}
Siano $f:S \mapsto T$ e $g:T \mapsto Q$ invertibili. Allora $g \circ f$ è invertibile
e la sua inversa è $(g \circ f) ^{-1} = f^{-1} \circ g ^{-1}$.
\end{thm}
%Inserire dimostrazione

%Operazione
\section{Operazioni}
Si definisce come \emph{operazione n-aria} su un insieme $S$, una funzione
$f:S^n \mapsto S$ con $n \geq 1$.
Se $f$ è un'operazione binaria su $S$, essa si può rappresentare anche mediante
la notazione infissa $x_1 f x_2$ invece di $f(x_1,x_2)$

%Inserire esempi!!!!

%Definizione Funzioni Monotone
\subsection{Funzioni monotone}
\begin{defi}
    Siano $(S,\leq _S)$ e $(T,\leq _T)$ due poset e sia $f:S \mapsto T$ una funzione allora:
\end{defi}
\begin{enumerate}
    \item $f$ è detta \emph{monotona non decrescente} quando $f(x) \leq_T f(y) \iff x \leq_S y$
    \item $f$ è detta \emph{monotona crescente} quando $f(x) <_T f(y) \iff x <_S y$
    \item $f$ è detta \emph{monotona non crescente} quando $f(x) \leq_T f(y) \iff x \geq_S y$
    \item $f$ è detta \emph{monotona decrescente} quando $f(x) <_T f(y) \iff x >_S y$
\end{enumerate}
%Capitolo sulle Funzioni
\section{Algebra Booleana}
Un \emph{algebra di boole} è un reticolo $(B,R)$ in cui valgono le seguenti proprietà:
\begin{enumerate}
    \item Limitato
    \item Complementato
    \item Distributivo
\end{enumerate}

L'algebra booleana si può definire anche in maniera assiomatica come segue:
\begin{defi}
    Sia $B$ un insieme, un algebra di Boole è una sestupla $(B,\cup,\cap,^{'},0,1)$
dove $\cup$ e $\cap$ sono due operazioni binarie su $B$, $^{'}$ è un operazione unaria su $B$,
$0$ e $1$ sono due elementi distinti di $B$.
\end{defi}

\begin{defi}
    In un algebra di Boole si dice \emph{duale} di un enunciato scambiando $\cup$
    con $\cap$ e $0$ con $1$
\end{defi}

\begin{thm}Nella algebra di Boole valgono le seguenti proprietà, presi qualsiasi $x,y,z \in B$:
\end{thm}
\begin{enumerate}
    \item $x \cap 0 = 0$
    \item $x \cup 1 = 1$
    \item $x \cup 0 = x$
    \item $x \cap 1 = x $
    \item $x \cap (x \cup y) = x$ (Assorbimento)
    \item $x \cup (x \cap y) = x$ (Assorbimento)
    \item $x \cup y = y \cup x $ (Commutativa)
    \item $x \cap y = y \cap x $ (Commutativa)
    \item $x \cup (y \cap z) = (x \cup y) \cap (x \cup z) $ (Distribitiva)
    \item $x \cap (y \cup z) = (x \cap y) \cup (x \cap z) $ (Distribuitiva)
    \item $x \cap (y \cap z) = (x \cap y) \cap z$ (Associatività)
    \item $x \cup (y \cup z) = (x \cup y) \cup z$ (Associatività)
    \item $(x \cap y)' = (x' \cup y')$ (Legge di De Morgan)
    \item $(x \cup y)' = x' \cap y' $ (Legge di de Morgan)
    \item $x \cap y = (x' \cup y')'$
    \item $x \cup y = (x' \cap y')'$
    \item $1' = 0$
    \item $0' = 1$
    \item $x \cup x' = 1$
    \item $x \cap x' = 0$
\end{enumerate}

%Capitolo sulla Algebra Booleana
%File per gli appunti sull'Induzione
\chapter{Induzione}
L'induzione è un importante strumento per la definizione di nuovi insiemi,come
ad esempio l'insieme delle FBF(Formule ben Formate), e la dimostrazione di determinate
proprietà di un insieme.

%Principio d'Induzione
%Dimostrazione per induzione(Da migliorare)
\section{Principio di Induzione}
Il principio di Induzione si utilizza per dimostrare la correttezza di determinate
proprietà dell'insieme dei numeri Naturali.

Il principio di induzione viene definito nel seguente modo:\newline
Data una proposizione $P(x)$ valida per $\forall x \in N$ bisogna:
\begin{enumerate}
  \item \textbf{Caso Base}: Sia $P(0)$ vero
  \item \textbf{Passo Induttivo}: Supposto $P(x)$ vero  bisogna verificare la verità di $P(x+1)$
\end{enumerate}

%Definizione Principio di Induzione Completo
Il principio di Induzione completo è definito nel seguente modo:
\begin{defi}
Sia A(n) una asserzione per ogni elemento $n \geq i \in \N$. Supponendo che:
\begin{itemize}
    \item $A(i)$ è vera (Caso Base)
    \item $\forall m \in \N$, se $A(k)$ è vera $\forall k$,con $0 < k < m$, ne segue
          che è vera $A(m)$
\end{itemize}
Allora $\forall n \in N$,$A(n)$ è vera
\end{defi}

%Esempi
%Esempio dimostrazione per induzione
\begin{thm}
$\sum _{k=0} ^ n (4k+1)= (2n+1)(n+1)$
\end{thm}

\begin{proof}
Caso Base $n = 0$:
\begin{equation*}
    \sum _{k = 0} ^ 0 (4k+1) = (2*0 +1)(0+1) \quad 1 = 1 \ \text{vero}
\end{equation*}
Caso passo:
\quad Ipotesi:$\sum _{k = 0} ^ n (4k+1) = (2n+1)(n+1)$\newline
\quad Tesi: $\sum _{k = 0} ^ {n+1} (4k+1) = (2n+3)(n+2)$
\begin{equation*}
\begin{split}
\sum _{k = 0} ^ {n+1} (4k+1) & = \sum _{k = 0} ^ n (4k+1) + 4(n+1) + 1 \\
                             & = (2n+1)(n+1) + 4n + 5\\
                             & = 2n^2+3n+1+4n+5 \\
                             & = (2n+3)(n+2)\\
\end{split}
\end{equation*}
\end{proof}

%Dimostrazione Tableaux Predicativo
Esempio:$\exists x (P(x) \lor Q(x)) \rightarrow (\exists x P(x) \lor \forall y Q(y))$
\begin{equation*}
\begin{prooftree}
\hypo{F \exists x (P(x) \lor Q(x)) \rightarrow (\exists x P(x) \lor \forall y Q(y))}
\infer1 {T \exists x (P(x) \lor Q(x)), F \exists x P(x) \lor \forall y Q(y)}
\infer1{T P(a) \lor Q(a), F \exists x P(x) \lor \forall y Q(y)}
\infer1 {T P(a) \lor Q(a), F \exists x P(x), F \forall y Q(y)}
\infer1 {T P(a) \lor Q(a), F \exists x P(x), F Q(b)}
\infer1 {T P(a),F \exists x P(x),F Q(b)/T Q(a),F Q(b),F \exists x P(x)}
\end{prooftree}
\end{equation*}
Il secondo ramo del tableau non potrà mai chiudere percui bisogna fare il T-Tableaux

\begin{equation*}
\begin{prooftree}
\hypo{T \exists x (P(x) \lor Q(x)) \rightarrow (\exists x P(x) \lor \forall y Q(y))}
\infer1 {F \exists x P(x) \lor Q(x)/T \exists x P(x) \lor \forall y Q(y)}
\infer1 {F \exists x P(x) \lor Q(x)/T \exists x P(x)/T \forall y Q(y)}
\end{prooftree}
\end{equation*}
Il tableaux non potrà mai chiudere in quanto il secondo e il terzo non generanno mai delle contraddizioni
per cui la formula è sodddisfacibile non tautologica.

Esempio:\newline
\begin{thm}
 $\displaystyle \sum_{i = 1}^n 2i-1 = n^2$
\end{thm}

\begin{proof}
Per $n = 1 \quad \displaystyle \sum_{i = 1}^1 2i-1 = 1^2 \quad 1 = 1$ è vero

Se $\displaystyle \sum_{i = 1}^n 2i-1 = n^2$ allora
$\displaystyle \sum_{i = 1}^{n+1} 2i-1 = (n+1)^2$

\begin{equation*}
\begin{split}
  \sum_{i=1}^{n+1} 2i-1 & = \sum_{i=1}^n 2i-1 + 2(n+1) - 1 \\
         & = n^2 + 2n + 1 \\
         & = (n+1)^2 \\
\end{split}
\end{equation*}
\end{proof}

\begin{thm}
    $\displaystyle n! \geq 2^{(n-1)} \quad \forall n \in \N$
\end{thm}
%Dimostrazione
\begin{proof}
Per $n = 0$ si ha $0! \geq 2^{0-1}$ ossia $ 1 \geq 1/2$ che è sempre verificato

Se $n! \geq 2^{(n-1)}$ allora $(n+1)! \geq 2^n$
\begin{equation*}
\begin{split}
(n+1)! \geq 2^{n+1-1} & = n! (n+1) \geq 2^{n-1} * 2 \\
                      & = n! \frac{(n+1)}{2} \geq 2^{n-1} \\
\end{split}
\end{equation*}
Essendo $\frac{(n+1)}{2} \geq 0 \forall n \in \N$ e $n! \geq 2^{n-1}$ verificato per ipotesi
si ricava che la proposizione è verificata.
\end{proof}

%Dimostrazione per induzione esempio5
\begin{thm}
    $\sum _ {k=0} ^ n (3k+1) = \frac{3n^2+5n+2}{2}$
\end{thm}

\begin{proof}
Caso base $n = 0$:
\begin{equation*}
    \sum _ {k = 0} ^ 0 (3k+1) = \frac{0+0+2}{2} \quad 1 = 1 \text{vero}
\end{equation*}
Caso passo:
\quad Ipotesi:$\sum _ {k=0} ^ n (3k+1) = \frac{3n^2+5n+2}{2}$ \newline
\quad Tesi:$\sum _ {k=0} ^ {n+1} (3k+1) = \frac{3(n+1)^2+5n+5+2}{2} = \frac{3n^2+11n+10}{2}$
\begin{equation*}
\begin{split}
\sum _ {k=0} ^ {n+1} (3k+1) & = \sum _ {k = 0} ^ n (3k+1) + 3(n+1)+1\\
                            & = \frac{3n^2+5n+2}{2} + 3n+4\\
                            & = \frac{3n^2+5n+2+6n+8}{2}\\
                            & = \frac{3n^2+11+10}{2}\\
\end{split}
\end{equation*}
\end{proof}

%Dimostrazione per Induzione esempio 6
\begin{thm}
    $\displaystyle \sum _{k=1} ^ n k^3 = \frac{n^2 (n+1)^2}{4}$
\end{thm}

%Dimostrazione
\begin{proof}
Caso base: $n = 1 \sum _{k=1} ^ 1 k^3 = \frac{1^2(1+1)^2}{4} 1 = \frac{4}{4} \text{vero}$

Ipotesi induttiva: $\sum _{k=1} ^ n k^3 = \frac{n^2 (n+1)^2}{4}$
Tesi induttiva: $\sum _{k=1} ^ {n+1} k^3 = \frac{(n+1)^2 (n+2)^2}{4}$
\begin{equation*}
\begin{split}
\sum _{k=1} ^{n+1} k^3 & = \sum _{k=1} ^ n k^3 + (n+1)^3 \\
                       & = \frac{n^2 (n+1)^2}{4} + (n+1)^3 \\
                       & = (n+1)^2 (\frac{n^2}{4} + (n+1)) \\
                       & = (n+1)^2 (\frac{n^2 + 4n + 4}{4}) \\
                       & = \frac{(n+1)^2 (n+2)^2}{4} \\
\end{split}
\end{equation*}
\end{proof}



%Definizione induttiva
%Definizione induttiva
\section{Definizione Induttiva}
L'induzione permette anche di definire nuovi insiemi nel seguente modo:
\begin{enumerate}
  \item si definisce un insieme di "oggetti base" appartenenti all'insieme.
  \item si definisce un insieme di operazioni di costruzione che, applicate ad elementi
        dell'insieme, producono nuovi elementi dell'insieme.
  \item nient'altro appartiene all'insieme definito.
\end{enumerate}

%Inserire Esempi
Esempio:Definizione induttiva di numeri naturali\newline
\begin{enumerate}
  \item $0 \in N$
  \item Se $x \in N$ allora $s(x) \in N$
  \item Nient'altro appartiene ai numeri naturali
\end{enumerate}

Esempio:espressione in Java
\begin{enumerate}
    \item le variabili e le costanti sono delle espressioni
    \item se $E_1$ e $E_2$ sono delle espressioni ed $op$ è un operatore binario,
          allora $E_1 op E_2$ è un espressione
    \item se $E_1$ e $E_2$ sono delle espressioni e $op$ è un operatore unario,
          allora $op E_1$ è un espressione
    \item nient'altro è un espressione
\end{enumerate}


%Ricorsione
%Ricorsione definizione
\section{Ricorsione}
La ricorsione è funzione definitoria che consiste nel definire un insieme
di elementi base e di definire gli altri elementi mediante il richiamo di se stessa,
fino ad arrivare ai casi base.

Esempio:
la definizione ricorsiva del fattoriale è definita come segue:
\begin{equation*}
    n! = \begin{cases} 1 \quad n = 0 \lor n = 1 \\ n * (n-1)! \quad n > 1\\
\end{cases}
\end{equation*}

la definizione del coefficiente binomiale è definita come segue:
\begin{equation*}
    \bigl( ^ n _ k \bigr) = \begin{cases} 1 \quad n = k \lor k = 0 \\
                             n \quad k = n-1 \lor k = 1 \\
                             (^{n-1} _{k-1}) + (^{n-1} _k) \quad \text{altrimenti} \\
                \end{cases}
\end{equation*}

la definizione di $somma:Z x Z \mapsto Z$ è la seguente:
\begin{equation*}
    somma(a,b) = \begin{cases} a \quad b = 0 \\
                               somma(b,a) \quad a < b \\
                               1 + somma(b-1) \quad b > 0\\
                               -1 + somma(b+1) \quad b < 0 \\
                  \end{cases}
\end{equation*}

%Capitolo Induzione
%File latex per il capitolo sulla logica proposizionale Classica
\chapter{Logica Proposizionale}
%Definizione di Logica
La logica è lo studio del ragionamento e dell’argomentazione e, in particolare,
dei	procedimenti inferenziali, rivolti a chiarire quali	procedimenti di pensiero siano validi e quali no.
Vi sono molteplici tipologie di logiche, come ad esempio la logica classica e le logiche costruttive,
tutte accomunate di essere composte da 3 elementi:

%Elementi di una Logica
\begin{itemize}
  \item \textbf{Linguaggio}:insieme di simboli utilizzati nella Logica per definire le cose
  \item \textbf{Sintassi}:insieme di regole che determina quali elementi appartengono o meno al linguaggio
  \item \textbf{Semantica}:permette di dare un significato alle formule del linguaggio e determinare
        se rappresentano o meno la verità.
\end{itemize}

Noi ci occupiamo della logica Classica che si compone in \textsc{logica proposizionale} e
\textit{logica predicativa}.

La logica proposizionale è un tipo di logica classica che presenta come caratteristica quella
di essere un linguaggio limitato in quanto si possono esprimere soltanto proposizioni senza
avere la possibilità di estenderla a una classe di persone.

%Sintassi proposizionale
\section{Linguaggio e Sintassi}
Un linguaggio predicativo $L$ è composto dai seguenti insiemi di simboli:
\begin{enumerate}
    \item Insieme di variabili individuali(infiniti) $x,y,z,\dots$
    \item Connettivi logici $\land \lor \neg \rightarrow \iff$
    \item Quantificatori esistenziali $\forall \exists$
    \item Simboli ( , )
    \item Costanti proposizionali $T,F$
    \item Simbolo di uguaglianza $=$, eventualmente assente
\end{enumerate}
Questa è la parte del linguaggio tipica di ogni linguaggio del primo ordine poi
ogni linguaggio definisce la propria segnatura ossia definisce in maniera autonomo:
\begin{enumerate}
    \item Insiemi di simboli di costante $a,b,c,\dots$
    \item Simboli di funzione con arieta $f,g,h,\dots$
    \item Simboli di predicato $P,Q,Z,\dots$ con arietà
\end{enumerate}

%Inserire Esempio
Esempio:Linguaggio della teoria degli insiemi \newline
Costante:$\emptyset$\newline
Predicati:$\in(x,y)$, $=(x,y)$

Esempio:Linguaggio della teoria dei Numeri \newline
Costante:$0$ \newline
Predicati:$<(x,y)$,$=(x,y)$ \newline
Funzioni:$succ(x)$,$+(x,y)$,$*(x,y)$

%Definizione di Termini e Formule ben formate
Per definire le formule ben formate della logica predicativa bisogna prima definire
l'insieme di termini e le formule atomiche.

\begin{defi}
    L'insieme $TERM$ dei termini è definito induttivamente come segue
    \begin{enumerate}
        \item Ogni variabile e costante sono dei Termini
        \item Se $t_1 \dots t_n$ sono dei termini e $f$ è un simbolo di funzione di arietà $n$
              allora $f(t_1,\dots,t_n)$ è un termine
    \end{enumerate}
\end{defi}

\begin{defi}
    L'insieme $ATOM$ delle formule atomiche è definito come:
    \begin{enumerate}
        \item $T$ e $F$ sono degli atomi
        \item Se $t_1$ e $t_2$ sono dei termini, allora $t_1 = t_2$ è un atomo
        \item Se $t_1,\dots,t_n$ sono dei termini e $P$ è un predicato a $n$ argomenti,
              allora $P(t_1,\dots,t_n)$ è un atomo.
    \end{enumerate}
\end{defi}

\begin{defi}
    L'insieme delle formule ben formate($FBF$) di $L$ è definito induttivamente come
    \begin{enumerate}
        \item Ogni atomo è una formula
        \item Se $A,B \in FBF$, allora $\neg A$, $A \land B$,$A \lor B$,$A \rightarrow B$
              e $A \iff B$ appartengono alle formule ben formate
        \item Se $A \in FBF$ e $x$ è una variabile, allora $\forall x A$ e $\exists x A$
              appartengono alle formule ben formate
        \item Nient'altro è una formula
    \end{enumerate}
\end{defi}

%Inserire Esempi

%Precedenza degli Operatori
La precedenza tra gli operatori logici è definita nella logica predicativa come segue
$\forall,\exists,\neg,\land,\lor,\rightarrow,\iff$ e si assume che gli operatori associno a destra.

%Inserire Esempi

%Variabili legate e chiuse
\begin{defi}
    L'insieme $var(t)$ delle variabili di un termine $t$ è definito come segue:
    \begin{itemize}
        \item $var(t) = \{t \}$, se $t$ è una variabile
        \item $var(t) = \emptyset$ se $t$ è una costante
        \item $var(f(t_1,\dots,t_n)) = \bigcup _{i = 1} ^n var(t_i)$
        \item $var(R(t_1,\dots,t_n)) = \bigcup _{i = 1} ^ n var(t_i)$
    \end{itemize}
\end{defi}

%Termini chiusi ed aperti
Si definisce come \emph{aperto} un termine che non contiene variabili altrimenti
si definisce il termine come \emph{chiuso}.\newline
Le variabili nei termini e nelle formule atomiche possono essere libere
 in quanto gli unici operatori che "legano" le variabili sono i quantificatori.

Il campo di azione dei quantificatori si riferisce soltanto alla parte in cui
si applica il quantificatore per cui una variabile si dice \emph{libera}
se non ricade nel campo di azione di un quantificatore altrimenti la variabile si dice \emph{vincolata}.


%Semantica Logica Proposizionale
\section{Semantica}
La semantica di una logica consente di dare un significato e un interpretazione
 alle formule del Linguaggio.\newline

\begin{defi}
Sia data una formula proposizionale $P$ e sia ${P_1,\dots,P_n}$, l'insieme degli
atomi che compaiono nella formula $A$.Si definisce come \emph{interpretazione} una
funzione $v:\{P_1,\dots,P_n\} \mapsto \{T,F\}$ che attribuisce un valore di verità
a ciascun atomo della formula $A$.
\end{defi}

I connettivi della Logica Proposizionale hanno i seguenti valori di verità:
%Tabella di Verità degli operatori
$\begin{array}{ccccccc}
\toprule
\text{A} & \text{B} & A \land B & A \lor B & \neg A & A \Rightarrow B & A \iff B \\
\midrule
    F & F & F & F & T & T & T \\
    F & T & F & T & T & T & F \\
    T & F & F & T & F & F & F \\
    T & T & T & T & F & T & T \\
\bottomrule
\end{array}$\newline

Essendo ogni formula $A$ definita mediante un unico albero sintattico, l'interpretazione $v$
è ben definito e ciò comporta che data una formula $A$ e un interpretazione $v$,
eseguita la definizione induttiva dei valori di verità, si ottiene un unico $v(A)$.

Una formula nella logica proposizionale può essere di tre diversi tipi:
%Tipologie di formule
\begin{description}
    \item[valida o tautologica]: la formula è soddisfatta da qualsiasi valutazione della Formula
    \item[Soddisfacibile non Tautologica]:la formula è soddisfatta da qualche valutazione
                        della formula ma non da tutte.
    \item[falsibicabile]:la formula non è soddisfatta da qualche valutazione della formula.
    \item[Contraddizione]:la formula non viene mai soddisfatta
\end{description}

\begin{thm}
$A$ è una formula valida se e solo se $\neg A$ è insoddisfacibile.
$A$ è soddisfacibile se e solo se $\neg A$ è falsibicabile
\end{thm}

%Fare la dimostrazione

Esempio:\newline
Formula $A \land \neg A$ \quad contraddizione

%Tabella di Verità
$\begin{array}{ccc}
\toprule A & \neg A & A \land \neg A \\
\midrule
        0 & 1 & 0 \\
        1 & 0 & 0 \\
\bottomrule
\end{array}$\newpage

Formula $Z = (A \land B) \lor C$  soddisfacibile non Tautologica

%Tabella di Verità
$\begin{array}{ccccc}
\toprule A & B & C & A \land B & (A \land B) \lor C \\
\midrule
         0 & 0 & 0 & 0 & 0 \\
         0 & 0 & 1 & 0 & 1 \\
         0 & 1 & 0 & 0 & 0 \\
         0 & 1 & 1 & 0 & 1 \\
         1 & 0 & 0 & 0 & 0 \\
         1 & 0 & 1 & 0 & 1 \\
         1 & 1 & 0 & 1 & 1 \\
         1 & 1 & 1 & 1 & 1 \\
\bottomrule
\end{array}$\newline

Formula $X = (A \land B) \Rightarrow (\neg A \land C)$ \quad Soddisfacibile  non tautologica\newline

%Tabella di Verità della formula X
$\begin{array}{ccccccc}
\toprule A & B & C & \neg A & A \land B & \neg A \land C & X\\
\midrule
         0 & 0 & 0 & 1 & 0 & 0 & 1 \\
         0 & 0 & 1 & 1 & 0 & 1 & 1 \\
         0 & 1 & 0 & 1 & 0 & 0 & 1 \\
         0 & 1 & 1 & 1 & 0 & 1 & 1 \\
         1 & 0 & 0 & 0 & 0 & 0 & 1 \\
         1 & 0 & 1 & 0 & 0 & 0 & 1 \\
         1 & 1 & 0 & 0 & 1 & 0 & 0 \\
         1 & 1 & 1 & 0 & 1 & 0 & 0 \\
\bottomrule
\end{array}$ \newline

Formula $Y = \neg(A \land B) \iff (A \lor B \Rightarrow C)$ soddisfacibile non Tautologica

%Tabella di Verità
$\begin{array}{cccccccc}
\toprule
A & B & C & A \land B & \neg(A \land B) & A \lor B & (A \lor B) \Rightarrow C & Y \\
\midrule
0 & 0 & 0 & 0 & 1 & 0 & 1 & 1 \\
0 & 0 & 1 & 0 & 1 & 0 & 1 & 1 \\
0 & 1 & 0 & 0 & 1 & 1 & 0 & 0 \\
0 & 1 & 1 & 0 & 1 & 1 & 1 & 1 \\
1 & 0 & 0 & 0 & 1 & 1 & 0 & 0 \\
1 & 0 & 1 & 0 & 1 & 1 & 1 & 1 \\
1 & 1 & 0 & 1 & 0 & 1 & 0 & 1 \\
1 & 1 & 1 & 1 & 0 & 1 & 1 & 0 \\
\bottomrule
\end{array}$

\subsection{Modelli e decidibilità}
Si definisce \emph{modello}, indicato con $M \models A$, tutte le valutazioni booleane
che rendono vera la formula $A$.
Si definisce \emph{contromodello}, indicato con , tutte le valutazioni booleane
che rendono falsa la formula $A$.

La logica proposizionale è decidibile (posso sempre verificare il significato di una formula).
Esiste infatti una procedura effettiva che stabilisce la validità o no di una formula, o se questa
ad esempio è una tautologia.
In particolare il verificare se una proposizione è tautologica o meno è l’operazione di decibi-
lità principale che si svolge nel calcolo proposizonale.

\begin{defi}
    Se $M \models A$ per tutti gli $M$, allora $A$ è una tautologia e si indica $\models A$
\end{defi}

\begin{defi}
    Se $M \models A$ per qualche $M$, allora $A$ è soddisfacibile
\end{defi}

\begin{defi}
Se $M \models A$ non è soddisfatta da nessun $M$, allora $A$ è insoddisfacibile
\end{defi}


%Sistema Deduttivo
\section{Sistema Deduttivo}
Il sistema deduttivo è un metodo di calcolo che manipola proposizioni, senza la
necessità di ricorrere ad altri aspetti della logica.\newline
Nella logica proposizionale, tramite i teoremi di completezza e correttezza, esiste
una corrispondenza tra le formule derivanti dal sistema deduttivo e le formule verificabili
tramite la semantica della logica.

I sistemi deduttivi della logica proposizionale sono i seguenti:
\begin{description}
    \item[Sistema deduttivo Hilbertiano]: non viene analizzato
    \item[Metodo dei Tableaux]
    \item [Risoluzione Proposizionale]:non viene analizzato !!!!
\end{description}

\begin{defi}
Una sequenza di formule $A_1,\dots,A_n$ di $\Lambda$ è una \emph{dimostrazione} se
per ogni $i$ compreso tra $1$ e $n$, $A_i$ è un assioma di $\Lambda$ oppure una
conseguenza diretta di una formula precedente.
\end{defi}

\begin{defi}
Una formula $A$ di una logica $\Lambda$ è detta \emph{teorema} di $\Lambda$,indicata
con $\vdash A$ se esiste una dimostrazione di $\Lambda$ che ha $A$ come ultima formula
\end{defi}

Una dimostrazione di una formula di una logica può venire tramite:
\begin{itemize}
  \item  \textbf{Metodo diretto}: Data un'ipotesi, attraverso una serie di passi
          si riesce a dimostrare la correttezza della Tesi
  \item \textbf{Metodo per assurdo}(non sempre accettato in tutte le logiche):
        Si nega la tesi ed attraverso una serie di passi si riesce a dimostrare
        la negazione delle ipotesi.
\end{itemize}

\begin{thm}
    Un apparato deduttivo $R$ è completo se, per ogni formula $A \in Fbf$, $\vdash A$
    implica $\models A$
\end{thm}

\begin{thm}
    Un apparato deduttivo $R$ è corretto se, per ogni formula $A \in Fbf$, $\models A$
    implica $\vdash A$
\end{thm}
\subsection{Tableau Proposizionali}
Il metodo dei Tableau è stato introdotto da Hintikka e Beth alla fine degli anni '50
e poi ripresi successivamente da Smullyan.
Per poter comprendere e capire i Tableaux dobbiamo introdurre una serie di definizioni:
\begin{defi}
Per ogni formula $A$, $\{A,\neg A\}$ è una coppia di formule complementari in cui
$A$ è il complemento di $\neg A$
\end{defi}

\begin{defi}
Un letterale è un atomo o la sua negazione.Se $p$ è un atomo allora $\{p,\neg p\}$
è una coppia di letterali complementari.
\end{defi}

I tableau sono degli alberi,la cui radice è l'enunciato in esame, e gli altri nodi
sono costruiti attraverso l'applicazione di una serie di regole,fino ad arrivare
alle formule atomiche come radici.

I tableaux proposizionali si dividono in due tipologie di formule(e quindi regole):
le $\alpha$ regole e le $\beta$ regole;le $\alpha$ formule sono di tipo congiuntivo
ed è soddisfatta se e soltanto se le sottoformule $\alpha_1$ e $\alpha_2$ sono entrambe soddisfatte
mentre le $\beta$ formule sono di tipo disgiuntivo e sono soddisfatte se e soltanto
se almeno una delle due sottoformule $\beta_1$ e $\beta_2$ è soddisfatta.

Le regole dei Tableau sono le seguenti:
%T AND
\begin{equation*}
%T AND
\begin{prooftree}
\hypo{S,T (A \land B)}
\infer1 {S,TA,TB}
\end{prooftree}
\quad T \land \qquad
%F AND
\begin{prooftree}
\hypo{S,F (A \land B)}
\infer1 {S,FA/S,FB}
\end{prooftree}
F \ \land
\end{equation*}

\begin{equation*}
%T OR
\begin{prooftree}
\hypo{S,T (A \lor B)}
\infer1 {S,TA / S,TB}
\end{prooftree}
\quad T \lor \qquad
%F OR
\begin{prooftree}
\hypo{S,F (A \lor B)}
\infer1 {S,FA,FB}
\end{prooftree}
F \ \lor
\end{equation*}

\begin{equation*}
%T NOT
\begin{prooftree}
\hypo{S,T (\neg A)}
\infer1 {S,FA}
\end{prooftree}
\quad T \neg \qquad
%F NOT
\begin{prooftree}
\hypo{S,F (\neg A)}
\infer1 {S,TA}
\end{prooftree}
F \ \neg
\end{equation*}

\begin{equation*}
%T ->
\begin{prooftree}
\hypo{S,T (A \rightarrow B)}
\infer1 {S,FA / S,TB}
\end{prooftree}
\quad T \rightarrow \qquad
%F ->
\begin{prooftree}
\hypo{S,F (A \rightarrow B)}
\infer1 {S,TA,FB}
\end{prooftree}
F \ \rightarrow
\end{equation*}

\begin{equation*}
%T <-->
\begin{prooftree}
\hypo{S,T (A \iff B)}
\infer1 {S,TA,TB/S,FA,FB}
\end{prooftree}
\quad T \iff \qquad
%F <-->
\begin{prooftree}
\hypo{S,F (A \iff B)}
\infer1{S,TA,FB/S,FA,TB}
\end{prooftree}
F \ \iff
\end{equation*}

Le $\beta$ regole sono quelle che creano due sottoformule indicate nella regola con $/$
mentre, per esclusione, le $\alpha$ regole sono quelle in cui si crea soltanto una sottoformula.

%Definizione induttiva di costruzione di un tableaux
Il tableaux di una formula $A$ inizialmente è composto da un solo nodo, la radice, etichettata
dalla formula $A$.
Il tableaux si costruisce induttivamente come segue:
si sceglie una foglia $l$ non etichettata dell'albero che verrà etichettata da un
insieme di formule $U(l)$ definite come:
\begin{enumerate}
  \item Se $U(l)$ è un insieme di letterali, si controlla se sono presenti una coppia
        di letterali complementati in $U(l)$;
        in caso siano presenti si marca la foglia come chiusa $\times$ altrimenti la foglia è aperta
  \item Se $U(l)$ non è un insieme di letterali si sceglie una formula in $U(l)$ tramite:
        \begin{itemize}
          \item Se la formula è un $\alpha$-formula $B$ si crea un nuovo nodo $l'$,
                figlio di $l$, e lo si etichetta come $U(l') = (U(l) - \{B\}) \cup \{\alpha_1,alpha_2\}$
          \item Se la formula è un $\beta$-formula $C$ si creano due nodi $l'$ e $l''$,
                figli di $l$ con $l'$ etichettato come: $U(l') = (U(l) - \{C\}) \cup \{\beta_1\}$
                mentre $l''$ è etichettata come $U(l'') = (U(l) - \{C\}) \cup \{\beta_2\}$
        \end{itemize}
\end{enumerate}
Il tableaux termina quando tutti i rami sono etichettati come chiusi e/o aperti.

%Definizione di Tableaux completo
\begin{defi}
Si definisce un tableaux \emph{completo} se la sua costruzione è complementata.
Il tableaux si dice \emph{chiuso} se tutte le foglie sono segnate come chiuse
altrimenti il tableaux è \emph{aperto}
\end{defi}

%Dimostrazione di tableau
Il metodo dei Tableau è un metodo dei sistemi deduttivi, che permette attraverso
l'applicazione di una serie di regole, di capire la tipologia della formula.
%Metodi per capire il tipo della Formula
\begin{tabular}{cccc}
\toprule Tipologia & Fare Tableau per & Chiuso? & Aperto? \\
\midrule
         Teorema & $\neg A$ & Si & No \\
         Soddisfacibile & $A$ & No & Si \\
         Contradditoria & $A$ & Si & No \\
\bottomrule
\end{tabular}

Esempio:$C \rightarrow (P \rightarrow ((Q \rightarrow \neg P) \lor (C \rightarrow P)))$
\begin{equation*}
\begin{prooftree}
\hypo{F C \rightarrow (P \rightarrow ((Q \rightarrow \neg P) \lor (C \rightarrow P)))}
\infer1 {TC,F (P \rightarrow ((Q \rightarrow \neg P) \lor (C \rightarrow P)))}
\infer1 {TC,TP,F (Q \rightarrow \neg P) \lor (C \rightarrow P)}
\infer1 {TC,TP,F (Q \rightarrow \neg P),F (C \rightarrow P)}
\infer1{TC,TP,F (Q \rightarrow \neg P),TC,FP}
\end{prooftree}
\end{equation*}
Il tableaux chiude in quanto non può essere contemporaneamente $TP$ e $FP$ per cui
la formula è una tautologia.

Esempio:
%Esempio dimostrazione per induzione
\begin{thm}
$\sum _{k=0} ^ n (4k+1)= (2n+1)(n+1)$
\end{thm}

\begin{proof}
Caso Base $n = 0$:
\begin{equation*}
    \sum _{k = 0} ^ 0 (4k+1) = (2*0 +1)(0+1) \quad 1 = 1 \ \text{vero}
\end{equation*}
Caso passo:
\quad Ipotesi:$\sum _{k = 0} ^ n (4k+1) = (2n+1)(n+1)$\newline
\quad Tesi: $\sum _{k = 0} ^ {n+1} (4k+1) = (2n+3)(n+2)$
\begin{equation*}
\begin{split}
\sum _{k = 0} ^ {n+1} (4k+1) & = \sum _{k = 0} ^ n (4k+1) + 4(n+1) + 1 \\
                             & = (2n+1)(n+1) + 4n + 5\\
                             & = 2n^2+3n+1+4n+5 \\
                             & = (2n+3)(n+2)\\
\end{split}
\end{equation*}
\end{proof}


%Completezza dei Tableaux
%Dimostrazione completezza Tableaux

%Dimostrazione correttezza Tableaux


%Equivalenze Logiche
%Paragrafo sulle equivalenze logiche
\section{Equivalenze logiche}
Nella logica predicativa si definisce due formule semanticamente equivalenti,
indicato con $P \equiv Q$, se hanno gli stessi modelli.
Le equivalenze della logica predicativa sono le seguenti:
\begin{enumerate}
    \item $\forall x P \equiv \neg \exists x \neg P$
    \item $\neg \forall x P \equiv \exists x \neg P$
    \item $\exists x P \equiv \neg \forall x \neg P$
    \item $\neg \exists x P \equiv \forall x \neg P$
    \item $\forall x \forall y P \equiv \forall y \forall x P$
    \item $\exists x \exists y P \equiv \exists y \exists x P$
    \item $\forall x(P_1 \land P_2) \equiv \forall x P_1 \land \forall x P_2$
    \item $\exists x(P_1 \lor P_2) \equiv \exists x P_1 \lor \exists x P_2$
\end{enumerate}

\subsection{Dimostrazione equivalenze logiche}
In questo sottoparagrafo vengono svolte le dimostrazioni delle equivalenze logiche
attraverso l'utilizzo del metodo dei Tableaux.
\begin{itemize}
    \item $\forall x P \equiv \neg \exists x \neg P$
          Da fare
    \item $\neg \forall x P \equiv \exists x \neg P$
          Da fare
    \item $\exists x P \equiv \neg \forall x \neg P$
          Da fare
    \item $\neg \exists x P \equiv \forall x \neg P$
          Da fare
    \item $\forall x \forall y P \equiv \forall y \forall x P$
          Da fare
    \item $\exists x \exists y P \equiv \exists y \exists x P$
          Da fare
    \item $\forall x(P_1 \land P_2) \equiv \forall x P_1 \land \forall x P_2$
          Da fare
    \item $\exists x(P_1 \lor P_2) \equiv \exists x P_1 \lor \exists x P_2$
          Da fare
\end{itemize}


%Completezza dei Connettivi
\section{Completezza di insiemi di Connettivi}
Un insieme di connettivi logici è completo se mediante i suoi connettivi si può
esprimere un qualunque altro connettivo.
Nella logica proposizionale valgono anche le seguenti equivalenze, utili per ridurre il linguaggio,:
\begin{align*}
    (A \rightarrow B) & \equiv & (\neg A \lor B) \\
    (A \lor B) & \equiv & \neg(\neg A \land \neg B) \\
    (A \land B) & \equiv & \neg(\neg A \lor \neg B) \\
    (A \iff B) & \equiv & (A \rightarrow B) \land (B \rightarrow A) \\
\end{align*}

L'insieme dei connettivi $\{ \neg,\lor,\land \}$, $\{ \neg,\land \}$ e $\{ \neg,\lor \}$ sono completi
e ciò è facilmente dimostrabile utilizzando le seguenti equivalenze logiche
%Capitolo Logica Proposizionale
%Capitolo sulla Logica Predicativa
\chapter{Logica Predicativa}
La logica Predicativa, detta anche logica del primo ordine, si ha la possibilità
di predicare le proprietà di una classe di individui.\newline
É una logica semidecidibile, in quanto è ricorsivamente enumerabile ma non ricorsivo,
per cui non sempre tramite una sequenza di passi si riesce a capire la tipologia di formula.

%Linguaggio e sintassi predicativa
\section{Linguaggio e Sintassi}
Un linguaggio predicativo $L$ è composto dai seguenti insiemi di simboli:
\begin{enumerate}
    \item Insieme di variabili individuali(infiniti) $x,y,z,\dots$
    \item Connettivi logici $\land \lor \neg \rightarrow \iff$
    \item Quantificatori esistenziali $\forall \exists$
    \item Simboli ( , )
    \item Costanti proposizionali $T,F$
    \item Simbolo di uguaglianza $=$, eventualmente assente
\end{enumerate}
Questa è la parte del linguaggio tipica di ogni linguaggio del primo ordine poi
ogni linguaggio definisce la propria segnatura ossia definisce in maniera autonomo:
\begin{enumerate}
    \item Insiemi di simboli di costante $a,b,c,\dots$
    \item Simboli di funzione con arieta $f,g,h,\dots$
    \item Simboli di predicato $P,Q,Z,\dots$ con arietà
\end{enumerate}

%Inserire Esempio
Esempio:Linguaggio della teoria degli insiemi \newline
Costante:$\emptyset$\newline
Predicati:$\in(x,y)$, $=(x,y)$

Esempio:Linguaggio della teoria dei Numeri \newline
Costante:$0$ \newline
Predicati:$<(x,y)$,$=(x,y)$ \newline
Funzioni:$succ(x)$,$+(x,y)$,$*(x,y)$

%Definizione di Termini e Formule ben formate
Per definire le formule ben formate della logica predicativa bisogna prima definire
l'insieme di termini e le formule atomiche.

\begin{defi}
    L'insieme $TERM$ dei termini è definito induttivamente come segue
    \begin{enumerate}
        \item Ogni variabile e costante sono dei Termini
        \item Se $t_1 \dots t_n$ sono dei termini e $f$ è un simbolo di funzione di arietà $n$
              allora $f(t_1,\dots,t_n)$ è un termine
    \end{enumerate}
\end{defi}

\begin{defi}
    L'insieme $ATOM$ delle formule atomiche è definito come:
    \begin{enumerate}
        \item $T$ e $F$ sono degli atomi
        \item Se $t_1$ e $t_2$ sono dei termini, allora $t_1 = t_2$ è un atomo
        \item Se $t_1,\dots,t_n$ sono dei termini e $P$ è un predicato a $n$ argomenti,
              allora $P(t_1,\dots,t_n)$ è un atomo.
    \end{enumerate}
\end{defi}

\begin{defi}
    L'insieme delle formule ben formate($FBF$) di $L$ è definito induttivamente come
    \begin{enumerate}
        \item Ogni atomo è una formula
        \item Se $A,B \in FBF$, allora $\neg A$, $A \land B$,$A \lor B$,$A \rightarrow B$
              e $A \iff B$ appartengono alle formule ben formate
        \item Se $A \in FBF$ e $x$ è una variabile, allora $\forall x A$ e $\exists x A$
              appartengono alle formule ben formate
        \item Nient'altro è una formula
    \end{enumerate}
\end{defi}

%Inserire Esempi

%Precedenza degli Operatori
La precedenza tra gli operatori logici è definita nella logica predicativa come segue
$\forall,\exists,\neg,\land,\lor,\rightarrow,\iff$ e si assume che gli operatori associno a destra.

%Inserire Esempi

%Variabili legate e chiuse
\begin{defi}
    L'insieme $var(t)$ delle variabili di un termine $t$ è definito come segue:
    \begin{itemize}
        \item $var(t) = \{t \}$, se $t$ è una variabile
        \item $var(t) = \emptyset$ se $t$ è una costante
        \item $var(f(t_1,\dots,t_n)) = \bigcup _{i = 1} ^n var(t_i)$
        \item $var(R(t_1,\dots,t_n)) = \bigcup _{i = 1} ^ n var(t_i)$
    \end{itemize}
\end{defi}

%Termini chiusi ed aperti
Si definisce come \emph{aperto} un termine che non contiene variabili altrimenti
si definisce il termine come \emph{chiuso}.\newline
Le variabili nei termini e nelle formule atomiche possono essere libere
 in quanto gli unici operatori che "legano" le variabili sono i quantificatori.

Il campo di azione dei quantificatori si riferisce soltanto alla parte in cui
si applica il quantificatore per cui una variabile si dice \emph{libera}
se non ricade nel campo di azione di un quantificatore altrimenti la variabile si dice \emph{vincolata}.


%Sistemi deduttivi predicativi
\section{Sistema Deduttivo}
Il sistema deduttivo è un metodo di calcolo che manipola proposizioni, senza la
necessità di ricorrere ad altri aspetti della logica.\newline
Nella logica proposizionale, tramite i teoremi di completezza e correttezza, esiste
una corrispondenza tra le formule derivanti dal sistema deduttivo e le formule verificabili
tramite la semantica della logica.

I sistemi deduttivi della logica proposizionale sono i seguenti:
\begin{description}
    \item[Sistema deduttivo Hilbertiano]: non viene analizzato
    \item[Metodo dei Tableaux]
    \item [Risoluzione Proposizionale]:non viene analizzato !!!!
\end{description}

\begin{defi}
Una sequenza di formule $A_1,\dots,A_n$ di $\Lambda$ è una \emph{dimostrazione} se
per ogni $i$ compreso tra $1$ e $n$, $A_i$ è un assioma di $\Lambda$ oppure una
conseguenza diretta di una formula precedente.
\end{defi}

\begin{defi}
Una formula $A$ di una logica $\Lambda$ è detta \emph{teorema} di $\Lambda$,indicata
con $\vdash A$ se esiste una dimostrazione di $\Lambda$ che ha $A$ come ultima formula
\end{defi}

Una dimostrazione di una formula di una logica può venire tramite:
\begin{itemize}
  \item  \textbf{Metodo diretto}: Data un'ipotesi, attraverso una serie di passi
          si riesce a dimostrare la correttezza della Tesi
  \item \textbf{Metodo per assurdo}(non sempre accettato in tutte le logiche):
        Si nega la tesi ed attraverso una serie di passi si riesce a dimostrare
        la negazione delle ipotesi.
\end{itemize}

\begin{thm}
    Un apparato deduttivo $R$ è completo se, per ogni formula $A \in Fbf$, $\vdash A$
    implica $\models A$
\end{thm}

\begin{thm}
    Un apparato deduttivo $R$ è corretto se, per ogni formula $A \in Fbf$, $\models A$
    implica $\vdash A$
\end{thm}
\subsection{Tableau Proposizionali}
Il metodo dei Tableau è stato introdotto da Hintikka e Beth alla fine degli anni '50
e poi ripresi successivamente da Smullyan.
Per poter comprendere e capire i Tableaux dobbiamo introdurre una serie di definizioni:
\begin{defi}
Per ogni formula $A$, $\{A,\neg A\}$ è una coppia di formule complementari in cui
$A$ è il complemento di $\neg A$
\end{defi}

\begin{defi}
Un letterale è un atomo o la sua negazione.Se $p$ è un atomo allora $\{p,\neg p\}$
è una coppia di letterali complementari.
\end{defi}

I tableau sono degli alberi,la cui radice è l'enunciato in esame, e gli altri nodi
sono costruiti attraverso l'applicazione di una serie di regole,fino ad arrivare
alle formule atomiche come radici.

I tableaux proposizionali si dividono in due tipologie di formule(e quindi regole):
le $\alpha$ regole e le $\beta$ regole;le $\alpha$ formule sono di tipo congiuntivo
ed è soddisfatta se e soltanto se le sottoformule $\alpha_1$ e $\alpha_2$ sono entrambe soddisfatte
mentre le $\beta$ formule sono di tipo disgiuntivo e sono soddisfatte se e soltanto
se almeno una delle due sottoformule $\beta_1$ e $\beta_2$ è soddisfatta.

Le regole dei Tableau sono le seguenti:
%T AND
\begin{equation*}
%T AND
\begin{prooftree}
\hypo{S,T (A \land B)}
\infer1 {S,TA,TB}
\end{prooftree}
\quad T \land \qquad
%F AND
\begin{prooftree}
\hypo{S,F (A \land B)}
\infer1 {S,FA/S,FB}
\end{prooftree}
F \ \land
\end{equation*}

\begin{equation*}
%T OR
\begin{prooftree}
\hypo{S,T (A \lor B)}
\infer1 {S,TA / S,TB}
\end{prooftree}
\quad T \lor \qquad
%F OR
\begin{prooftree}
\hypo{S,F (A \lor B)}
\infer1 {S,FA,FB}
\end{prooftree}
F \ \lor
\end{equation*}

\begin{equation*}
%T NOT
\begin{prooftree}
\hypo{S,T (\neg A)}
\infer1 {S,FA}
\end{prooftree}
\quad T \neg \qquad
%F NOT
\begin{prooftree}
\hypo{S,F (\neg A)}
\infer1 {S,TA}
\end{prooftree}
F \ \neg
\end{equation*}

\begin{equation*}
%T ->
\begin{prooftree}
\hypo{S,T (A \rightarrow B)}
\infer1 {S,FA / S,TB}
\end{prooftree}
\quad T \rightarrow \qquad
%F ->
\begin{prooftree}
\hypo{S,F (A \rightarrow B)}
\infer1 {S,TA,FB}
\end{prooftree}
F \ \rightarrow
\end{equation*}

\begin{equation*}
%T <-->
\begin{prooftree}
\hypo{S,T (A \iff B)}
\infer1 {S,TA,TB/S,FA,FB}
\end{prooftree}
\quad T \iff \qquad
%F <-->
\begin{prooftree}
\hypo{S,F (A \iff B)}
\infer1{S,TA,FB/S,FA,TB}
\end{prooftree}
F \ \iff
\end{equation*}

Le $\beta$ regole sono quelle che creano due sottoformule indicate nella regola con $/$
mentre, per esclusione, le $\alpha$ regole sono quelle in cui si crea soltanto una sottoformula.

%Definizione induttiva di costruzione di un tableaux
Il tableaux di una formula $A$ inizialmente è composto da un solo nodo, la radice, etichettata
dalla formula $A$.
Il tableaux si costruisce induttivamente come segue:
si sceglie una foglia $l$ non etichettata dell'albero che verrà etichettata da un
insieme di formule $U(l)$ definite come:
\begin{enumerate}
  \item Se $U(l)$ è un insieme di letterali, si controlla se sono presenti una coppia
        di letterali complementati in $U(l)$;
        in caso siano presenti si marca la foglia come chiusa $\times$ altrimenti la foglia è aperta
  \item Se $U(l)$ non è un insieme di letterali si sceglie una formula in $U(l)$ tramite:
        \begin{itemize}
          \item Se la formula è un $\alpha$-formula $B$ si crea un nuovo nodo $l'$,
                figlio di $l$, e lo si etichetta come $U(l') = (U(l) - \{B\}) \cup \{\alpha_1,alpha_2\}$
          \item Se la formula è un $\beta$-formula $C$ si creano due nodi $l'$ e $l''$,
                figli di $l$ con $l'$ etichettato come: $U(l') = (U(l) - \{C\}) \cup \{\beta_1\}$
                mentre $l''$ è etichettata come $U(l'') = (U(l) - \{C\}) \cup \{\beta_2\}$
        \end{itemize}
\end{enumerate}
Il tableaux termina quando tutti i rami sono etichettati come chiusi e/o aperti.

%Definizione di Tableaux completo
\begin{defi}
Si definisce un tableaux \emph{completo} se la sua costruzione è complementata.
Il tableaux si dice \emph{chiuso} se tutte le foglie sono segnate come chiuse
altrimenti il tableaux è \emph{aperto}
\end{defi}

%Dimostrazione di tableau
Il metodo dei Tableau è un metodo dei sistemi deduttivi, che permette attraverso
l'applicazione di una serie di regole, di capire la tipologia della formula.
%Metodi per capire il tipo della Formula
\begin{tabular}{cccc}
\toprule Tipologia & Fare Tableau per & Chiuso? & Aperto? \\
\midrule
         Teorema & $\neg A$ & Si & No \\
         Soddisfacibile & $A$ & No & Si \\
         Contradditoria & $A$ & Si & No \\
\bottomrule
\end{tabular}

Esempio:$C \rightarrow (P \rightarrow ((Q \rightarrow \neg P) \lor (C \rightarrow P)))$
\begin{equation*}
\begin{prooftree}
\hypo{F C \rightarrow (P \rightarrow ((Q \rightarrow \neg P) \lor (C \rightarrow P)))}
\infer1 {TC,F (P \rightarrow ((Q \rightarrow \neg P) \lor (C \rightarrow P)))}
\infer1 {TC,TP,F (Q \rightarrow \neg P) \lor (C \rightarrow P)}
\infer1 {TC,TP,F (Q \rightarrow \neg P),F (C \rightarrow P)}
\infer1{TC,TP,F (Q \rightarrow \neg P),TC,FP}
\end{prooftree}
\end{equation*}
Il tableaux chiude in quanto non può essere contemporaneamente $TP$ e $FP$ per cui
la formula è una tautologia.

Esempio:
%Esempio dimostrazione per induzione
\begin{thm}
$\sum _{k=0} ^ n (4k+1)= (2n+1)(n+1)$
\end{thm}

\begin{proof}
Caso Base $n = 0$:
\begin{equation*}
    \sum _{k = 0} ^ 0 (4k+1) = (2*0 +1)(0+1) \quad 1 = 1 \ \text{vero}
\end{equation*}
Caso passo:
\quad Ipotesi:$\sum _{k = 0} ^ n (4k+1) = (2n+1)(n+1)$\newline
\quad Tesi: $\sum _{k = 0} ^ {n+1} (4k+1) = (2n+3)(n+2)$
\begin{equation*}
\begin{split}
\sum _{k = 0} ^ {n+1} (4k+1) & = \sum _{k = 0} ^ n (4k+1) + 4(n+1) + 1 \\
                             & = (2n+1)(n+1) + 4n + 5\\
                             & = 2n^2+3n+1+4n+5 \\
                             & = (2n+3)(n+2)\\
\end{split}
\end{equation*}
\end{proof}


%Completezza dei Tableaux
%Dimostrazione completezza Tableaux

%Dimostrazione correttezza Tableaux


%Semantica Predicativa
\section{Semantica}
La semantica di una logica consente di dare un significato e un interpretazione
 alle formule del Linguaggio.\newline

\begin{defi}
Sia data una formula proposizionale $P$ e sia ${P_1,\dots,P_n}$, l'insieme degli
atomi che compaiono nella formula $A$.Si definisce come \emph{interpretazione} una
funzione $v:\{P_1,\dots,P_n\} \mapsto \{T,F\}$ che attribuisce un valore di verità
a ciascun atomo della formula $A$.
\end{defi}

I connettivi della Logica Proposizionale hanno i seguenti valori di verità:
%Tabella di Verità degli operatori
$\begin{array}{ccccccc}
\toprule
\text{A} & \text{B} & A \land B & A \lor B & \neg A & A \Rightarrow B & A \iff B \\
\midrule
    F & F & F & F & T & T & T \\
    F & T & F & T & T & T & F \\
    T & F & F & T & F & F & F \\
    T & T & T & T & F & T & T \\
\bottomrule
\end{array}$\newline

Essendo ogni formula $A$ definita mediante un unico albero sintattico, l'interpretazione $v$
è ben definito e ciò comporta che data una formula $A$ e un interpretazione $v$,
eseguita la definizione induttiva dei valori di verità, si ottiene un unico $v(A)$.

Una formula nella logica proposizionale può essere di tre diversi tipi:
%Tipologie di formule
\begin{description}
    \item[valida o tautologica]: la formula è soddisfatta da qualsiasi valutazione della Formula
    \item[Soddisfacibile non Tautologica]:la formula è soddisfatta da qualche valutazione
                        della formula ma non da tutte.
    \item[falsibicabile]:la formula non è soddisfatta da qualche valutazione della formula.
    \item[Contraddizione]:la formula non viene mai soddisfatta
\end{description}

\begin{thm}
$A$ è una formula valida se e solo se $\neg A$ è insoddisfacibile.
$A$ è soddisfacibile se e solo se $\neg A$ è falsibicabile
\end{thm}

%Fare la dimostrazione

Esempio:\newline
Formula $A \land \neg A$ \quad contraddizione

%Tabella di Verità
$\begin{array}{ccc}
\toprule A & \neg A & A \land \neg A \\
\midrule
        0 & 1 & 0 \\
        1 & 0 & 0 \\
\bottomrule
\end{array}$\newpage

Formula $Z = (A \land B) \lor C$  soddisfacibile non Tautologica

%Tabella di Verità
$\begin{array}{ccccc}
\toprule A & B & C & A \land B & (A \land B) \lor C \\
\midrule
         0 & 0 & 0 & 0 & 0 \\
         0 & 0 & 1 & 0 & 1 \\
         0 & 1 & 0 & 0 & 0 \\
         0 & 1 & 1 & 0 & 1 \\
         1 & 0 & 0 & 0 & 0 \\
         1 & 0 & 1 & 0 & 1 \\
         1 & 1 & 0 & 1 & 1 \\
         1 & 1 & 1 & 1 & 1 \\
\bottomrule
\end{array}$\newline

Formula $X = (A \land B) \Rightarrow (\neg A \land C)$ \quad Soddisfacibile  non tautologica\newline

%Tabella di Verità della formula X
$\begin{array}{ccccccc}
\toprule A & B & C & \neg A & A \land B & \neg A \land C & X\\
\midrule
         0 & 0 & 0 & 1 & 0 & 0 & 1 \\
         0 & 0 & 1 & 1 & 0 & 1 & 1 \\
         0 & 1 & 0 & 1 & 0 & 0 & 1 \\
         0 & 1 & 1 & 1 & 0 & 1 & 1 \\
         1 & 0 & 0 & 0 & 0 & 0 & 1 \\
         1 & 0 & 1 & 0 & 0 & 0 & 1 \\
         1 & 1 & 0 & 0 & 1 & 0 & 0 \\
         1 & 1 & 1 & 0 & 1 & 0 & 0 \\
\bottomrule
\end{array}$ \newline

Formula $Y = \neg(A \land B) \iff (A \lor B \Rightarrow C)$ soddisfacibile non Tautologica

%Tabella di Verità
$\begin{array}{cccccccc}
\toprule
A & B & C & A \land B & \neg(A \land B) & A \lor B & (A \lor B) \Rightarrow C & Y \\
\midrule
0 & 0 & 0 & 0 & 1 & 0 & 1 & 1 \\
0 & 0 & 1 & 0 & 1 & 0 & 1 & 1 \\
0 & 1 & 0 & 0 & 1 & 1 & 0 & 0 \\
0 & 1 & 1 & 0 & 1 & 1 & 1 & 1 \\
1 & 0 & 0 & 0 & 1 & 1 & 0 & 0 \\
1 & 0 & 1 & 0 & 1 & 1 & 1 & 1 \\
1 & 1 & 0 & 1 & 0 & 1 & 0 & 1 \\
1 & 1 & 1 & 1 & 0 & 1 & 1 & 0 \\
\bottomrule
\end{array}$

\subsection{Modelli e decidibilità}
Si definisce \emph{modello}, indicato con $M \models A$, tutte le valutazioni booleane
che rendono vera la formula $A$.
Si definisce \emph{contromodello}, indicato con , tutte le valutazioni booleane
che rendono falsa la formula $A$.

La logica proposizionale è decidibile (posso sempre verificare il significato di una formula).
Esiste infatti una procedura effettiva che stabilisce la validità o no di una formula, o se questa
ad esempio è una tautologia.
In particolare il verificare se una proposizione è tautologica o meno è l’operazione di decibi-
lità principale che si svolge nel calcolo proposizonale.

\begin{defi}
    Se $M \models A$ per tutti gli $M$, allora $A$ è una tautologia e si indica $\models A$
\end{defi}

\begin{defi}
    Se $M \models A$ per qualche $M$, allora $A$ è soddisfacibile
\end{defi}

\begin{defi}
Se $M \models A$ non è soddisfatta da nessun $M$, allora $A$ è insoddisfacibile
\end{defi}


%Equivalenze Logiche
%Paragrafo sulle equivalenze logiche
\section{Equivalenze logiche}
Nella logica predicativa si definisce due formule semanticamente equivalenti,
indicato con $P \equiv Q$, se hanno gli stessi modelli.
Le equivalenze della logica predicativa sono le seguenti:
\begin{enumerate}
    \item $\forall x P \equiv \neg \exists x \neg P$
    \item $\neg \forall x P \equiv \exists x \neg P$
    \item $\exists x P \equiv \neg \forall x \neg P$
    \item $\neg \exists x P \equiv \forall x \neg P$
    \item $\forall x \forall y P \equiv \forall y \forall x P$
    \item $\exists x \exists y P \equiv \exists y \exists x P$
    \item $\forall x(P_1 \land P_2) \equiv \forall x P_1 \land \forall x P_2$
    \item $\exists x(P_1 \lor P_2) \equiv \exists x P_1 \lor \exists x P_2$
\end{enumerate}

\subsection{Dimostrazione equivalenze logiche}
In questo sottoparagrafo vengono svolte le dimostrazioni delle equivalenze logiche
attraverso l'utilizzo del metodo dei Tableaux.
\begin{itemize}
    \item $\forall x P \equiv \neg \exists x \neg P$
          Da fare
    \item $\neg \forall x P \equiv \exists x \neg P$
          Da fare
    \item $\exists x P \equiv \neg \forall x \neg P$
          Da fare
    \item $\neg \exists x P \equiv \forall x \neg P$
          Da fare
    \item $\forall x \forall y P \equiv \forall y \forall x P$
          Da fare
    \item $\exists x \exists y P \equiv \exists y \exists x P$
          Da fare
    \item $\forall x(P_1 \land P_2) \equiv \forall x P_1 \land \forall x P_2$
          Da fare
    \item $\exists x(P_1 \lor P_2) \equiv \exists x P_1 \lor \exists x P_2$
          Da fare
\end{itemize}


%Teorie logiche del primo Ordine
\section{Teorie del Primo Ordine}
Una teoria è un insieme di formule di un linguaggio del primo ordine $L$ e la teoria
la definiamo a partire dalla relazione $\models$

\begin{defi}
    Sia $\sum$ un insieme di formule di $L$, che chiameremo assiomi, si definisce
\emph{teoria}, l'insieme $T_ {\sum}$ delle formule $\phi$ di $L$ tali che $\sum \models \phi$
\end{defi}

\begin{defi}
    Una teoria T è un insieme di enunciati chiuso rispetto alla conseguenza logica
    ovvero $T \models \phi$ implica $\phi \in T$
\end{defi}

\begin{defi}
    Una teoria del primo ordine T è \emph{completa} se per ogni formula $\phi \in L$
è verificata una e una sola delle due:$T \models \phi$ o $T \models \neg \phi$.
\end{defi}

Per rappresentare un particolare dominio, ad esempio i numeri naturali, i grafi,
dobbiamo definire degli assiomi che permettono di catturare la struttura e il comportamento
degli oggetti del dominio che intendiamo trattare.

Una teoria, come l'aritmetica di Peano, la teoria dei gruppi ed eccetera, è un insieme
di assiomi che descrivono certe proprietà degli oggetti che si definiscono.\newline
Gli assiomi hanno il compito di restringere la classe dei modelli della logica
del primo ordine ai modelli degli oggetti che si vogliono trattare.

Consideriamo il linguaggio della teoria dei numeri, definito negli appunti, per
rappresentare l'\emph{aritmetica di Peano} si usano i seguenti assiomi:
\begin{enumerate}
    \item $\forall x \neg (s(x) = 0)$
    \item $\forall x \neg(x < 0)$
    \item $\forall x,y (x < s(y) \rightarrow (x < y \lor x = y))$
    \item $\forall x,y (x < y \lor x = y \lor x > y)$
    \item $\forall x \ x + 0 = x$
    \item $\forall x,y \ x + s(y) = s(x+y)$
    \item $\forall x x * 0 = 0$
    \item $\forall x,y x * s(y) = (x * y) + x$
    \item $\forall x,y (s(x) = s(y)) \rightarrow x = y$
    \item $P(0)$ dove $P$ indica i numeri Pari
    \item $\neg D(0)$ dove $D$ indica i numeri Dispari
    \item $\forall x (P(x) \rightarrow \neg P(s(x)))$
    \item $\forall x (D(x) \rightarrow \neg D(S(x)))$
\end{enumerate}


%Traduzione da italiano a linguaggio formale
%Paragrafo sulla traduzione dall'italiano a un linguaggio formale della logica del 1° ordine
\section{Traduzione in linguaggio formale}
La traduzione in linguaggio formale della logica predicativa consiste nel formalizzare
le frasi della lingua naturale, in particolare l'italiano per noi italiani, in
formule della logica proposizionale attraverso la definizione della realtà da rappresentare.

Per rappresentare le frasi del linguaggio naturale in frasi formali bisogna definire:
\begin{enumerate}
    \item quali sono le eventuali costanti della frase da tradurre
    \item quali sono le eventuali funzioni della frase da tradurre
    \item quali sono i predicati della frase da tradurre
\end{enumerate}

Le costanti sono rappresentati nel linguaggio naturale da sostantivi mentre i
predicati e le funzioni sono rappresentati da forme verbali.
Alcuni esempi di rappresentazione da italiano a linguaggio formale sono i seguenti:

Esempio:
\begin{itemize}
    \item Essere padre di: è una relazione non transitiva,irriflessiva e asimmetrica.
    \item Essere parenti: è una relazione simmetrica,transitiva e irriflessiva.
    \item Essere sposati: è una relazione non transitiva,irriflessiva e simmetrica.
    \item $< \ \subseteq N \times N$:è una relazione asimmetrica,transitiva e irriflessiva.
    \item $\leq \ \subseteq N \times N$:è una relazione riflessiva,transitiva e antisimmetrica.
\end{itemize}

Esercizio:Gli studenti che non si iscrivono all'appello di Fondamenti non possono svolgere l'esame

Costanti:$Fondamenti$
Predicati:$Studente(x)$,$Iscrivere(x,y)$,$Svolgere(x,y)$,$Esame(y)$
Funzioni:
\begin{equation*}
\forall x (Studente(x) \land \neg Iscrivere(x,Fondamenti) \rightarrow
\exists y (Esame(y) \land \neg Svolgere(x,y)))
\end{equation*}

Esercizio:Tutti i professori fanno esami

Costanti: non presenti \newline
Predicati:$Professore(x)$,$Fare(x,y)$,$Esame(y)$ \newline
Funzioni: non presenti
\begin{equation*}
    \forall x (Professore(x) \rightarrow \exists y(Esame(y) \land Fare(x,y)))
\end{equation*}

Esercizio: Se uno studente non è iscritto via Sifa ad un appello non può fare l'esame

Costanti: non presenti \newline
Predicati:$Studente(x)$,$Iscritto(x,y)$,$Appello(y)$,$Esame(x)$ \newline
Funzioni: non presenti
\begin{equation*}
    \forall x (Studenti(x) \land \exists y(Appello(y) \land \neg Iscritto(x,y)) \rightarrow \neg Esame(x))
\end{equation*}

Esercizio: il voto di un esame universitario va da 0 a 30 e lode

Costanti:$0$ e $30L$ \newline
Predicati:$Esame(x)$,$>=(x,y)$,$<=(x,y)$ \newline
Funzioni:$voto(x)$
\begin{equation*}
    \forall x (Esame(x) \rightarrow voto(x) >= 0 \land voto(x) <= 30L)
\end{equation*}

Esercizio:Tutti i docenti sono sposati con una donna antipatica

Costanti: non presenti \newline
Predicati:$Docente(x)$,$Donna(y)$,$Sposati(x,y)$,$Antipatica(y)$ \newline
Funzioni: non presenti
\begin{equation*}
    \forall x (Docente(x) \rightarrow \exists y(Donna(y) \land Antipatica(y) \land Sposati(x,y)))
\end{equation*}

Esercizio:Marco ha un capo magnanimo

Costanti:$Marco$ \newline
Predicati:$Capo(x,y)$,$Magnanimo(x)$ \newline
Funzioni: non presenti \newline
\begin{equation*}
    Capo(x,Marco) \land Magnanimo(x)
\end{equation*}

Esercizio:L'everest è la montagna più alta al mondo

Costanti:$Everest$ \newline
Predicati:$Montagna(x)$,$<(x,y)$ \newline
Funzioni:$altezza(x)$
\begin{equation*}
    \forall x (Montagna(x) \rightarrow altezza(x) < altezza(Everest))
\end{equation*}

Esercizio:Se ogni amico di Mario è amico di Diego e Pietro non è amico di
          Mario, allora Pietro non è amico di Diego

Costanti:$Mario$,$Diego$,$Pietro$ \newline
Predicati:$Amico(x,y)$ \newline
Funzioni:non presenti
\begin{equation*}
    \forall x (((Amico(x,Mario) \rightarrow Amico(x,Diego)) \land \neg Amico(Pietro,Mario))
                \rightarrow \neg Amico(Pietro,Diego))
\end{equation*}


%Capitolo Logica Predicativa
\end{document}
