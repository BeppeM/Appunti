\documentclass[a4paper,12pt, oneside]{book}

%\usepackage{fullpage}
\usepackage[italian]{babel}
\usepackage[utf8]{inputenc}
\usepackage{amssymb}
\usepackage{amsthm}
\usepackage{graphics}
\usepackage{amsfonts}
\usepackage{listings}
\usepackage{amsmath}
\usepackage{amstext}
\usepackage{engrec}
\usepackage{rotating}
\usepackage[safe,extra]{tipa}
\usepackage{showkeys}
\usepackage{multirow}
\usepackage{hyperref}
\usepackage{microtype}
\usepackage{enumerate}
\usepackage{braket}
\usepackage{marginnote}
\usepackage{pgfplots}
\usepackage{cancel}
\usepackage{polynom}
\usepackage{booktabs}
\usepackage{enumitem}
\usepackage{framed}
\usepackage{pdfpages}
\usepackage{pgfplots}
\usepackage[cache=false]{minted}
\usepackage{fancyhdr}
\pagestyle{fancy}
\fancyhead[LE,RO]{\slshape \rightmark}
\fancyhead[LO,RE]{\slshape \leftmark}
\fancyfoot[C]{\thepage}



\title{Basi di Dati}
\author{UniShare\\\\Davide Cozzi\\\href{https://t.me/dlcgold}{@dlcgold}\\\\Gabriele De Rosa\\\href{https://t.me/derogab}{@derogab} \\\\Federica Di Lauro\\\href{https://t.me/f_dila}{@f\textunderscore dila}}
\date{}

\pgfplotsset{compat=1.13}
\begin{document}
\maketitle

\definecolor{shadecolor}{gray}{0.80}

\newtheorem{teorema}{Teorema}
\newtheorem{definizione}{Definizione}
\newtheorem{esempio}{Esempio}
\newtheorem{corollario}{Corollario}
\newtheorem{lemma}{Lemma}
\newtheorem{osservazione}{Osservazione}
\newtheorem{nota}{Nota}
\newtheorem{esercizio}{Esercizio}
\tableofcontents
\renewcommand{\chaptermark}[1]{%
\markboth{\chaptername
\ \thechapter.\ #1}{}}
\renewcommand{\sectionmark}[1]{\markright{\thesection.\ #1}}
\chapter{Introduzione}
\textbf{Questi appunti sono presi a ldurante le esercitazioni in laboratorio. Per quanto sia stata fatta una revisione è altamente probabile (praticamente certo) che possano contenere errori, sia di stampa che di vero e proprio contenuto. Per eventuali proposte di correzione effettuare una pull request. Link: } \url{https://github.com/dlcgold/Appunti}.\\
\textbf{Grazie mille e buono studio!}
\chapter{Lezione 1}
Il corso si divide in 7 parti:
\begin{enumerate}
\item introduzione generale
\item metodologie e modelli per il progetto delle basi di dati
\item progettazione concettuale
\item modello razionale
\item progettazione logica
\item linguaggio SQL
\item algebra relazionale
\end{enumerate}
Le \textbf{informazioni} fanno parte delle risorse di un'azienda. Una \textbf{base di dati} \textit{è un insieme organizzato di dati utilizzato per il supporto allo svolgimento di attività.} L'\textbf{informatica} è la \textit{scienza del trattamento relazionale, specialmente per mezzo di macchine automatiche, dell'informazione, considerata come supporto alla conoscenza umana e all'informazione}. Un \textbf{sistema informativo} è \textit{un componente di un'organizzazione che gestisce le informazioni d'interesse}. Si divide in:
\begin{itemize}
\item acquisizione e memorizzazione
\item aggiornamento
\item interrogazione
\item elaborazione
\end{itemize}
Un sistema informativo è una porzione automatizzata del sistema informatico. Un sistema informatico gestisce il sistema informativo in modo automatizzato, ne garantisce la memorizzazione, l'aggiornamento dei dati per riflettere le loro variazioni, l'accessibilità dei dati. Un sistema informatico può essere distribuito sul territorio. Le informazioni vengono gestite in vari modi, mediante idee formali, con il linguaggio naturale, graficamente con schemi e con numeri o codici. Anche il mezzo può variare dalla carta ai dispositivi elettronici. Pian piano si è arrivato ad avere una codifica standard per l'informazione, a fini organizzativi. Le informazioni nei sistemi informatici sono rappresentate in modo essenziale mediante i \textbf{dati}, questa è la \textit{sintassi}. La \textit{semantica} invece si ottiene con le intestazioni delle tabelle. I dati sono una risorsa strategica perché sono stabili nel tempo. Solitamente i dati sono immutati durante una migrazione tra un sistema e un altro. Un \textbf{Data Base} è una collezione di dati usati per rappresentare un'informazione di interesse di un sistema informativo. Un \textbf{DBMS} è un software che gestisce un database. Un base di dati è anche un insieme di archivi in cui ogni dato è rappresentato logicamente una e una sola volta è può essere usato da un insieme di applicazioni e da più utenti su vari criteri di riservatezza. Si hanno le seguenti caratteristiche:
\begin{itemize}
\item i dati sono molti
\item i dati hanno un formato definito
\item i dati sono permanenti
\item i dati sono raggruppati per insiemi omogenei di dati
\item esistono relazioni specifiche tra gli insiemi di dati
\item la ridondanza minima è controllata ed è assicurata la consistenza delle informazioni
\item i dati sono disponibili per utenze diverse  e concorrenti
\item i dati sono controllati e  protetti da
malfunzionamenti hardware e software
\item i dati sono indipendenti dal programma
\end{itemize}
Si hanno tre fasi di creazione di una tabella:
\begin{enumerate}
\item definizione
\item creazione e popolazione
\item manipolazione
\end{enumerate}
Si garantiscono:
\begin{itemize}
\item privatezza
\item affidabilità
\item efficienza
\item efficacia
\end{itemize}
Un organizzazione è divisa in vari settori e ogni settore ha un suo sottosistema informativo (non necessariamente disgiunto). Le basi dati solitamente sono condivise. Una base di dati gestisce i meccanismi di autorizzazione e ha un controllo della concorrenza, oltre ad essere una risorsa integrata e condivisa. Una base di dati deve essere conservata a lungo termine e si ha una gestione delle \textbf{transazioni}. Una \textbf{transazione} \textit{è un insieme di operazioni da considerare indivisibile, "atomico", corretto anche in presenza di concorrenza e con effetti definitivi}. La sequenza di operazioni sulla base di dati deve essere eseguita nella sua interezza (sono quindi atomiche) e l'effetto di transazioni concorrenti deve essere coerente. La conclusione positiva di una transazione corrisponde ad un impegno, \textit{commit}, a mantenere traccia del risultato in modo definitivo, anche in presenza di guasti e di esecuzione concorrente.\\
I DBMS devono essere efficienti, utilizzando al meglio memoria e tempo. Devono anche essere efficaci e produttivi. \\
Si hanno delle caratteristiche nell'apporoccio alla base dati:
\begin{itemize}
\item natura autodescrittiva di un sistema di basi
di dati: il sistema di basi di dati memorizza i dati e
anche una descrizione completa della sua
struttura (catalogo) queste informazioni sono
chiamate metadati. Questo consente al DBMS di
lavorare con qualsiasi applicazione
\item separazione tra programmi e dati: chiamata
indipendenza tra programmi e dati. E’ possibile
cambiare la struttura dati senza cambiare i
programmi
\item astrazione dei dati: Si usa un modello dati
per nascondere dettagli e presentare all'utente
una vista concettuale del database
\item supporto di viste multiple dei dati: Ogni
utente può usare una vista (view) differente del
database, contenente solo i dati di interesse per
quell'utente
\item condivisione dei dati e gestione delle
transazioni con utenti multipli. Permette a più
utenti concorrenti di accedere contemporaneamente
al database. Il controllo della concorrenza
garantisce che più utenti impegnati ad aggiornare gli
stessi dati lo facciano in maniera controllata: il
risultato degli aggiornamenti è corretto
\end{itemize}
I DBMS estendono le funzionalità dei file
system, fornendo più servizi ed in maniera integrata.
\\In ogni base di dati si ha:
\begin{itemize}
\item lo \textbf{schema}, sostanzialmente invariante nel
tempo, che ne descrive la struttura, il
significato (aspetto intensionale, ovvero la descrizione
"astratta" delle proprietà, ed è invariante
nel tempo).
\item l'\textbf{istanza}, i valori attuali, che possono
cambiare anche molto rapidamente (aspetto estensionale "concreto", che varia nel
tempo al variare della situazione di ciò
che stiamo descrivendo)
\end{itemize}
Si hanno due tipi di modelli:
\begin{itemize}
\item \textbf{modelli logici}, adottati nei DBMS esistenti per l'organizzazione dei dati, utilizzati dai programmi e indipendenti dalle strutture fisiche
\item \textbf{modelli concettuali} che permettono di rappresentare i dati in modo indipendente da ogni sistema, che cercano di descrivere i concetti del mondo reali e sono usati delle fasi preliminari di progettazione. Il più diffuso è il modello \textbf{Entity-Relationship}
\end{itemize}
ecco l'architettura di un DBMS:
$$utente$$
$$\downarrow$$
$$schema\,\,\,logico$$
$$\downarrow$$
$$schema\,\,\,interno$$
$$\downarrow$$
$$database$$
Si hanno 2 schemi:
\begin{enumerate}
\item \textbf{schema logico:} descrizione della base
di dati nel modello logico (ad esempio,
la struttura della tabella)
\item \textbf{schema interno (o fisico):}
rappresentazione dello schema logico
per mezzo di strutture memorizzazione
(file; ad esempio, record con puntatori,
ordinati in un certo modo)
\end{enumerate}
Il livello logico è indipendente da quello
fisico infatti una tabella è utilizzata nello stesso modo
qualunque sia la sua realizzazione fisica
(che può anche cambiare nel tempo). \textbf{Noi tratteremo solo il livello logico}.
%immagine architettura standard
Ecco i tre schemi dell'architettura ANSI/SPARC:
\begin{enumerate}
\item \textbf{schema logico: }descrizione dell'intera base di
dati nel modello logico “principale” del DBMS
\item \textbf{schema fisico: }rappresentazione dello
schema logico per mezzo di strutture fisiche
di memorizzazione
\item\textbf{ schema esterno:} descrizione di parte della
base di dati in un modello logico (“viste”
parziali, derivate, anche in modelli diversi)
\end{enumerate}
L'accesso ai dati avviene solo mediante il livello esterno (che a volte coincide con quello logico) e si hanno 2 forme di indipendenza:
\begin{enumerate}
\item \textbf{indipendenza fisica} 
\item \textbf{indipendenza logica}
\end{enumerate}
Il livello logico e quello esterno sono
indipendenti da quello fisico.\\
Un linguaggio interattivo per gestire una base dati è SQL, che funziona tramite \textbf{query}. Si hanno linguaggi per la definizione di dati, i DLL, \textit{Data Definition Languages}, e linguaggi di manipolazione dei dati, DML, \textit{Data Manipulation Languages}. I primi definiscono e i secondi interrogano e aggiornano. SQL può svolgere le operazioni di entrambi.\\
SI hanno due tipi di utenti:
\begin{enumerate}
\item \textbf{utenti finali (terminalisti):} eseguono applicazioni
predefinite (transazioni)
\item \textbf{utenti casuali:} eseguono operazioni non previste
a priori, usando linguaggi interattivi
\end{enumerate}
inoltre si hanno:
\begin{itemize}
\item progettisti e realizzatori di DBMS
\item progettisti della base di dati e
amministratori della base di dati (DBA), ersona o gruppo di persone
responsabile del controllo centralizzato
e della gestione del sistema, delle
prestazioni, dell'affidabilità, delle
autorizzazioni. Le funzioni del DBA includono quelle di
progettazione, anche se in progetti
complessi ci possono essere distinzioni
\item progettisti e programmatori di applicazioni
\end{itemize}
\end{document}