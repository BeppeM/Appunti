\documentclass[a4paper,12pt, oneside]{book}

%\usepackage{fullpage}
\usepackage[italian]{babel}
\usepackage[utf8]{inputenc}
\usepackage{amssymb}
\usepackage{amsthm}
\usepackage{graphics}
\usepackage{amsfonts}
\usepackage{amsmath}
\usepackage{amstext}
\usepackage{engrec}
\usepackage{rotating}
\usepackage{verbatim}
\usepackage[safe,extra]{tipa}
\usepackage{showkeys}
\usepackage{multirow}
\usepackage{hyperref}
\usepackage{microtype}
\usepackage{enumerate}
\usepackage{braket}
\usepackage{marginnote}
\usepackage{pgfplots}
\usepackage{cancel}
\usepackage{polynom}
\usepackage{booktabs}
\usepackage{enumitem}
\usepackage{framed}
\usepackage{pdfpages}
 \usepackage[usenames,dvipsnames]{pstricks}
 \usepackage{epsfig}
 \usepackage{pst-grad} % For gradients
 \usepackage{pst-plot} % For axes
 \usepackage[space]{grffile} % For spaces in paths
 \usepackage{etoolbox} % For spaces in paths
 \makeatletter % For spaces in paths
 \patchcmd\Gread@eps{\@inputcheck#1 }{\@inputcheck"#1"\relax}{}{}
 \makeatother
\usepackage{pgfplots}
\usepackage{fancyhdr}
\pagestyle{fancy}
\fancyhead[LE,RO]{\slshape \rightmark}
\fancyhead[LO,RE]{\slshape \leftmark}
\fancyfoot[C]{\thepage}


\title{Formulario RSO}
\author{UniShare\\\\Davide Cozzi\\\href{https://t.me/dlcgold}{@dlcgold}}
\date{}

\pgfplotsset{compat=1.13}
\begin{document}
\maketitle

\definecolor{shadecolor}{gray}{0.80}

\newtheorem{teorema}{Teorema}
\newtheorem{definizione}{Definizione}
\newtheorem{esempio}{Esempio}
\newtheorem{corollario}{Corollario}
\newtheorem{lemma}{Lemma}
\newtheorem{osservazione}{Osservazione}
\newtheorem{nota}{Nota}
\newtheorem{esercizio}{Esercizio}
\tableofcontents

\renewcommand{\chaptermark}[1]{%
	\markboth{\chaptername
		\ \thechapter.\ #1}{}}
\renewcommand{\sectionmark}[1]{\markright{\thesection.\ #1}}
\chapter{Reti}
\section{Introduzione}
ritardo di trasmissione  (tempo per trasmettere tutti i bit):
$$ritardo\_trasmissione=\frac{L}{R}=\left[\frac{bit}{bps}\right]$$
ritardo tra $N$ collegamenti ($N-1$ router):
$$ritardo\_end\_to\_end=N\frac{L}{R}=\left[\frac{bit}{bps}\right]$$
ritardo di accodamento dell'n-esimo pacchetto (tempo che il pacchetto passa in coda):
$$ritardo\_accodamento=(n-1)\frac{L}{R}=\left[\frac{bit}{bps}\right]$$
ritardo di propagazione (tempo attraversare collegamento tra due router):
$$ritardo\_propagazione=\frac{distanza}{velolcita}=\frac{x}{v}=\left[\frac{m}{\frac{m}{s}}\right]$$
intensità di traffico, se $i>1$ ritardo di coda tende a infinito:
$$intesita\_traffico=\frac{L\cdot a}{R}=\left[\frac{bit\cdot \frac{pacchetti}{s}}{bps}\right]$$
ritardo end-to-end:
$$ritardo\_end\_to\_end=N(ritardo\_elaborazione+ritardo\_trasmissione+ritardo_propagazione)$$
\section{Livello di Trasporto}
TCP usa 32 bit di intestazione\\
dimensione massima file per la quale numeri di sequenza TCP non si ripetono:
$$dimensione\_massima=2^{32}=4gb$$
estimatedRTT, $\alpha=0,125$:
$$estimatedRTT=(1-\alpha)\cdot estimatedRTT+\alpha\cdot sampleRTT$$
deviazione standard RTT, $\beta=0,25$:
$$devRTT=(1-\beta)\cdot devRTT+\beta\cdot (sampleRTT-estimatedRTT)$$
intervallo:
$$timeoutInteval=estimatedRTT+4\cdot sampleRTT$$
latenza:
$$latenza=2\cdot RTT+\frac{dimensione\_pacchetto}{R}$$
se invio $n$ caratteri con sequenza $a$ e riscontro $b$ il successivo pacchetto ha sequenza $b$ e riscontro $a+n$\\
throughput\_medio, con $W$ ampiezza della finestra, :
$$throughput\_medio=\frac{3}{4}\frac{W}{RTT}=\frac{3}{4}\frac{N\cdot L}{RTT}=\frac{1,22\cdot MSS}{RTT\sqrt{L}}$$
utilizzazione del canale:
$$U=\frac{N\cdot \frac{L}{R}}{RTT+\frac{L}{R}}$$
assenza di stallo:
$$W\cdot \frac{MSS}{R}=\frac{MSS}{R}+RTT\to W=1+\frac{R}{MSS}RTT$$
si ha fair con velocità $\frac{R}{K}$, con $K$ connessioni
\newpage
\section{Livello di Rete}
tabelle di routing:
$$tabelle=N\_router+1$$
numero frammenti generati da un datagramma, in TCP $header=40$:
$$N\_frammenti=\frac{byte\_datagramma}{MTU-header}$$
percentuale overhead:
$$percentuale\_overhead\_header=\frac{header}{byte\_datagramma+header}$$
parte di rete dell'indirizzo:
$$2^x>host\to bit\_rete=32-x$$
numero schede di rete:
\\ ultime tre cifre della mask in binario, conto gli 1, gli sommo 24. faccio 2 elevato a (32 - quella cifra) e sottraggo 3\\
\section{Data Link}
efficienza aloha slotted:
$$efficienza\_aloha=Np(1-p)^{N-1}\to \frac{1}{e}=0,37$$
efficienza aloha puro:
$$efficienza\_aloha=Np(1-p)^{2(N-1)}\to \frac{1}{2e}=0,18$$
CSMA/CD:
$$t\_segnale\_jam=\frac{bit\_segnale}{velocita}$$
$r$ bit di CRC rappresentano un generatore G di
$r + 1$ bit, in grado di rilevare errori a raffica fino
a $r + 1$ bit\\
segnale di jam è di 48 bit\\
header ethernet 26\\
efficienza CSMA/CD:
$$effcienza= \frac{1}{1+5\frac{t_{prop}}{t_{trans}}}$$
tempi di attesa tra frame 96 bit\\
dimensione minima frame 72 byte (46 payload + 26 header)
\section{Wireless}
\begin{center}
	\begin{tabular}{|c|c|c|}
		\hline
		802.11b  & 2.4 GHz       & 1-11 Mbit/s \\
		\hline
		802.11g  & 2.4 GHz       & 54 Mbit/s   \\
		\hline
		802.11a  & 5.8 GHz       & 54 Mbit/s   \\
		\hline
		802.11n  & 2.4 e 5.8 GHz & 150 Mbit/s  \\
		\hline
		802.11ac & 5.8 GHz       & 800 Mbit/s  \\
		\hline
	\end{tabular}
\end{center}
massimo 3 AP vicini per non avere interferenza mutua
\chapter{Sistemi Operativi}
\section{Page Table}
indirizzo fisico:
$$dimensione\_indirizzamento\_fisico=2^{bit}$$
pagine virtuali:
$$N\_pagine=\frac{memoria\_virtuale}{dimensione\_pagine}$$
pagine virtuali in memoria:
$$N\_pagine\_memoria\_in\_memoria=\frac{dimensione\_indirizzamento\_fisico}{dimensione\_pagine}$$
grandezza memoria fisica:
$$grandezza=page\_size\cdot 2^{bit\_entry}$$
grandezza page table:
$$grandezza=\frac{2^{indirizzo\_virtuale}}{dimensione\_pagina}\cdot dimensione\_riga$$
\section{Memoria Virtuale}
tempo di accesso effettivo:
$$tempo=(1-p)\cdot ma+p\cdot t_{page\_fault}$$

\end{document}