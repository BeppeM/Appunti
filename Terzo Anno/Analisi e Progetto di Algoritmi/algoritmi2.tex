\documentclass[a4paper,12pt, oneside]{book}

% \usepackage{fullpage}
\usepackage[italian]{babel}
\usepackage[utf8]{inputenc}
\usepackage{amssymb}
\usepackage{amsthm}
\usepackage{graphics}
\usepackage{amsfonts}
\usepackage{listings}
\usepackage{amsmath}
\usepackage{amstext}
\usepackage{engrec}
\usepackage{rotating}
\usepackage{verbatim}
\usepackage[safe,extra]{tipa}
\usepackage{showkeys}
\usepackage{multirow}
\usepackage{hyperref}
\usepackage{microtype}
\usepackage{enumerate}
\usepackage{braket}
\usepackage{marginnote}
\usepackage{pgfplots}
\usepackage{cancel}
\usepackage{polynom}
\usepackage{booktabs}
\usepackage{enumitem}
\usepackage{framed}
\usepackage{pdfpages}
\usepackage{pgfplots}
\usepackage{algorithmic}
\usepackage{algpseudocode}
\usepackage[cache=false]{minted}

\usepackage{tikz}\usetikzlibrary{er}\tikzset{multi  attribute /.style={attribute ,double  distance =1.5pt}}\tikzset{derived  attribute /.style={attribute ,dashed}}\tikzset{total /.style={double  distance =1.5pt}}\tikzset{every  entity /.style={draw=orange , fill=orange!20}}\tikzset{every  attribute /.style={draw=MediumPurple1, fill=MediumPurple1!20}}\tikzset{every  relationship /.style={draw=Chartreuse2, fill=Chartreuse2!20}}\newcommand{\key}[1]{\underline{#1}}



\usepackage{fancyhdr}
\pagestyle{fancy}
\fancyhead[LE,RO]{\slshape \rightmark}
\fancyhead[LO,RE]{\slshape \leftmark}
\fancyfoot[C]{\thepage}



\title{Analisi e Progetto di Algoritmi}
\author{UniShare\\\\Davide Cozzi\\\href{https://t.me/dlcgold}{@dlcgold}\\\\Gabriele De Rosa\\\href{https://t.me/derogab}{@derogab} \\\\Federica Di Lauro\\\href{https://t.me/f_dila}{@f\textunderscore dila}}
\date{}

\pgfplotsset{compat=1.13}
\begin{document}
\maketitle

\definecolor{shadecolor}{gray}{0.80}
\setlist{leftmargin = 2cm}
\newtheorem{teorema}{Teorema}
\newtheorem{definizione}{Definizione}
\newtheorem{esempio}{Esempio}
\newtheorem{corollario}{Corollario}
\newtheorem{lemma}{Lemma}
\newtheorem{osservazione}{Osservazione}
\newtheorem{nota}{Nota}
\newtheorem{esercizio}{Esercizio}
\algdef{SE}[DOWHILE]{Do}{doWhile}{\algorithmicdo}[1]{\algorithmicwhile\ #1}
\tableofcontents
\renewcommand{\chaptermark}[1]{%
  \markboth{\chaptername
    \ \thechapter.\ #1}{}}
\renewcommand{\sectionmark}[1]{\markright{\thesection.\ #1}}
\chapter{Introduzione}
\textbf{Questi appunti sono presi a lezione. Per quanto sia stata fatta una revisione è altamente probabile (praticamente certo) che possano contenere errori, sia di stampa che di vero e proprio contenuto. Per eventuali proposte di correzione effettuare una pull request. Link: } \url{https://github.com/dlcgold/Appunti}.\\
\textbf{Grazie mille e buono studio!}
\\
\begin{comment}

  \begin{algorithm}
    \If {$i \gets 1$}
    \State $test$
    \Else
    \State $bho$
    \EndIf
    \While {$test$}
    \State $cose$
    \EndWhile
    \For {$cose...$}
    \State $altre cose$
    \EndFor

    \Function{Increment}{$a$}
    \State $a \gets a+1$
    \State \Return $a$
    \EndFunction
  \end{algorithm}

  \begin{algorithm}
    \Do
    \State ciao
    \doWhile {$ciao$}
  \end{algorithm}
\end{comment}
\chapter{Introduzione al corso}
\section{Argomenti}
Si hanno diversi tipi di problemi:
\begin{itemize}
  \item \textbf{problemi di ottimo} dove si cercano singole soluzioni
  efficienti (massimi o minimi) tra molte soluzioni possibili. Si
  usa anche la \textbf{programmazione greedy}, dove si sceglie in base
  ai costi locali per ottenere massimi e minimi senza però guardare i
  costi complessivi.   
  \item \textbf{problemi non risolubili in tempi accettabili}, per i
  quali si usa la \textbf{programmazione dinamica}, che cerca di
  individuare sotto-strutture ottime per risolvere il problema,
  cercando la soluzione migliore memorizzando le altre soluzioni e
  utilizzandole. Si cerca comunque la soluzione meno dispendiosa in
  termini di tempo.
  \item \textbf{problemi NP-completi}, ovvero problemi per cui non si
  può trovare un algoritmo o non si può trovare un algoritmo con una
  complessità asintotica polinomiale. Si useranno anche tecniche non
  deterministiche. Si cercherà di studiare uno dei 10 problemi più
  difficili della matematica: $P\subseteq NP$?
\end{itemize}
Studieremo poi i grafi non pesati con gli algoritmi \textit{BFS}
(per cercare in ampiezza) e \textit{DFS} (per cercare in
profondità). Studieremo anche i grafi pesati con problemi di cammino
minimo.\\
\section{Ripasso Algoritmi 1}
Innazitutto due algoritmi con lo stesso scopo si possono confrontare
in base a tempo e spazio, scegliendo anche in base alle esigenze
hardware. Per lo spazio si calcola quanto spazio viene richiesto da
variabili e strutture dati, soprattutto queste ultime che dipendono
dalla dimensione dell'input. Per quanto riguarda il tempo si usano le
tecniche di conto soprattutto basate sui cicli e, in generale, su
tutte operazioni da effettuare. Il tempo si basa sull'input $n$ e si
indica con $T(n)$ e si esprime in forma asintotica, interessandoci
quindi unicamente all'ordine di grandezza. Si hanno il caso peggiore,
indicato con l'O-grande e quello migliore indicato con l'o-piccolo
a seconda di $n$.\\
Si ricorda poi la tecnica della ricorsione con algoritmi che si
muovono su se stessi mediante dei ``passi'' arrivando ad un caso base
di uscita. Per calcolare i tempi di un algoritmo ricorsivo si ha
$T(n)=F(n)+T(n-1)$ con $F$ che rappresenta le istruzioni delle
subroutines. Questa equazione di ricorrenza non è facilmente
calcolabile ma può essere espansa muovendosi sui passi fino a che non
si arriva a qualcosa di calcolabile grazie al caso 0, questo è il
metodo iterativo (anche se si ha anche il metodo per
sostituzione). Per gli algoritmi ricorsivi si hanno anche i divide et
impera (dove il problema P è diviso in sottoproblemi risolti
separatemente, con la divide, e poi combinati alla fine, con la
combina) dove i tempi non sono sempre calcolabili ma se lo sono
si usa il metodo dell'esperto (studiando le tre possibili casistiche).
\begin{shaded}
  \subsection{Equazioni di Ricorrenza}
  Le equazioni di ricorrenza hanno solitamente la seguente forma:
  $$\begin{cases}
    T(n)=T(n-1)+f(n) \\
    T(1)=\Theta(1)
  \end{cases}$$
  Esistono tre metodi per risolvere le equazioni di ricorrenza:
  \begin{itemize}
    \item Iterativo (detto anche Albero di ricorsione)
    \item Sostituzione
    \item Esperto (detto anche Principale)
  \end{itemize}
  \subsubsection{Metodo Iterativo}
  Si può usare sia per algoritmi ricorsivi e per Divide et Impera.
  Ad ogni passo si prende il valore a destra dell'uguaglianza e
  lo si sostituisce, arrivando, dopo $k$ passi ad una formula
  generale. Sempre $k$ ci darà il caso base. Posso rappresentare
  questo  metodo con l'albero delle chiamate ricorsive,
  guardando quanto è alto l'albero e quanto impiega ad ogni livello
  \begin{esempio}
    Calcolo i tempi di:
    $$\begin{cases}
      T(N)=T(n-1)+8 \\
      T(1)=6
    \end{cases}$$
    procedo nella seguente maniera:
    $$T(n)=T(n-1)+8=[T(n-2)+8]+8=T(n-2)+2\cdot 8$$
    $$=[T(n-3)+8]+2\cdot 8= T(n-3)+3\cdot 8$$
    $$=[T(n-4)+8]+3\cdot 8=T(n-4)+4\cdot 8$$
    $$=T(n-k)+k\cdot 8$$
    per $k=n-1$ si ha:
    $$T[n-(n-1)]+(n-1)\cdot 8=T(1)+(n-1)\cdot 8=6+(n-1)\cdot 8=\Theta(n)$$
  \end{esempio}
  \textit{Altri esempi su sito e appunti di Chiodini}
  \subsubsection{Metodo per Sostituzione}
  Si ipotizza un tempo di calcolo (si possono usare gli
  asintotici con $O$ e $\Omega$ lo si dimostra per induzione
  \begin{esempio}
    $$\begin{cases}
      T(n)=2\cdot T\left(\frac{n}{\floor{2}}\right)+n & n>1 \\
      T(1)                                            & n=1
    \end{cases}
    $$
    Ipotizzo $O(n\cdot \log n)$ e dimostro per induzione:
    $$T(n)=O(n\cdot \log n)\leq c\cdot n\cdot \log n$$
    Serve una dimostrazione forte:
    ipotizzo $T(m)$ vera per $1\leq m\leq n-1$ quindi si ha:
    $$T(n)=2\cdot T\left(\frac{n}{\floor{2}}\right)+n\leq 2\cdot \left[c\cdot \frac{n}{\floor{2}}\cdot \log \frac{n}{\floor{2}}\right]+n$$
    $$=c\cdot n\cdot \log \frac{n}{2}+n=c\cdot n\cdot (\log_2 n-\log_2 2)+n$$
    $$=c\cdot n\cdot \log_2 n-c\cdot n+n\leq c\cdot n\cdot \log n \mbox{ se } c\geq 1$$
    \newpage
    Analizzo ora il caso base:\\
    $T(1)=1$ quindi voglio $1\leq c\cdot \log_2 1$ ovvero $1\leq c\cdot 0$ ovvero mai.
    testo fino a che non trovo $T(3)=2\cdot T(1)+3=29+3=5$ che mi va bene, infatti $5\leq c\cdot 3\cdot \log_2 3$
  \end{esempio}
  \subsubsection{Metodo dell'Esperto}
  Posso usare questo metodo solo nel caso di un'equazione di ricorrenza di questo tipo:
  $$\begin{cases}
    T(n)=a\cdot T\left(\frac{n}{b}\right) +f(n) \\
    T(1)=\Theta(1)
  \end{cases}$$
  dove:
  \begin{itemize}
    \item $a\cdot T\left(\frac{n}{b}\right)$ è l'Impera ed è $\sim n^{\log_b a}$
    \item $f(n)$ è il divide e il combina (ovvero la parte iterativa)
  \end{itemize}
  Si definiscono tre casi:
  \begin{itemize}
    \item \textbf{caso 1:} $n^{\log_b a}>f(n)$ quindi $T(n)\sim n^{\log_b a}$. Si hanno le seguenti condizioni necessarie: $f(n)=O(n^{log_b a -\epsilon})$ ( con $\epsilon>0$) e quindi $T(n)=\Theta(n^{log_b a -\epsilon})$
    \item \textbf{caso 2:} $n^{\log_b a}\cong f(n)$ quindi $T(n)\sim f(n)\cdot \log n$. Si hanno le seguenti condizioni necessarie $f(n)=\Theta(n^{\log_b a})$ e quindi $T(n)=\Theta(n^{\log_b a})$
    \item \textbf{caso 3:} $n^{\log_b a}< f(n)$ quindi $T(n)\sim f(n)$. Si hanno le seguenti condizioni necessarie: $f(n)=\Omega(n^{\log_b a +\epsilon})$ (con $\epsilon>0$) e $a\cdot f\left(\frac{n}{b}
    \right)\leq k\cdot f(n)$ (con $k<1$) quindi $T(n)=\Theta(f(n))$
  \end{itemize}

  \begin{esempio}
    Risolvo:
    $$T(n)=9\cdot T\left(\frac{n}{3}\right)+n$$
    Si ha: $f(n)=n$, $a=9$ e $b=3$.\\
    Ho che $n^{\log_3 9}=n^2$ quindi ho il primo caso:\\
    $f(n)=O(n^{\log_b a -\epsilon})=O(n^{2-\epsilon})$
    Posso dire che $\exists \epsilon:\, O(n^{2-\epsilon})=n$?\\
    Si $\forall \epsilon<1$, per esempio $\epsilon=\frac{1}{2}$. Quindi il Metodo dell'esperto è applicabile (nel primo caso) e si ha quindi $T(n)=\Theta(n)$
  \end{esempio}
  \newpage
  \begin{esempio}
    Si può analizzare meglio il MergeSort:
    $$T(n)\cong 2\cdot T\left(\frac{n}{2}\right)+\Theta(n)$$
    Si ha: $f(n)=\Theta(n)$ e $n^{\log_b a}=n^{\log_2 2 }=n$\\
    Posso applicare il Metodo dell'esperto nel secondo caso avendo così: $$T(n)=\Theta(n\cdot \log n)$$
  \end{esempio}
  \begin{esempio}
    $$T(n)=3\cdot T\left(\frac{n}{4}\right)+n\cdot \log n$$
    Si ha: $f(n)=n\cdot \log n$ e $n^{\log_b a}=n^{\log_4 3}$ e siamo nel terzo caso:
    $$f(n)=\Omega(n^{\log_4 3+\epsilon}$$
    se pongo $\epsilon=1-\log_4 3$ ottengo $n$.
    Il terzo caso richiede una doppia verifica:
    $$3\cdot \frac{n}{4}\cdot\log \frac{n}{4}\leq k\cdot n\log n$$
    che vale per $k=\frac{3}{4}$ infatti si ha:
    $$\frac{3}{4}\cdot n\cdot\log \frac{n}{4}\leq \frac{3}{4} \cdot n\cdot \log n$$
    Si hanno quindi entrambi i requisiti e si può asserire che $T(n)=\Theta(n\cdot\log n)$
  \end{esempio}
  \begin{esempio}
    Calcolo i tempi di:
    $$T(n)=2\cdot T\left(\frac{n}{2}\right)+n\cdot\log n$$
    Si ha: $n^{\log_b a }=n^{\log_2 2 }=n$ e $f(n)=n\cdot\log n$.
    Provo a procedere col terzo caso, dimostrando che: $$n\cdot\log n=\Omega(n^{\log_b a +\epsilon})=\Omega(n^{1+\epsilon})=\Omega(n\cdot n^\epsilon)$$
    Ma tale $\epsilon$ non esiste in quanto $n^\epsilon>\log n$ infatti:
    $$\lim_{n\rightarrow \infty}\frac{n\cdot\log n}{n\cdot n^\epsilon}=0,\,\,\, \forall\, \epsilon>0$$
    Bisogna quindi applicare un altro metodo per risolvere l'equazione di ricorrenza
  \end{esempio}
  \begin{esempio}
    Calcolo la seguente equazione di ricorrenza:
    \[\
      \begin{cases}
        T(n)=1 & n=1\\
        T(n)= 2\cdot T(\frac{n}{2})+1 & n>1
      \end{cases}
    \]
    Quindi avrò un albero binario di soli 1 di profonfità $2^k$
    Quindi $T(n)=\sum_{i=0}^k2^i=2^{k+1}-1$ con $k=\log n$ in quanto si avranno
    in totale $n=2^k=2\cdot2^k-1$. Quindi ottengo $2n-1$ quindi
    avrò $\Theta(n)$.
  \end{esempio}
\end{shaded}
Abbiamo poi visto alcune strutture dati: \textit{array, list, stack, queue,
  tree (e binary-tree) e heap}. 
\chapter{Programmazione Dinamica}
Partiamo dall'algoritmo che calcola la lista di Fibonacci:
\begin{shaded}
  \begin{algorithm}
    \Function{$FIB$}{$n$}
    \If {$n=1$}
    \State $return\,\,n$
    \Else
    \State $return\,\,FIB(n-1)+FIB(n-2)$
    \EndIf
    \EndFunction
  \end{algorithm}
\end{shaded}
\\
Si vede che non sappiamo calcolarne la compplessità, che non è
polinomiale ma magari esponenziale o addirittura fattoriale.\\
Sia $T(n)$ il costo della chiamata alla funzione. Se $n=0$ o $n=1$ ho
$T(n)=1$. Andando avanti avrò $T(n)=1+T(n-1)+T(n-2)$ che non è
risolvibile con le tecniche che conosciamo. Vediamo come risolverla:
riscriviamo l'equazione non omogenea:
\[T(n)-T(n-1)-T(n-2)=1\]
e facciamo una piccola approssimazione:
\[T(n)-T(n-1)-T(n-2)=0\]
ottenendo un'equazione lineare omogenea a cui sommerò qualcosa per
ottenre il risultato della non omogenea. Quindi risolvo l'omogenea
ipotizzando un valore per $T(n)$, per esempio $T(n)=r^n$, e
testiamolo, diventa:
\[r^n-r^{n-1}-r^{n-2}=0\]
moltiplico da entrambe le parti per $r^2$ perché posso:
\[r^2\cdot r^n-r\cdot r^n-r^n=0\]
\[r^2-r-1=0\]
che è un'equazione di secondo grado con soluzioni
$r=\frac{1\pm\sqrt{5}}{2}$
Quindi
\[T(n)-T(n-1)-T(n-2)=0\]
ha due soluzioni:
\[C_1\left(\frac{1+\sqrt{5}}{2}\right)^{n}\]
\[C_2\left(\frac{1-\sqrt{5}}{2}\right)^{n}\]
quindi:
\[T_0(n)=C_1\left(\frac{1+\sqrt{5}}{2}\right)^{n} + C_2
  \left(\frac{1-\sqrt{5}}{2}\right)^{n}\]
Ora cerco la soluzione particolare, sostituisco in:
\[T(n)-T(n-1)-T(n-2)=1\]
Tutte le $T(\cdot)$ con $k$ ottenendo $k-k-k=1\to k = -1$.\\
Quindi la soluzione finale è:
\[T(n)=C_1\left(\frac{1+\sqrt{5}}{2}\right)^{n} +C_2
  \left(\frac{1-\sqrt{5}}{2}\right)^{n}-1=\Theta\left(\left(\frac{1+\sqrt{5}}{2}\right)^{n}\right)\]
\textit{che è la sezione aurea}\\
Miglioriamo l'algoritmo introducendo un array di $n$ celle $F$,
inizializzarlo
\begin{shaded}
  \begin{algorithm}
    \State $F[1\ldots n]$
    \For {$i\gets1\,\,to\,\, n$}
    \State $F[i]\gets empty$
    \EndFor
  \end{algorithm}
\end{shaded}
e procedere con la ricorsione con annotazione, che scrive i vari step
su un array (sprecando quindi memoria) e modificando fibonacci per
ottenre la versione con annotazione:
\begin{shaded}
  \Function{$FIBANN$}{$n$}
  \If {$f[i]== empty$}
  \If {$n\leq 1$}
  \State $F[n]\gets n$
  \Else
  \State $F[n]=FIBANN(n-1)+FIBANN(n-2)$
  \EndIf
  \State $return\,\,F[n]$
  \EndIf
  \EndFunction
\end{shaded}
Quindi se si richiede qualcosa di già usato lo si ritorna prendendolo
dall'array.\\
Iterativamente sarebbe:
\begin{shaded}
  \begin{algorithm}
    \Function{$FIBIT$}{$n$}
    \State $F[0] \gets 0$
    \State $F[1] \gets 1$
    \For {$i\gets 2\,\,to\,\,n$}
    \State $F[i] \gets F[i-1]+F[i-2]$
    \EndFor
    \State $return\,\,F[n]$
    \EndFunction
  \end{algorithm}
\end{shaded}
\subsubsection{Un Nuovo Problema}
da fare a casa
\end{document}