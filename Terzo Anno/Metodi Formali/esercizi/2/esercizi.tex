\documentclass[a4paper,12pt, oneside]{book}

% \usepackage{fullpage}
\usepackage[italian]{babel}
\usepackage[utf8]{inputenc}
\usepackage{amssymb}
\usepackage{amsthm}
\usepackage{graphics}
\usepackage{amsfonts}
\usepackage{listings}
\usepackage{amsmath}
\usepackage{amstext}
\usepackage{engrec}
\usepackage{rotating}
\usepackage{verbatim}
\usepackage[safe,extra]{tipa}
\usepackage{showkeys}
\usepackage{multirow}
\usepackage{hyperref}
\usepackage{microtype}
\usepackage{fontspec}
\usepackage{enumerate}
\usepackage{braket}
\usepackage{marginnote}
\usepackage{pgfplots}
\usepackage{cancel}
\usepackage{polynom}
\usepackage{booktabs}
\usepackage{enumitem}
\usepackage{framed}
\usepackage{pdfpages}
\usepackage{pgfplots}
\usepackage{algorithm}
% \usepackage{algpseudocode}
\usepackage[cache=false]{minted}
\usepackage{mathtools}
\usepackage[noend]{algpseudocode}
\usepackage{varwidth,pst-tree,realscripts}
\usepackage{bidi}
\usetikzlibrary{arrows,shapes,snakes,automata,backgrounds,petri}
\usetikzlibrary{automata,positioning}
\psset{showbbox=false,treemode=D,linewidth=0.3pt,treesep=2ex,levelsep=0.5cm}
\newcommand{\LFTw}[2]{%
  \Tr[ref=#1]{\psframebox[linestyle=none,framesep=4pt]{%
      \begin{varwidth}{15ex}\center #2\end{varwidth}}}}
\newcommand{\LFTr}[2]{\Tr[ref=#1]{\psframebox[linestyle=none,framesep=4pt]{#2}}}

\def\pstreehooki{\psset{thislevelsep=*0pt}}
\def\pstreehookiii{\psset{thislevelsep=*0pt}}
\def\pstreehookv{\psset{thislevelsep=*0pt}}

\usepackage{fancyhdr}
\pagestyle{fancy}

\fancyfoot[C]{\thepage}


\title{Metodi Formali}
\author{Davide Cozzi, 829827}
\date{}

\pgfplotsset{compat=1.13}
\begin{document}
\maketitle

\definecolor{shadecolor}{gray}{0.80}
\setlist{leftmargin = 2cm}
\newtheorem{teorema}{Teorema}
\newtheorem{definizione}{Definizione}
\newtheorem{esempio}{Esempio}
\newtheorem{corollario}{Corollario}
\newtheorem{lemma}{Lemma}
\newtheorem{osservazione}{Osservazione}
\newtheorem{nota}{Nota}
\newtheorem{esercizio}{Esercizio}
\algdef{SE}[DOWHILE]{Do}{doWhile}{\algorithmicdo}[1]{\algorithmicwhile\ #1}

\renewcommand{\chaptermark}[1]{%
  \markboth{\chaptername
    \ \thechapter.\ #1}{}}
\renewcommand{\sectionmark}[1]{\markright{\thesection.\ #1}}
\newcommand{\floor}[1]{\lfloor #1 \rfloor}
\section*{Esercizio Semafori}
Per sincronizzare le due componenti \textit{semaforo} ho pensato ad una
soluzione simile, in parte, a quella dell'esempio della \textit{mutua
  esclusione}.\\
Si hanno quindi le due componenti semaforo che, prese singolarmente, si
comportano nella maniera classica, diventando sequenzialmente rossi ($R$), verdi
($V$) e gialli-verdi ($GV$). Le transizioni sono state nominate con la lettera
iniziale $D$ che significa \textit{``diventa''}. Le due componenti sono
segnalate sia con i pedici che con i due colori: blu e viola. È stata aggiunta
poi la componente di sincronizzazione $S$ che impedisce alle due transizioni che
comportano il diventare verde di scattare contemporaneamente (se scatta uno si
merde la marcatura su $S$ che permetterebbe lo scatto dell'altro). Partendo dal
caso iniziale in cui entrambi i semafori sono rossi si ha che solo uno può
diventare verde mentre l'altro potrà diventare verde solo nel momento in cui
anche l'altro torna rosso (in quanto fino ad allora $S$ non sarà marcato), cosa
che accade dopo aver compito la regolare sequenza di verde e giallo-verde.
\begin{center}

  \begin{tikzpicture}[node distance=2.8cm,>=stealth',bend angle=45,auto]

    \tikzstyle{place}=[circle,thick,draw=black,minimum size=6mm]
    \tikzstyle{transition}=[rectangle,thick,draw=black,minimum size=4mm]
     
    \node [transition] (dv1) {$DV_1$};
    \node [](sep) [right of = dv1]{};
    \node [transition] (dv2) [right of = sep] {$DV_2$};
    \node [place] (v1) [below of = dv1, label=left:$V_1$] {};
    \node [place] (v2) [below of = dv2, label=right:$V_2$] {};
    \node [transition] (dgv1) [below of = v1] {$DGV_1$};
    \node [transition] (dgv2) [below of = v2] {$DGV_2$};
    \node [place] (s) [right of = dgv1, label=above:$S$, tokens=1] {};
    \node [place] (r2) [right of = dgv2, label=right:$R_2$, tokens=1] {};
    \node [place] (r1) [left of = dgv1, label=left:$R_1$, tokens=1] {};
    \node [place] (gv1) [below of = dgv1, label=left:$GV_1$] {};
    \node [place] (gv2) [below of = dgv2, label=right:$GV_2$] {};
    \node [transition] (dr1) [below of = gv1] {$DR_1$};
    \node [transition] (dr2) [below of = gv2] {$DR_2$};
    \path[->,ultra thick, draw=blue, ]
    (dv1) edge node {} (v1)
    (v1) edge node {} (dgv1)
    (dgv1) edge node {} (gv1)
    (gv1) edge node {} (dr1)
    (dr1) edge node {} (r1)
    (r1) edge node {} (dv1)
    \path[->,ultra thick, draw=violet]
    (dv2) edge node {} (v2)
    
    (v2) edge node {} (dgv2)
    
    (dgv2) edge node {} (gv2)
    
    (gv2) edge node {} (dr2)
    
    (dr2) edge node {} (r2)
    
    (r2) edge node {} (dv2)
    \path[->,ultra thick, draw=black]
    (dr1) edge node {} (s)
    (dr2) edge node {} (s)
    (s) edge node {} (dv1)
    (s) edge node {} (dv2);
    
  \end{tikzpicture}
  
\end{center}
\end{document}
% LocalWords:  img jpg above left node edge right of below bend distance black
% LocalWords:  LocalWords thick draw transition pre palce place sep label very
% LocalWords:  tokens
