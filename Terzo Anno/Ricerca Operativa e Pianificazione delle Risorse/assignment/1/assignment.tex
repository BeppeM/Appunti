\documentclass[a4paper,12pt, oneside]{book}

% \usepackage{fullpage}
\usepackage[italian]{babel}
\usepackage[utf8]{inputenc}
\usepackage{amssymb}
\usepackage{amsthm}
\usepackage{graphics}
\usepackage{amsfonts}
\usepackage{listings}
\usepackage{amsmath}
\usepackage{amstext}
\usepackage{engrec}
\usepackage{rotating}
\usepackage[safe,extra]{tipa}
\usepackage{showkeys}
\usepackage{multirow}
\usepackage{hyperref}
\usepackage{mathtools}
\usepackage{microtype}
\usepackage{enumerate}
\usepackage{braket}
\usepackage{marginnote}
\usepackage{pgfplots}
\usepackage{cancel}
\usepackage{polynom}
\usepackage{booktabs}
\usepackage{enumitem}
\usepackage{framed}
\usepackage{algorithm}
\usepackage{algpseudocode}
\usepackage{pdfpages}
\usepackage{pgfplots}
\usepackage[cache=false]{minted}

\usepackage[usenames,dvipsnames]{pstricks}
\usepackage{epsfig}
\usepackage{pst-grad} % For gradients
\usepackage{pst-plot} % For axes
\usepackage[space]{grffile} % For spaces in paths
\usepackage{etoolbox} % For spaces in paths
\makeatletter % For spaces in paths
\patchcmd\Gread@eps{\@inputcheck#1 }{\@inputcheck"#1"\relax}{}{}
\makeatother

\usepackage{tikz}\usetikzlibrary{er}\tikzset{multi  attribute /.style={attribute ,double  distance =1.5pt}}\tikzset{derived  attribute /.style={attribute ,dashed}}\tikzset{total /.style={double  distance =1.5pt}}\tikzset{every  entity /.style={draw=orange , fill=orange!20}}\tikzset{every  attribute /.style={draw=MediumPurple1, fill=MediumPurple1!20}}\tikzset{every  relationship /.style={draw=Chartreuse2, fill=Chartreuse2!20}}\newcommand{\key}[1]{\underline{#1}}

\usepackage{fancyhdr}
\pagestyle{fancy}
\fancyhead[LE,RO]{\slshape \rightmark}
\fancyhead[LO,RE]{\slshape \leftmark}
\fancyfoot[C]{\thepage}
\usepackage{tikz}
\usetikzlibrary{automata,positioning}


\title{Assignment 1}
\author{Davide Cozzi, 829827}
\date{}

\pgfplotsset{compat=1.13}
\begin{document}
\maketitle

\definecolor{shadecolor}{gray}{0.80}
\setlist{leftmargin = 2cm}
\newtheorem{teorema}{Teorema}
\newtheorem{definizione}{Definizione}
\newtheorem{esempio}{Esempio}
\newtheorem{corollario}{Corollario}
\newtheorem{lemma}{Lemma}
\newtheorem{osservazione}{Osservazione}
\newtheorem{nota}{Nota}
\newtheorem{esercizio}{Esercizio}

\renewcommand{\chaptermark}[1]{%
  \markboth{\chaptername
    \ \thechapter.\ #1}{}}
\renewcommand{\sectionmark}[1]{\markright{\thesection.\ #1}}
\chapter{esercizio 1}
Innazitutto definisco i due insiemi degli indici. Per le cotture abbiamo:
\[C=\{G,V,GR\}\]
dove:
\begin{itemize}
  \item $G=griglia$
  \item $V=vapore$
  \item $GR=gratinati$
\end{itemize}
per i piatti:
\[P=\{P,Z,T,O\}\]
dove:
\begin{itemize}
  \item $P=peperoni$
  \item $Z=zucchine$
  \item $T=totani$
  \item $O=orate$
\end{itemize}

Quindi per $c\in C$ e $p\in P$ definisco $x_{c,p}$ il numero di piatti
per cottura.\\
Sappiamo poi che il costo delle verdure è pari a 5, dei totani è 7 e
delle orate è 11.\\ 
Definisco quindi la funzione obiettivo:
\[\min(z)=\bigg\{5\cdot\sum_{i\in C}x_{i,P}+5\cdot\sum_{i\in
    C}x_{i,Z}+7\cdot\sum_{i\in C}x_{i,T}+11\cdot\sum_{i\in
    C}x_{i,O}\Bigg\}\]
Dove con le varie sommatorie rappresentiamo la somma delle varie
cotture per ogni piatto.\\
Analizziamo ora i vincoli.\\
Innanzitutto abbiamo che la somma dei piatti deve essere almeno
100. Si ha quindi che la somma di ogni piatto per ogni cottura deve
essere $\geq 100$:
\[\sum_{i\in C}x_{i,P}+\sum_{i\in
    C}x_{i,Z}+\sum_{i\in C}x_{i,T}+\sum_{i\in
    C}x_{i,O}\geq 100\]
Ovvero:
\[\sum_{i\in C}\sum_{j\in P}x_{i,j} \geq 100\]
Un altro vincolo consiste nell'avere i piatti cotti alla griglia di
numero inferiore a 60:
\[\sum_{i\in P}x_{G,i}\leq 60\]
Abbiamo poi dicitura di indicare che il numero di ogni cottura deve
essere uguale. Posso quindi esprimerlo con due vincoli, dicendo che le
cotture alla griglia sono uguali sia a quelle al vapore che quelle
gratinate (implicando quindi che le gratinate siano uguali a quella al
vapore):
\[\sum_{i\in P}x_{G,i}=\sum_{i\in P}x_{V,i}\]
\[\sum_{i\in P}x_{G,i}=\sum_{i\in P}x_{GR,i}\]
Rappresentiamo poi che le orate possono essere unicamente cotte alla
griglia (specificando che per qualsiasi altra cottura il numero di
piatti sarà nullo):
\[\forall i\in \{V,GR\}:\,\,x_{i,O}=0\]
Bisogna poi rappresentare che un piatto non può essere di numero
negativo (per motivi logici):
\[\forall i\in P,\,\,\forall j\in C:\,\, x_{i,j}\geq 0\]
\newpage
Siamo arrivati quindi a poter scrivere il nostro modello matematico:
\begin{shaded}
  \textbf{Funzione obiettivo:}
  \[\min(z)=\Bigg\{5\cdot\sum_{i\in C}x_{i,P}+5\cdot\sum_{i\in
      C}x_{i,Z}+7\cdot\sum_{i\in C}x_{i,T}+11\cdot\sum_{i\in
      C}x_{i,O}\Bigg\}\]
  \textbf{vincoli:}
  \[
    \begin{dcases}
      \sum_{i\in C}\sum_{j\in P}x_{i,j} \geq 100\\
      \sum_{i\in P}x_{G,i}\leq 60\\
      \sum_{i\in P}x_{G,i}=\sum_{i\in P}x_{V,i}\\
      \sum_{i\in P}x_{G,i}=\sum_{i\in P}x_{GR,i}\\
      \forall i\in \{V, GR\}:\,\,x_{i,O}=0\\
      \forall i\in P,\,\,\forall j\in C:\,\, x_{i,j}\geq 0
    \end{dcases}
  \]
\end{shaded}
\chapter{Esercizio 2}
Innazitutto, procedendo con la risoluzione grafica, cerchiamo la
regione ammissibile data dai vincoli.
Prendendo il vincolo $x_1+2\cdot x_2 \leq 4$ cerchiamo l'area sottesa
alla retta $x_1+2\cdot x_2 = 4$ (che ha $q=2$ e $m=4$):
\begin{center}
  \begin{pspicture}(0,-4.3089447)(12.817888,4.3089447)
    \rput(4.808944,-2.2910554){\psaxes[linecolor=black, linewidth=0.04, tickstyle=full, axesstyle=axes, labels=all, ticks=all, dx=1.0cm, dy=1.0cm](0,0)(0,0)(6,6)}
    \psline[linecolor=black, linewidth=0.04](0.008944397,2.1089447)(12.808945,-4.2910557)
    \psline[linecolor=black, linewidth=0.04, arrowsize=0.05291667cm 2.0,arrowlength=1.4,arrowinset=0.0]{->}(4.808944,3.7089446)(4.808944,4.1089444)
    \psline[linecolor=black, linewidth=0.04, arrowsize=0.05291667cm 2.0,arrowlength=1.4,arrowinset=0.0]{->}(10.808945,-2.2910554)(11.208944,-2.2910554)
    \rput[bl](11.208944,-2.6910555){$x_1$}
    \rput[bl](4.0089445,4.1089444){$x_2$}
  \end{pspicture}
\end{center}
\newpage
Prendiamo poi il secondo vincolo $x_1+\frac{1}{3}\cdot x_2\leq 2$. Procediamo
nella stessa maniera di sopra, disegnando la retta ($x_1+\frac{1}{3}$
che ha $q=6$ e $m=-3$) e ragionando sull'area sottesa:
\begin{center}
  
  \begin{pspicture}(0,-4.2063246)(7.5,4.2063246)
    \rput(0.8,-3.0){\psaxes[linecolor=black, linewidth=0.04, tickstyle=full, axesstyle=axes, labels=all, ticks=all, dx=1.0cm, dy=1.0cm](0,0)(0,0)(6,6)}
    \psline[linecolor=black, linewidth=0.04, arrowsize=0.05291667cm 2.0,arrowlength=1.4,arrowinset=0.0]{->}(0.8,3.0)(0.8,3.3999999)
    \psline[linecolor=black, linewidth=0.04, arrowsize=0.05291667cm 2.0,arrowlength=1.4,arrowinset=0.0]{->}(6.8,-3.0)(7.2,-3.0)
    \rput[bl](7.2,-3.4){$x_1$}
    \rput[bl](0.0,3.3999999){$x_2$}
    \psline[linecolor=black, linewidth=0.04](0.4,4.2)(3.2,-4.2000003)
  \end{pspicture}

\end{center}
L'ultimo vincolo mi specifica che sono nel primo quadrante del piano
cartesiano (con ascisse e ordinate positive).\\
Uniamo quindi le aree di interesse dei vincoli per ottenere la regione
ammissibile:
\begin{center}
  \begin{pspicture}(0,-3.5319617)(7.5,3.5319617)
    \definecolor{colour0}{rgb}{0.38431373,0.40784314,0.92941177}
    \rput(0.8,-3.0680382){\psaxes[linecolor=black, linewidth=0.04, tickstyle=full, axesstyle=axes, labels=all, ticks=all, dx=1.0cm, dy=1.0cm](0,0)(0,0)(6,6)}
    \psline[linecolor=black, linewidth=0.04, arrowsize=0.05291667cm 2.0,arrowlength=1.4,arrowinset=0.0]{->}(0.8,2.9319618)(0.8,3.3319616)
    \psline[linecolor=black, linewidth=0.04, arrowsize=0.05291667cm 2.0,arrowlength=1.4,arrowinset=0.0]{->}(6.8,-3.0680382)(7.2,-3.0680382)
    \rput[bl](7.2,-3.4680383){$x_1$}
    \rput[bl](0.0,3.3319616){$x_2$}
    \psline[linecolor=black, linewidth=0.04](0.8,2.9319618)(2.8,-3.0680382)
    \psline[linecolor=black, linewidth=0.04](0.8,-1.0680383)(4.8,-3.0680382)
    \pspolygon[linecolor=black, linewidth=0.04](0.8,-3.0680382)(2.8,-3.0680382)(2.4,-1.8680383)(0.8,-1.0680383)
    \pspolygon[linecolor=black, linewidth=0.04, fillstyle=solid,fillcolor=colour0](0.8,-3.0680382)(2.8,-3.0680382)(2.4,-1.8680383)(0.8,-1.0680383)
  \end{pspicture}
\end{center}
\newpage
I punti di interesse sono quindi $(0,2)$, $(2,0)$ e il punto di
intersezione tra le due rette $x_1+\frac{1}{3}\cdotx_2=2$ e $x_1+2\cdot x_2 =
4$.\\
Calcoliamo quindi questo punto mettendo a sistema le due rette:
\[
  \begin{cases}
    x_1+2\cdot x_2 = 4\\
    x_1+\frac{1}{3}\cdot x_2=2
  \end{cases}
\]
Risolviamo il sistema:
\[\left(
    \begin{matrix}
      1 & 2 & | & 4\\
      1  & \frac{1}{3} & | & 2
    \end{matrix}\right)=\left(
    \begin{matrix}
      1 & 2 & | & 4\\
      0 & -\frac{5}{3} & | & -2
    \end{matrix}\right)
\]
Quindi $-\frac{5}{3}\cdot x_2 = -2$, ovvero $x_2 =
\frac{6}{5}$. Sostituendo $x_2$ nella prima equazione otteniamo
$x_1=\frac{8}{5}$.
Abbiamo quindi i 3 punti di interesse: $(0,2)$, $(2,0)$ e
$\left(\frac{8}{5}, \frac{6}{5}\right)$:
\begin{center}
  \begin{pspicture}(0,-3.7)(7.5,3.7)
    \definecolor{colour0}{rgb}{0.38431373,0.40784314,0.92941177}
    \rput(0.8,-2.9){\psaxes[linecolor=black, linewidth=0.04, tickstyle=full, axesstyle=axes, labels=all, ticks=all, dx=1.0cm, dy=1.0cm](0,0)(0,0)(6,6)}
    \psline[linecolor=black, linewidth=0.04, arrowsize=0.05291667cm 2.0,arrowlength=1.4,arrowinset=0.0]{->}(0.8,3.1)(0.8,3.5)
    \psline[linecolor=black, linewidth=0.04, arrowsize=0.05291667cm 2.0,arrowlength=1.4,arrowinset=0.0]{->}(6.8,-2.9)(7.2,-2.9)
    \rput[bl](7.2,-3.3){$x_1$}
    \rput[bl](0.0,3.5){$x_2$}
    \psline[linecolor=black, linewidth=0.04](0.8,3.1)(2.8,-2.9)
    \psline[linecolor=black, linewidth=0.04](0.8,-0.9)(4.8,-2.9)
    \pspolygon[linecolor=black, linewidth=0.04](0.8,-2.9)(2.8,-2.9)(2.4,-1.7)(0.8,-0.9)
    \pspolygon[linecolor=black, linewidth=0.04, fillstyle=solid,fillcolor=colour0](0.8,-2.9)(2.8,-2.9)(2.4,-1.7)(0.8,-0.9)
    \psdots[linecolor=black, dotsize=0.4](0.8,-0.9)
    \psdots[linecolor=black, dotsize=0.4](2.4,-1.7)
    \psdots[linecolor=black, dotsize=0.4](2.8,-2.9)
    \rput[bl](2.6,-1.7){$\left(\frac{8}{5}, \frac{6}{5}\right)$}
    \rput[bl](-0.2,-0.6){$(0,2)$}
    \rput[bl](2.8,-4.0){$(2,0)$}
  \end{pspicture}
\end{center}
Procediamo ora con la ricerca del massimo. Prendiamo la funzione
obiettivo e calcoliamo la retta passante per l'origine. In questo caso
avremmo $5\cdot x_1+4\cdot x_2 = 0$, che ha $q=0$ e
$m=\frac{-5}{4}$. Per vedere quale dei tre punti è massimo vediamo le
tre rette parallele a quella passante per l'origine è più lontana
dall'origine stessa. Per farlo sostituiamo le coordinate dei punti in
$5\cdot x_1+4\cdot x_2 $ ottenendo le tre $q$ che rappresentano le 3
rette parallele.
\newpage
Per $(0,2)$ avremo $q=8$, per $(2,0)$ $q=10$ e per
$\left(\frac{8}{5}, \frac{6}{5}\right)$ $q=\frac{64}{5}$. Disegnamo
quindi le rette parallele, muovendoci aumentando $q$ (che corrisponde
alla $z$ della funzione obiettivo):
\begin{center}
  \begin{pspicture}(0,-5.306292)(10.315545,5.306292)
    \definecolor{colour0}{rgb}{0.38431373,0.40784314,0.92941177}
    \rput(3.615545,-1.2937077){\psaxes[linecolor=black, linewidth=0.04, tickstyle=full, axesstyle=axes, labels=all, ticks=all, dx=1.0cm, dy=1.0cm](0,0)(0,0)(6,6)}
    \psline[linecolor=black, linewidth=0.04, arrowsize=0.05291667cm 2.0,arrowlength=1.4,arrowinset=0.0]{->}(3.615545,4.706292)(3.615545,5.1062922)
    \psline[linecolor=black, linewidth=0.04, arrowsize=0.05291667cm 2.0,arrowlength=1.4,arrowinset=0.0]{->}(9.615545,-1.2937077)(10.015545,-1.2937077)
    \rput[bl](10.015545,-1.6937077){$x_1$}
    \rput[bl](2.815545,5.1062922){$x_2$}
    \psline[linecolor=black, linewidth=0.04](3.615545,4.706292)(5.6155453,-1.2937077)
    \psline[linecolor=black, linewidth=0.04](3.615545,0.7062923)(7.6155453,-1.2937077)
    \pspolygon[linecolor=black, linewidth=0.04](3.615545,-1.2937077)(5.6155453,-1.2937077)(5.215545,-0.09370773)(3.615545,0.7062923)
    \pspolygon[linecolor=black, linewidth=0.04, fillstyle=solid,fillcolor=colour0](3.615545,-1.2937077)(5.6155453,-1.2937077)(5.215545,-0.09370773)(3.615545,0.7062923)
    \psdots[linecolor=black, dotsize=0.2](3.615545,0.7062923)
    \psdots[linecolor=black, dotsize=0.2](5.215545,-0.09370773)
    \psdots[linecolor=black, dotsize=0.2](5.6155453,-1.2937077)
    \rput[bl](5.6155453,-0.09370773){$\left(\frac{8}{5}, \frac{6}{5}\right)$}
    \psline[linecolor=black, linewidth=0.04, linestyle=dashed, dash=0.17638889cm 0.10583334cm](0.015545044,3.1062922)(6.815545,-5.293708)
    \psline[linecolor=black, linewidth=0.04, linestyle=dashed, dash=0.17638889cm 0.10583334cm](0.995545,4.3862925)(7.795545,-4.0137076)
    \psline[linecolor=black, linewidth=0.04, linestyle=dashed, dash=0.17638889cm 0.10583334cm](1.735545,4.186292)(8.535545,-4.213708)
    \psline[linecolor=black, linewidth=0.04, linestyle=dashed, dash=0.17638889cm 0.10583334cm](1.2955451,3.6062922)(8.095545,-4.793708)
  \end{pspicture}
\end{center}
Come si dimostra sia graficamente che matematicamente col calcolo
del termine noto $q$ si ha che il punto di massimo cercato è il punto:
\[P=\left(\frac{8}{5}, \frac{6}{5}\right)\]
quindi:
\begin{shaded}
 il valore che massimizza la funzione obiettivo è $\left(\frac{8}{5},
   \frac{6}{5}\right)$ con $z=\frac{64}{5}$
\end{shaded}
\end{document}
