\documentclass[a4paper,12pt, oneside]{book}

% \usepackage{fullpage}
\usepackage[italian]{babel}
\usepackage[utf8]{inputenc}
\usepackage{amssymb}
\usepackage{amsthm}
\usepackage{graphics}
\usepackage{amsfonts}
\usepackage{listings}
\usepackage{amsmath}
\usepackage{amstext}
\usepackage{engrec}
\usepackage{rotating}
\usepackage[safe,extra]{tipa}
\usepackage{showkeys}
\usepackage{multirow}
\usepackage{hyperref}
\usepackage{mathtools}
\usepackage{microtype}
\usepackage{enumerate}
\usepackage{braket}
\usepackage{marginnote}
\usepackage{pgfplots}
\usepackage{cancel}
\usepackage{polynom}
\usepackage{booktabs}
\usepackage{enumitem}
\usepackage{framed}
\usepackage{algorithm}
\usepackage{algpseudocode}
\usepackage{pdfpages}
\usepackage{pgfplots}
\usepackage[cache=false]{minted}

\usepackage[usenames,dvipsnames]{pstricks}
\usepackage{epsfig}
\usepackage{pst-grad} % For gradients
\usepackage{pst-plot} % For axes
\usepackage[space]{grffile} % For spaces in paths
\usepackage{etoolbox} % For spaces in paths
\makeatletter % For spaces in paths
\patchcmd\Gread@eps{\@inputcheck#1 }{\@inputcheck"#1"\relax}{}{}
\makeatother

\usepackage{tikz}\usetikzlibrary{er}\tikzset{multi  attribute /.style={attribute ,double  distance =1.5pt}}\tikzset{derived  attribute /.style={attribute ,dashed}}\tikzset{total /.style={double  distance =1.5pt}}\tikzset{every  entity /.style={draw=orange , fill=orange!20}}\tikzset{every  attribute /.style={draw=MediumPurple1, fill=MediumPurple1!20}}\tikzset{every  relationship /.style={draw=Chartreuse2, fill=Chartreuse2!20}}\newcommand{\key}[1]{\underline{#1}}

\usepackage{fancyhdr}
\pagestyle{fancy}
\fancyhead[LE,RO]{\slshape \rightmark}
\fancyhead[LO,RE]{\slshape \leftmark}
\fancyfoot[C]{\thepage}
\usepackage{tikz}
\usetikzlibrary{automata,positioning}


\title{Assignment 2}
\author{Davide Cozzi, 829827}
\date{}

\pgfplotsset{compat=1.13}
\begin{document}
\maketitle

\definecolor{shadecolor}{gray}{0.80}
\setlist{leftmargin = 2cm}
\newtheorem{teorema}{Teorema}
\newtheorem{definizione}{Definizione}
\newtheorem{esempio}{Esempio}
\newtheorem{corollario}{Corollario}
\newtheorem{lemma}{Lemma}
\newtheorem{osservazione}{Osservazione}
\newtheorem{nota}{Nota}
\newtheorem{esercizio}{Esercizio}

\renewcommand{\chaptermark}[1]{%
  \markboth{\chaptername
    \ \thechapter.\ #1}{}}
\renewcommand{\sectionmark}[1]{\markright{\thesection.\ #1}}
\chapter{Esercizio 1}
\section{Parte a}
Innanzitutto dobbiamo scrivere il problema in forma standard
introducendo 3 variabili di slack $s_1, s_2, s_3$. Otteniamo quindi la
seguente funzione obiettivo:
\[\max z=2x_1-x_2+x_3\]
soggetta ai seguenti vincoli:
\[3x_1+x_2+x_3+s_1\qquad\qquad\leq 60\]
\[x_1-x_2+2x_3\qquad+ s_2\qquad\leq 10\]
\[x_1+x_2-x_3 \qquad\qquad+s_3\leq 20\]
con i seguenti vincoli di dominio:
\[x_1,x_2,x_3,s_1,s_2,s_3\geq 0\]
Scriviamo quindi il tableau, ricordandoci di invertire il segno dei
coefficienti della funzione obiettivo:
\begin{center}
  \begin{tabular}{c|cccccc}
    & $x_1$ & $x_2$ & $x_3$ & $s_1$ & $s_2$ & $s_3$\\
    \hline
    0 & -2 & 1 & -1 & 0 & 0 & 0\\
    \hline
    60 & 3 & 1 & 1 & 1 & 0 & 0\\
    10 & 1 & -1 & 2 & 0 & 1 & 0\\
    20 & 1 & 1 & -1 & 0 & 0 & 1
  \end{tabular}
\end{center}
abbiamo quindi le variabili di slack in base e le variabili del
problema originale fuori base.\\
Nella prima riga, detta $riga_0$, si leggono i tassi di miglioramento.
Si capirà di essere giunti alla soluzione ottimale quando non si avranno
coefficienti negativi come tassi di miglioramento.\\
Partiamo con la \textbf{prima iterazione}.\\
Scegliamo la variabile da far entrare in base e scegliamo quella col
tasso di miglioramento più grande, in questo caso $x_1$ che ha tasso
di miglioramento pari a $-2$, e chiamiamo \textbf{colonna pivot}
la colonna sottostante:
\[
  \left[\begin{matrix}
    3\\
    1\\
    1
  \end{matrix}\right]
\]
Iniziamo ora a cercare la variabile uscente. Si parte col \textbf{test
  del minimo rapporto}. Prendo i rapporti tra i termini noti e i
corrispettivi coefficienti positivi (tutti nel nostro caso) della
colonna pivot e ne cerco il minimo:
\[\min\Bigg\{\frac{60}{3}, \frac{10}{1},\frac{20}{1}\Bigg\}\]
e notiamo che il minimo, $10$, corrisponde al rapporto tra il termini
della seconda riga, che diventa quindi la nostra \textbf{riga pivot}:
\[\left[
    10,1,-1,2,0,1,0
  \right]\]
L'incrocio tra riga e colonna pivot corrisponde al nostro
\textbf{elemento di pivot}, che, in questo caso, è pari a 1.
Divido ora la riga pivot per l'elemento pivot (anche se in questo caso
è un'operazione banale essendo il pivot pari a 1) e riscrivo il
tableau con la nuova riga pivot:
\begin{center}
  \begin{tabular}{c|cccccc}
    & $x_1$ & $x_2$ & $x_3$ & $s_1$ & $s_2$ & $s_3$\\
    \hline
    0 & -2 & 1 & -1 & 0 & 0 & 0\\
    \hline
    60 & 3 & 1 & 1 & 1 & 0 & 0\\
    10 & 1 & -1 & 2 & 0 & 1 & 0\\
    20 & 1 & 1 & -1 & 0 & 0 & 1
  \end{tabular}
\end{center}
manipolo ora il tableau in modo da annullare tutti i valori della
colonna pivot tranne il pivot stesso. Effettuo quindi le seguenti
operazioni tra righe:
\[riga_0=riga_0+2riga_2\]
\[riga_1=riga_1-3riga_2\]
\[riga_3=riga_3-riga_2\]
ottenendo quindi un nuovo tableau:
\begin{center}
  \begin{tabular}{c|cccccc}
    & $x_1$ & $x_2$ & $x_3$ & $s_1$ & $s_2$ & $s_3$\\
    \hline
    20 & 0 & -1 & 3 & 0 & 2 & 0\\
    \hline
    30 & 0 & 4 & -5 & 1 & -3 & 0\\
    10 & 1 & -1 & 2 & 0 & 1 & 0\\
    10 & 0 & 2 & -3 & 0 & -1 & 1
  \end{tabular}
\end{center}
Qundi ora in base abbiamo $x_1,s_1,s_3$ mentre $s_2$ esce dalla
base.\\
Sulla riga 0 abbiamo i nuovi tassi di miglioramento e, avendo ancora
valori negativi, sappiamo che non abbiamo raggiunto la soluzione
ottimale. Quindi il primo valore della $riga_0$, 20, non è il valore
ottimale della funzione obiettivo.\\
Procedo quindi con la \textbf{seconda iterazione}.\\
Il tasso di miglioramento maggiore è $-1$ quindi vogliamo far entrare
in base $x_2$. La colonna pivot sarà quindi:
\[
  \left[\begin{matrix}
    4\\
    -1\\
    2
  \end{matrix}\right]
\]
Facciamo nuovamente il test del rapporto minimo, ricordando di
prendere solo i valori strettamente positivi della colonna pivot:
\[\min\Bigg\{\frac{30}{4}, \frac{10}{2}\Bigg\}\]
e il minimo, 5, corriponde ai rapporti tra i valori della terza riga
che qindi diventa la nostra riga pivot:
\[[10,0.2,-3,0,-1,1]\]
Il nostro elemento pivot corriponde quindi a 2.\\
Divido ora la riga pivot per l’elemento pivot e riscrivo il tableau:
\begin{center}
  \begin{tabular}{c|cccccc}
    & $x_1$ & $x_2$ & $x_3$ & $s_1$ & $s_2$ & $s_3$\\
    \hline
    20 & 0 & -1 & 3 & 0 & 2 & 0\\
    \hline
    30 & 0 & 4 & -5 & 1 & -3 & 0\\
    10 & 1 & -1 & 2 & 0 & 1 & 0\\
    5 & 0 & 1 & $-\frac{3}{2}$ & 0 & $-\frac{1}{2}$ & $\frac{1}{2}$
  \end{tabular}
\end{center}
manipolo ora il tableau in modo da annullare tutti i valori della colonna pivot
tranne il pivot stesso. Effettuo quindi le seguenti operazioni tra
righe:
\[riga_0=riga_0+riga_3\]
\[riga_1=riga_1-4riga_3\]
\[riga_2=riga_2+riga_3\]
ottengo quindi il nuovo tableau:
\begin{center}
  \begin{tabular}{c|cccccc}
    & $x_1$ & $x_2$ & $x_3$ & $s_1$ & $s_2$ & $s_3$\\
    \hline
    25 & 0 & 0 & $\frac{3}{2}$ & 0 & $\frac{3}{2}$ & $\frac{1}{2}$\\
    \hline
    10 & 0 & 0 & 7 & 1 & -1 & -2\\
    15 & 1 & 0 & $\frac{1}{2}$ & 0 & $\frac{1}{2}$ & $\frac{1}{2}$\\
    5 & 0 & 1 & $-\frac{3}{2}$ & 0 & $-\frac{1}{2}$ & $\frac{1}{2}$
  \end{tabular}
\end{center}
quindi $x_2$ è entrato in base mentre $s_3$ esce dalla base.\\
\textbf{Non si hanno più tassi di miglioramento negativi, abbiamo
  trovato la soluzione ottimale}.\\
Quindi siamo giunti alla risoluzione:
\begin{shaded}
  \noindent Il punto di massimo è dato da $(15,5,0,10,0,0)$\\
  Il massimo è $25$
\end{shaded}
\section{Parte b}
Dobbiamo scrivere il problema duale della funzione obiettivo:
\[\max z=2x_1-x_2+x_3\]
soggetta ai seguenti vincoli:
\[3x_1+x_2+x_3\leq 60\]
\[x_1-x_2+2x_3\leq 10\]
\[x_1+x_2-x_3 \leq 20\]
con i seguenti vincoli di dominio:
\[x_1,x_2,x_3\geq 0\]
Innazitutto bisogna invertire cosa si ricerca nella funzione obiettivo
del problema primale. Nel nostro caso il problema primale cerca il
massimo quindi il duale cercherà il minimo. Inoltre i termini noti
dei vincoli del primale diventeranno i coefficienti della funzione
obiettivo del duale e i coefficienti della funzione obiettivo del
problema primale diventeranno i termini noti dei vincoli. Nei vincoli
si cambia anche il ``verso'' quindi $\leq$ diventerà $\geq$, mentre i
vincoli di dominio conservano il ``verso''. Infine i coefficienti dei
vincoli del problema duale non sono altro che la matrice trasposta dei
coefficienti del problema primale.
\newpage
Si ottiene quindi la seguente
funzione obiettivo per il problema duale:
\[\min z^*=60y_1+10y_2+20y_3\]
Con i seguenti vincoli:
\[3y_1+y_2+y_3\geq 2\]
\[y_1-y_2+y_3\geq -1\]
\[y_1+2y_2-y_3\geq 1\]
con i seguenti vincoli di dominio:
\[y_1,y_2,y_3\geq 0\]
Prodedo ora cercando il punto di ottimo del duale col teorema degli
scarti.\\
Procedo usando il teorema, ricordando che $x_1=15$, $x_2=10$ e
$x_3=0$.\\
Applico quindi le condizioni di complementarietà:
\begin{itemize}
  \item $y_1\cdot(3x_1+x_2+x_3-60)=0$, sostituisco le $x_i$ e ottengo
  $y_1\cdot (-10)=0$ quindi $y_1=0$ che è la \textbf{prima condizione}
  \item $y_2\cdot(x_1-x_2+2x_3-10)=0$  sostituisco le $x_i$ e ottengo
  $y_2\cdot (0)=0$ quindi\textbf{ non posso dedurre alcuna condizione di
  complementarietà} su $y_2$
  \item $y_3\cdot(x_1+x_2-x_3-20)=0$, sostituisco le $x_i$ e ottengo
  $y_3\cdot (0)=0$ \textbf{quindi non posso dedurre alcuna condizione di
    complementarietà} su $y_3$
  \item $(3y_1+y_2+y_3-2)\cdot x_1=0$ che corrisponde a
  $(3y_1+y_2+y_3-2)\cdot 15=0$ quindi $3y_1+y_2+y_3-2=0$ è la
  \textbf{seconda condizione}
  \item $(y_1-y_2+y_3+1)\cdot x_2=0$ che corrisponde a
  $(y_1-y_2+y_3+1)\cdot 5=0$ quindi $y_1-y_2+y_3+1=0$ è la
  \textbf{terza condizione}
  \item $(y_1+2y_2-y_3-1)\cdot x_3=0$ che corrisponde a
  $(y_1+2y_2-y_3-1)\cdot 0=0$ \textbf{quindi non è possibile dedurre alcuna
    condizione di complementarietà sul primo vincolo duale}
\end{itemize}
\newpage
risolvo quindi il sistema con le 3 condizioni trovate:
\[
  \begin{cases}
    y_i=0\\
    3y_1+y_2+y_3-2=0\\
    y_1-y_2+y_3+1=0
  \end{cases}
\]
\[\Downarrow\]
\[
  \begin{cases}
    y_i=0\\
    3y_1+y_2+y_3=2\\
    y_1-y_2+y_3=-1
  \end{cases}
\]
\[\Downarrow\]
\[
  \begin{cases}
    y_i=0\\
    y_2+y_3=2\\
    y_2+y_3=-1
  \end{cases}
\]
Ottengo quindi:
\begin{shaded}
  \[
    \begin{cases}
      y_1=0\\
      y_2=\frac{3}{2}\\
      y_3=\frac{1}{2}
    \end{cases}
  \]
  e sostituendo nella funzione obiettivo del problema duale ottengo
  esattamente $z^*=25$ quindi la soluzione è congruente. Inoltre,
  sempre sostitutendo, verifico che tutti i vincoli (compresi quelli
  di dominio) sono rispettati, verificando così la correttezza dei
  risultati ottenuti.
\end{shaded}
\chapter{Esercizio 2}
Partiamo dalla seguente funzione obiettivo:
\[\max z=3x_1+4x_2\]
soggetta ai seguenti vincoli:
\[x_1+2x_2\leq 4\]
\[x_1+\frac{1}{3}x_2\leq 2\]
con i seguenti vincoli di dominio:
\[x_1,x_2\geq 0\]
Sappiamo, grazie al testo, che si ha il seguente tableau finale:
\begin{center}
  \begin{tabular}{c|cccc}
    & $x_1$ & $x_2$ & $s_1$ & $s_2$\\
    \hline
    $\frac{64}{5}$ & 0 & 0 & $\frac{7}{5}$ & $\frac{18}{5}$\\
    \hline
    $\frac{6}{5}$ & 0 & 1 & $\frac{3}{5}$ & $-\frac{3}{5}$\\
    $\frac{8}{5}$ & 1 & 0 & $-\frac{1}{5}$ & $\frac{6}{5}$\\
  \end{tabular}
\end{center}
Procediamo per punti:
\begin{itemize}
  \item cerchiamo gli intervalli di ammissibilità dei termini noti
  $b_i$:
  \begin{itemize}
    \item il primo è $b_1=4$. Sappiamo che al primo vincolo è
    associata la variabile di slack $s_1$ la quale può modificare il
    valore del termine noto. Prendendo il tableau finale dobbiamo
    ottenere che i termini noti restino positivi al più di $\Delta_1$
    volte il valore della variabile di slack in quella riga. Ottengo quindi
    il sistema:
    \[
      \begin{cases}
        \frac{6}{5}+\frac{3}{5}\Delta_1\geq 0\\
        \frac{8}{5}-\frac{1}{5}\Delta_1\geq 0
      \end{cases}
    \]
    Risolvendo il sistema ottengo $-2\leq \Delta_1\leq 8$ e quindi
    l'intervallo di ammissibilità di $b_1$:
    \begin{shaded}
      $b_1\in [4-2, 4+8]=[2,12]$
    \end{shaded}
    \item il secondo è $b_2=2$. Sappiamo che al secondo vincolo è
    associata la variabile di slack $s_2$  la quale può modificare il
    valore del termine noto. Prendendo il tableau finale dobbiamo
    ottenere che i termini noti restino positivi al più di $\Delta_2$
    volte il valore della variabile di slack in quella riga. Ottengo quindi
    il sistema:
    \[
      \begin{cases}
        \frac{6}{5}-\frac{3}{5}\Delta_2\geq 0\\
        \frac{8}{5}+\frac{6}{5}\Delta_2\geq 0
      \end{cases}
    \]
    Risolvendo il sistema ottengo $-\frac{4}{3}\leq \Delta_2\leq 2$ e quindi
    l'intervallo di ammissibilità di $b_2$:
    \begin{shaded}
      $b_2\in [2-\frac{4}{3}, 2+2]=[\frac{2}{3},4]$
    \end{shaded}
  \end{itemize}
  \item cerchiamo gli intervalli di ammissibilità dei coefficienti
  della funzione obiettivo relativi alle variabili in base $c_1$:
  \begin{itemize}
    \item il primo è $c_1=3$. Riscrivo il tableau ponendo un
    $-\Delta_1$, nella $riga_0$ in corrispondenza alla variabile di cui
    sto studiando il coefficiente. Cerco quindi un modo per riportare
    la variabile in base e studio le trasformazioni che vengono
    propagate. Il tableu sarà quindi:
    \begin{center}
      \begin{tabular}{c|cccc}
        & $x_1$ & $x_2$ & $s_1$ & $s_2$\\
        \hline
        $\frac{64}{5}$ & $-\Delta_1$ & 0 & $\frac{7}{5}$ & $\frac{18}{5}$\\
        \hline
        $\frac{6}{5}$ & 0 & 1 & $\frac{3}{5}$ & $-\frac{3}{5}$\\
        $\frac{8}{5}$ & 1 & 0 & $-\frac{1}{5}$ & $\frac{6}{5}$\\
      \end{tabular}
    \end{center}
    e procedo facendo $riga_0=riga_0+\Delta\cdot riga_2$ in quanto
    $x_1$ è presente per la $riga_2$, rimandando così $x_1$ in base.
    Ottengo quindi:
    \begin{center}
      \begin{tabular}{c|cccc}
        & $x_1$ & $x_2$ & $s_1$ & $s_2$\\
        \hline
        $\frac{64}{5}+\Delta_1\cdot \frac{8}{5}$ & 0 & 0 & $\frac{7}{5}-\Delta_1\cdot \frac{1}{5}$ & $\frac{18}{5}+\Delta_1\cdot \frac{6}{5}$\\
        \hline
        $\frac{6}{5}$ & 0 & 1 & $\frac{3}{5}$ & $-\frac{3}{5}$\\
        $\frac{8}{5}$ & 1 & 0 & $-\frac{1}{5}$ & $\frac{6}{5}$\\
      \end{tabular}
    \end{center}
    ma sappiamo che per mantenere ottima la soluzione devo avere solo
    termini positivi in $riga_0$ quindi:
    \[
      \begin{cases}
        \frac{7}{5}-\Delta_1\cdot \frac{1}{5}\geq 0\\
        \frac{18}{5}+\Delta_1\cdot \frac{6}{5}\geq 0
      \end{cases}
    \]
    Risolvendo il sistema ottengo $-3\leq \Delta_1\leq 7$ e quindi
    l'intervallo di ammissibilità di $c_1$:
    \begin{shaded}
      $c_1\in [3-3,3+7]=[0,10]$
    \end{shaded}
    \item il secondo è $c_2=4$. Riscrivo il tableau ponendo un
    $-\Delta_2$, nella $riga_0$ in corrispondenza alla variabile di cui
    sto studiando il coefficiente. Cerco quindi un modo per riportare
    la variabile in base e studio le trasformazioni che vengono
    propagate. Il tableu sarà quindi:
    \begin{center}
      \begin{tabular}{c|cccc}
        & $x_1$ & $x_2$ & $s_1$ & $s_2$\\
        \hline
        $\frac{64}{5}$ & 0 & $-\Delta_2$ & $\frac{7}{5}$ & $\frac{18}{5}$\\
        \hline
        $\frac{6}{5}$ & 0 & 1 & $\frac{3}{5}$ & $-\frac{3}{5}$\\
        $\frac{8}{5}$ & 1 & 0 & $-\frac{1}{5}$ & $\frac{6}{5}$\\
      \end{tabular}
    \end{center}
    e procedo facendo $riga_0=riga_0+\Delta\cdot riga_1$ in quanto
    $x_2$ è presente per la $riga_1$, rimandando così $x_2$ in base.
    Ottengo quindi:
    \begin{center}
      \begin{tabular}{c|cccc}
        & $x_1$ & $x_2$ & $s_1$ & $s_2$\\
        \hline
        $\frac{64}{5}+\Delta_2\cdot \frac{6}{5}$ & 0 & 0 & $\frac{7}{5}+\Delta_2\cdot \frac{3}{5}$ & $\frac{18}{5}-\Delta_2\cdot \frac{3}{5}$\\
        \hline
        $\frac{6}{5}$ & 0 & 1 & $\frac{3}{5}$ & $-\frac{3}{5}$\\
        $\frac{8}{5}$ & 1 & 0 & $-\frac{1}{5}$ & $\frac{6}{5}$\\
      \end{tabular}
    \end{center}
    ma sappiamo che per mantenere ottima la soluzione devo avere solo
    termini positivi in $riga_0$ quindi:
    \[
      \begin{cases}
        \frac{7}{5}+\Delta_2\cdot \frac{3}{5}\geq 0\\
        \frac{18}{5}-\Delta_2\cdot \frac{3}{5}\geq 0
      \end{cases}
    \]
    Risolvendo il sistema ottengo $-\frac{7}{3}\leq \Delta_2\leq 6$ e quindi
    l'intervallo di ammissibilità di $c_2$:
    \begin{shaded}
      $c_2\in [4-\frac{3}{7},4+6]=[\frac{5}{3},10]$
    \end{shaded}
  \end{itemize}
  \item cerchiamo gli intervalli di ammissibilità dei coefficienti
  della funzione obiettivo relativi alle variabili fuori base $k_i$.\\
  \textbf{Per rispondere alla domanda non si deve fare nulla in quanto
  le variabili della funzione obiettivo sono in base. ``Per sport''
  calcoliamo lo stesso quelle delle variabili fuori base, anche se non
  sono nella funzione obiettivo}. \\
  In questo caso ci basta verificare che i valori corrispondenti alle
  variabili fuori base nella $riga_0$ restino positivi al più di un
  $\Delta_i$, in modo di restare nella condizione di avere la
  soluzione ottimale. Si ha quindi:
  \begin{itemize}
    \item $\frac{7}{5}+\Delta_1\geq 0\Longrightarrow \Delta_1\geq -\frac{7}{5}$
    \item $\frac{18}{5}+\Delta_2\geq 0\Longrightarrow \Delta_2\geq -\frac{18}{5}$
  \end{itemize}
  quindi si ottiene che:
  \begin{shaded}
    $k_1\in[-\frac{7}{5}, +\infty]$\\
    $k_2\in[-\frac{18}{5}, +\infty]$
  \end{shaded}
\end{itemize}
\end{document}