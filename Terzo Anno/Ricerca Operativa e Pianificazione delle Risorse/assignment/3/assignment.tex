\documentclass[a4paper,12pt, oneside]{book}

% \usepackage{fullpage}
\usepackage[italian]{babel}
\usepackage[utf8]{inputenc}
\usepackage{amssymb}
\usepackage{amsthm}
\usepackage{graphics}
\usepackage{amsfonts}
\usepackage{listings}
\usepackage{amsmath}
\usepackage{amstext}
\usepackage{engrec}
\usepackage{rotating}
\usepackage[safe,extra]{tipa}
\usepackage{showkeys}
\usepackage{multirow}
\usepackage{hyperref}
\usepackage{mathtools}
\usepackage{microtype}
\usepackage{enumerate}
\usepackage{braket}
\usepackage{marginnote}
\usepackage{pgfplots}
\usepackage{cancel}
\usepackage{polynom}
\usepackage{booktabs}
\usepackage{enumitem}
\usepackage{framed}
\usepackage{algorithm}
\usepackage{algpseudocode}
\usepackage{pdfpages}
\usepackage{pgfplots}
\usepackage[cache=false]{minted}

\usepackage[usenames,dvipsnames]{pstricks}
\usepackage{epsfig}
\usepackage{pst-grad} % For gradients
\usepackage{pst-plot} % For axes
\usepackage[space]{grffile} % For spaces in paths
\usepackage{etoolbox} % For spaces in paths
\makeatletter % For spaces in paths
\patchcmd\Gread@eps{\@inputcheck#1 }{\@inputcheck"#1"\relax}{}{}
\makeatother

\usepackage{tikz}\usetikzlibrary{er}\tikzset{multi  attribute /.style={attribute ,double  distance =1.5pt}}\tikzset{derived  attribute /.style={attribute ,dashed}}\tikzset{total /.style={double  distance =1.5pt}}\tikzset{every  entity /.style={draw=orange , fill=orange!20}}\tikzset{every  attribute /.style={draw=MediumPurple1, fill=MediumPurple1!20}}\tikzset{every  relationship /.style={draw=Chartreuse2, fill=Chartreuse2!20}}\newcommand{\key}[1]{\underline{#1}}

\usepackage{fancyhdr}
\pagestyle{fancy}
\fancyhead[LE,RO]{\slshape \rightmark}
\fancyhead[LO,RE]{\slshape \leftmark}
\fancyfoot[C]{\thepage}
\usepackage{tikz}
\usetikzlibrary{automata,positioning}


\title{Assignment 3}
\author{Davide Cozzi, 829827}
\date{}

\pgfplotsset{compat=1.13}
\begin{document}
\maketitle

\definecolor{shadecolor}{gray}{0.80}
\setlist{leftmargin = 2cm}
\newtheorem{teorema}{Teorema}
\newtheorem{definizione}{Definizione}
\newtheorem{esempio}{Esempio}
\newtheorem{corollario}{Corollario}
\newtheorem{lemma}{Lemma}
\newtheorem{osservazione}{Osservazione}
\newtheorem{nota}{Nota}
\newtheorem{esercizio}{Esercizio}

\renewcommand{\chaptermark}[1]{%
  \markboth{\chaptername
    \ \thechapter.\ #1}{}}
\renewcommand{\sectionmark}[1]{\markright{\thesection.\ #1}}
\chapter{Esercizio 1}
\section{Parte a}
Iniziamo disegnando la regione ammissibile del problema di
programmazione lineare intera.\\
Disegnamo sul piano cartesiano le due rette che rappresentano i vincoli,
ovvero:
\[-2x_1+5x_2=10\]
\[6x_1-5x_2=30\]
ottenendo:

\begin{center}
  \psscalebox{1.0 1.0} % Change this value to rescale the drawing.
  {
    \begin{pspicture}(0,-2.759881)(6.714386,2.759881)
      \psline[linecolor=black, linewidth=0.04, arrowsize=0.05291667cm 2.0,arrowlength=1.4,arrowinset=0.0]{->}(1.607428,-2.3470771)(6.807428,-2.3470771)
      \psline[linecolor=black, linewidth=0.04](0.0074279783,-2.3470771)(6.007428,0.052922823)
      \rput[bl](6.407428,-2.7470772){$x_1$}
      \rput[bl](1.607428,2.4529228){$x_2$}
      \psline[linecolor=black, linewidth=0.04](4.407428,-2.7470772)(0.40742797,2.0529227)
      \psdots[linecolor=black, dotsize=0.1](4.047428,-2.3470771)
      \psdots[linecolor=black, dotsize=0.1](2.027428,0.11292282)
      \psdots[linecolor=black, dotsize=0.1](2.007428,-1.5470772)
      \psdots[linecolor=black, dotsize=0.1](3.0674279,-1.1470772)
      \psline[linecolor=black, linewidth=0.04, arrowsize=0.05291667cm 2.0,arrowlength=1.4,arrowinset=0.0]{->}(2.007428,-2.7470772)(2.007428,2.4529228)(2.007428,2.4529228)(2.007428,2.852923)
      \rput[bl](3.967428,-2.7470772){5}
      \rput[bl](1.707428,-1.4470772){2}
      \rput[bl](1.707428,-0.04707718){6}
    \end{pspicture}
  }

\end{center}
\newpage
Essendo però un problema di programmazione lineare intera non avremo
l'area ammissibile data unicamente dai vincoli bensì avremo i punti di
coordinate intere in quest'area, ovvero:
\begin{center}
  
  \psscalebox{1.0 1.0} % Change this value to rescale the drawing.
  {
    \begin{pspicture}(0,-2.759881)(6.714386,2.759881)
      \definecolor{colour0}{rgb}{0.039215688,0.023529412,0.9254902}
      \definecolor{colour1}{rgb}{0.015686275,0.015686275,0.019607844}
      \psline[linecolor=black, linewidth=0.04, arrowsize=0.05291667cm 2.0,arrowlength=1.4,arrowinset=0.0]{->}(1.607428,-2.3470771)(6.807428,-2.3470771)
      \psline[linecolor=black, linewidth=0.04](0.0074279783,-2.3470771)(6.007428,0.052922823)
      \rput[bl](6.407428,-2.7470772){$x_1$}
      \rput[bl](1.607428,2.4529228){$x_2$}
      \psline[linecolor=black, linewidth=0.04](4.407428,-2.7470772)(0.40742797,2.0529227)
      \psdots[linecolor=black, dotsize=0.1](4.047428,-2.3470771)
      \psdots[linecolor=black, dotsize=0.1](2.027428,0.11292282)
      \psdots[linecolor=black, dotsize=0.1](2.007428,-1.5470772)
      \psdots[linecolor=black, dotsize=0.1](3.0674279,-1.1470772)
      \psline[linecolor=black, linewidth=0.04, arrowsize=0.05291667cm 2.0,arrowlength=1.4,arrowinset=0.0]{->}(2.007428,-2.7470772)(2.007428,2.4529228)(2.007428,2.4529228)(2.007428,2.852923)
      \rput[bl](3.967428,-2.7470772){5}
      \rput[bl](1.707428,-1.4470772){2}
      \rput[bl](1.707428,-0.04707718){6}
      \psdots[linecolor=colour0, dotsize=0.1](2.407428,-1.5470772)
      \psdots[linecolor=colour0, dotsize=0.1](2.807428,-1.5470772)
      \psdots[linecolor=colour0, dotsize=0.1](3.207428,-1.5470772)
      \psdots[linecolor=colour0, dotsize=0.1](3.607428,-1.9470772)
      \psdots[linecolor=colour0, dotsize=0.1](2.007428,-1.5470772)
      \psdots[linecolor=colour0, dotsize=0.1](2.007428,-1.9470772)
      \psdots[linecolor=colour0, dotsize=0.1](2.007428,-2.3470771)
      \psdots[linecolor=colour0, dotsize=0.1](2.407428,-2.3470771)
      \psdots[linecolor=colour0, dotsize=0.1](2.807428,-2.3470771)
      \psdots[linecolor=colour0, dotsize=0.1](3.207428,-2.3470771)
      \psdots[linecolor=colour0, dotsize=0.1](3.607428,-2.3470771)
      \psdots[linecolor=colour0, dotsize=0.1](4.007428,-2.3470771)
      \psdots[linecolor=colour0, dotsize=0.1](3.207428,-1.9470772)
      \psdots[linecolor=colour0, dotsize=0.1](2.807428,-1.9470772)
      \psdots[linecolor=colour0, dotsize=0.1](2.407428,-1.9470772)
      \rput[bl](3.547428,-2.7270772){\textcolor{colour1}{4}}
      \rput[bl](3.127428,-2.7270772){3}
      \rput[bl](2.747428,-2.7270772){2}
      \rput[bl](2.367428,-2.7470772){1}
      \rput[bl](1.7474279,-2.667077){0}
      \rput[bl](1.7474279,-2.0270772){\textcolor{colour1}{1}}
    \end{pspicture}
  } 
\end{center}
\section{Parte b}
Procediamo ora con la risoluzione del problema.\\
Iniziamo risolvendo il \textit{rilassamento lineare} del
problema. Chiamiamo $P_0$ questo problema.\\
Dovendo risolvere il rilassamento lineare avremo a che fare con:
\[\min z=x_1-3x_2\]
\begin{center}
  soggetto ai vincoli
\end{center}
\[-2x_1+5x_2\leq 10\]
\[6x_1-5x_2\leq 30\]
\[x_1,x_2\geq 0\]
I punti di nostro interesse sono $(0,0)$, ovvero l'origine, $(0,2)$,
ovvero l'incrocio tra il primo vincolo e l'asse $x_2$, $(5,0)$, ovvero
l'incrocio tra il secondo vincolo e l'asse $x_1$, e il punto di
incontro tra i due vincoli. Calcoliamo questo punto di incontro:
\[
  \begin{cases}
    -2x_1+5x_2= 10\\
    6x_1-5x_2=30
  \end{cases}\Longrightarrow
  x_1=\frac{5}{2},\,\,\,x_2=3\Longrightarrow (\frac{5}{2},3)
\]
\newpage
Procediamo quindi con la risoluzione grafica mediante il metodo del
simplesso.\\
Partiamo valutando il punto $(0,0)$, qui si ha $z=0$. Come vertici
adiacenti ha $(0,2)$ e $(5,0)$. Grazie alla funzione obiettivo notiamo
che $z$ decresce (stiamo cercando il minimo) se ci spostiamo verso
$(0,2)$, dove $z=-6$, e cresce se ci spostiamo verso $(5,0)$ dove
$z=5$.\\
Arrivato in $(0,2)$ ho solamente $(\frac{5}{2},3)$ come vertice
ammissibile da verificare. Si ha che, in $(\frac{5}{2},3)$,
$z=-\frac{13}{2}$, abbiamo trovato quindi la soluzione
ottimale. Graficamente si avrebbe:
\begin{center}
  
  \psscalebox{1.0 1.0} % Change this value to rescale the drawing.
  {
    \begin{pspicture}(0,-2.759881)(6.3223224,2.759881)
      \definecolor{colour1}{rgb}{0.015686275,0.015686275,0.019607844}
      \definecolor{colour2}{rgb}{0.23921569,0.23921569,0.94509804}
      \definecolor{colour3}{rgb}{0.39607844,0.7647059,0.96862745}
      \definecolor{colour4}{rgb}{0.039215688,0.039215688,0.047058824}
      \psline[linecolor=black, linewidth=0.04, arrowsize=0.05291667cm 2.0,arrowlength=1.4,arrowinset=0.0]{->}(1.2153643,-2.3470771)(6.4153643,-2.3470771)
      \psline[linecolor=black, linewidth=0.04](1.6153644,-1.5470772)(5.6153646,0.052922823)
      \rput[bl](6.015364,-2.7470772){$x_1$}
      \rput[bl](1.2153643,2.4529228){$x_2$}
      \psline[linecolor=black, linewidth=0.04](4.015364,-2.7470772)(0.01536438,2.0529227)
      \psdots[linecolor=black, dotsize=0.1](3.6553643,-2.3470771)
      \psdots[linecolor=black, dotsize=0.1](1.6353644,0.11292282)
      \psdots[linecolor=black, dotsize=0.1](1.6153644,-1.5470772)
      \psdots[linecolor=black, dotsize=0.1](2.6753645,-1.1470772)
      \psline[linecolor=black, linewidth=0.04, arrowsize=0.05291667cm 2.0,arrowlength=1.4,arrowinset=0.0]{->}(1.6153644,-2.7470772)(1.6153644,2.4529228)(1.6153644,2.4529228)(1.6153644,2.852923)
      \rput[bl](3.5753644,-2.7470772){5}
      \rput[bl](1.3153644,-1.4470772){2}
      \rput[bl](1.3153644,-0.04707718){6}
      \rput[bl](3.1553643,-2.7270772){\textcolor{colour1}{4}}
      \rput[bl](2.7353644,-2.7270772){3}
      \rput[bl](2.3553643,-2.7270772){2}
      \rput[bl](1.9753643,-2.7470772){1}
      \rput[bl](1.3553643,-2.667077){0}
      \rput[bl](1.3553643,-2.0270772){\textcolor{colour1}{1}}
      \psline[linecolor=colour1, linewidth=0.04, arrowsize=0.05291667cm 2.0,arrowlength=1.4,arrowinset=0.0]{->}(1.8385804,-1.3231351)(2.3921485,-1.0710192)
      \psdots[linecolor=colour1, dotsize=0.2](2.7153645,-1.1470772)
      \rput[bl](2.4153643,-0.80707717){\textcolor{colour1}{$(\frac{5}{2},3)$}}
      \pspolygon[linecolor=colour2, linewidth=0.04, fillstyle=solid,fillcolor=colour3](3.6553643,-2.3270772)(1.6353644,-2.3470771)(1.6153644,-1.5270772)(2.7153645,-1.1270772)(3.6953645,-2.367077)
      \psline[linecolor=colour4, linewidth=0.04, arrowsize=0.05291667cm 2.0,arrowlength=1.4,arrowinset=0.0]{->}(1.2808977,-2.207037)(1.269831,-1.6071174)
    \end{pspicture}
  }

\end{center}
Possiamo quindi dire che nel nodo $P_0$ abbiamo $z_0=-\frac{13}{5}$,
$UB_0=6$ (ovvero l'upperbound intero) e $z^*=+\infty$ (ovvero la migliore
soluzione intera fino a $P_o$, posta a $+\infty$ in quanto non ancora
trovata).\\
Posso quindi procedere con il Metodo Branch\&Bound.\\
Avendo come soluzione ottima di $P_0$ il punto $(\frac{5}{2},3)$
cerchiamo la soluzione intera partizionando su $x_1$, specificando gli
intervalli secondo le formule:
\[P_1:x_j\leq \lfloor x_j^*\rfloor\]
\[P_2:x_j\geq \lfloor x_j^*\rfloor+1\]
ottenendo quindi, nel nostro caso, i seguenti vincoli per il problema
$P_1$:
\[-2x_1+5x_2\leq 10\]
\[6x_1-5x_2\leq 30\]
\[x_1\leq 2\]
\[x_1,x_2\in\mathbb{N}\]
e per $P_2$:
\[-2x_1+5x_2\leq 10\]
\[6x_1-5x_2\leq 30\]
\[x_1\geq 3\]
\[x_1,x_2\in\mathbb{N}\]
Iniziamo valutando $P_1$. Disegnandolo si ottiene:
\begin{center}
  
  \psscalebox{1.0 1.0} % Change this value to rescale the drawing.
  {
    \begin{pspicture}(0,-2.759881)(6.3223224,2.759881)
      \definecolor{colour1}{rgb}{0.015686275,0.015686275,0.019607844}
      \definecolor{colour4}{rgb}{0.039215688,0.039215688,0.047058824}
      \definecolor{colour3}{rgb}{0.39607844,0.7647059,0.96862745}
      \psline[linecolor=black, linewidth=0.04, arrowsize=0.05291667cm 2.0,arrowlength=1.4,arrowinset=0.0]{->}(1.2153643,-2.3470771)(6.4153643,-2.3470771)
      \psline[linecolor=black, linewidth=0.04](1.6153644,-1.5470772)(5.6153646,0.052922823)
      \rput[bl](6.015364,-2.7470772){$x_1$}
      \rput[bl](1.2153643,2.4529228){$x_2$}
      \psline[linecolor=black, linewidth=0.04](4.015364,-2.7470772)(0.01536438,2.0529227)
      \psdots[linecolor=black, dotsize=0.1](3.6553643,-2.3470771)
      \psdots[linecolor=black, dotsize=0.1](1.6353644,0.11292282)
      \psdots[linecolor=black, dotsize=0.1](1.6153644,-1.5470772)
      \psdots[linecolor=black, dotsize=0.1](2.6753645,-1.1470772)
      \psline[linecolor=black, linewidth=0.04, arrowsize=0.05291667cm 2.0,arrowlength=1.4,arrowinset=0.0]{->}(1.6153644,-2.7470772)(1.6153644,2.4529228)(1.6153644,2.4529228)(1.6153644,2.852923)
      \rput[bl](3.5753644,-2.7470772){5}
      \rput[bl](1.3153644,-1.4470772){2}
      \rput[bl](1.3153644,-0.04707718){6}
      \rput[bl](3.1553643,-2.7270772){\textcolor{colour1}{4}}
      \rput[bl](2.7353644,-2.7270772){3}
      \rput[bl](2.3553643,-2.7270772){2}
      \rput[bl](1.9753643,-2.7470772){1}
      \rput[bl](1.3553643,-2.667077){0}
      \rput[bl](1.3553643,-2.0270772){\textcolor{colour1}{1}}
      \psline[linecolor=colour4, linewidth=0.04](2.4153643,-2.3470771)(2.4153643,1.2529228)
      \pspolygon[linecolor=colour4, linewidth=0.04, fillstyle=solid,fillcolor=colour3](2.4553645,-1.2470772)(1.6353644,-1.5470772)(1.5953643,-2.3470771)(2.4353645,-2.3270772)(2.4153643,-1.2470772)
      \psdots[linecolor=colour4, dotsize=0.2](2.4553645,-1.2870772)
    \end{pspicture}
  }

\end{center}
Risolviamo $P_1$ e $P_2$ mediante il bounding, risolvendone i
rilassamenti lineari. Per il problema
$P_1$ si ha:
\[-2x_1+5x_2\leq 10\]
\[6x_1-5x_2\leq 30\]
\[x_1\leq 2\]
\[x_1,x_2\geq 0\]
e per $P_2$:
\[-2x_1+5x_2\leq 10\]
\[6x_1-5x_2\leq 30\]
\[x_1\geq 3\]
\[x_1,x_2\geq 0\]
Partiamo quindi da $(0,0)$ che ha $z=0$. Come vertici adiacenti abbiamo
$(0,2)$, con $z=-6$, e $(2,0)$, con $z=2$. Ci spostiamo in
$(0,2)$. Calcoliamo l'incorcio tra $-2x_1+5x_2=10$ e $x_1=2$ e
otteniamo il punto $(2,\frac{14}{2})$, che ha $z=-\frac{32}{5}$,
ovvero la soluzione ottima di $P_1$, anche se ancora non intera.\\
Per $P_1$ si avrà quindi $z_1=-\frac{32}{5}$, $UB_1=6$ e
$z_1^*=-\infty$. Non possiamo usare il fathoming per chiudere questo
problema. 
\newpage
Passiamo ora a $P_2$ che invece si presenta come:
\begin{center}
  
  \psscalebox{1.0 1.0} % Change this value to rescale the drawing.
  {
    \begin{pspicture}(0,-2.759881)(6.3223224,2.759881)
      \definecolor{colour1}{rgb}{0.015686275,0.015686275,0.019607844}
      \definecolor{colour4}{rgb}{0.039215688,0.039215688,0.047058824}
      \definecolor{colour3}{rgb}{0.39607844,0.7647059,0.96862745}
      \psline[linecolor=black, linewidth=0.04, arrowsize=0.05291667cm 2.0,arrowlength=1.4,arrowinset=0.0]{->}(1.2153643,-2.3470771)(6.4153643,-2.3470771)
      \psline[linecolor=black, linewidth=0.04](1.6153644,-1.5470772)(5.6153646,0.052922823)
      \rput[bl](6.015364,-2.7470772){$x_1$}
      \rput[bl](1.2153643,2.4529228){$x_2$}
      \psline[linecolor=black, linewidth=0.04](4.015364,-2.7470772)(0.01536438,2.0529227)
      \psdots[linecolor=black, dotsize=0.1](3.6553643,-2.3470771)
      \psdots[linecolor=black, dotsize=0.1](1.6353644,0.11292282)
      \psdots[linecolor=black, dotsize=0.1](1.6153644,-1.5470772)
      \psdots[linecolor=black, dotsize=0.1](2.6753645,-1.1470772)
      \psline[linecolor=black, linewidth=0.04, arrowsize=0.05291667cm 2.0,arrowlength=1.4,arrowinset=0.0]{->}(1.6153644,-2.7470772)(1.6153644,2.4529228)(1.6153644,2.4529228)(1.6153644,2.852923)
      \rput[bl](3.5753644,-2.7470772){5}
      \rput[bl](1.3153644,-1.4470772){2}
      \rput[bl](1.3153644,-0.04707718){6}
      \rput[bl](3.1553643,-2.7270772){\textcolor{colour1}{4}}
      \rput[bl](2.7353644,-2.7270772){3}
      \rput[bl](2.3553643,-2.7270772){2}
      \rput[bl](1.9753643,-2.7470772){1}
      \rput[bl](1.3553643,-2.667077){0}
      \rput[bl](1.3553643,-2.0270772){\textcolor{colour1}{1}}
      \psline[linecolor=colour4, linewidth=0.04](2.8153644,-2.3470771)(2.8153644,2.0529227)
      \pspolygon[linecolor=colour4, linewidth=0.04, fillstyle=solid,fillcolor=colour3](3.6553643,-2.3470771)(2.8353643,-2.3470771)(2.8353643,-1.3470771)(3.6553643,-2.3070772)
    \end{pspicture}
  }

\end{center}
Partiamo quindi, per il metodo del simplesso, da $(3,0)$, che ha
$z=3$. Come vertici adiacenti abbiamo $(5,0)$, con $z=5$ e l'incrocio
tra $6x_1+5x_2=30$ e $x_1=3$. Calcolato il punto si ottiene che è
$(3,\frac{12}{5})$, con $z=-\frac{21}{5}$, ovvero la soluzione ottima,
anche se ancora non intera, per $P_2$. In $P_2$ si ha quindi
$z_2=\frac{21}{5}$, $UB_2=4$ e $z_2^*=-\infty$.
Non possiamo usare il fathoming per chiudere questo problema.\\
Avendo ancora entrambi nodi attivi procediamo col branching di
$P_1$. In questo caso, avendo $x_2=\frac{14}{5}$, procediamo col creare
$P_3$ con i seguenti vincoli (usando la stessa tecnica usata sopra):
\[-2x_1+5x_2\leq 10\]
\[6x_1-5x_2\leq 30\]
\[x_1\leq 2\]
\[x_2\leq 2\]
\[x_1,x_2\in\mathbb{N}\]
e per $P_4$:
\[-2x_1+5x_2\leq 10\]
\[6x_1-5x_2\leq 30\]
\[x_1\leq 2\]
\[x_2\geq 3\]
\[x_1,x_2\in\mathbb{N}\]
\newpage
Procedendo col bounding otteniamo i rispettivi rilassamenti
lineari. Per $P_3$:
\[-2x_1+5x_2\leq 10\]
\[6x_1-5x_2\leq 30\]
\[x_1\leq 2\]
\[x_2\leq 2\]
\[x_1,x_2\geq 0\]
e per $P_4$:
\[-2x_1+5x_2\leq 10\]
\[6x_1-5x_2\leq 30\]
\[x_1\leq 2\]
\[x_2\geq 3\]
\[x_1,x_2\geq 0\]
Partiamo da $P_3$ che si presenta così:
\begin{center}
  \psscalebox{1.0 1.0} % Change this value to rescale the drawing.
  {
    \begin{pspicture}(0,-2.759881)(6.3223224,2.759881)
      \definecolor{colour1}{rgb}{0.015686275,0.015686275,0.019607844}
      \definecolor{colour4}{rgb}{0.039215688,0.039215688,0.047058824}
      \definecolor{colour3}{rgb}{0.39607844,0.7647059,0.96862745}
      \psline[linecolor=black, linewidth=0.04, arrowsize=0.05291667cm 2.0,arrowlength=1.4,arrowinset=0.0]{->}(1.2153643,-2.3470771)(6.4153643,-2.3470771)
      \psline[linecolor=black, linewidth=0.04](1.6153644,-1.5470772)(5.6153646,0.052922823)
      \rput[bl](6.015364,-2.7470772){$x_1$}
      \rput[bl](1.2153643,2.4529228){$x_2$}
      \psline[linecolor=black, linewidth=0.04](4.015364,-2.7470772)(0.01536438,2.0529227)
      \psdots[linecolor=black, dotsize=0.1](3.6553643,-2.3470771)
      \psdots[linecolor=black, dotsize=0.1](1.6353644,0.11292282)
      \psdots[linecolor=black, dotsize=0.1](1.6153644,-1.5470772)
      \psdots[linecolor=black, dotsize=0.1](2.6753645,-1.1470772)
      \psline[linecolor=black, linewidth=0.04, arrowsize=0.05291667cm 2.0,arrowlength=1.4,arrowinset=0.0]{->}(1.6153644,-2.7470772)(1.6153644,2.4529228)(1.6153644,2.4529228)(1.6153644,2.852923)
      \rput[bl](3.5753644,-2.7470772){5}
      \rput[bl](1.3153644,-1.4470772){2}
      \rput[bl](1.3153644,-0.04707718){6}
      \rput[bl](3.1553643,-2.7270772){\textcolor{colour1}{4}}
      \rput[bl](2.7353644,-2.7270772){3}
      \rput[bl](2.3553643,-2.7270772){2}
      \rput[bl](1.9753643,-2.7470772){1}
      \rput[bl](1.3553643,-2.667077){0}
      \rput[bl](1.3553643,-2.0270772){\textcolor{colour1}{1}}
      \psline[linecolor=colour4, linewidth=0.04](2.4153643,-2.3470771)(2.4153643,2.0529227)
      \psline[linecolor=colour4, linewidth=0.04](1.6153644,-1.5470772)(5.2153645,-1.5470772)
      \pspolygon[linecolor=colour4, linewidth=0.04, fillstyle=solid,fillcolor=colour3](1.6153644,-1.5470772)(2.4153643,-1.5470772)(2.4153643,-2.3470771)(1.6153644,-2.3470771)
    \end{pspicture}
  }
\end{center}
Partiamo valutando il punto $(0,0)$, con $z=0$. Come vertici adiacenti
ha il vertice $(0,2)$, con $z=-6$, e il vertice $(2,0)$, con
$z=2$. Mi sposto in $(0,2)$ e valuto l'unico vertice adiacente
rimasto, ovvero $(2,2)$, con $z=-4$, valore che mi fa restare in
$(0,2)$. Abbiamo quindi che in $(0,2)$ si ha la soluzione ottima,
intera, e, quindi, per $P_3$ si ha $z_3=-6$, $UB_3=-6$ e $z_3^*=-6$.
Possiamo chiudere il branch  grazie alla terza regola di
fathoming\\
Valutiamo $P_4$, disegnandolo si ottiene:
\begin{center}
  \psscalebox{1.0 1.0} % Change this value to rescale the drawing.
  {
    \begin{pspicture}(0,-2.759881)(6.3223224,2.759881)
      \definecolor{colour1}{rgb}{0.015686275,0.015686275,0.019607844}
      \definecolor{colour4}{rgb}{0.039215688,0.039215688,0.047058824}
      \psline[linecolor=black, linewidth=0.04, arrowsize=0.05291667cm 2.0,arrowlength=1.4,arrowinset=0.0]{->}(1.2153643,-2.3470771)(6.4153643,-2.3470771)
      \psline[linecolor=black, linewidth=0.04](1.6153644,-1.5470772)(5.6153646,0.052922823)
      \rput[bl](6.015364,-2.7470772){$x_1$}
      \rput[bl](1.2153643,2.4529228){$x_2$}
      \psline[linecolor=black, linewidth=0.04](4.015364,-2.7470772)(0.01536438,2.0529227)
      \psdots[linecolor=black, dotsize=0.1](3.6553643,-2.3470771)
      \psdots[linecolor=black, dotsize=0.1](1.6353644,0.11292282)
      \psdots[linecolor=black, dotsize=0.1](1.6153644,-1.5470772)
      \psdots[linecolor=black, dotsize=0.1](2.6753645,-1.1470772)
      \psline[linecolor=black, linewidth=0.04, arrowsize=0.05291667cm 2.0,arrowlength=1.4,arrowinset=0.0]{->}(1.6153644,-2.7470772)(1.6153644,2.4529228)(1.6153644,2.4529228)(1.6153644,2.852923)
      \rput[bl](3.5753644,-2.7470772){5}
      \rput[bl](1.3153644,-1.4470772){2}
      \rput[bl](1.3153644,-0.04707718){6}
      \rput[bl](3.1553643,-2.7270772){\textcolor{colour1}{4}}
      \rput[bl](2.7353644,-2.7270772){3}
      \rput[bl](2.3553643,-2.7270772){2}
      \rput[bl](1.9753643,-2.7470772){1}
      \rput[bl](1.3553643,-2.667077){0}
      \rput[bl](1.3553643,-2.0270772){\textcolor{colour1}{1}}
      \psline[linecolor=colour4, linewidth=0.04](2.4153643,-2.3470771)(2.4153643,2.0529227)
      \psline[linecolor=colour4, linewidth=0.04](1.6153644,-1.1470772)(5.2153645,-1.1470772)
    \end{pspicture}
  }

\end{center}
Notando subito che $P_4$ non ha soluzioni ammissibili (non si hanno punti
nella regione ammissibile) e quindi posso
usare la prima regola di fathoming per chiudere il
branch.\\
Abbiamo ancora il nodo $P_2$ da valutare. $P_2$ aveva soluzione
pottima in $(3,\frac{12}{5})$ quindi procediamo col metodo di branch
su $x_2$. Per $P_5$ si ottiene:
\[-2x_1+5x_2\leq 10\]
\[6x_1-5x_2\leq 30\]
\[x_1\geq 3\]
\[x_2\geq 2\]
\[x_1,x_2\in \mathbb{N}\]
e per $P_6$:
\[-2x_1+5x_2\leq 10\]
\[6x_1-5x_2\leq 30\]
\[x_1\geq 3\]
\[x_2\geq 3\]
\[x_1,x_2\in \mathbb{N}\]
Procedendo col bounding otteniamo i rispettivi rilassamenti lineari.
Per $P_5$ si ottiene:
\[-2x_1+5x_2\leq 10\]
\[6x_1-5x_2\leq 30\]
\[x_1\geq 3\]
\[x_2\geq 2\]
\[x_1,x_2\geq 0\]
e per $P_6$:
\[-2x_1+5x_2\leq 10\]
\[6x_1-5x_2\leq 30\]
\[x_1\geq 3\]
\[x_2\geq 3\]
\[x_1,x_2\geq 0\]
Partiamo da $P_5$ che si presenta così:
\begin{center}
  
  \psscalebox{1.0 1.0} % Change this value to rescale the drawing.
  {
    \begin{pspicture}(0,-3.399967)(8.91405,3.399967)
      \definecolor{colour1}{rgb}{0.015686275,0.015686275,0.019607844}
      \definecolor{colour4}{rgb}{0.039215688,0.039215688,0.047058824}
      \definecolor{colour3}{rgb}{0.39607844,0.7647059,0.96862745}
      \psline[linecolor=black, linewidth=0.04, arrowsize=0.05291667cm 2.0,arrowlength=1.4,arrowinset=0.0]{->}(1.2153643,-2.987163)(6.4153643,-2.987163)
      \psline[linecolor=black, linewidth=0.04](1.6153644,-2.187163)(5.6153646,-0.5871631)
      \rput[bl](6.015364,-3.3871632){$x_1$}
      \rput[bl](1.2153643,1.8128369){$x_2$}
      \psline[linecolor=black, linewidth=0.04](4.015364,-3.3871632)(0.01536438,1.4128369)
      \psdots[linecolor=black, dotsize=0.1](3.6553643,-2.987163)
      \psdots[linecolor=black, dotsize=0.1](1.6353644,-0.5271631)
      \psdots[linecolor=black, dotsize=0.1](1.6153644,-2.187163)
      \psdots[linecolor=black, dotsize=0.1](2.6753645,-1.7871631)
      \psline[linecolor=black, linewidth=0.04, arrowsize=0.05291667cm 2.0,arrowlength=1.4,arrowinset=0.0]{->}(1.6153644,-3.3871632)(1.6153644,1.8128369)(1.6153644,1.8128369)(1.6153644,2.212837)
      \rput[bl](3.5753644,-3.3871632){5}
      \rput[bl](1.3153644,-2.087163){2}
      \rput[bl](1.3153644,-0.6871631){6}
      \rput[bl](3.1553643,-3.3671632){\textcolor{colour1}{4}}
      \rput[bl](2.7353644,-3.3671632){3}
      \rput[bl](2.3553643,-3.3671632){2}
      \rput[bl](1.9753643,-3.3871632){1}
      \rput[bl](1.3553643,-3.307163){0}
      \rput[bl](1.3553643,-2.6671631){\textcolor{colour1}{1}}
      \psline[linecolor=colour4, linewidth=0.04](2.8153644,-2.987163)(2.8153644,1.4128369)
      \psline[linecolor=colour4, linewidth=0.04](1.6153644,-2.187163)(5.2153645,-2.187163)
      \pspolygon[linecolor=colour4, linewidth=0.04, fillstyle=solid,fillcolor=colour3](2.8353643,-2.207163)(3.0553644,-2.187163)(3.6953645,-3.027163)(2.8353643,-2.987163)(2.8353643,-2.227163)
    \end{pspicture}
  }

  \\
\end{center}
partiamo valutando il punto $(3,0)$, con $z=3$. Come vertici adiacenti
abbiamo $(5,0)$, con $z=5$, e $(3,2)$, con $z=-3$. Ci spostiamo quindi
in $(3,2)$ e valutiamo l'unico vertice adiacente rimasto, ovvero
l'incrocio tra $6x_1+5x_2=30$ e $x_2=2$. Questo incrocio è
rappresentato dal punto $(\frac{10}{3},2)$ che ha $z=-\frac{8}{3}$,
valore che ci fa restare in $(3,2)$, che è la soluzione ottima intera
di $P_5$. Per $P_5$ si ha quindi $z_5=-3$, $UB_5=-3$ e
$z_5^*=-6$. Possiamo quindi chiudere questo branch per la prima
regola di fathoming in quanto si ha un upperbound inferiore a
$z^*$. \newpage
Valutiamo infine $P_6$:
\begin{center}
  
  \psscalebox{1.0 1.0} % Change this value to rescale the drawing.
  {
    \begin{pspicture}(0,-2.759881)(6.3223224,2.759881)
      \definecolor{colour1}{rgb}{0.015686275,0.015686275,0.019607844}
      \definecolor{colour4}{rgb}{0.039215688,0.039215688,0.047058824}
      \psline[linecolor=black, linewidth=0.04, arrowsize=0.05291667cm 2.0,arrowlength=1.4,arrowinset=0.0]{->}(1.2153643,-2.3470771)(6.4153643,-2.3470771)
      \psline[linecolor=black, linewidth=0.04](1.6153644,-1.5470772)(5.6153646,0.052922823)
      \rput[bl](6.015364,-2.7470772){$x_1$}
      \rput[bl](1.2153643,2.4529228){$x_2$}
      \psline[linecolor=black, linewidth=0.04](4.015364,-2.7470772)(0.01536438,2.0529227)
      \psdots[linecolor=black, dotsize=0.1](3.6553643,-2.3470771)
      \psdots[linecolor=black, dotsize=0.1](1.6353644,0.11292282)
      \psdots[linecolor=black, dotsize=0.1](1.6153644,-1.5470772)
      \psdots[linecolor=black, dotsize=0.1](2.6753645,-1.1470772)
      \psline[linecolor=black, linewidth=0.04, arrowsize=0.05291667cm 2.0,arrowlength=1.4,arrowinset=0.0]{->}(1.6153644,-2.7470772)(1.6153644,2.4529228)(1.6153644,2.4529228)(1.6153644,2.852923)
      \rput[bl](3.5753644,-2.7470772){5}
      \rput[bl](1.3153644,-1.4470772){2}
      \rput[bl](1.3153644,-0.04707718){6}
      \rput[bl](3.1553643,-2.7270772){\textcolor{colour1}{4}}
      \rput[bl](2.7353644,-2.7270772){3}
      \rput[bl](2.3553643,-2.7270772){2}
      \rput[bl](1.9753643,-2.7470772){1}
      \rput[bl](1.3553643,-2.667077){0}
      \rput[bl](1.3553643,-2.0270772){\textcolor{colour1}{1}}
      \psline[linecolor=colour4, linewidth=0.04](2.8153644,-2.3470771)(2.8153644,2.0529227)
      \psline[linecolor=colour4, linewidth=0.04](1.6353644,-1.1670772)(5.2353644,-1.1670772)
    \end{pspicture}
  }

\end{center}
Come si vede non si hanno soluzioni ammissibili (non si hanno punti
nella regione ammissibile) e quindi il branch
viene chiuso per la seconda regola di fathoming.\\
Siamo quindi giunti alla conclusione che il problema lineare intero
ha:
\begin{shaded}
  punto di minimo in $(0,2)$ e minimo pari a $z=-6$
\end{shaded}
Graficamente si avrebbe:
\begin{center}
  \begin{tikzpicture}[shorten >=1pt,node distance=4cm,on grid,auto]
    \node[state] (q_0) [text width=2cm] {$P_{0}$:
      $z_0=-\frac{13}{2}$, $UB_0=-6$, $z_0^*=-\infty$};  
    \node[state] (q_1) [below left=of q_0, text width=2cm] {$P_{1}$:
      $z_1=-\frac{32}{5}$, $UB_1=-6$, $z_0^*=-\infty$};
    \node[state] (q_2) [below right =of q_0, text width=2cm] {$P_{2}$:
      $z_2=-\frac{21}{5}$, $UB_0=-4$, $z_0^*=-\infty$};
    \node[state, accepting] (q_3) [above left=of q_1, text
    width=2cm] {$P_{3}$: $z_3=-6$, $UB_0=-6$, $z_0^*=-6$, $F3$};
    \node[state] (q_4) [below left =of q_1, text width=2cm] {$P_{4}$:
      $z_4=NIL$, $UB_4=NIL$, $z_4^*=-6$, $F2$};
    \node[state] (q_5) [below left =of q_2, text width=2cm] {$P_{5}$:
      $z_5=-3$, $UB_5=-3$, $z_5^*=-6$, $F1$};
    \node[state] (q_6) [below right =of q_2, text width=2cm] {$P_{6}$:
      $z_6=NIL$, $UB_6=NIL$, $z_6^*=-6$, $F2$};
    \path[->]
    (q_0) edge  node {} (q_1)
    (q_0) edge  node {} (q_2)
    (q_1) edge  node {} (q_3)
    (q_1) edge  node {} (q_4)
    (q_2) edge  node {} (q_5)
    (q_2) edge  node {} (q_6);
  \end{tikzpicture}
\end{center}
\chapter{Esercizio 2}
\section{Parte a}
Dobbiamo applicare un'iterazione del metodo del gradiente effettuando
la line-search in modo esatto, a partire dal punto $A^T=(-1,4)$ sul
problema di minimizzazione non lineare:
\[\min f(x_1,x_2)=2x_1^2+x_1x_2+2(x_2-3)^2\]
Iniziamo quindi definendo $x^0=x^A$, $k=0$, $\varepsilon_1=0.01$ e
$\varepsilon_2=0.1$.\\
Innazitutto calcolo il gradiente della funzione. Calcolo quindi la
derivata parziali:
\[\frac{\partial f(x_1,x_2)}{\partial x_1}=4x_1+x_2\]
\[\frac{\partial f(x_1,x_2)}{\partial x_2}=x_1+4(x_2-3)\]
Ottenendo quindi:
\[\nabla f(x_1,x_2)=[
  \begin{matrix}
    4x_1+x_2 & x_1+4(x_2-3)
  \end{matrix}]
\]
Studio quindi la direzione di discesa studiando il gradiente nel punto
iniziale $x^0=x^A$:
\[d^0=-\nabla f(x^0)=-[
  \begin{matrix}
    4(-1)+(4) & (-1)+4((4)-3)
  \end{matrix}]=[
  \begin{matrix}
    0 & -3
  \end{matrix}]
\]
Seguendo il metodo del gradiente cerchiamo ora il punto
$x^1$. Sappiamo che:
\[x^{k+1}=x^k+\alpha^k d^k\]
quindi:
\[x^1=[
  \begin{matrix}
    -1 & 4
  \end{matrix}]+\alpha^0[
  \begin{matrix}
    0 & -3
  \end{matrix}]\Longrightarrow x^1=[
  \begin{matrix}
    -1 & 4-3\alpha^0
  \end{matrix}]
\]
\newpage
Ma bisogna calcolare $\alpha^0$. Calcolo quindi il minimo della
funzione $f$ lungo $d^0$, ovvero calcolo $f(x_1)$:
\[f(x^1)=2(-1)^2+(-1)(4-3\alpha^0)+2((4-3\alpha^0)-3)^2\]
\[=2-4+3\alpha^0+2(-3\alpha^0+1)^2\]
\[=-2+3\alpha^0+2(9\alpha^{{0}^2}-6\alpha^0+1)\]
\[=18\alpha^{{0}^2}-9\alpha^0\]
ma per avere un punto di minimo ci serve:
\[\frac{d f(x^1)}{d\alpha^0}=0\]
quindi la derivata di $18\alpha^{{0}^2}-9\alpha^0$ nulla, ovvero
otteniamo come punto di minimo:
\[36\alpha^0-9=0\Longrightarrow\alpha^0=\frac{1}{4}\]
Possiamo quindi calcolare effettivamente $x^1$:
\[x^1=[
  \begin{matrix}
    -1 & 4
  \end{matrix}]+\frac{1}{4}[
  \begin{matrix}
    0 & -3
  \end{matrix}]\Longrightarrow x^1=[
  \begin{matrix}
    -1 & \frac{13}{4}
  \end{matrix}]
\]
Procediamo ora con la verifica dei 2 criteri d'arresto. Innazitutto
valutiamo la funzione nei punti $x^0$ e $x^1$:
\[f(x^0)=2(-1)^2+(-1)(4)+2((4)-3)^2= 2-4+2=0\]
\[f(x^1)=2(-1)^2+(-1)(\frac{13}{4})+2((\frac{13}{4})-3)^2=2-\frac{13}{4}
  +\frac{1}{8}=-\frac{9}{8}\]
calcolo inoltre il gradiente in $x^1$:
\[\nabla f(x^1)=[\begin{matrix}
    -\frac{3}{4} & 0
  \end{matrix}]\]
Verifico quindi i due criteri:
\begin{enumerate}
  \item verifichiamo $\vert f(x^{k+1}) - f(x^k)\vert < \varepsilon_1$:
  \[\vert -\frac{9}{8}-0\vert <0.01\]
  ma $\frac{9}{8}\not < 0.01$ quindi il criterio d'arresto non è
  verificato
  \item verifichiamo $\Vert \nabla f(x^1)\Vert< 0.1$:
  \[\sqrt{\frac{9}{16}+0}=\frac{3}{4}\]
  ma $\frac{3}{4}\not < 0.1$ quindi il criterio d'arresto non è
  verificato
\end{enumerate}
\begin{shaded}
  Possiamo dire che con una sola iterazione non si riesce a calcolare
  una soluzione ottima del problema
\end{shaded}
\section{Parte b}
Dobbiamo applicare un'iterazione del metodo di Newton a partire dal
punto $A^T=(-1,4)$ sul problema di minimizzazione non lineare:
\[\min f(x_1,x_2)=2x_1^2+x_1x_2+2(x_2-3)^2\]
Notiamo innazitutto che abbiamo a che fare con una funzione quadratica
quindi il metodo convergerà con una sola iterazione.\\
Iniziamo quindi definendo $x^0=x^A$, $k=0$, $\varepsilon_1=0.01$ e
$\varepsilon_2=0.1$.\\
Innazitutto calcolo il gradiente della funzione. Calcolo quindi la
derivata parziali:
\[\frac{\partial f(x_1,x_2)}{\partial x_1}=4x_1+x_2\]
\[\frac{\partial f(x_1,x_2)}{\partial x_2}=x_1+4(x_2-3)\]
Ottenendo quindi:
\[\nabla f(x_1,x_2)=[
  \begin{matrix}
    4x_1+x_2 & x_1+4(x_2-3)
  \end{matrix}]
\]
Studio poi il gradiente nel punto iniziale $x^0=x^A$:
\[\nabla f(x^0)=[
  \begin{matrix}
    4(-1)+(4) & (-1)+4((4)-3)
  \end{matrix}]=[
  \begin{matrix}
    0 & 3
  \end{matrix}]
\]
Procedo poi col calcolo della matrice Hessiana. Calcolo quindi le
derivate parziali seconde:
\[\frac{\partial f(x_1,x_2)}{\partial x_1x_1}=4,\,\,\,\,\frac{\partial
    f(x_1,x_2)}{\partial x_1x_2}=1\] 
\[\frac{\partial f(x_1,x_2)}{\partial x_2x_1}=1,\,\,\,\,\frac{\partial
    f(x_1,x_2)}{\partial x_2x_2}=4\] 
Ottengo quindi la mia matrice Hessiana per un generico punto:
\[H_f(x)=\left[
    \begin{matrix}
      4 & 1\\
      1 & 4
    \end{matrix}
  \right]
\]
\newpage
Procedo quindi col calcolo dell'inversa della matrice. Calcolo
innazitutto il determinante che sarà:
\[det(H_f(x))=(4\cdot 4)-(1\cdot 1)=15\]
procedo poi col calcolo dei cofattori:
\[c_{11}=(-1)^{1+1}\cdot 4 =4\]
\[c_{12}=(-1)^{1+2}\cdot 1 =-1\]
\[c_{21}=(-1)^{2+1}\cdot 1 =-1\]
\[c_{22}=(-1)^{2+2}\cdot 4 =4\]
Sappiamo quindi che la matrice inversa è la matrice dei cofattori,
trasposta, divisa per il determinante:
\[H_f(x)^{-1}=\frac{1}{15}\left[
    \begin{matrix}
      4 & -1\\
      -1 & 4
    \end{matrix}
  \right]^T=\frac{1}{15}\left[
    \begin{matrix}
      4 & -1\\
      -1 & 4
    \end{matrix}
  \right]
\]
\[=\left[
    \begin{matrix}
      \frac{4}{15} & - \frac{1}{15}\\
      - \frac{1}{15} &  \frac{4}{15}
    \end{matrix}\right]
\]
Sappiamo ora che il punto $x^{k+1}$ è dato da:
\[x^{k+1}=x^k-H_f(x^k)^{-1}\cdot\nabla f(x^k)\]
quindi, nel nostro caso:
\[x^{1}=x^0-H_f(x^0)^{-1}\cdot\nabla f(x^0)\]
ovvero:
\[x^1=\left[
    \begin{matrix}
      -1 \\
      4 
    \end{matrix}
  \right]-\left[
    \begin{matrix}
      \frac{4}{15} & - \frac{1}{15}\\
      - \frac{1}{15} &  \frac{4}{15}
    \end{matrix}\right]\cdot \left[
  \begin{matrix}
    0\\
    3
  \end{matrix}\right]\]
Effettuo il prodotto riga per colonna tra l'Hessiana e il gradiente
nel punto iniziale:
\[\left[
    \begin{matrix}
      \frac{4}{15} & - \frac{1}{15}\\
      - \frac{1}{15} &  \frac{4}{15}
    \end{matrix}\right]\cdot \left[
    \begin{matrix}
      0\\
      3
    \end{matrix}\right]=\left[
    \begin{matrix}
      0-\frac{1}{5}\\
      0+\frac{4}{5}
    \end{matrix}
  \right]=\left[
    \begin{matrix}
      -\frac{1}{5}\\
      \frac{4}{5}
    \end{matrix}
  \right]\]
ed effettuo la sottrazione:
\[x^1=\left[
    \begin{matrix}
      -1 \\
      4 
    \end{matrix}
  \right]-\left[
    \begin{matrix}
      -\frac{1}{5}\\
      \frac{4}{5}
    \end{matrix}
  \right]=\left[
    \begin{matrix}
      -1+\frac{1}{5}\\
      4-\frac{4}{5}
    \end{matrix}
  \right]=\left[
    \begin{matrix}
      -\frac{4}{5}\\
      \frac{16}{5}
    \end{matrix}
  \right]
\]
Ho quindi trovato il punto $x^1$.\\
Calcolo il gradiente in $x^1$:
\[\nabla f(x^1)=[
  \begin{matrix}
    4(-\frac{4}{5})+\frac{16}{5} & -\frac{4}{5}+4(\frac{16}{5}-3)
  \end{matrix}]=[
  \begin{matrix}
    0 & 0
  \end{matrix}]
\]
Sappiamo quindi che è un punto di ottimo.\\
Calcolo ora gli autovalori della matrice Hessiana, scrivo quindi la
matrice:
\[\left[
    \begin{matrix}
      4-\lambda & 1\\
      1 & 4-\lambda
    \end{matrix}
  \right]
\]
ne calcolo il determinante:
\[d=(4-\lambda^2)-1\]
per trovare gli autovalori voglio $d=0$ quindi:
\[(4-\lambda^2)-1=0\to
  \begin{cases}
    \lambda = 3\\
    \lambda = 5
  \end{cases}
\]
essendo entrambi positivi stiamo valutando un punto di minimo.\\
\begin{shaded}
  Il punto di minimo è $x^1=\left(-\frac{4}{5}, \frac{16}{5}\right)$
\end{shaded}
\end{document}