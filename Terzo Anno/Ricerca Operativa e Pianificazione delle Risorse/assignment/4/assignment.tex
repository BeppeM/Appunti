\documentclass[a4paper,12pt, oneside]{book}

% \usepackage{fullpage}
\usepackage[italian]{babel}
\usepackage[utf8]{inputenc}
\usepackage{amssymb}
\usepackage{amsthm}
\usepackage{graphics}
\usepackage{amsfonts}
\usepackage{listings}
\usepackage{amsmath}
\usepackage{amstext}
\usepackage{engrec}
\usepackage{rotating}
\usepackage[safe,extra]{tipa}
\usepackage{showkeys}
\usepackage{multirow}
\usepackage{hyperref}
\usepackage{mathtools}
\usepackage{microtype}
\usepackage{enumerate}
\usepackage{braket}
\usepackage{marginnote}
\usepackage{pgfplots}
\usepackage{cancel}
\usepackage{polynom}
\usepackage{booktabs}
\usepackage{enumitem}
\usepackage{framed}
\usepackage{algorithm}
\usepackage{algpseudocode}
\usepackage{pdfpages}
\usepackage{pgfplots}
\usepackage[cache=false]{minted}

\usepackage[usenames,dvipsnames]{pstricks}
\usepackage{epsfig}
\usepackage{pst-grad} % For gradients
\usepackage{pst-plot} % For axes
\usepackage[space]{grffile} % For spaces in paths
\usepackage{etoolbox} % For spaces in paths
\makeatletter % For spaces in paths
\patchcmd\Gread@eps{\@inputcheck#1 }{\@inputcheck"#1"\relax}{}{}
\makeatother

\usepackage{tikz}\usetikzlibrary{er}\tikzset{multi  attribute /.style={attribute ,double  distance =1.5pt}}\tikzset{derived  attribute /.style={attribute ,dashed}}\tikzset{total /.style={double  distance =1.5pt}}\tikzset{every  entity /.style={draw=orange , fill=orange!20}}\tikzset{every  attribute /.style={draw=MediumPurple1, fill=MediumPurple1!20}}\tikzset{every  relationship /.style={draw=Chartreuse2, fill=Chartreuse2!20}}\newcommand{\key}[1]{\underline{#1}}

\usepackage{fancyhdr}
\pagestyle{fancy}
\fancyhead[LE,RO]{\slshape \rightmark}
\fancyhead[LO,RE]{\slshape \leftmark}
\fancyfoot[C]{\thepage}
\usepackage{tikz}
\usetikzlibrary{automata,positioning}


\title{Assignment 4}
\author{Davide Cozzi, 829827}
\date{}

\pgfplotsset{compat=1.13}
\begin{document}
\maketitle

\definecolor{shadecolor}{gray}{0.80}
\setlist{leftmargin = 2cm}
\newtheorem{teorema}{Teorema}
\newtheorem{definizione}{Definizione}
\newtheorem{esempio}{Esempio}
\newtheorem{corollario}{Corollario}
\newtheorem{lemma}{Lemma}
\newtheorem{osservazione}{Osservazione}
\newtheorem{nota}{Nota}
\newtheorem{esercizio}{Esercizio}

\renewcommand{\chaptermark}[1]{%
  \markboth{\chaptername
    \ \thechapter.\ #1}{}}
\renewcommand{\sectionmark}[1]{\markright{\thesection.\ #1}}
\chapter{Esercizio 1}
Si ha la funzione:
\[f(x,y,z)=3x+4y+11\]
col dominio:
\[M=\{(x,y,z)\in\mathbb{R}^3|\,\,(x-2)^2+(y-3)^2+(z-4)^2=1\}\]
definisco la funzione $g(x,y,z)$ come la funzione rappresentante il dominio:
\[g(x,y,z)=(x-2)^2+(y-3)^2+(z-4)^2-1\]
so che il punto $(x_0,y_0,z_0)$ candidato ad essere un punto di
estremo relativo per $f$ su $M$ deve rispettare due condizioni:
\begin{enumerate}
  \item $g(x_0,y_0,z_0)=0$
  \item $\nabla g(x_0,y_0,z_0)\neq 0$
\end{enumerate}
procedo quindi con lo studio della Lagrangiana, ricordando di avere a
che fare con un solo vincolo. Ricordando che la Lagrangiana è della
forma:
\[L(x_1,\ldots,x_n,\lambda_1,\ldots,\lambda_n)=f(x_1,\ldots,x_n)+\sum_{i=1}^m
  \lambda_ig(x_1,\ldots x_n)\]
con $m$ numero di vincoli, si ha, nel nostro caso, che:
\[L(x,y,z,\lambda)=f(x,y,z)+\lambda\cdot g(x,y,z)\]
\[=3x+4y+11+\lambda[(x-2)^2+(y-3)^2+(z-4)^2-1]\]
cerco quindi i punti stazionari della Lagrangiana, ovvero quei punti
$(x_0,y_0,z_0)$ tali che:
\[\nabla L(x_0,y_0,z_o,\lambda_0)=0\]
calcolo quindi il gradiente delle Lagrangiana:
\[\left[
    \begin{matrix}
      \frac{\partial L}{\partial x}\\
      \frac{\partial L}{\partial y}\\
      \frac{\partial L}{\partial z}\\
      \frac{\partial L}{\partial \lambda}\\
    \end{matrix}
  \right]=\left[
    \begin{matrix}
      3+\lambda\cdot [2(x-2)]\\
      4+\lambda\cdot [2(y-3)]\\
      \lambda\cdot [2(z-4)]\\
      (x-2)^2+(y-3)^2+(z-4)^2-1
    \end{matrix}
  \right]=\left[
    \begin{matrix}
      3+2\lambda x-4\lambda\\
      4+2\lambda y-6\lambda\\
      2\lambda z-8\lambda\\
      (x-2)^2+(y-3)^2+(z-4)^2-1
    \end{matrix}
  \right]
\]
per avere nullo il gradiente devo quindi risolvere:
\[
  \begin{cases}
    3+2\lambda x-4\lambda=0\\
    4+2\lambda y-6\lambda=0\\
    2\lambda z-8\lambda=0\\
    (x-2)^2+(y-3)^2+(z-4)^2-1=0
  \end{cases}
\]
risolviamo quindi il sistema di equazioni:
\[
  \begin{cases}
    3+2\lambda x-4\lambda=0\\
    4+2\lambda y-6\lambda=0\\
    2\lambda z-8\lambda=0\\
    (x-2)^2+(y-3)^2+(z-4)^2-1=0
  \end{cases}\to
  \begin{cases}
    3+2\lambda x-4\lambda=0\\
    2+\lambda y-3\lambda=0\\
    2z-8=0\\
    (x-2)^2+(y-3)^2+(z-4)^2-1=0
  \end{cases}
\]
\[
  \to\begin{cases}
    3+2\lambda x-4\lambda=0\\
    2+\lambda y-3\lambda=0\\
    z=4\\
    (x-2)^2+(y-3)^2+(z-4)^2-1=0
  \end{cases}\to
  \begin{cases}
    2\lambda x=4\lambda-3\\
    \lambda y=3\lambda-2\\
    2z-8=0\\
    (x-2)^2+(y-3)^2+(z-4)^2-1=0
  \end{cases}
\]
\[
  \to\begin{cases}
    x=2-\frac{3}{2\lambda}\\
    y=3-\frac{2}{\lambda}\\
    z=4\\
    (x-2)^2+(y-3)^2+(z-4)^2-1=0
  \end{cases}
  \to\begin{cases}
    x=2-\frac{3}{2\lambda}\\
    y=3-\frac{2}{\lambda}\\
    z=4\\
    (\frac{3}{2\lambda})^2+(\frac{2}{\lambda})^2-1=0
  \end{cases}
\]
\[
\to
  \begin{cases}
    x=2-\frac{3}{2\lambda}\\
    y=3-\frac{2}{\lambda}\\
    z=4\\
    \frac{9}{4\lambda^4}+\frac{4}{\lambda^2}=1
  \end{cases}\to
  \begin{cases}
    x=2-\frac{3}{2\lambda}\\
    y=3-\frac{2}{\lambda}\\
    z=4\\
    \frac{25}{4\lambda^4}=1
  \end{cases}\to
  \begin{cases}
    x=2-\frac{3}{2\lambda}\\
    y=3-\frac{2}{\lambda}\\
    z=4\\
    \lambda^2=\frac{25}{4}
  \end{cases}
  \to
  \begin{cases}
    x=2-\frac{3}{2\lambda}\\
    y=3-\frac{2}{\lambda}\\
    z=4\\
    \lambda=\pm\frac{5}{2}
  \end{cases}
\]
\newpage
Abbiamo quindi due casi:
\begin{enumerate}
  \item $\lambda=\frac{5}{2}$ da cui si ricavano:
  \begin{itemize}
    \item $x=2-\frac{3}{2\lambda}\to x=2-\frac{3}{5}\to
    x=\frac{7}{5}$
    \item $y=3-\frac{2}{\lambda}\to y=3-\frac{4}{5}
    \to y=\frac{11}{5}$ 
  \end{itemize}
  avendo sempre $z=4$ abbiamo trovato il primo punto:
  \[\left(\frac{7}{5}, \frac{11}{5}, 4\right)\]
  
  \item $\lambda=-\frac{5}{2}$ da cui si ricavano:
  \begin{itemize}
    \item $x=2-\frac{3}{2\lambda}\to x=2+\frac{3}{5}\to
    x=\frac{13}{5}$
    \item $y=3-\frac{2}{\lambda}\to y=3+\frac{4}{5}
    \to y=\frac{19}{5}$ 
  \end{itemize}
  avendo sempre $z=4$ abbiamo trovato il secondo punto:
  \[\left(\frac{13}{5}, \frac{19}{5}, 4\right)\]
\end{enumerate}
Non ci resta che valutare la funzione $f$ nei due punti per stabilire
se sono punti di massimo e minimo:
\begin{itemize}
  \item $f\left(\frac{7}{5}, \frac{11}{5},
    4\right)=\frac{21}{5}+\frac{44}{5}+11=24$
  \item $f\left(\frac{13}{5}, \frac{19}{5},
    4\right)=\frac{39}{5}+\frac{76}{5}+11=34$
\end{itemize}
Si ha quindi che:
\begin{shaded}
  Il punto $\left(\frac{7}{5}, \frac{11}{5}, 4\right)$ \textbf{è un
    punto di minimo} mentre il punto $[\left(\frac{13}{5}, \frac{19}{5},
    4\right)$ \textbf{è un punto di massimo}
\end{shaded}
\chapter{Esercizio 2}
\section{Parte Obbligatoria}
Si ha la funzione su cui cercare gli eventuali punti candidati ad
essere punti di minimo:
\[f(x_1,x_2)=x_1\]
soggetta ai seguenti vincoli:
\[(x_1-4)^2+x_2^2\leq 16\]
\[(x_1-3)^2+(x_2-2)^2=13\]
Abbiamo quindi un vincolo di disuguaglianza e uno di uguaglianza:
\begin{itemize}
  \item $ (x_1-4)^2+x_2^2-16\leq 0$ \textbf{vincolo di disuguaglianza}
  e chiamo $h$ la funzione:
  \[h(x_1,x_2)= (x_1-4)^2+x_2^2-16\]
  per l'\textbf{ammissibilità primale} (una delle condizioni di \textit{KKT})
  \item $(x_1-3)^2+(x_2-2)^2-13=0$ \textbf{vincolo di uguaglianza} e
  chiamo $g$ la funzione:
  \[g(x_1,x_2)=(x_1-3)^2+(x_2-2)^2-13\]
  per l'\textbf{ammissibilità primale} (una delle condizioni di \textit{KKT})
\end{itemize}
\newpage
Procedo ora calcolando i gradienti di $f$, $h$ e $g$:
\begin{itemize}
  \item $\nabla f(x_1,x_2)=\left[
    \begin{matrix}
      1\\
      0
    \end{matrix}
  \right]$
  \item $\nabla h(x_1,x_2)=\left[
    \begin{matrix}
      2(x-4)\\
      2x_2
    \end{matrix}
  \right]$
  \item $\nabla g(x_1,x_2)=\left[
    \begin{matrix}
      2(x_1-3)\\
      2(x_2-2)
    \end{matrix}
  \right]$
\end{itemize}
Imposto quindi la \textbf{condizione di stazionarietà per problemi di
  minimo}, avendo un moltiplicatore $\mu$ (uno solo avendo un solo
vincolo di disuguaglianza, $p=1$) e un moltiplicatore $\lambda$ (uno solo
avendo un solo vincolo di uguaglianza, $m=1$):
\[\nabla f(x^*)+\sum_{i=1}^p\mu_i^*\cdot
  h_i(x^*)+\sum_{j=1}^m\lambda_j^*\cdot g_j(x^*)=0\]
\[\Downarrow\]
\[\nabla f(x_1,x_2)+\mu\cdot\nabla h(x_1,x_2)+\lambda\cdot\nabla
  g(x_1,x_2)=0\]
\[\Downarrow\]
\[\left[
    \begin{matrix}
      1\\
      0
    \end{matrix}
  \right]+\mu\cdot \left[
    \begin{matrix}
      2(x-4)\\
      2x_2
    \end{matrix}
  \right]+\lambda\cdot \left[
    \begin{matrix}
      2(x_1-3)\\
      2(x_2-2)
    \end{matrix}
  \right]=0
\]
\[\Downarrow\]
\[
  \begin{cases}
    1+2\mu(x_1-4)+2\lambda(x_1-3)=0\\
    \mu(2x_2)+2\lambda(x_2-2)=0
  \end{cases}
\]
ricordiamo anche le altre 2 condizioni di \textit{KKT} rimanenti:
\begin{enumerate}
  \item \textbf{complementarietà}: $\mu\cdot h(x^*)=0$
  \item \textbf{ammissibilità duale}: $\mu\geq 0$
\end{enumerate}
Sappiamo che il vincolo di uguaglianza è sempre attivo, mentre quello
di disuguaglianza può essere attivo ($h(x^*)=0$) o inattivo
($h(x^*)<0$). Per far valere la condizione di complementarietà quindi
nel caso il vincolo sia attivo bisognerà verificare la validità
dell'ammissibilità duale (che comporterà la validità della
complementarietà). Nel caso il vincolo non sia attivo impongo a
priori $\mu=0$, così si ha la validità dell'ammissibilità duale e
anche della complementarietà.
\newpage
\subsection{Vincolo di disuguaglianza inattivo}
Imposto un sistema dove devono essere valide le condizioni di
\textit{KKT}, ricordando l'imposizione di $\mu=0$: 
\[
  \begin{cases}
     1+2\mu(x_1-4)+2\lambda(x_1-3)=0\\
     \mu(2x_2)+2\lambda(x_2-2)=0\\
     \mu=0\\
     \mu[(x_1-4)^2+x_2^2-16]=0\\
     (x_1-4)^2+x_2^2-16 < 0\\
     (x_1-3)^2+(x_2-2)^2-13=0
  \end{cases}
\]
risolvo il sistema:
\[
  \begin{cases}
    1+2\mu(x_1-4)+2\lambda(x_1-3)=0\\
    \mu(2x_2)+2\lambda(x_2-2)=0\\
    \mu=0\\
    \mu[(x_1-4^2)+x_2^2-16]=0\\
    (x_1-4)^2+x_2^2-16 < 0\\
    (x_1-3)^2+(x_2-2)^2-13=0
  \end{cases}\to\begin{cases}
    1+2\lambda(x_1-3)=0\\
    2\lambda(x_2-2)=0\\
    \mu=0\\
    0=0\\
    (x_1-4)^2+x_2^2-16 < 0\\
    (x_1-3)^2+(x_2-2)^2-13=0
  \end{cases}
\]
\[
  \to
  \begin{cases}
    1+2\lambda x_1-6\lambda=0\\
    2\lambda x_2-4\lambda=0\\
    \mu=0\\
    0=0\\
    (x_1-4)^2+x_2^2-16 < 0\\
    (x_1-3)^2+(x_2-2)^2-13=0
  \end{cases}\to
  \begin{cases}
    2\lambda x_1=6\lambda-1\\
    2\lambda x_2=4\lambda\\
    \mu=0\\
    0=0\\
    (x_1-4)^2+x_2^2-16 < 0\\
    (x_1-3)^2+(x_2-2)^2-13=0
  \end{cases}
\]
\[\to
  \begin{cases}
    x_1=3-\frac{1}{2\lambda}\\
    x_2=2\\
    \mu=0\\
    0=0\\
    (x_1-4)^2+x_2^2-16 < 0\\
    (x_1-3)^2+(x_2-2)^2-13=0
  \end{cases}\to\begin{cases}
    x_1=3-\frac{1}{2\lambda}\\
    x_2=2\\
    \mu=0\\
    0=0\\
    (-\frac{1}{2\lambda}-1)^2+4-16 < 0\\
    (\frac{1}{2\lambda})^2-13=0
  \end{cases}
\]
\[\to
  \begin{cases}
    x_1=3-\frac{1}{2\lambda}\\
    x_2=2\\
    \mu=0\\
    0=0\\
    (-\frac{1}{2\lambda}-1)^2-12 < 0\\
    \frac{1}{4\lambda^2}=13
  \end{cases}\to\begin{cases}
    x_1=3-\frac{1}{2\lambda}\\
    x_2=2\\
    \mu=0\\
    0=0\\
    (-\frac{1}{2\lambda}-1)^2-12 < 0\\
    52\lambda^2=1
  \end{cases}\to\begin{cases}
    x_1=3-\frac{1}{2\lambda}\\
    x_2=2\\
    \mu=0\\
    0=0\\
    (-\frac{1}{2\lambda}-1)^2-12 < 0\\
    \lambda =\pm \frac{\sqrt{13}}{26}
  \end{cases}
\]
abbiamo quindi 2 alternative:
\begin{enumerate}
  \item $\lambda = \frac{\sqrt{13}}{26}$ quindi:
  \begin{itemize}
    \item $x_1=3-\frac{1}{2\cdot \frac{\sqrt{13}}{26}}\to
    x_1=3-\frac{1}{\frac{\sqrt{13}}{13}}\to x_1=3-\sqrt{13}$
    \item
    $h(x_1,x_2)=(-\frac{1}{2\frac{\sqrt{13}}{26}}-1)^2-12=(-\sqrt{13}-1)^2-12=
    2+2\sqrt{13}$ \\(infatti la penultima equazione deriva da
    $h(x_1,x_2)=(x_1-4)^2+x_2^2-16$) 
  \end{itemize}
  ma $h(x_1,x_2)=2+2\sqrt{13}\geq 0$ quindi la condizione di
  ammissibilità primale $h(x^*)\leq 0$ non è rispettata, quindi il
  punto $(3-\sqrt{13},2)$ \textit{non è ammissibile}
  
  \item $\lambda = -\frac{\sqrt{13}}{26}$ quindi:
  \begin{itemize}
    \item $x_1=3-\frac{1}{-2\cdot \frac{\sqrt{13}}{26}}\to
    x_1=3+\frac{1}{\frac{\sqrt{13}}{13}}\to x_1=3+\sqrt{13}$
    \item
    $h(x_1,x_2)=(-\frac{1}{-2\frac{\sqrt{13}}{26}}-1)^2-12=(\sqrt{13}-1)^2-12=
    2-2\sqrt{13}$ \\(infatti la penultima equazione deriva da
    $h(x_1,x_2)=(x_1-4)^2+x_2^2-16$) 
  \end{itemize}
  in questo caso si ha che $h(x_1,x_2)=2-2\sqrt{13}< 0$, quindi la
  dondizione di ammissibilità primale è rispettata. La quarta
  equazione ci conferma che anche l'altra condizione di ammissibilità
  è rispettata $g(x_1,x_2)=0$. Inoltre le condizioni di
  complementarietà e ammissibilità duale sono rispettate dal fatto che
  è stato imposto il vincolo di disuguaglianza inattivo mettendo
  $\mu=0$. \\
 \textbf{ Quindi il punto} $(3+\sqrt{13},2)$,\textbf{ con}
  $f(3+\sqrt{13},2)=3+\sqrt{13}$\textbf{ è un }\textbf{candidato punto di
    minimo} 
\end{enumerate}
\subsection{Vincolo di disuguaglianza attivo}
Imposto un sistema dove devono essere valide le condizioni di
\textit{KKT}, rciordando di avere il vincolo di disuguaglianza attivo:
\[
  \begin{cases}
     1+2\mu(x_1-4)+2\lambda(x_1-3)=0\\
     \mu(2x_2)+2\lambda(x_2-2)=0\\
     \mu\geq 0\\
     \mu[(x_1-4)^2+x_2^2-16]=0\\
     (x_1-4)^2+x_2^2-16 = 0\\
     (x_1-3)^2+(x_2-2)^2-13=0
  \end{cases}
\]
risolvo il sistema:
\[
  \begin{cases}
    1+2\mu(x_1-4)+2\lambda(x_1-3)=0\\
    \mu(2x_2)+2\lambda(x_2-2)=0\\
    \mu\geq 0\\
    \mu[(x_1-4)^2+x_2^2-16]=0\\
    (x_1-4)^2+x_2^2-16 = 0\\
    (x_1-3)^2+(x_2-2)^2-13=0
  \end{cases}\to
  \begin{cases}
    1+2\mu(x_1-4)+2\lambda(x_1-3)=0\\
    \mu(2x_2)+2\lambda(x_2-2)=0\\
    \mu\geq 0\\
    \mu[(x_1-4)^2+x_2^2-16]=0\\
    (x_1-4)^2+x_2^2-16 = 0\\
    (x_1-3)^2+(x_2-2)^2-13=0
  \end{cases}
\]
\[\to
  \begin{cases}
    1+2\mu(x_1-4)+2\lambda(x_1-3)=0\\
    \mu(2x_2)+2\lambda(x_2-2)=0\\
    \mu\geq 0\\
    \mu[(x_1-4)^2+x_2^2-16]=0\\
    x_1^2-8x_1+16+x_2^2-16 = 0\\
    (x_1-3)^2+(x_2-2)^2-13=0
  \end{cases}\to
  \begin{cases}
    1+2\mu(x_1-4)+2\lambda(x_1-3)=0\\
    \mu(2x_2)+2\lambda(x_2-2)=0\\
    \mu\geq 0\\
    \mu[(x_1-4)^2+x_2^2-16]=0\\
    x_2^2=-x_1^2+8x_1 \\
    (x_1-3)^2+(x_2-2)^2-13=0
  \end{cases}
\]
\[\to
  \begin{cases}
    1+2\mu(x_1-4)+2\lambda(x_1-3)=0\\
    \mu(2x_2)+2\lambda(x_2-2)=0\\
    \mu\geq 0\\
    \mu[(x_1-4)^2+x_2^2-16]=0\\
    x_2^2=-x_1^2+8x_1 \\
    x_1^2-6x_1+9+x_2^2-4x_2+4-13=0
  \end{cases}\to
   \begin{cases}
    1+2\mu(x_1-4)+2\lambda(x_1-3)=0\\
    \mu(2x_2)+2\lambda(x_2-2)=0\\
    \mu\geq 0\\
    \mu[(x_1-4)^2+x_2^2-16]=0\\
    x_2^2=-x_1^2+8x_1 \\
    x_1^2-6x_1+9+-x_1^2+8x_1-4x_2+4-13=0
  \end{cases}
\]
\[\to
  \begin{cases}
    1+2\mu(x_1-4)+2\lambda(x_1-3)=0\\
    \mu(2x_2)+2\lambda(x_2-2)=0\\
    \mu\geq 0\\
    \mu[(x_1-4)^2+x_2^2-16]=0\\
    x_2^2=-x_1^2+8x_1 \\
    2x_1-4x_2=0
  \end{cases}\to
  \begin{cases}
    1+2\mu(x_1-4)+2\lambda(x_1-3)=0\\
    \mu(2x_2)+2\lambda(x_2-2)=0\\
    \mu\geq 0\\
    \mu[(x_1-4)^2+x_2^2-16]=0\\
    x_2^2=-x_1^2+8x_1 \\
    x_1=2x_2
  \end{cases}
\]
\[\to
  \begin{cases}
    1+2\mu(x_1-4)+2\lambda(x_1-3)=0\\
    \mu(2x_2)+2\lambda(x_2-2)=0\\
    \mu\geq 0\\
    \mu[(x_1-4)^2+x_2^2-16]=0\\
    x_2^2=-4x_2^2+16x_2 \\
    x_1=2x_2
  \end{cases}\to
  \begin{cases}
    1+2\mu(x_1-4)+2\lambda(x_1-3)=0\\
    \mu(2x_2)+2\lambda(x_2-2)=0\\
    \mu\geq 0\\
    \mu[(x_1-4)^2+x_2^2-16]=0\\
    5x_2^2=16x_2 \\
    x_1=2x_2
  \end{cases}
\]
Si arriva quindi a poter dire che $x_2$ può valere 0 oppure
$\frac{16}{5}$:
\begin{enumerate}
  \item se $x_2=0$ si ricava che $x_1=2x_2\to x_1=0$
  \item se $x_2=\frac{16}{5}$ si ricava che $x_1=2x_2\to x_1=\frac{32}{5}$
\end{enumerate}
riprendiamo ad analizzare quindi il sistema nei due casi.
\subsubsection{Caso 1}
In questo caso si ha $x_1=0$ e $x_2=0$, riprendiamo quindi a risolvere
il sistema:
\[\begin{cases}
    1+2\mu(x_1-4)+2\lambda(x_1-3)=0\\
    \mu(2x_2)+2\lambda(x_2-2)=0\\
    \mu\geq 0\\
    \mu[(x_1-4)^2+x_2^2-16]=0\\
    x_1=0 \\
    x_2=0
  \end{cases}\to
  \begin{cases}
    1-8\mu-6\lambda=0\\
    -4\lambda=0\\
    \mu\geq 0\\
    \mu[0]=0\\
    x_1=0 \\
    x_2=0
  \end{cases}\to
    \begin{cases}
    \mu=\frac{1}{8}\\
    \lambda=0\\
    \frac{1}{8}\geq 0\\
    0=0\\
    x_1=0 \\
    x_2=0
  \end{cases}
\]
Si hanno tutte le condizioni verificate quindi:\\
\textbf{il punto} $(0,0)$, con $f(0,0)=0$, \textbf{è
  un candidato punto di minimo}

\newpage
\subsection{Caso 2}
In questo caso si ha $x_1=\frac{32}{5}$ e $x_2=\frac{16}{5}$,
riprendiamo quindi a risolvere il sistema:
\[\begin{cases}
    1+2\mu(x_1-4)+2\lambda(x_1-3)=0\\
    \mu(2x_2)+2\lambda(x_2-2)=0\\
    \mu\geq 0\\
    \mu[(x_1-4)^2+x_2^2-16]=0\\
    x_1=\frac{32}{5} \\
    x_2=\frac{16}{5}
  \end{cases}\to
  \begin{cases}
    1+2\mu(\frac{32}{5}-4)+2\lambda(\frac{32}{5}-3)=0\\
    \mu(2\frac{16}{5})+2\lambda(\frac{16}{5}-2)=0\\
    \mu\geq 0\\
    \mu[(\frac{32}{5}-4)^2+\left(\frac{16}{5}\right)^2-16]=0\\
    x_1=\frac{32}{5} \\
    x_2=\frac{16}{5}
  \end{cases}
\]
\[\to
  \begin{cases}
    1+2\mu(\frac{12}{5})+2\lambda(\frac{17}{5})=0\\
    \mu(\frac{32}{5})+2\lambda(\frac{16}{5})=0\\
    \mu\geq 0\\
    \mu[(\frac{12}{5})^2+\left(\frac{16}{5}\right)^2-16]=0\\
    x_1=\frac{32}{5} \\
    x_2=\frac{16}{5}
  \end{cases}\to
  \begin{cases}
    1+\frac{24}{5}\mu+\frac{34}{5}\lambda=0\\
    \frac{32}{5}\mu+\frac{32}{5}\lambda=0\\
    \mu\geq 0\\
    \mu[\frac{144}{25}+\frac{256}{25}-16]=0\\
    x_1=\frac{32}{5} \\
    x_2=\frac{16}{5}
  \end{cases}
\]
\[\to
  \begin{cases}
    1+\frac{24}{5}\mu+\frac{34}{5}\lambda=0\\
    \frac{32}{5}\mu=-\frac{12}{5}\lambda\\
    \mu\geq 0\\
    \mu[16-16]=0\\
    x_1=\frac{32}{5} \\
    x_2=\frac{16}{5}
  \end{cases}\to
  \begin{cases}
    1+\frac{24}{5}\mu+\frac{34}{5}\lambda=0\\
    \mu=-\frac{12}{32}\lambda\\
    \mu\geq 0\\
    0=0\\
    x_1=\frac{32}{5} \\
    x_2=\frac{16}{5}
  \end{cases}\to
  \begin{cases}
    1+\frac{24}{5}\left(-\frac{12}{32}\lambda\right)+\frac{34}{5}\lambda=0\\
    \mu=-\frac{12}{32}\lambda\\
    \mu\geq 0\\
    0=0\\
    x_1=\frac{32}{5} \\
    x_2=\frac{16}{5}
  \end{cases}
\]
\[\to
  \begin{cases}
    1-\frac{9}{5}\lambda+\frac{34}{5}\lambda=0\\
    \mu=-\frac{12}{32}\lambda\\
    \mu\geq 0\\
    0=0\\
    x_1=\frac{32}{5} \\
    x_2=\frac{16}{5}
  \end{cases}\to
  \begin{cases}
    1+5\lambda=0\\
    \mu=-\frac{12}{32}\lambda\\
    \mu\geq 0\\
    0=0\\
    x_1=\frac{32}{5} \\
    x_2=\frac{16}{5}
  \end{cases}\to
  \begin{cases}
    \lambda=-\frac{1}{5}\\
    \mu=-\frac{12}{32}\lambda\\
    \mu\geq 0\\
    0=0\\
    x_1=\frac{32}{5} \\
    x_2=\frac{16}{5}
  \end{cases}\to
  \begin{cases}
    \lambda=-\frac{1}{5}\\
    \mu=\frac{3}{40}\\
    \frac{3}{40}\geq 0\\
    0=0\\
    x_1=\frac{32}{5} \\
    x_2=\frac{16}{5}
  \end{cases}
\]
Si hanno tutte le condizioni verificate quindi:\\
\textbf{il punto} $\left(\frac{32}{5}, \frac{16}{5}\right)$, con
$f\left(\frac{32}{5}, \frac{16}{5}\right)=\frac{32}{5}$, \textbf{è  
  un candidato punto di minimo}
\newpage
\subsubsection{Conclusioni}
\begin{shaded}
  Si sono ottenuti 3 punti candidati ad essere punto di minimo:
  \begin{enumerate}
    \item $(3+\sqrt{13},2)$, con $f(3+\sqrt{13},2)=3+\sqrt{13}$
    \item $(0,0)$, con $f(0,0)=0$
    \item $\left(\frac{32}{5}, \frac{16}{5}\right)$, con
    $f\left(\frac{32}{5}, \frac{16}{5}\right)=\frac{32}{5}$ 
  \end{enumerate}
  Grazie alle valutazioni della funzione nei punti si ha che il punto
  $(0,0)$ è \textbf{punto di minimo}
\end{shaded}
\section{Parte Facoltativa}
\textit{Essendo la maggior parte dei conti identici alla parte
  obbligatoria alcuni di essi verranno omessi per praticità.}\\
Questa volta ci occupiamo di ricercare i candidati punti di
massimo. \\
Imposto quindi la \textbf{condizione di stazionarietà per problemi di
  massimo}, avendo un moltiplicatore $\mu$ (uno solo avendo un solo
vincolo di disuguaglianza, $p=1$) e un moltiplicatore $\lambda$ (uno solo
avendo un solo vincolo di uguaglianza, $m=1$):
\[\nabla f(x^*)-\sum_{i=1}^p\mu_i^*\cdot
  h_i(x^*)-\sum_{j=1}^m\lambda_j^*\cdot g_j(x^*)=0\]
\[\Downarrow\]
\[\nabla f(x_1,x_2)-\mu\cdot\nabla h(x_1,x_2)-\lambda\cdot\nabla
  g(x_1,x_2)=0\]
\[\Downarrow\]
\[\left[
    \begin{matrix}
      1\\
      0
    \end{matrix}
  \right]-\mu\cdot \left[
    \begin{matrix}
      2(x-4)\\
      2x_2
    \end{matrix}
  \right]-\lambda\cdot \left[
    \begin{matrix}
      2(x_1-3)\\
      2(x_2-2)
    \end{matrix}
  \right]=0
\]
\[\Downarrow\]
\[
  \begin{cases}
    1-2\mu(x_1-4)+-\lambda(x_1-3)=0\\
    -\mu(2x_2)-2\lambda(x_2-2)=0
  \end{cases}
\]
ricordiamo anche le altre 2 condizioni di \textit{KKT} rimanenti:
\begin{enumerate}
  \item \textbf{complementarietà}: $\mu\cdot h(x^*)=0$
  \item \textbf{ammissibilità duale}: $\mu\geq 0$
\end{enumerate}
Sappiamo che il vincolo di uguaglianza è sempre attivo, mentre quello
di disuguaglianza può essere attivo ($h(x^*)=0$) o inattivo
($h(x^*)<0$). Per far valere la condizione di complementarietà quindi
nel caso il vincolo sia attivo bisognerà verificare la validità
dell'ammissibilità duale (che comporterà la validità della
complementarietà). Nel caso il vincolo non sia attivo impongo a
priori $\mu=0$, così si ha la validità dell'ammissibilità duale e
anche della complementarietà.
\subsection{Vincolo di disuguaglianza inattivo}
Imposto un sistema dove devono essere valide le condizioni di
\textit{KKT}, ricordando l'imposizione di $\mu=0$: 
\[
  \begin{cases}
     1-2\mu(x_1-4)-2\lambda(x_1-3)=0\\
     -\mu(2x_2)-2\lambda(x_2-2)=0\\
     \mu=0\\
     \mu[(x_1-4)^2+x_2^2-16]=0\\
     (x_1-4)^2+x_2^2-16 < 0\\
     (x_1-3)^2+(x_2-2)^2-13=0
  \end{cases}
\]
risolvo il sistema:
\[
  \begin{cases}
    1-2\mu(x_1-4)-2\lambda(x_1-3)=0\\
    -\mu(2x_2)-2\lambda(x_2-2)=0\\
    \mu=0\\
    \mu[(x_1-4^2)+x_2^2-16]=0\\
    (x_1-4)^2+x_2^2-16 < 0\\
    (x_1-3)^2+(x_2-2)^2-13=0
  \end{cases}\to\begin{cases}
    1-2\lambda(x_1-3)=0\\
    -2\lambda(x_2-2)=0\\
    \mu=0\\
    0=0\\
    (x_1-4)^2+x_2^2-16 < 0\\
    (x_1-3)^2+(x_2-2)^2-13=0
  \end{cases}
\]
\[
  \to
  \begin{cases}
    1-2\lambda x_1+6\lambda=0\\
    -2\lambda x_2+4\lambda=0\\
    \mu=0\\
    0=0\\
    (x_1-4)^2+x_2^2-16 < 0\\
    (x_1-3)^2+(x_2-2)^2-13=0
  \end{cases}\to
  \begin{cases}
    2\lambda x_1=6\lambda+1\\
    2\lambda x_2=4\lambda\\
    \mu=0\\
    0=0\\
    (x_1-4)^2+x_2^2-16 < 0\\
    (x_1-3)^2+(x_2-2)^2-13=0
  \end{cases}
\]
\[\to
  \begin{cases}
    x_1=3+\frac{1}{2\lambda}\\
    x_2=2\\
    \mu=0\\
    0=0\\
    (x_1-4)^2+x_2^2-16 < 0\\
    (x_1-3)^2+(x_2-2)^2-13=0
  \end{cases}\to\begin{cases}
    x_1=3+\frac{1}{2\lambda}\\
    x_2=2\\
    \mu=0\\
    0=0\\
    (\frac{1}{2\lambda}-1)^2+4-16 < 0\\
    (\frac{1}{2\lambda})^2-13=0
  \end{cases}
\]
\[\to
  \begin{cases}
    x_1=3+\frac{1}{2\lambda}\\
    x_2=2\\
    \mu=0\\
    0=0\\
    (\frac{1}{2\lambda}-1)^2-12 < 0\\
    \frac{1}{4\lambda^2}=13
  \end{cases}\to\begin{cases}
    x_1=3+\frac{1}{2\lambda}\\
    x_2=2\\
    \mu=0\\
    0=0\\
    (\frac{1}{2\lambda}-1)^2-12 < 0\\
    52\lambda^2=1
  \end{cases}\to\begin{cases}
    x_1=3+\frac{1}{2\lambda}\\
    x_2=2\\
    \mu=0\\
    0=0\\
    (\frac{1}{2\lambda}-1)^2-12 < 0\\
    \lambda =\pm \frac{\sqrt{13}}{26}
  \end{cases}
\]
abbiamo quindi 2 alternative:
\begin{enumerate}
  \item $\lambda = -\frac{\sqrt{13}}{26}$ quindi:
  \begin{itemize}
    \item $x_1=3+\frac{1}{-2\cdot \frac{\sqrt{13}}{26}}\to
    x_1=3+\frac{1}{-\frac{\sqrt{13}}{13}}\to x_1=3-\sqrt{13}$
    \item
    $h(x_1,x_2)=(\frac{1}{-2\frac{\sqrt{13}}{26}}-1)^2-12=(-\sqrt{13}-1)^2-12=
    2+2\sqrt{13}$ \\(infatti la penultima equazione deriva da
    $h(x_1,x_2)=(x_1-4)^2+x_2^2-16$) 
  \end{itemize}
  ma $h(x_1,x_2)=2+2\sqrt{13}\geq 0$ quindi la condizione di
  ammissibilità primale $h(x^*)\leq 0$ non è rispettata, quindi il
  punto $(3-\sqrt{13},2)$ \textit{non è ammissibile}
  
  \item $\lambda = \frac{\sqrt{13}}{26}$ quindi:
  \begin{itemize}
    \item $x_1=3+\frac{1}{2\cdot \frac{\sqrt{13}}{26}}\to
    x_1=3+\frac{1}{\frac{\sqrt{13}}{13}}\to x_1=3+\sqrt{13}$
    \item
    $h(x_1,x_2)=(\frac{1}{2\frac{\sqrt{13}}{26}}-1)^2-12=(\sqrt{13}-1)^2-12=
    2-2\sqrt{13}$ \\(infatti la penultima equazione deriva da
    $h(x_1,x_2)=(x_1-4)^2+x_2^2-16$) 
  \end{itemize}
  in questo caso si ha che $h(x_1,x_2)=2-2\sqrt{13}< 0$, quindi la
  dondizione di ammissibilità primale è rispettata. La quarta
  equazione ci conferma che anche l'altra condizione di ammissibilità
  è rispettata $g(x_1,x_2)=0$. Inoltre le condizioni di
  complementarietà e ammissibilità duale sono rispettate dal fatto che
  è stato imposto il vincolo di disuguaglianza inattivo mettendo
  $\mu=0$. \\
 \textbf{ Quindi il punto} $(3+\sqrt{13},2)$,\textbf{ con}
  $f(3+\sqrt{13},2)=3+\sqrt{13}$\textbf{ è un }\textbf{candidato punto di
    massimo} 
\end{enumerate}
\subsection{Vincolo di disuguaglianza attivo}
Si danno per scontati (in quanto identici a quelli fatti per cercare i
punti minimi candidati) i primi passaggi e si analizzano direttamente i
due casi.
\subsubsection{Caso 1}
In questo caso si ha $x_1=0$ e $x_2=0$, riprendiamo quindi a risolvere
il sistema:
\[\begin{cases}
    1-2\mu(x_1-4)-2\lambda(x_1-3)=0\\
    -\mu(2x_2)-2\lambda(x_2-2)=0\\
    \mu\geq 0\\
    \mu[(x_1-4)^2+x_2^2-16]=0\\
    x_1=0 \\
    x_2=0
  \end{cases}\to
  \begin{cases}
    1+8\mu+6\lambda=0\\
    4\lambda=0\\
    \mu\geq 0\\
    \mu[0]=0\\
    x_1=0 \\
    x_2=0
  \end{cases}\to
    \begin{cases}
    \mu=-\frac{1}{8}\\
    \lambda=0\\
    -\frac{1}{8}\geq 0 \mbox { impossibile}\\
    0=0\\
    x_1=0 \\
    x_2=0
  \end{cases}
\]
\textbf{Si ha quindi che l'ammissibilità duale non è verificata e quindi il
  punto $(0,0)$ non è un candidato punto di massimo}
\subsection{Caso 2}
In questo caso si ha $x_1=\frac{32}{5}$ e $x_2=\frac{16}{5}$,
riprendiamo quindi a risolvere il sistema:
\[\begin{cases}
    1-2\mu(x_1-4)-2\lambda(x_1-3)=0\\
    -\mu(2x_2)-2\lambda(x_2-2)=0\\
    \mu\geq 0\\
    \mu[(x_1-4)^2+x_2^2-16]=0\\
    x_1=\frac{32}{5} \\
    x_2=\frac{16}{5}
  \end{cases}\to
  \begin{cases}
    1-2\mu(\frac{32}{5}-4)-2\lambda(\frac{32}{5}-3)=0\\
    -\mu(2\frac{16}{5})-2\lambda(\frac{16}{5}-2)=0\\
    \mu\geq 0\\
    \mu[(\frac{32}{5}-4)^2+\left(\frac{16}{5}\right)^2-16]=0\\
    x_1=\frac{32}{5} \\
    x_2=\frac{16}{5}
  \end{cases}
\]
\[\to
  \begin{cases}
    1-2\mu(\frac{12}{5})-2\lambda(\frac{17}{5})=0\\
    -\mu(\frac{32}{5})-2\lambda(\frac{16}{5})=0\\
    \mu\geq 0\\
    \mu[(\frac{12}{5})^2+\left(\frac{16}{5}\right)^2-16]=0\\
    x_1=\frac{32}{5} \\
    x_2=\frac{16}{5}
  \end{cases}\to
  \begin{cases}
    1-\frac{24}{5}\mu-\frac{34}{5}\lambda=0\\
    -\frac{32}{5}\mu-\frac{32}{5}\lambda=0\\
    \mu\geq 0\\
    \mu[\frac{144}{25}+\frac{256}{25}-16]=0\\
    x_1=\frac{32}{5} \\
    x_2=\frac{16}{5}
  \end{cases}
\]
\[\to
  \begin{cases}
    1-\frac{24}{5}\mu-\frac{34}{5}\lambda=0\\
    \frac{32}{5}\mu=-\frac{12}{5}\lambda\\
    \mu\geq 0\\
    \mu[16-16]=0\\
    x_1=\frac{32}{5} \\
    x_2=\frac{16}{5}
  \end{cases}\to
  \begin{cases}
    1-\frac{24}{5}\mu-\frac{34}{5}\lambda=0\\
    \mu=-\frac{12}{32}\lambda\\
    \mu\geq 0\\
    0=0\\
    x_1=\frac{32}{5} \\
    x_2=\frac{16}{5}
  \end{cases}\to
  \begin{cases}
    1-\frac{24}{5}\left(-\frac{12}{32}\lambda\right)-\frac{34}{5}\lambda=0\\
    \mu=-\frac{12}{32}\lambda\\
    \mu\geq 0\\
    0=0\\
    x_1=\frac{32}{5} \\
    x_2=\frac{16}{5}
  \end{cases}
\]
\[\to
  \begin{cases}
    1+\frac{9}{5}\lambda-\frac{34}{5}\lambda=0\\
    \mu=-\frac{12}{32}\lambda\\
    \mu\geq 0\\
    0=0\\
    x_1=\frac{32}{5} \\
    x_2=\frac{16}{5}
  \end{cases}\to
  \begin{cases}
    1-5\lambda=0\\
    \mu=-\frac{12}{32}\lambda\\
    \mu\geq 0\\
    0=0\\
    x_1=\frac{32}{5} \\
    x_2=\frac{16}{5}
  \end{cases}\to
  \begin{cases}
    \lambda=\frac{1}{5}\\
    \mu=-\frac{12}{32}\lambda\\
    \mu\geq 0\\
    0=0\\
    x_1=\frac{32}{5} \\
    x_2=\frac{16}{5}
  \end{cases}\to
  \begin{cases}
    \lambda=\frac{1}{5}\\
    \mu=-\frac{3}{40}\\
    -\frac{3}{40}\geq 0 \mbox{ impossibile}\\
    0=0\\
    x_1=\frac{32}{5} \\
    x_2=\frac{16}{5}
  \end{cases}
\]
\textbf{Si ha quindi che l'ammissibilità duale non è verificata e quindi il
  punto $\left(\frac{32}{5},\frac{16}{5}\right)$ non è un candidato
  punto di massimo}
\subsubsection{Conclusioni}
\begin{shaded}
  Si ha quindi che solo il punto $(3+\sqrt{13},2)$, con
  $f(3+\sqrt{13},2)=3+\sqrt{13}$, è un \textbf{candidato punto di
    massimo} e quindi è il \textbf{punto di massimo}
\end{shaded}
\chapter{Esercizio 3}
\begin{itemize}
  \item \textit{Qual è lo scopo della lista tabù in una metaeuristica
    di tipo tabù search?}\\
  Dato che usando la tabù search si hanno alcune \textit{mosse} dette
  \textbf{mosse di non miglioramento} si hanno i \textbf{tabù}, ovvero
  si proibiscono le ultime mosse dell'algoritmo stesso per evitare di
  analizzare inutilemente più volte lo stesso ottimo. Queste mosse
  vietate sono contenute nella \textbf{tabù list}.\\
  A livello di algoritmo si ha che nell'iterazione dello stesso si
  usano innazitutto mosse assenti nella tabù list, identificate
  mediante un \textit{metodo di ricerca locale}. Se, però, una mossa
  nella tabù list è migliore di tutte le altre trovate può essere
  usata, grazie al \textbf{criterio di aspirazione}.\\
  Alla fine dell'iterazione la mossa corrente viene caricata nella
  tabù list, alla quale vengono eventualmente rimossi gli elementi più
  vecchi.\\
  L'uso della tabù list consente quindi una maggior esplorazione dello
  spazio delle soluzioni.
  \item \textit{Qual è lo scopo della temperatura in una metaeuristica
    di tipo simulated annealing?}
  Con il simulated annealing si cercano soluzioni successive in modo
  casuale a partire da quella corrente, cercando comunque una
  soluzione nel suo intorno. Se la nuova soluzione non risulta
  migliore di quella corrente può essere comunque accettata in base ad
  una certa probabilità che è direttamente proporzionale ad un valore
  $T$ detto \textbf{temperatura} (termine che ricorda la metallurgica,
  alla quale si ispira l'intero algoritmo). La probabilità è inoltre
  inversamente proporzionale alla distanza tra le due soluzioni. \\
  All'inizio dell'algoritmo si ha una \textit{temperatura} molto alta
  in modo da accettare anche mosse di peggioramento al fine di
  allontanare l'analisi dai punti di ottimo locale. La
  \textit{temperatura} cala per tutta la durata dell'algoritmo nel
  monento in cui si hanno un certo numero di iterazioni tutte alla
  stessa temperatura. Alla fine dell'algortimo la temperatura sarà
  scesa fino ad un valore prossimo a 0 (in modo che non vengano più
  accettate mosse di peggioramento). Dopo aver iterato un certo numero
  di volte con la temperatura minima scelta si può dire di aver
  ottenuto la miglior soluzione.\\
  Per una buona esecuzione si parte con una \textit{temperatura} in
  grado di accettare almeno la metà delle mosse peggiorative possibili
  e durante l'esecuzione viene diminuita lentamente.\\
  Si vede come la velocità di convergenza al punto di ottimo globale è
  direttamente influenzata dalla \textit{temperatura} e dal ``rate'' a
  cui viene diminuita.
  \item \textit{A cosa servono gli operatori genetici negli algoritmi
    genetici?} \\
  Innazitutto si ha che gli operatori genetici sono 2:
  \textbf{crossover} e \textbf{mutazione}. Questi operatori vengono
  usati per migliorare la ricerca di soluzioni mediante un algoritmo
  genetico, generando i discendenti di una popolazione secondo
  determinati criteri. Con il crossover infatti i \textit{figli}
  vengono generati ricombinando il materiale genetico degli individui
  che costituiscono il\textit{ matingpool}, consentendo di ottenere nuovi
  individui il cui codice genetico è mutuato in parte da un genitore e
  in parte dall’altro. Con la \textbf{mutazione}, invece, si mantiene
  la diversità genetica per cercare di esplorare anche zone dello
  spazio di ricerca non presenti nella popolazione attuale.
\end{itemize}
\end{document}