\documentclass[a4paper,12pt, oneside]{book}

%\usepackage{fullpage}
\usepackage[italian]{babel}
\usepackage[utf8]{inputenc}
\usepackage{amssymb}
\usepackage{amsthm}
\usepackage{graphics}
\usepackage{amsfonts}
\usepackage{listings}
\usepackage{amsmath}
\usepackage{amstext}
\usepackage{engrec}
\usepackage{rotating}
\usepackage[safe,extra]{tipa}
\usepackage{showkeys}
\usepackage{multirow}
\usepackage{hyperref}
\usepackage{microtype}
\usepackage{enumerate}
\usepackage{braket}
\usepackage{marginnote}
\usepackage{pgfplots}
\usepackage{cancel}
\usepackage{polynom}
\usepackage{booktabs}
\usepackage{enumitem}
\usepackage{framed}
\usepackage{pdfpages}
\usepackage{pgfplots}
\usepackage[cache=false]{minted}

\usepackage{tikz}\usetikzlibrary{er}\tikzset{multi  attribute /.style={attribute ,double  distance =1.5pt}}\tikzset{derived  attribute /.style={attribute ,dashed}}\tikzset{total /.style={double  distance =1.5pt}}\tikzset{every  entity /.style={draw=orange , fill=orange!20}}\tikzset{every  attribute /.style={draw=MediumPurple1, fill=MediumPurple1!20}}\tikzset{every  relationship /.style={draw=Chartreuse2, fill=Chartreuse2!20}}\newcommand{\key}[1]{\underline{#1}}

\usepackage{fancyhdr}
\pagestyle{fancy}
\fancyhead[LE,RO]{\slshape \rightmark}
\fancyhead[LO,RE]{\slshape \leftmark}
\fancyfoot[C]{\thepage}



\title{Ricerca Operativa e Pianificazione delle Risorse}
\author{UniShare\\\\Davide Cozzi\\\href{https://t.me/dlcgold}{@dlcgold}\\\\Gabriele De Rosa\\\href{https://t.me/derogab}{@derogab} \\\\Federica Di Lauro\\\href{https://t.me/f_dila}{@f\textunderscore dila}}
\date{}

\pgfplotsset{compat=1.13}
\begin{document}
\maketitle

\definecolor{shadecolor}{gray}{0.80}
\setlist{leftmargin = 2cm}
\newtheorem{teorema}{Teorema}
\newtheorem{definizione}{Definizione}
\newtheorem{esempio}{Esempio}
\newtheorem{corollario}{Corollario}
\newtheorem{lemma}{Lemma}
\newtheorem{osservazione}{Osservazione}
\newtheorem{nota}{Nota}
\newtheorem{esercizio}{Esercizio}
\tableofcontents
\renewcommand{\chaptermark}[1]{%
	\markboth{\chaptername
		\ \thechapter.\ #1}{}}
\renewcommand{\sectionmark}[1]{\markright{\thesection.\ #1}}
\chapter{Introduzione}
\textbf{Questi appunti sono presi a lezione. Per quanto sia stata
  fatta una revisione è altamente probabile (praticamente certo)
  che possano contenere errori, sia di stampa che di vero e proprio
  contenuto. Per eventuali proposte di correzione effettuare una
  pull request. Link: } \url{https://github.com/dlcgold/Appunti}.\\
\textbf{Grazie mille e buono studio!}
\chapter{Introduzione alla Ricerca Operativa}
La\textbf{ Ricerca Operativa} è essenziale nel \textit{problem
  solving} e nell'ambito del \textit{decision making}.
Sostanzialmente quindi si studia l'ottimizzazione, massimizzando le
performances, l'accuratezza dei costi etc$\ldots$ per raggiungere un
obiettivo. \\ \textit{Sulle slides ci sono vari esempi introduttivi di
  vita reale}\\
Un altro problema studiato dalla riceca operativa sono le previsioni,
mediante algoritmi predittivi che studiano i \textit{pesi} delle
osservazioni (cosa utile nel \textbf{Machine Learning} in quanto sono
un uso di base delle \textbf{Reti Neurali}, \textit{vari esempi
  introduttivi sulle slides}).\\
\textbf{La ricerca operativa si occupa di formalizzare un problema in
  un modello matematico e calcolare una soluzione ottimo o
  approssimata}. Essa costituisce un approccio scientifico alla
risoluzione di problemi complessi da ricondurre alla matematica
applicata. È utile in ambiti economici, logistici, di progettazione di
servizi e di sistemi di trasporto e, ovviamente, nelle tecnologie.
\textit{È la branca della matematica più applicata}.\\
Il \textit{primo passo} consiste nel costruire un modello traducendo il
problema reale in linguaggio anturale in un linguaggio matematico, che
non è ambiguo. Il \textit{secondo passo} consiste nella costruzione delle
soluzioni del modello tramite algoritmi e programmi di calcolo. Il
\textit{terzo passo}, ovvero l'ultimo, è l'interpretazione e la
valutazione delle soluzioni del modello rispetto a quelle del problema
reale.\\
La ricerca operativa ha origini nel 1800 in un ambiente puramente
matematico. È stata resa ``\textit{algoritmica}'' con la Macchina di
Turing. \textbf{La ricerca operativa usa anche tecniche numeriche e
  non solo analitiche}.\\
Negli ultimi hanno si sono sviluppati, mediante il concetto di
\textbf{gradiente}, nuovi algoritmi per il \textbf{deep network}.\\

\end{document}