\section{DAG ed Alberi}
Si definisce come \emph{DAG}(Directed Acyclic Graph), un grafo orientato senza cicli
mentre si definisce come \emph{foresta}, un grafo non orientato aciclico.
%Inserire esempi degli DAG

Si definisce come \emph{Albero libero}, un DAG connesso con un solo nodo sorgente, detto \emph{radice},
in cui ogni nodo diverso dalla radice ha un solo nodo entrante.\newline
I nodi privi di archi entranti sono detti \emph{foglie} dell'albero.

%Inserire esempi e proprietà degli Alberi
%Esempi di Alberi
\begin{forest}
    [20
        [10
            [3]
            [5]
        ]
        [40
            [25]
            [80]
        ]
    ]
\end{forest}

%Esempio Alberi di Ricerca
\begin{forest}
    [20
        [10
            [3]
            [5]
        ]
        [40
            [25]
            [80]
        ]
    ]
\end{forest}
%Esempi Alberi!!!!

%Inserire caratteristiche e Nomenclatura degli Alberi
Le tipologie di Alberi sono:
\begin{description}
    \item[Albero Binario]: albero con al più due rami per ogni nodo diverso dalle foglie.
    \item[Albero Strettamento Binario]: albero da cui da ogni nodo, diverso dalle foglie, partono proprio due rami.
    \item[Albero Bilanciato]: albero in cui tutti i cammini hanno la stessa lunghezza
\end{description}



%Alberi binari di Ricerca e metodo di visita degli alberi
\subsection{Alberi binari di ricerca e metodo di visita}
Dato un insieme di nodi in cui è definita una relazione d'ordine, si definisce come
\emph{albero di ricerca} un albero in cui tutti i nodi della radice sinistra sono
minori della radice e tutti i nodi a destra della radice sono maggiori e ogni sottoalbero
è anch'esso un albero di ricerca.


Il metodo di visità di Albero di Ricerca è il seguente:\newline
Partendo dalla radice per visitare tutti i nodi dell'albero bisogna:
\begin{itemize}
    \item visita del sottoalbero sinistro
    \item visita della radice
    \item visita del sottoalbero destro
\end{itemize}
Si continua ad applicare la visita al sottoalbero fino a quanto si arriva a visitare la radice
attraverso la ricorsione

%Fare esempio
