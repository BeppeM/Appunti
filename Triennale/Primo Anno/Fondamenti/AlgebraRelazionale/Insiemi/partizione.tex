\section{Partizione di un Insieme}
Dato un insieme non vuoto $S$, una partizione di $S$ è una famiglia $F$ di sottoinsiemi di $S$ tale che
\begin{enumerate}
    \item ogni elemento di $S$ appartiene a qualche elemento di $F$, ossia $\cup F = S$
    \item due elementi qualunque di $F$ sono disgiunti ossia $\cap F = \emptyset$
\end{enumerate}
La partizione non può avere come elemento l'insieme vuoto in quanto esso non
appartiene agli elementi dell'insieme A.
\begin{align*}
A = \{ 1,2,3 \}\\
Par(A) = \{ \{ 1 \},\{2,3 \} \}\\
B = \{5,10,\emptyset,23,45\}\\
Par(B) = \{ \{\emptyset,23\}, \{5,10\},\{45\}\} \\
\end{align*}
