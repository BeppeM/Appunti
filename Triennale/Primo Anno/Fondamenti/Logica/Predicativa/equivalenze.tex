%Paragrafo sulle equivalenze logiche
\section{Equivalenze logiche}
Nella logica predicativa si definisce due formule semanticamente equivalenti,
indicato con $P \equiv Q$, se hanno gli stessi modelli.
Le equivalenze della logica predicativa sono le seguenti:
\begin{enumerate}
    \item $\forall x P \equiv \neg \exists x \neg P$
    \item $\neg \forall x P \equiv \exists x \neg P$
    \item $\exists x P \equiv \neg \forall x \neg P$
    \item $\neg \exists x P \equiv \forall x \neg P$
    \item $\forall x \forall y P \equiv \forall y \forall x P$
    \item $\exists x \exists y P \equiv \exists y \exists x P$
    \item $\forall x(P_1 \land P_2) \equiv \forall x P_1 \land \forall x P_2$
    \item $\exists x(P_1 \lor P_2) \equiv \exists x P_1 \lor \exists x P_2$
\end{enumerate}

\subsection{Dimostrazione equivalenze logiche}
In questo sottoparagrafo vengono svolte le dimostrazioni delle equivalenze logiche
attraverso l'utilizzo del metodo dei Tableaux.
\begin{itemize}
    \item $\forall x P \equiv \neg \exists x \neg P$
          Da fare
    \item $\neg \forall x P \equiv \exists x \neg P$
          Da fare
    \item $\exists x P \equiv \neg \forall x \neg P$
          Da fare
    \item $\neg \exists x P \equiv \forall x \neg P$
          Da fare
    \item $\forall x \forall y P \equiv \forall y \forall x P$
          Da fare
    \item $\exists x \exists y P \equiv \exists y \exists x P$
          Da fare
    \item $\forall x(P_1 \land P_2) \equiv \forall x P_1 \land \forall x P_2$
          Da fare
    \item $\exists x(P_1 \lor P_2) \equiv \exists x P_1 \lor \exists x P_2$
          Da fare
\end{itemize}
