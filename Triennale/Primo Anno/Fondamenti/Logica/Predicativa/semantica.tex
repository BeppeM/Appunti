\section{Semantica}
La semantica della logica predicativa è più complessa della semantica proposizionale
in quanto nella logica predicativa si ha la possibilità di predicare su oggetti e le loro proprietà.
L'interpretazione semantica di formule predicative si serve delle \emph{strutture},
oggetto matematico che traduce formule predicative in espressioni che hanno un significato
specifico relativamente alla realtà che si sta rappresentando.

\begin{defi}
    Una struttura per il linguaggio $L$ è una coppia $U = (D,I)$ in cui:
    $D$(dominio) è un insieme non vuoto di elementi del dominio
    $I$(interpretazione) è una funzione che associa a simboli e formule del linguaggio
    un significato  a partire dalla segnatura del linguaggio.
    $I$ associa:
    \begin{itemize}
        \item a ogni simbolo di costante $c$ un elemento $c^I \in D$
        \item a ogni simbolo di funzione n-aria $f$ una funzione $f^I:D^n \mapsto D$
        \item a ogni simbolo di predicato n-ario $P$ una relazione n-aria $P^I \subseteq D^n$
    \end{itemize}
\end{defi}

\begin{defi}
    Sia $Var$ l'insieme delle variabili di un linguaggio predicativo $L$, un assegnazione $\eta$
    delle variabili in una struttura $U = (D,I)$ è una funzione $\eta:Var \mapsto D$
\end{defi}
Un assegnazione $\eta$ è una maniera di associare un valore alle variabili del linguaggio.

\begin{defi}
    Sia $U =(D,I)$ una struttura per $L$ e sia $\eta$ una assegnazione.Estendiamo
    tale estensione a un'assegnazione $\bar{\eta}(I,\eta)$ sui termini definita come:
    \begin{itemize}
        \item Per ogni variabile $x$, $x^{I,\eta} = x^{\eta}$
        \item Per ogni costante $c$, $c ^{I,\eta} = c^I$
        \item Se $t_1,\dots,t_n$ sono dei termini e $f$ è un simbolo di funzione n-aria
              allora $f(t_1,\dots,t_n)^{I,\eta} = f^I(t_1 ^{I,\eta},\dots,t_n ^{I,\eta})$
    \end{itemize}
\end{defi}

Se un termine è chiuso non si necessità della funzione $\eta$ in quanto l'Interpretazione
è unica e non dipende dall'interpretazione data tramite la funzione $\eta$.

Una formula $P$, in una struttura $U$ rispetto a un assegnazione $\eta$, si dice \emph{soddisfacibile}
, denotata con $U,\eta \models P$, se e solo se è vera l'interpretazione di $P$
in una struttura $U$ in cui ad ogni variabile $x$ è valutata come $x^{\eta}$.

La soddisfacibilità di una formula è definita induttivamente come segue:
\begin{defi}
    Sia $U = (D,I)$ una struttura per un linguaggio $L$ e $\eta$ un'assegnazione in $U$
    \begin{itemize}
        \item $(U,\eta) \models T$ e $(U,\eta) \not \models F$
        \item se $A$ è una formula atomica del tipo $P(t_1,\dots,t_n)$, allora
              $(U,\eta) \models P(t_1,\dots,t_n) \iff (t_1^{I,\eta},\dots,t_n^{I,\eta}) \in P^I$
        \item $(U,\eta) \models \neg A \iff (U,\eta) \not \models A$
        \item $(U,\eta) \models (A \circ B) \iff (U,\eta) \models A \circ (U,\eta) \models B$
        \item $(U,\eta) \models \forall x A \iff \forall d \in D$ è verificato che $U \models A(\eta[d/x])$
        \item $(U,\eta) \models \exists x A \iff \exists d \in D$ per cui $U \models A(\eta[d/x])$
    \end{itemize}
\end{defi}

Esempio:
$x \mapsto s(0)$
$y \mapsto 0$
$z \mapsto s(0)$
Allora x + y = z con questi valori è vera

%Definizione di Modello
\begin{defi}
    Se per una formula $A \in L$, $(U,\eta) \models A$ è verificato se per ogni
    assegnazione alle variabili, allora scriviamo $U \models A$ e diciamo che $U$
    è un modello di $A$.
\end{defi}

%tautologia
\begin{defi}
Una formula $A \in L$ è una \emph{tautologia} se e solo se è vera in tutte le strutture
di $U$ e lo scriviamo $U \models A$
\end{defi}

Si può estendere la definizione di soddisfacibile anche un insieme di formule con
la definizione seguente:
\begin{defi}
Un insieme di formule $\Gamma$ è soddisfacibile se esiste una struttura $U$ e un
assegnazione $\eta$ tale che $(U,\eta) \models A$ per ogni $A \in \Gamma$
\end{defi}
