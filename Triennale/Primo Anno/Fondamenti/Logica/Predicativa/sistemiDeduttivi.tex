\section{Sistemi Deduttivi predicativi}
Comprendere la tipologia della formula tramite la sua semantica nella logica predicativa
è più complesso rispetto alla semantica della logica proposizionale per cui l'apparato deduttivo
e la sua correttezza e completezza rispetto alla semantica assumono particolare rilevanza.

\begin{defi}
Una sostituzione è una funzione $\sigma:VAR \mapsto TERM$ definita induttivamente:
\end{defi}
\begin{enumerate}
    \item $T\sigma = T$ e $F \sigma = F$
    \item se $c$ è una costante allora $c \sigma = c$
    \item se $x$ è una variabile allora $x \sigma = x$
    \item se $f$ è un simbolo di funzione di arietà $n$ allora
          $f(t_1,\dots,t_n)\sigma = f(t_1\sigma,\dots,t_n\sigma)$
\end{enumerate}
La sostituzione $\sigma$ di un termine $t$ al posto di un simbolo di variabile $x$
è indicata da $\sigma = \{t/x \}$.

\begin{lem}
Se $t$ è un termine e $\sigma$ è una sostituzione, allora $t\sigma$ è un termine
\end{lem}
%Da fare dimostrazione

\begin{defi}
    Data una formula $S$, la formula $S\sigma$ è definita nel seguente modo:
\end{defi}
\begin{enumerate}
    \item se $S = P(t_1,\dots,t_n)$ è una formula atomica allora
          allora $S\sigma = P(t_1\sigma,\dots,t_n\sigma)$
    \item se $S = (\neg A)$ allora $S \sigma = \neg(A \sigma)$
    \item se $S = (A \circ B)$ allora $S \sigma = A\sigma \circ B \sigma$
    \item se $S = (\forall x A)$ allora $S \sigma = \forall x (A  \sigma _ x)$
    \item se $S = (\exists x A)$ allora $S \sigma = \exists x (A \sigma _ x)$
\end{enumerate}
%Tableau predicativi
\subsection{Tableau Predicativi}
I tableau predicativi è un apparato deduttivo che permette di stabilire la tipologia
di una formula del linguaggio mediante l'applicazione di una serie di regole.
Mantiene le stesse modalità di dimostrazione dei tableaux proposizionale e le stesse
definizioni di espansioni del Tableaux.

\begin{equation*}
\begin{prooftree}%T EXISTS
\hypo{S,T \exists x A(x)}
\infer1 {S,A(a) \ \text{con a un nuovo simbolo di variabile}}
\end{prooftree}
\quad T \exists \qquad
\begin{prooftree} %F EXISTS
\hypo{S,F \exists x A(x)}
\infer1 {S,F A(a),F \exists A(x)}
\end{prooftree}
F \ \exists
\end{equation*}

\begin{equation*}
\begin{prooftree}%T FORALL
\hypo{S,T \forall x A(x)}
\infer1 {S,T A(a),T \forall x A(x)}
\end{prooftree}
\quad T \forall \qquad
\begin{prooftree}%F FORALL
\hypo{S,F \forall x A(x)}
\infer1 {S,F A(a) \ \text{con a un nuovo simbolo di variabile}}
\end{prooftree}
F \ \forall
\end{equation*}

Nelle tableuax regole $T \exists$ e $F \forall$ bisogna introdurre un nuovo simbolo di
variabile, in quanto non si poteva conoscere prima dell'applicazione della regole
il valore della variabile, mentre nelle altre due si può utilizzare qualsiasi variabile.
Il fatto di portarsi dietro la formula nell'applicazione delle regole dei tableaux
porta alla semidecidibilità.

Nei tableau predicativi l'ordine di applicazione delle regole, quando è possibile
sceglierlo, cambia la chiusura o meno del Tableaux.

Le euristiche nell'applicazione delle regole dei Tableaux sono le seguenti:
\begin{itemize}
    \item Applicare prima le regole che non ramificano il tableaux
    \item Applicare prima le regole che vincolano all'introduzione di nuovi parametri
    \item Quando si ha la possibilità di scegliere il parametro, conviene sceglierlo uno già scelto.
\end{itemize}

%Esempi
%Esempio dimostrazione per induzione
\begin{thm}
$\sum _{k=0} ^ n (4k+1)= (2n+1)(n+1)$
\end{thm}

\begin{proof}
Caso Base $n = 0$:
\begin{equation*}
    \sum _{k = 0} ^ 0 (4k+1) = (2*0 +1)(0+1) \quad 1 = 1 \ \text{vero}
\end{equation*}
Caso passo:
\quad Ipotesi:$\sum _{k = 0} ^ n (4k+1) = (2n+1)(n+1)$\newline
\quad Tesi: $\sum _{k = 0} ^ {n+1} (4k+1) = (2n+3)(n+2)$
\begin{equation*}
\begin{split}
\sum _{k = 0} ^ {n+1} (4k+1) & = \sum _{k = 0} ^ n (4k+1) + 4(n+1) + 1 \\
                             & = (2n+1)(n+1) + 4n + 5\\
                             & = 2n^2+3n+1+4n+5 \\
                             & = (2n+3)(n+2)\\
\end{split}
\end{equation*}
\end{proof}

%Dimostrazione Tableaux Predicativo
Esempio:$\exists x (P(x) \lor Q(x)) \rightarrow (\exists x P(x) \lor \forall y Q(y))$
\begin{equation*}
\begin{prooftree}
\hypo{F \exists x (P(x) \lor Q(x)) \rightarrow (\exists x P(x) \lor \forall y Q(y))}
\infer1 {T \exists x (P(x) \lor Q(x)), F \exists x P(x) \lor \forall y Q(y)}
\infer1{T P(a) \lor Q(a), F \exists x P(x) \lor \forall y Q(y)}
\infer1 {T P(a) \lor Q(a), F \exists x P(x), F \forall y Q(y)}
\infer1 {T P(a) \lor Q(a), F \exists x P(x), F Q(b)}
\infer1 {T P(a),F \exists x P(x),F Q(b)/T Q(a),F Q(b),F \exists x P(x)}
\end{prooftree}
\end{equation*}
Il secondo ramo del tableau non potrà mai chiudere percui bisogna fare il T-Tableaux

\begin{equation*}
\begin{prooftree}
\hypo{T \exists x (P(x) \lor Q(x)) \rightarrow (\exists x P(x) \lor \forall y Q(y))}
\infer1 {F \exists x P(x) \lor Q(x)/T \exists x P(x) \lor \forall y Q(y)}
\infer1 {F \exists x P(x) \lor Q(x)/T \exists x P(x)/T \forall y Q(y)}
\end{prooftree}
\end{equation*}
Il tableaux non potrà mai chiudere in quanto il secondo e il terzo non generanno mai delle contraddizioni
per cui la formula è sodddisfacibile non tautologica.

