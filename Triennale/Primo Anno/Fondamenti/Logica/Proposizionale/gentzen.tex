%Parte sulla deduzione di Gentzen(detta anche Deduzione Naturale)
\section{Deduzione di Gentzen}
Il sistema deduttivo di Gentzen, introdotto come si può dedurre dal nome dal logico
tedesco Gerhard Gentzen nel 1930, presenta delle analogie con il sistema dei Tableaux.
Può sembrare non famigliare in quanto ha soltanto un tipo di assioma e molte regole
di inferenza, al contrario delle teorie matematiche, e per cui si può rappresentare
la deduzione mediante alberi, cosa che la rende simile ai tableaux.

\begin{defi}
Si definisce \emph{assioma} del sistema di Gentzen, un insieme di espressioni $U$
in cui compare una coppia di letterali complementari, mentre le \emph{regole di inferenza}
sono della forma:
\begin{equation*}
\begin{prooftree}
\hypo{U_1 \cup \{\alpha_1,\alpha_2\}}
\infer1{U_1 \cup \{\alpha\}}
\end{prooftree}
\end{equation*}
\begin{equation*}
\begin{prooftree}
\hypo{U_1 \cup \{\beta_1\}}
\hypo{U_2 \cup \{\beta_2\}}
\infer2{U_1 \cup U_2 \cup \{\beta\}}
\end{prooftree}
\end{equation*}
dove le $\alpha$ e le $\beta$ scomposizioni sono mostrate nella tabella qui di seguito.
\end{defi}

Nel sistema Gentzen sono definite delle regole, come avvenuto nei Tableaux, per
riuscire a desumere le regole di inferenza e rappresentano l'inverso dei tableaux,
 come si può vedere dalla seguente tabella di regole del sistema di Gentzen:
$\begin{array}{ccc}
\toprule
\alpha & \alpha_1 & \alpha_2 \\
\midrule
A \cup B & A & B \\
\neg(A \cap B) & \neg A & \neg B \\
A \rightarrow B & \neg A & B \\
\end{array}$
$\begin{array}{ccc}
\toprule
\beta & \beta_1 & \beta_2 \\
\midrule
A \cap B & A & B \\
\neg(A \cup B) & \neg A & \neg B \\
\neg(A \rightarrow B) & \neg A & B \\
\end{array}$
%Inserire le regole di Gentzen

%Esempio sistema di Gentzen
Esempio:Dimostrare in Gentzen $(p \lor q) \rightarrow (q lor p)$
\begin{proof}

\end{proof}
%Analogie tra Gentzen e i Tableaux
\begin{thm}
Sia $U$ un insieme di formule e $\bar{U}$ l'insieme dei complementi delle formule in $U$.
Esiste una dimostrazione di $U$ in $G$ se e soltanto se c'è un tableaux chiuso per $\bar{U}$.
Come caso particolare esiste una dimostrazione di $A$ in $G$ se e soltanto se c'è
un tableaux chiuso per $\neg A$.
\end{thm}

%Dimostrazione analogie tra Gentzen e i tableaux

%Correttezza e completezza di Gentzen
\begin{thm}
Una formula $A$ è valida se e soltanto se c'è una dimostrazione di $A$ in $G$.
\end{thm}


%Inserire Esercizi
