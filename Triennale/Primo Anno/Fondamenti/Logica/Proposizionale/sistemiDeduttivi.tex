\section{Sistema Deduttivo}
Il sistema deduttivo è un metodo di calcolo che manipola proposizioni, senza la
necessità di ricorrere ad altri aspetti della logica.\newline
Nella logica proposizionale, tramite i teoremi di completezza e correttezza, esiste
una corrispondenza tra le formule derivanti dal sistema deduttivo e le formule verificabili
tramite la semantica della logica.

I sistemi deduttivi della logica proposizionale sono i seguenti:
\begin{description}
    \item[Sistema deduttivo Hilbertiano]: non viene analizzato
    \item[Metodo dei Tableaux]
    \item [Risoluzione Proposizionale]:non viene analizzato !!!!
\end{description}

\begin{defi}
Una sequenza di formule $A_1,\dots,A_n$ di $\Lambda$ è una \emph{dimostrazione} se
per ogni $i$ compreso tra $1$ e $n$, $A_i$ è un assioma di $\Lambda$ oppure una
conseguenza diretta di una formula precedente.
\end{defi}

\begin{defi}
Una formula $A$ di una logica $\Lambda$ è detta \emph{teorema} di $\Lambda$,indicata
con $\vdash A$ se esiste una dimostrazione di $\Lambda$ che ha $A$ come ultima formula
\end{defi}

Una dimostrazione di una formula di una logica può venire tramite:
\begin{itemize}
  \item  \textbf{Metodo diretto}: Data un'ipotesi, attraverso una serie di passi
          si riesce a dimostrare la correttezza della Tesi
  \item \textbf{Metodo per assurdo}(non sempre accettato in tutte le logiche):
        Si nega la tesi ed attraverso una serie di passi si riesce a dimostrare
        la negazione delle ipotesi.
\end{itemize}

\begin{thm}
    Un apparato deduttivo $R$ è completo se, per ogni formula $A \in Fbf$, $\vdash A$
    implica $\models A$
\end{thm}

\begin{thm}
    Un apparato deduttivo $R$ è corretto se, per ogni formula $A \in Fbf$, $\models A$
    implica $\vdash A$
\end{thm}
\subsection{Tableau Proposizionali}
Il metodo dei Tableau è stato introdotto da Hintikka e Beth alla fine degli anni '50
e poi ripresi successivamente da Smullyan.
Per poter comprendere e capire i Tableaux dobbiamo introdurre una serie di definizioni:
\begin{defi}
Per ogni formula $A$, $\{A,\neg A\}$ è una coppia di formule complementari in cui
$A$ è il complemento di $\neg A$
\end{defi}

\begin{defi}
Un letterale è un atomo o la sua negazione.Se $p$ è un atomo allora $\{p,\neg p\}$
è una coppia di letterali complementari.
\end{defi}

I tableau sono degli alberi,la cui radice è l'enunciato in esame, e gli altri nodi
sono costruiti attraverso l'applicazione di una serie di regole,fino ad arrivare
alle formule atomiche come radici.

I tableaux proposizionali si dividono in due tipologie di formule(e quindi regole):
le $\alpha$ regole e le $\beta$ regole;le $\alpha$ formule sono di tipo congiuntivo
ed è soddisfatta se e soltanto se le sottoformule $\alpha_1$ e $\alpha_2$ sono entrambe soddisfatte
mentre le $\beta$ formule sono di tipo disgiuntivo e sono soddisfatte se e soltanto
se almeno una delle due sottoformule $\beta_1$ e $\beta_2$ è soddisfatta.

Le regole dei Tableau sono le seguenti:
%T AND
\begin{equation*}
%T AND
\begin{prooftree}
\hypo{S,T (A \land B)}
\infer1 {S,TA,TB}
\end{prooftree}
\quad T \land \qquad
%F AND
\begin{prooftree}
\hypo{S,F (A \land B)}
\infer1 {S,FA/S,FB}
\end{prooftree}
F \ \land
\end{equation*}

\begin{equation*}
%T OR
\begin{prooftree}
\hypo{S,T (A \lor B)}
\infer1 {S,TA / S,TB}
\end{prooftree}
\quad T \lor \qquad
%F OR
\begin{prooftree}
\hypo{S,F (A \lor B)}
\infer1 {S,FA,FB}
\end{prooftree}
F \ \lor
\end{equation*}

\begin{equation*}
%T NOT
\begin{prooftree}
\hypo{S,T (\neg A)}
\infer1 {S,FA}
\end{prooftree}
\quad T \neg \qquad
%F NOT
\begin{prooftree}
\hypo{S,F (\neg A)}
\infer1 {S,TA}
\end{prooftree}
F \ \neg
\end{equation*}

\begin{equation*}
%T ->
\begin{prooftree}
\hypo{S,T (A \rightarrow B)}
\infer1 {S,FA / S,TB}
\end{prooftree}
\quad T \rightarrow \qquad
%F ->
\begin{prooftree}
\hypo{S,F (A \rightarrow B)}
\infer1 {S,TA,FB}
\end{prooftree}
F \ \rightarrow
\end{equation*}

\begin{equation*}
%T <-->
\begin{prooftree}
\hypo{S,T (A \iff B)}
\infer1 {S,TA,TB/S,FA,FB}
\end{prooftree}
\quad T \iff \qquad
%F <-->
\begin{prooftree}
\hypo{S,F (A \iff B)}
\infer1{S,TA,FB/S,FA,TB}
\end{prooftree}
F \ \iff
\end{equation*}

Le $\beta$ regole sono quelle che creano due sottoformule indicate nella regola con $/$
mentre, per esclusione, le $\alpha$ regole sono quelle in cui si crea soltanto una sottoformula.

%Definizione induttiva di costruzione di un tableaux
Il tableaux di una formula $A$ inizialmente è composto da un solo nodo, la radice, etichettata
dalla formula $A$.
Il tableaux si costruisce induttivamente come segue:
si sceglie una foglia $l$ non etichettata dell'albero che verrà etichettata da un
insieme di formule $U(l)$ definite come:
\begin{enumerate}
  \item Se $U(l)$ è un insieme di letterali, si controlla se sono presenti una coppia
        di letterali complementati in $U(l)$;
        in caso siano presenti si marca la foglia come chiusa $\times$ altrimenti la foglia è aperta
  \item Se $U(l)$ non è un insieme di letterali si sceglie una formula in $U(l)$ tramite:
        \begin{itemize}
          \item Se la formula è un $\alpha$-formula $B$ si crea un nuovo nodo $l'$,
                figlio di $l$, e lo si etichetta come $U(l') = (U(l) - \{B\}) \cup \{\alpha_1,alpha_2\}$
          \item Se la formula è un $\beta$-formula $C$ si creano due nodi $l'$ e $l''$,
                figli di $l$ con $l'$ etichettato come: $U(l') = (U(l) - \{C\}) \cup \{\beta_1\}$
                mentre $l''$ è etichettata come $U(l'') = (U(l) - \{C\}) \cup \{\beta_2\}$
        \end{itemize}
\end{enumerate}
Il tableaux termina quando tutti i rami sono etichettati come chiusi e/o aperti.

%Definizione di Tableaux completo
\begin{defi}
Si definisce un tableaux \emph{completo} se la sua costruzione è complementata.
Il tableaux si dice \emph{chiuso} se tutte le foglie sono segnate come chiuse
altrimenti il tableaux è \emph{aperto}
\end{defi}

%Dimostrazione di tableau
Il metodo dei Tableau è un metodo dei sistemi deduttivi, che permette attraverso
l'applicazione di una serie di regole, di capire la tipologia della formula.
%Metodi per capire il tipo della Formula
\begin{tabular}{cccc}
\toprule Tipologia & Fare Tableau per & Chiuso? & Aperto? \\
\midrule
         Teorema & $\neg A$ & Si & No \\
         Soddisfacibile & $A$ & No & Si \\
         Contradditoria & $A$ & Si & No \\
\bottomrule
\end{tabular}

Esempio:$C \rightarrow (P \rightarrow ((Q \rightarrow \neg P) \lor (C \rightarrow P)))$
\begin{equation*}
\begin{prooftree}
\hypo{F C \rightarrow (P \rightarrow ((Q \rightarrow \neg P) \lor (C \rightarrow P)))}
\infer1 {TC,F (P \rightarrow ((Q \rightarrow \neg P) \lor (C \rightarrow P)))}
\infer1 {TC,TP,F (Q \rightarrow \neg P) \lor (C \rightarrow P)}
\infer1 {TC,TP,F (Q \rightarrow \neg P),F (C \rightarrow P)}
\infer1{TC,TP,F (Q \rightarrow \neg P),TC,FP}
\end{prooftree}
\end{equation*}
Il tableaux chiude in quanto non può essere contemporaneamente $TP$ e $FP$ per cui
la formula è una tautologia.

Esempio:
%Esempio dimostrazione per induzione
\begin{thm}
$\sum _{k=0} ^ n (4k+1)= (2n+1)(n+1)$
\end{thm}

\begin{proof}
Caso Base $n = 0$:
\begin{equation*}
    \sum _{k = 0} ^ 0 (4k+1) = (2*0 +1)(0+1) \quad 1 = 1 \ \text{vero}
\end{equation*}
Caso passo:
\quad Ipotesi:$\sum _{k = 0} ^ n (4k+1) = (2n+1)(n+1)$\newline
\quad Tesi: $\sum _{k = 0} ^ {n+1} (4k+1) = (2n+3)(n+2)$
\begin{equation*}
\begin{split}
\sum _{k = 0} ^ {n+1} (4k+1) & = \sum _{k = 0} ^ n (4k+1) + 4(n+1) + 1 \\
                             & = (2n+1)(n+1) + 4n + 5\\
                             & = 2n^2+3n+1+4n+5 \\
                             & = (2n+3)(n+2)\\
\end{split}
\end{equation*}
\end{proof}


%Completezza dei Tableaux
%Dimostrazione completezza Tableaux

%Dimostrazione correttezza Tableaux
