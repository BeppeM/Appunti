\documentclass[a4paper]{report}
\usepackage[T1]{fontenc}
\usepackage[utf8]{inputenc}
\usepackage[italian]{babel}

\begin{document}
L'induzione permette anche di definire nuovi insiemi nel seguente modo:
\begin{enumerate}
  \item si definisce un insieme di "oggetti base" appartenenti all'insieme.
  \item si definisce un insieme di operazioni di costruzione che, applicate ad elementi
        dell'insieme, producono nuovi elementi dell'insieme.
  \item nient'altro appartiene all'insieme definito.
\end{enumerate}

%Inserire Esempi
Esempio:Definizione induttiva di numeri naturali\newline
\begin{enumerate}
  \item $0 \in N$
  \item Se $x \in N$ allora $s(x) \in N$
  \item Nient'altro appartiene ai numeri naturali
\end{enumerate}

Esempio:espressione in Java
\begin{enumerate}
    \item le variabili e le costanti sono delle espressioni
    \item se $E_1$ e $E_2$ sono delle espressioni ed $op$ è un operatore binario,
          allora $E_1 op E_2$ è un espressione
    \item se $E_1$ e $E_2$ sono delle espressioni e $op$ è un operatore unario,
          allora $op E_1$ è un espressione
    \item nient'altro è un espressione
\end{enumerate}

La ricorsione è funzione definitoria che consiste nel definire un insieme
di elementi base e di definire gli altri elementi mediante il richiamo di se stessa,
fino ad arrivare ai casi base.

Esempio:
la definizione ricorsiva del fattoriale è definita come segue:
\begin{equation*}
    n! = \begin{cases} 1 \quad n = 0 \lor n = 1 \\ n * (n-1)! \quad n > 1\\
\end{cases}
\end{equation*}

la definizione del coefficiente binomiale è definita come segue:
\begin{equation*}
    \bigl( ^ n _ k \bigr) = \begin{cases} 1 \quad n = k \lor k = 0 \\
                             n \quad k = n-1 \lor k = 1 \\
                             (^{n-1} _{k-1}) + (^{n-1} _k) \quad \text{altrimenti} \\
                \end{cases}
\end{equation*}

la definizione di $somma:Z x Z \mapsto Z$ è la seguente:
\begin{equation*}
    somma(a,b) = \begin{cases} a \quad b = 0 \\
                               somma(b,a) \quad a < b \\
                               1 + somma(b-1) \quad b > 0\\
                               -1 + somma(b+1) \quad b < 0 \\
                  \end{cases}
\end{equation*}

\end{document}
